% LaTeX source for textbook ``How to think like a computer scientist''
% Copyright (C) 1999  Allen B. Downey

% This LaTeX source is free software; you can redistribute it and/or
% modify it under the terms of the GNU General Public License as
% published by the Free Software Foundation (version 2).

% This LaTeX source is distributed in the hope that it will be useful,
% but WITHOUT ANY WARRANTY; without even the implied warranty of
% MERCHANTABILITY or FITNESS FOR A PARTICULAR PURPOSE.  See the GNU
% General Public License for more details.

% Compiling this LaTeX source has the effect of generating
% a device-independent representation of a textbook, which
% can be converted to other formats and printed.  All intermediate
% representations (including DVI and Postscript), and all printed
% copies of the textbook are also covered by the GNU General
% Public License.

% This distribution includes a file named COPYING that contains the text
% of the GNU General Public License.  If it is missing, you can obtain
% it from www.gnu.org or by writing to the Free Software Foundation,
% Inc., 59 Temple Place - Suite 330, Boston, MA 02111-1307, USA.


\documentclass[a4paper]{book}
\usepackage[ngerman, english]{babel}
\usepackage[latin1]{inputenc}
\usepackage[T1]{fontenc}
\usepackage{lmodern}
\usepackage{epsfig}
\usepackage{makeidx}
\usepackage{url}
\usepackage{float}
\usepackage{fancyhdr}
\usepackage{multicol}
\usepackage{hyperref}
\usepackage{longtable}
\usepackage{ifthen}
\usepackage{boxedminipage}
\usepackage{marginnote}
\usepackage{eso-pic}
%\usepackage{todonotes}
\usepackage[disable]{todonotes}
\usepackage{paralist} 

%\usepackage{calc}
%\newsavebox{\fcolbox} \newlength{\fcolwidth}
%\newenvironment{boxedminipage}[2][c]
%  {\setlength{\fcolwidth}{#2-2\fboxsep-2\fboxrule}%
%   \begin{lrbox}{\fcolbox}%
%   \begin{minipage}[#1]{\fcolwidth}}
%  {\end{minipage}\end{lrbox}\fbox{\usebox{\fcolbox}}}

% conditional compilation of the document for different languages

\newboolean{German}
\setboolean{German}{true}
\selectlanguage{ngerman}

\renewcommand{\marginnotevadjust}{-3em} 
\newcommand{\hint}{\marginnote{ \includegraphics[width=30pt]{figs/Hint.pdf}}}

% the exercise environment

\newcounter{exercisenum}                                                  
     
% by default, the exercise number includes the chapter number             
% this way, an exercise label is a complete, unique exercise id           
     
\renewcommand{\theexercisenum}{{\thechapter}.\arabic{exercisenum}}  

% Standard font size for exercise/problem text                            

\newenvironment{exercisesize}{\begin{small}}{\end{small}}                 

\newcommand{\exerciseheader}[2]{                                          
     
  \begin{exercisesize}                                                    
     
  % Use alphabetic chars for subparts of exercises,                            
  % and roman numerals for subparts of them.
     
  \def\theenumi{\alph{enumi}}                                             
  \def\labelenumi{\theenumi.}                                             
  \def\theenumii{\roman{enumii}}                                          
  \def\labelenumii{\theenumii.}
  
  \ifthenelse {\boolean{German}}{{\bf �bung {#1}{#2}}\hspace{0.1in}  }{{\bf Exercise {#1}{#2}}\hspace{0.1in}}
                                        
%  {\bf Exercise {#1}{#2}}\hspace{0.1in}                 
}                                                                         

\newcommand{\startexercise}[1]{%
  \refstepcounter{exercisenum}                                            
  \exerciseheader{\theexercisenum}{#1}                                    
}                                                                         

\newcommand{\stopexercise}{%                                                   
  {\hfill}                                                               
  \end{exercisesize}      
}                                                         
     
\newcommand{\normaldif}{}                                                 
     
\newcommand{\bigdif}{\dag{}}                                              
     
\newcommand{\verybigdif}{\ddag{}}             

\newenvironment{exercise}{\startexercise{\normaldif{}}}{\stopexercise}    
     
\newenvironment{hardexercise}{\startexercise{\bigdif{}}}{\stopexercise}   
     
%% end of the exercise environment


%%------------------------------------------------------------
% formatting commands

\sloppy
\setlength{\topmargin}{0.125in}
\setlength{\oddsidemargin}{0.875in}
\setlength{\evensidemargin}{0.875in}

\setlength{\headsep}{3ex}
\setlength{\textheight}{8in}

\setlength{\parindent}{0.0in}
\setlength{\parskip}{1.7ex plus 0.5ex minus 0.5ex}
\renewcommand{\baselinestretch}{1.02}

% see LaTeX Companion page 62
\setlength{\topsep}{-0.0\parskip}
\setlength{\partopsep}{-0.5\parskip}
\setlength{\itemindent}{0.0in}
\setlength{\listparindent}{0.0in}

% see LaTeX Companion page 26
% these are copied from /usr/local/teTeX/share/texmf/tex/latex/base/book.cls
% all I changed is afterskip

\makeatletter
\renewcommand{\section}{\@startsection 
    {section} {1} {0mm}%
    {-3.5ex \@plus -1ex \@minus -.2ex}%
    {0.7ex \@plus.2ex}%
    {\normalfont\Large\bfseries}}
\renewcommand\subsection{\@startsection {subsection}{2}{0mm}%
    {-3.25ex\@plus -1ex \@minus -.2ex}%
    {0.3ex \@plus .2ex}%
    {\normalfont\large\bfseries}}
\renewcommand\subsubsection{\@startsection {subsubsection}{3}{0mm}%
    {-3.25ex\@plus -1ex \@minus -.2ex}%
    {0.3ex \@plus .2ex}%
    {\normalfont\normalsize\bfseries}}

\makeatother

\newcommand{\beforeverb}{\vspace{0.6\parskip}}
\newcommand{\afterverb}{\vspace{0.6\parskip}}

\newcommand{\adjustpage}[1]{\enlargethispage{#1\baselineskip}}
\newcommand{\clearemptydoublepage}{\newpage{\pagestyle{empty}\cleardoublepage}}
\newcommand{\blankpage}{\pagestyle{empty}\vspace*{1in}\newpage}

\newcommand{\beforefig}{\vspace{1.3\parskip}}
\newcommand{\afterfig}{\vspace{-0.2\parskip}}
\newcommand{\myfig}[1]{
    \beforefig
    \centerline{\epsfig{#1,scale=0.8}}
    \afterfig
}

\newcommand{\beforechapter}{
%    \clearemptydoublepage 
    \cleardoublepage 
    \setcounter{exercisenum}{0}
}

\pagestyle{fancyplain}

\renewcommand{\chaptermark}[1]{\markboth{#1}{}}
\renewcommand{\sectionmark}[1]{\markright{\thesection\ #1}{}}

\lhead[\fancyplain{}{\bfseries\thepage}]%
      {\fancyplain{}{\bfseries\rightmark}}
\rhead[\fancyplain{}{\bfseries\leftmark}]%
      {\fancyplain{}{\bfseries\thepage}}
\cfoot{}

% turn off the rule under the header
%\setlength{\headrulewidth}{0pt}

% the following is a brute-force way to prevent the headers
% from getting transformed into all-caps
\renewcommand\MakeUppercase{}


\sloppy
\setlength{\topmargin}{0.75in}
\setlength{\headsep}{0.5in}
\setlength{\oddsidemargin}{1.0in}
\setlength{\evensidemargin}{.95in}
\makeindex


%%-----------------------------------------------------------
% beginning of the document

\begin{document}

%\title {How to think like a computer scientist}
%\author {Allen B. Downey, Thomas Scheffler}
%\date {C Version, First Edition}
%\maketitle

%\vspace{2in}
%\begin{center}
%{\Large How to think like a computer scientist}

%C Version, First Edition
%\vspace{0.25in}


%\begin{flushright}
%\vspace*{2.5in}

%{\huge How to Think Like a Computer Scientist}

%\vspace{1in}

%{\LARGE C Version}

%\vfill

%\end{flushright}

%--verso------------------------------------------------------

%\clearemptydoublepage
%\cleardoublepage

%%--title page--------------------------------------------------
%\pagebreak
\thispagestyle{empty}

\begin{flushright}
\vspace*{2.5in}

{\huge How to Think Like a Computer Scientist}

\vspace{0.25in}

{\LARGE C Version}

\vspace{1in}

{\Large Allen B. Downey, Thomas Scheffler}

{ C-Version und deutsche �bersetzung: Thomas Scheffler}

\vspace{1in}

{\Large Version 0.9.5}

{\small 6. Dezember 2016}
\todo{Flowchart!}
\vfill

\end{flushright}



Copyright (C) 1999  Allen B. Downey, 2016 Thomas Scheffler
%\end{center}
\vspace{0.25in}

This book is an Open Source Textbook (OST).  Permission is
granted to reproduce, store or transmit the text of this
book by any means, electrical, mechanical, or biological,
in accordance with the terms of the GNU General Public License as
published by the Free Software Foundation (version 2).

This book is distributed in the hope that it will be useful,
but WITHOUT ANY WARRANTY; without even the implied warranty of
MERCHANTABILITY or FITNESS FOR A PARTICULAR PURPOSE.  See the GNU
General Public License for more details.

The original form of this book is LaTeX source code.
Compiling this LaTeX source has the effect of generating
a device-independent representation of a textbook, which
can be converted to other formats and printed.  All intermediate
representations (including DVI and Postscript), and all printed
copies of the textbook are also covered by the GNU General
Public License.

The LaTeX source for this book, and more information about
the Open Source Textbook project, is available from

%\begin{verbatim}
     \url{https://prof.beuth-hochschule.de/scheffler/}
%\end{verbatim}

or by writing to Allen B. Downey, 5850 Mayflower Hill,
Waterville, ME 04901.

The GNU General Public License is available from
www.gnu.org or by writing to the Free Software Foundation,
Inc., 59 Temple Place - Suite 330, Boston, MA 02111-1307, USA.

This book was typeset by the author using LaTeX and dvips,
which are both free, open-source programs.

\selectlanguage{ngerman}
\frontmatter
\tableofcontents

\mainmatter
%!TEX root = Main_german.tex


\selectlanguage{ngerman}
\chapter{Achtung, jetzt kommt ein Programm!}
\label{chap01}

Das Ziel dieses Buches ist es, das Verst�ndnis daf�r zu wecken,
wie Informatiker denken. Ich mag es, wie Informatiker denken,
weil sie sich dabei der unterschiedlichen Ans�tze aus der Mathematik,
der Ingenieurwissenschaften und der Sprachwissenschaften bedienen,
um mit viel Kreativit�t, Ausdauer und Beharrlichkeit etwas Neues noch
nicht Dagewesenes zu schaffen.  

Informatiker benutzen dabei, wie die Mathematiker, \emph{formale Sprachen}
um ihre Ideen und Berechnungen aufzuschreiben. 
Sie entwerfen Dinge und konstruieren komplexe Systeme, die sie
aus einzelnen Komponenten zusammenbauen, und m�ssen dabei
verschiedene Alternativen bewerten, abw�gen und ausw�hlen.
Sie  beobachten  das Verhalten dieser Systeme wie Wissenschaftler:
sie formulieren Hypothesen und testen ihre Vorhersagen.
  
Die wichtigste F�higkeit eines Informatikers besteht darin, \textbf{Probleme
zu l�sen}.
Dazu muss er diese Probleme erkennen, geschickt formulieren, 
kreativ �ber m�gliche L�sungen nachdenken und diese
klar, �bersichtlich und nachvollziehbar ausdr�cken und darstellen k�nnen.
So wie es sich herausstellt, ist  der Prozess des Erlernen einer Programmiersprache
eine exzellente M�glichkeit, sich in der F�higkeit des Probleml�sens
zu �ben. Deshalb hei�t dieses Kapitel ``Achtung, jetzt kommt ein Programm!''

%As it turns out, the process of learning to program is an
%excellent opportunity to practice problem-solving skills.  That's why
%this chapter is called ``The way of the program.''

Das Erlernen des Programmierens ist f�r sich allein genommen bereits 
eine n�tzliche F�higkeit. Je mehr wir uns in der F�higkeit �ben, um so 
offensichtlicher wird es werden, dass wir das Programmieren auch als
ein Mittel zum Zweck nutzen k�nnen. Dass es sich dabei um ein sehr leistungsf�higes
Mittel handelt, wird hoffentlich im Laufe
des Buches noch klarer werden. 
 

\section{Was ist eine Programmiersprache?}
\index{Programmiersprache}
\index{Sprache!Programmierung}

Die Programmiersprache, welche wir in diesem Kurs lernen werden, hei�t 
C und wurde in den fr�hen 1970er Jahren von  Dennis M. Ritchie 
in den Bell Laboratories entwickelt.  C ist eine so genannte 
{\bf Hochsprache} oder {\bf  High-level Sprache}. Andere Hochsprachen, die
in der Programmierung verwendet werden, sind Pascal, Python, C++ und Java.

Aus dem Namen ``Hochsprache'' kann man ableiten, dass
 auch sogenannte {\bf Low-level Sprachen} existieren. Diese nennt man 
Maschinensprache oder auch Assembler. 
Computer k�nnen nur Programme in Maschinensprache ausf�hren.
Es ist daher notwendig, die Programme, welche in einer Hochsprache geschrieben
wurden, in Maschinensprache zu �bersetzen.
Diese �bersetzung ben�tigt Zeit und einen zus�tzlichen Arbeitsschritt, was
einen klitzekleinen Nachteil gegen�ber Low-level Sprachen darstellt.

\index{portabel}
\index{High-level Sprache}
\index{Low-level Sprache}
\index{Sprache!high-level}
\index{Sprache!low-level}

Allerdings sind die Vorteile von Hochsprachen enorm.
So ist es, erstens, {\em viel} einfacher in einer Hochsprache zu programmieren.
Mit ``einfacher'' meine ich, dass es weniger Zeit in Anspruch nimmt ein Programm
zu schreiben. Das Programm ist k�rzer, einfacher zu lesen und mit einer h�heren
Wahrscheinlichkeit auch korrekt. Das hei�t, es tut, was wir von dem Programm erwarten.
High-level Sprachen verf�gen �ber eine zweite wichtige Eigenschaft, sie sind {\bf portabel}.
Das bedeutet, dass unser Programm auf unterschiedlichen Arten von 
Computern ausgef�hrt werden kann. Maschinensprachen sind jeweils nur f�r
eine bestimmte Computerarchitektur definiert. Programme, die in 
Low-level Sprachen erstellt wurden, m�ssten komplett neu geschrieben werden, wenn
in unserem Computer statt einem Intel-kompatiblen Prozessor ein Prozessor von  ARM
verwendet w�rde. Ein Programm in einer Hochsprache muss einfach nur neu
�bersetzt werden.

Aufgrund dieser Vorteile wird die �berwiegende
Anzahl von Programmen in Hochsprachen geschrieben.
Low-level Sprachen werden nur noch f�r wenige Spezialanwendungen verwendet.

\index{Kompilieren}
\index{Interpretieren}

F�r die �bersetzung unseres Programms ben�tigen wir eine bestimmte 
Software auf unserem Computer - den �bersetzer.
Es existieren grunds�tzlich zwei Wege ein Programm zu �bersetzen:
{\bf Interpretieren} oder {\bf Kompilieren}.  Ein \emph{Interpreter}
ist ein Programm welches ein High-level Programm
interpretiert und ausf�hrt.  Dazu �bersetzt der Interpreter das
Programm Zeile-f�r-Zeile und f�hrt nach jeder �bersetzten
Programmzeile die darin enthaltenen Kommandos sofort aus.

\myfig{figure=figs/interpret.eps}

Ein \emph{Compiler} ist ein Programm, welches ein High-level Programm
im Ganzen einliest und �bersetzt. Dabei wird stets das gesamte Programm
komplett �bersetzt, bevor die einzelnen Kommandos des Programms
ausgef�hrt werden k�nnen. Die Programmiersprache C verwendet einen
Compiler f�r die �bersetzung der Befehle in Maschinensprache.

Durch das Kompilieren entsteht eine neue, ausf�hrbare Datei.
Es wird dazu ein Programm zuerst kompiliert und das so �bersetzte 
Programm in einem zweiten, separaten Schritt zur Ausf�hrung gebracht.
Man bezeichnet in diesem Fall das High-level
Programm als den  {\bf Source code} oder {\bf Quelltext}, und das
�bersetzte Programm nennt man den {\bf Object code} oder das
{\bf ausf�hrbare Programm}.

%(Pr�fungsfrage: Interpreter / Compiler, anzahl von Dateien)

Angenommen, wir schreiben unser erstes Programm in C.
Wir k�nnen daf�r einen ganz einfachen Texteditor benutzen,
um das Programm aufzuschreiben (ein Texteditor ist ein ganz einfaches
Textverarbeitungsprogramm, welches in der Regel nicht einmal
verschiedene Schriftarten darstellen kann). 
Wenn wir das Programm aufgeschrieben haben, m�ssen 
wir es auf der Festplatte des Computers speichern, zum Beispiel unter
dem Namen {\tt program.c}, wobei ``program''
ein beliebiger, selbstgew�hlter Dateiname ist. Die Dateiendung {\tt .c} 
ist wichtig, weil sie einen Hinweis darauf gibt, dass es sich bei dieser
Datei um Quellcode in der Programmiersprache C handelt.

Danach k�nnen wir den Texteditor schlie�en und den Compiler aufrufen
(der genaue Ablauf h�ngt dabei von der verwendeten Programmierumgebung ab).
Der Compiler lie�t den Quelltext, �bersetzt ihn und erzeugt eine neue
Datei mit dem Namen {\tt program.o}, welches den Objektcode enth�lt,
oder die Datei {\tt program.exe}, welche das ausf�hrbare Programm enth�lt. 

\myfig{figure=figs/compile.eps}

Im n�chsten Schritt k�nnen wir das Programm ausf�hren lassen. Dazu
wird das Programm in den Hauptspeicher des Rechners geladen (von
der Festplatte in den Arbeitsspeicher kopiert) und danach werden die 
einzelnen Anweisungen des Programms ausgef�hrt.  

Dieser Prozess klingt erst einmal sehr kompliziert. 
Allerdings sind in den meisten Programmierumgebungen (auch
Entwicklungsumgebungen genannt) viele dieser Schritte automatisiert.
In der Regel schreibt man dort sein Programm, klickt mit der Maus auf einen 
Bildschirmsymbol oder gibt ein einzelnes Kommando ein und das Programm
wird �bersetzt und ausgef�hrt.
Allerdings ist es immer gut zu wissen, welche Schritte im Hintergrund
stattfinden. So kann man, im Fall dass etwas schief geht, herausfinden,
wo der Fehler steckt.

% Leftover: when is compilation better than interpretation?
% �bungsaufgabe: Vorteile von kompilierten Programmen:
% Weitergabe des fertigen Programms, Unver�nderbarkeit, Geschwindigkeit,...

\section{Was ist ein Programm?}

Ein Programm ist eine Abfolge von Befehlen (engl.: \emph{instructions}), welche
angeben, wie eine Berechnung durchgef�hrt wird. 
Diese Berechnung kann mathematischer Art sein, wie zum Beispiel
das L�sen eines Gleichungssystems oder die Ermittlung der Quadratwurzel
eines Polynoms. Es kann aber auch eine symbolische Berechnung
sein, wie die Aufgabe, in einem Dokument einen bestimmten Text zu finden
und zu ersetzen. Erstaunlicherweise kann dies auch das Kompilieren eines
Programmes sein.

\index{Anweisung}

Die Programmbefehle, welche sich  {\bf Anweisungen} (engl.:  {\bf  \emph{statements}}) 
nennen, sehen in unterschiedlichen Programmiersprachen verschieden aus.
Es existieren aber in allen Sprachen die gleichen, wenigen Basiskategorien, 
aus denen ein Computerprogramm aufgebaut ist. Es ist deshalb nicht schwer 
eine neue Programmiersprache zu lernen, wenn man bereits eine andere gut
beherrscht.
Die meisten Programmiersprachen unterst�tzen die folgenden Befehlskategorien:

\begin{description}

\item[Input:] Daten von der Tastatur, aus einer Datei oder von einem angeschlossenen Ger�t
in das Programm einlesen.

\item[Output:] Daten auf dem Monitor darstellen, in eine Datei schreiben oder an ein 
angeschlossenes Ger�t ausgeben. 

\item[Mathematik:] Durchf�hren von grundlegenden mathematischen Operationen, wie zum
Beispiel Addition und Multiplikation.

\item[Testen und Vergleichen:] �berpr�fen, ob bestimmte Bedingungen erf�llt sind
und die Steuerung der Ausf�hrung bestimmter Abfolgen von Anweisungen in Abh�ngigkeit
von diesen Bedingungen. 

\item[Wiederholung:] Bestimmte Aktionen werden mehrfach, manchmal mit geringen �nderungen,
nacheinander ausgef�hrt. 

\end{description}

Das w�re dann schon fast alles.
Jedes Programm, das wir in diesem Kurs kennenlernen, ist unabh�ngig von seiner
Komplexit�t, aus einzelnen Anweisungen aufgebaut, welche diese Operationen unterst�tzen.
Daher besteht ein Ansatz der Programmierung darin, einen gro�en komplizierten Prozess in
immer kleinere und kleinere Unteraufgaben zu unterteilen, bis die einzelnen Aufgaben 
so klein und unbedeutend werden, dass sie mit einem dieser grundlegenden Befehle
ausgef�hrt werden kann.

\section{Was ist \textit{debugging}?}
\index{Debugging}
\index{Bug}

Programmieren ist ein komplexer Prozess, und da es von Menschen
durchgef�hrt wird, ist es mehr als wahrscheinlich, dass 
sich hier und dort Fehler einstellen.
Programmierer bezeichnen einen Softwarefehler �blicherweise
als  {\bf Error} oder auch {\bf Bug} und den Prozess des Aufsp�rens und Korrigierens
des Fehlers als {\bf Debugging}.

Es gibt verschiedene Arten von Fehlern, die in einem 
Programm auftreten k�nnen. Es ist sinnvoll, die
Unterscheidung zwischen diesen Arten zu kennen, um
Fehler in eigenen Programmen schneller entdecken 
und beheben zu k�nnen.
Programmfehler k�nnen sich zu unterschiedlichen Zeiten
bemerkbar machen. Man unterscheidet zwischen Fehlern 
beim Kompilieren und beim Ausf�hren des Programms. 

\subsection{Fehler beim Kompilieren (Compile-time errors)}
\index{Compile-time error}
\index{Error!compile-time}

Der Compiler kann ein Programm nur �bersetzen wenn dieses
Programm den formalen Regeln der Programmiersprache
entspricht. 
Diese Regeln, die {\bf Syntax}, beschreiben die Struktur des 
Programms und der darin enthaltenen Anweisungen.
Ein Programm muss syntaktisch korrekt sein, anderenfalls 
schl�gt die Kompilierung fehl und das Programm
kann nicht ausgef�hrt werden.


\index{Syntax}

So beginnt zum Beispiel jeder deutsche Satz mit einem gro�en 
Buchstaben und endet mit einem Punkt.
\emph{dieser Satz enth�lt einen Syntaxfehler. Dieser Satz ebenfalls}

F�r die meisten Menschen sind ein paar Syntaxfehler in einem
Text kein gr��eres Problem. Wir k�nnen diese Texte trotzdem
in ihrer Bedeutung verstehen, weil wir �ber Erfahrung und
Weltwissen verf�gen.

Compiler sind nicht so tolerant. Selbst ein einzelner Syntaxfehler
irgendwo in unserem Programm f�hrt dazu, dass der Compiler
eine Fehlermeldung (engl.: \emph{error message}) auf dem Bildschirm anzeigt
und die weitere Arbeit des �bersetzens einstellt. Das erzeugte
Programm kann nicht ausgef�hrt werden.

Zu allem �berfluss gibt es sehr viele Syntaxregeln in C, 
und die Fehlermeldungen des Compilers sind oft nicht besonders
hilfreich f�r den Programmieranf�nger. 
Der ber�hmte Physiker Niels Bohr hat einmal gesagt: 
``Ein Experte ist jemand, der in einem begrenzten Bereich 
schon alle m�glichen Fehler gemacht hat.'' 
W�hrend der ersten paar Wochen unserer Karriere als
C-Programmiererin oder C-Programmierer werden wir voraussichtlich viel Zeit damit 
zubringen, Syntaxfehler in selbst erstellten Programmen zu finden.
In dem Ma�e wie unsere Erfahrung zunimmt, werden wir
weniger Fehler machen und die gemachten Fehler schneller finden.



\subsection{Fehler beim Ablauf des Programms (Run-time errors)}
\label{run-time}
\index{Run-time error}
\index{Error!run-time}
\index{Sprache!safe}

Eine zweite Kategorie von Programmfehlern sind die so genannten Laufzeitfehler (engl.: \emph{run-time errors}).
Sie werden so genannt, weil der Fehler erst auftritt, wenn unser Programm ausgef�hrt wird: zur Laufzeit.

Auch wenn wir unser Programm syntaktisch richtig aufgeschrieben haben, k�nnen
Fehler auftreten die zum Abbruch des Programms f�hren. 
C ist keine  {\bf sichere} Sprache, wie zum Beispiel Java, wo Laufzeitfehler relativ 
selten sind.
C ist eine relativ hardwarenahe Programmiersprache. Vielleicht
die hardwaren�heste von allen h�heren Programmiersprachen.
Die meisten Laufzeitfehler in C treten deshalb auf, weil die Sprache selbst
keine Schutzmechanismen gegen den direkten Zugriff auf den Speicher des
Computers bietet. So kann es vorkommen, dass unser Programm wichtige 
Speicherbereiche versehentlich �berschreibt.

Die Hardwaren�he von C hat ihre guten, wie ihre schlechten Seiten.
Viele Algorithmen sind dadurch in C besonders effizient und leistungsf�hig 
umsetzbar. Gleichzeitig sind wir als Programmierer selbst 
daf�r verantwortlich, dass unser Programm auch nur das tut, was
wir beabsichtigen. Daf�r m�ssen wir manchmal sorgf�ltiger arbeiten
als Programmierer anderer Sprachen.

Bei den einfachen Programmen, die wir in den n�chsten Wochen schreiben
werden, ist es allerdings unwahrscheinlich, dass wir in unserem Programm
einen Laufzeitfehler provozieren.


\subsection{Logische Fehler und Semantik}
\index{Semantik}
\index{Logische Fehler}
\index{Fehler!Logische}

Der dritte Fehlertyp, dem wir begegnen werden, ist der {\bf logische} 
oder {\bf semantische} Fehler.  
Wenn in unserem Programm ein logischer Fehler steckt, so wird
es zun�chst  kaum auffallen. Es l�sst sich kompilieren und ausf�hren,
ohne dass der Computer irgendwelche Fehlermeldungen produziert.
Es wird aber leider nicht die richtigen Dinge tun. 
Es wird einfach irgend etwas anderes tun, oder auch gar nichts.
Insbesondere wird es nicht das tun, wof�r wir das Programm
geschrieben haben.

Das  Programm, das wir geschrieben haben, ist nicht
das Programm, das wir schreiben wollten. Die Bedeutung des 
Programms - seine Semantik - ist falsch.
Die Gr�nde daf�r k�nnen vielf�ltig sein. Am Anfang sind es 
vor allem unklare Vorstellungen dar�ber, was das Programm tun soll und
wie ich das mit den Mitteln der Programmiersprache C erreichen kann.
Es kann aber auch sein, dass unser Algorithmus fehlerhaft ist,
oder wir nicht alle Voraussetzungen und Bedingungen vollst�ndig gepr�ft haben.

Logische Fehler zu finden ist meistens ziemlich schwer. Es erfordert
Zeit und Geduld herauszufinden, wo der Fehler steckt. 
Dazu kann es notwendig sein r�ckw�rts zu arbeiten: Wir schauen
uns die Resultate unseres Programms an und �berlegen, was
zu diesen Ergebnissen gef�hrt haben kann.

\subsection{Experimentelles Debugging}

Eine der wichtigsten F�higkeiten, die wir als angehende Programmierer
erlernen m�ssen, ist das Debuggen von Programmen.
Obwohl es gelegentlich auch frustrierend sein kann, so
ist das Aufsp�hren von Programmfehlern ein intellektuell
anspruchsvoller, herausfordernder und interessanter 
Teil des Programmierens.

In vielerlei Hinsicht ist Debugging mit der Arbeit eines Kriminalisten vergleichbar.
Man muss Hinweisen nachgehen und  Zusammenh�nge
herstellen zwischen den Prozessen innerhalb des Programms
und den Resultaten, die sichtbar sind.

Debugging ist gleichfalls den experimentellen Wissenschaften
�hnlich. Sobald wir eine Idee haben, was in unserem Programm
falsch gelaufen sein sollte, ver�ndern wir dieses und beobachten erneut.
Wir bilden Hypothesen �ber das Verhalten des Programms.
Stimmt unsere Hypothese, dann k�nnen wir das Ergebnis
der Modifikation vorhersagen und wir sind dem Ziel, eines
funktionsf�higen Programms, einen Schritt n�her gekommen.

Wenn unsere Hypothese falsch war, m�ssen wir eine neue bilden.
Wie bereits Sherlock Holmes sagte,  ``...when you have eliminated the
impossible, whatever remains, however improbable, must be the truth''
(aus Arthur Conan Doyle, {\em Das Zeichen der Vier}).

\index{Holmes, Sherlock}
\index{Doyle, Arthur Conan}

Einige Leute betrachten Programmieren und Debuggen als
ein und dieselbe Sache.
Man k�nnte auch sagen, Programmieren ist der Prozess, ein Programm 
so lange zu debuggen bis am Ende ein funktionsf�higes
Programm entstanden ist, das unseren Vorstellungen entspricht.
Dahinter steckt die Idee, dass wir immer mit einem
funktionsf�higen Programm starten, welches \emph{irgendeine} Funktion
realisiert.
Danach machen wir kleine Modifikationen, entfernen die Fehler und
testen unser Programm, so dass wir zu jeder Zeit ein funktionsf�higes 
Programm haben, welches am Ende eine neue Funktion realisiert. 
 
So ist zum Beispiel Linux ein Betriebssystem, welches Millionen von 
Programmzeilen enth�lt. Begonnen wurde es aber als ein einfaches
Programm, dass Linus Torvalds benutzt hat um den Intel 80386 kennenzulernen.  
So berichtet Larry Greenfield: 
``One of Linus's earlier projects was a program that would switch
between printing AAAA and BBBB.  This later evolved to Linux''
(aus {\em The Linux Users' Guide}, Beta Version 1).

\index{Linux}

In sp�teren Kapiteln werde ich einige praktische Hinweise zum Debugging
und anderen Programmierpraktiken geben. 

\section{Formale und nat�rliche Sprachen}
\index{Formale Sprache}
\index{Nat�rliche Sprache}
\index{Sprache!Formal}
\index{Sprache!Nat�rliche}
\index{Algorithmus}

Als {\bf nat�rliche Sprachen} bezeichnen wir alle Sprachen, die
von Menschen gesprochen werden, wie zum Beispiel Englisch, Spanisch 
und Deutsch. Diese Sprachen wurden nicht von Menschen konstruiert 
(obwohl Menschen versuchen ihnen Ordnung und Struktur zu verleihen),
sie sind das Resultat eines nat�rlichen Evolutionsprozesses. 

{\bf Formale Sprachen}  sind vom Menschen f�r spezielle 
Anwendungszwecke konstruiert worden.
So benutzen zum Beispiel Mathematiker spezielle 
Notationsformen, um den Zusammenhang zwischen Symbolen, Zahlen
und Mengen darzustellen. 
Chemiker benutzen eine formale Sprache, um die chemische Struktur von Molek�len zu notieren.  Und
f�r uns ganz wichtig:

\begin{quote}
{\bf Programmiersprachen sind formale Sprachen, mit denen man das 
Verhalten einer Maschine steuert.}
\end{quote}

\begin{quote}
Programmiersprachen dienen der Beschreibung von Berechnungen
und der Formulierung von Algorithmen.
Ein \textbf{Algorithmus} ist eine aus abz�hlbar vielen Schritten bestehende 
eindeutige Handlungsanweisung zur L�sung einer Klasse von Problemen.
\end{quote}

Wie ich bereits angedeutet hatte, tendieren formale Sprachen dazu,
strikte Syntaxregeln zu besitzen.
So ist zum Beispiel $3+3=6$ ein syntaktisch korrekter mathematischer
Ausdruck, $3=:6\$$ hingegen nicht.  $H_2O$ ist eine syntaktisch korrekte
chemische Formel, $_2Z$ ist es nicht.

Syntaxregeln betreffen zwei Bereiche der Sprache: deren Symbole und Struktur.
Die Symbole stellen die Basiselemente einer Sprache bereit, dazu geh�ren
die W�rter und Bezeichner, Zahlen und chemische Elemente.
Eines der Probleme mit dem Ausdruck {\tt 3=:6\$} ist, dass {\tt \$} kein
legales Symbol in der Mathematik darstellt (so weit ich wei�, jedenfalls ...).  
Gleichfalls ist, $_2Z$ nicht legal, weil es kein chemisches Element mit
der Abk�rzung $Z$ gibt.

Der zweite Fall f�r die Anwendung von Syntaxregeln betrifft die Struktur
eines Ausdrucks, das hei�t die Art und Weise, wie die Symbole der Sprache
angeordnet werden. 

Der Ausdruck {\tt 3=:6\$} ist auch deshalb nicht korrekt, weil es nicht 
erlaubt ist, ein Divisionszeichen unmittelbar nach einem Gleichheitsszeichen
zu schreiben.
In gleicher Weise werden in molekularen Formeln die Mengenverh�ltnisse
eines Elements als tiefer gestellte Zahl nach dem Elementname angegeben
und nicht davor.

Wenn wir einen Satz in einer nat�rlichen Sprache lesen oder
 einen Ausdruck in einer formalen Sprache erfassen wollen, m�ssen
wir seine Struktur herausfinden (bei nat�rlichen Sprachen macht unser
Gehirn, das meistens ganz unbewu�t von selbst). Diesen Prozess 
bezeichnen Informatiker als {\bf Parsen}.

\index{parsen}

Wenn wir zum Beispiel den Satz h�ren  ``Die W�rfel sind gefallen.'' 
erkennen wir,  ``Die W�rfel'' als das Subjekt und ``sind gefallen'' als das
Pr�dikat.  
Nachdem wir den Satz geparst haben, k�nnen wir herausfinden, was
dieser Satz bedeutet. Wir k�nnen seine Bedeutung (seine Semantik)
verstehen. Angenommen wir wissen,
was ein W�rfel ist und was es bedeutet zu fallen, so k�nnen wir damit
die generelle Bedeutung dieses Satzes verstehen. 

\subsection{Unterschiede formaler und nat�rlicher Sprachen}

Obwohl formale und nat�rliche Sprachen viele Gemeinsamkeiten 
aufweisen: Symbole, Struktur, Syntax und Semantik, so 
existieren doch auch viele Unterschiede zwischen den Sprachen.

\index{Vieldeutigkeit}
\index{Redundanz}
\index{Wortw�rtlichkeit}

\begin{description}

\item[Vieldeutigkeit:] Nat�rliche Sprachen sind voll von Mehrdeutigkeiten. 
Wir Menschen erkennen die Bedeutung von Aussagen in nat�rlicher Sprache 
�blicherweise anhand von Hinweisen aus dem Kontext und unserem Erfahrungswissen.
So ist zum Beispiel die Aussage ``wilde Tiere jagen'' nicht eindeutig. Es kann bedeuten,
dass der J�ger wilde Tiere jagt, aber auch, dass wilde Tiere ihre Beute jagen.
Formale Sprachen sind �blicherweise so konstruiert, dass sie keine Mehrdeutigkeiten
aufweisen. Jede Aussage ist eindeutig interpretierbar. Aus {\tt 2+2}  l�sst sich  
immer und eindeutigerweise der Wert {\tt 4} ableiten. 
Die Bedeutung (Semantik) einer Aussage ist nicht von ihrem Kontext abh�ngig.


\item[Redundanz:] Um mit den vorhandenen Mehrdeutigkeiten in nat�rlichen
Sprachen umzugehen und Missverst�ndnisse zu reduzieren, verwenden
diese Sprachen oft das Mittel der Redundanz. 
Das hei�t, Informationen werden mehrfach wiederholt, zum Teil in anderen
Formulierungen, obwohl dies eigentlich f�r das Verst�ndnis der Bedeutung
nicht notwendig w�re. Als Resultat sind nat�rliche Sprachen oft wortreich und
ausschweifend. W�hrend formale Sprachen wenig oder keine Redundanz 
aufweisen und dadurch knapp und pr�zise ausfallen.

\item[Wortw�rtlichkeit:] Nat�rliche  Sprachen beinhalten oft Redensarten und
Metaphern.  Wenn ich sage, ``Die W�rfel sind gefallen.'', dann sind  
wahrscheinlich nirgends W�rfel im Spiel, und es ist auch nichts heruntergefallen. 
Formale Sprachen hingegen, meinen wortw�rtlich genau das, was geschrieben steht.

\end{description}

Viele Menschen, die ganz selbstverst�ndlich eine nat�rliche Sprache verwenden
(wir alle), haben oft Schwierigkeiten im Umgang mit formalen Sprachen.
In vieler Art ist der Unterschied zwischen formalen und nat�rlichen Sprachen
wie der Unterschied zwischen Poesie und Prosa -- nur noch viel ausgepr�gter: 


\index{Poesie}
\index{Prosa}

\begin{description}

\item[Poesie:] W�rter werden wegen ihrer Bedeutung, manchmal aber auch
nur wegen ihres Klangs benutzt. Gedichte werden zum Teil wegen ihres
Effekts oder der emotionalen Reaktion beim Leser geschrieben.
Mehrdeutigkeiten kommen oft vor und werden vom Dichter stellenweise
als Stilmittel bewusst eingesetzt.

\item[Prosa:] Die w�rtliche Bedeutung eines Textes ist von Bedeutung und
seine Struktur tr�gt dazu bei, das Verst�ndnis seiner Bedeutung zu erfassen.
Prosatexte lassen sich leichter analysieren als Gedichte, trotzdem enthalten
sie oft Mehrdeutigkeiten.

\item[Programm:] Die Bedeutung eines Computerprogramms ist eindeutig,
unzweifelhaft und w�rtlich. Ein Programm l�sst sich alleinig durch
die Analyse der Symbole und der Struktur erfassen und verstehen. 

\end{description}

\subsection{Tipps zum Lesen von Programmen}
\label{sec:program reading}

F�r das Erlernen einer Sprache ist es wichtig, nicht nur das Schreiben 
zu lernen, sondern auch viele Texte in dieser Sprache zu lesen.
Im Folgenden habe ich einige Vorschl�ge zusammengetragen, wie man an das Lesen
von Programmen (und Texten in anderen formalen Sprachen) 
herangehen sollte:

Zuallererst ist zu beachten, dass Texte in formalen Sprachen
viel kompakter  (wort�rmer, weniger ausschweifend) sind als Texte nat�rlicher
Sprachen, weil die Texte keine Redundanzen enthalten.
Es dauert also in der Regel viel l�nger einen Text in einer formalen Sprache
zu lesen, da dieser pro Texteinheit einfach viel mehr Information enth�lt.
Hilfreich sind hier oft Anmerkungen der Programmierer die Erkl�rungen in 
nat�rlicher Sprache enthalten. Sie sollten sich auch angew�hnen, selbst solche
Anmerkungen zu verfassen.

Wichtig ist auch die Struktur eines Programms. Es ist keine gute Idee,
ein Programm als linearen Text von oben nach unten und links nach rechts
durcharbeiten zu wollen. Gr��ere Programme sind �blicherweise 
in sinnvolle Module gegliedert, die nicht unbedingt in der Reihenfolge
des Quelltextes ausgef�hrt werden. Wir m�ssen versuchen das Programm
in seiner Struktur zu erfassen. Dazu m�ssen wir zuerst wichtige Symbole
erfassen und erkennen und uns ein mentales Bild vom Zusammenspiel
der einzelnen Elemente bilden. Dabei kann es hilfreich sein, sich
diese Elemente und ihre Abh�ngigkeiten kurz zu skizzieren.

Zum Schluss m�chte ich noch darauf hinweisen, dass selbst kleine
Details wichtig sind. Tippfehler und schwache Zeichensetzung, 
Dinge, die in nat�rlichen Sprachen als Formfehler gelten, k�nnen
in formalen Sprachen gro�e Auswirkungen auf die Bedeutung eines
Textes haben.


\section{Das erste Programm}
\label{hello}
\index{Hello, World!}

Die Tradition verlangt, dass man das erste Programm, welches 
man in einer neuen Programmiersprache schreibt, ``Hello, World!'' nennt.
Es ist ein einfaches Programm, welches nichts weiter tun soll, 
als die Worte ``Hello, World!'' auf dem Bildschirm auszugeben. 
In C sieht dieses Programm wie folgt aus:

\begin{verbatim}
  #include <stdio.h>
  #include <stdlib.h>

  /* main: generate some simple output */

  int main(void)
  {
        printf("Hello, World!\n");
        return EXIT_SUCCESS;
  }

\end{verbatim}
%
Manche Leute beurteilen die Qualit�t einer Programmiersprache danach,
wie einfach es ist, das ``Hello, World!'' Programm zu erstellen. 
Nach diesem Standard schl�gt sich C noch vergleichsweise gut.  
Allerdings enth�lt bereits dieses einfache Programm einige  
Merkmale, die es schwierig machen das komplette Programme einem
Programmieranf�nger zu erkl�ren.

Deshalb werden wir an dieser Stelle erst einmal einige von ihnen ignorieren,
wie zum Beispiel die ersten zwei Programmzeilen. 

\index{Kommentar}
\index{Anweisung!Kommentar}

Die dritte Programmzeile f�ngt mit einem {\tt /*} an und endet mit  {\tt */}. 
Das zeigt uns, dass es sich bei dieser Zeile um einen {\bf Kommentar} handelt.  
Ein Kommentar ist eine kurze Anmerkungen zu einem Teil des Programms 
(siehe Abschnitt \ref{sec:program reading}).
Diese Anmerkung wird �blicherweise dazu benutzt, zu erkl�ren, was das Programm
tut. Kommentare k�nnen irgendwo im Quelltext des Programms stehen. 
Wenn der Compiler ein {\tt /*} sieht, dann ignoriert er von da an alles bis er
das dazugeh�rige {\tt */} findet. Die enthaltenen Anmerkungen sind daher 
streng genommen nicht Teil unseres Programms.

In der vierten Programmzeile f�llt das Wort {\tt main} auf.  {\tt main} ist ein 
spezieller Name, der angibt, wo in einem Programm die Ausf�hrung der
Befehle beginnt. Wenn das Programm startet, wird die
jeweils erste {\bf Anweisung} in {\tt main} ausgef�hrt, danach werden der
Reihe nach alle weiteren Anweisungen ausgef�hrt, bis das Programm zum
letzten Programmbefehl kommt und beendet wird.


\index{printf()}
\index{Anweisung!printf}	

Es existiert keine Beschr�nkung hinsichtlich der Anzahl von Anweisungen,
die unser Programm in {\tt main} enthalten kann.
In unserem Beispiel sind das nur zwei Anweisungen. 
Die erste ist eine {\bf Ausgabeanweisung}. Sie wird dazu benutzt, eine 
Nachricht auf dem Bildschirm anzuzeigen (zu ``drucken''). 
In C wird die {\tt printf} Anweisung benutzt, um Dinge auf dem Bildschirm
des Computers auszugeben. Die Zeichen zwischen den Anf�hrungszeichen
werden ausgegeben. 

Auff�llig ist dabei der {\tt \textbackslash n} am Ende der Nachricht. 
Dabei handelt es sich um ein spezielles Zeichen, genannt \emph{newline}, 
welches an das Ende einer Textzeile angef�gt wird und den 
Cursor veranlasst auf die n�chste Zeile des Bildschirms zu wechseln.
Wenn unser Programm jetzt das n�chste Mal etwas ausgeben m�chte,
so erscheint der neue Text auf einer neuen Bildschirmzeile.
Am Ende der Anweisung finden wir ein Semikolon ({\tt ;}).
Damit wird die Anweisung abgeschlossen -- es muss am
Ende jeder Anweisung stehen.

Mit der letzten Anweisung verlassen wir das Programm und geben die
Kontrolle an das Betriebssystem zur�ck. Die {\tt return} Anweisung 
wird verwendet um einen Programmteil (in C Funktion genannt) zu beenden 
und die Kontrolle an die Funktion zur�ckzugeben, welche die aktuelle Funktion
gestartet (aufgerufen) hat.
Dabei k�nnen wir eine Nachricht an die aufrufende Funktion 
�bergeben (in unserem Fall das Betriebssystem) und teilen mit, dass 
das Programm erfolgreich beendet wurde. 

Es gibt einige weitere Dinge, die wir �ber die Syntax von C-Programmen
wissen m�ssen:\\
C  benutzt geschweifte Klammern (\{ und
\}) um Gruppen von Anweisungen zu bilden. 
In unserem Programm befindet sich die Ausgabeanweisung {\tt printf} 
innerhalb geschweifter Klammern. Damit wird angezeigt,
dass sie sich {\em innerhalb} der Definition von {\tt main} befindet. 
Wir stellen auch fest, dass die Anweisungen im Programm
einger�ckt sind. Dabei handelt es sich nicht um eine strikte
Vorgabe des Compilers, sondern um eine �bereinkunft zwischen
Programmierern, die uns das Lesen eines Programms erleichtern soll.
So kann man leichter visuell erfassen, welche Programmteile 
zusammengeh�ren und von anderen Teilen abh�ngig sind.
Im Anhang \ref{Coding Style} habe ich dazu einige wichtige Regeln
zusammengetragen, die man m�glichst von Anfang an ber�cksichtigen sollte.

Zu diesem Zeitpunkt w�re es eine gute Idee sich an einen Computer
zu setzen und das Programm zu kompilieren und auszuf�hren.
Leider kann ich an dieser Stelle nicht genauer darauf eingehen, 
wie man das macht, da sich die einzelnen Computersysteme 
stark voneinander unterscheiden. Ich gehe davon aus, dass 
ein Seminarbetreuer, Freund oder eine Internet-Suchmaschine
hier weiterhelfen k�nnen.

Wie ich bereits erw�hnte, ist der C-Compiler sehr pedantisch, 
wenn es um die Einhaltung der Syntaxregeln geht.
Wenn wir auch nur den kleinsten Tippfehler bei der Eingabe
des Programms machen, ist die Gefahr gro�, dass sich das Programm
nicht erfolgreich kompilieren l�sst. Wenn wir zum Beispiel {\tt sdtio.h} statt
{\tt stdio.h} eingegeben haben, werden wir eine Fehlermeldung 
wie die folgende erhalten:

\begin{verbatim}
   hello_world.c:1:19: error: sdtio.h: No such file or directory
\end{verbatim}
%

Diese Fehlermeldung enth�lt eine Vielzahl von Informationen, 
leider sind sie in einem kompakten, schwer zu interpretierenden
Format verfasst. Ein freund\-licherer Compiler w�rde statt dessen
schreiben:

\begin{quote}
``In Zeile 1 des Quelltextes mit dem Dateinamen \texttt{hello\_world.c} haben
Sie versucht eine Headerdatei mit dem Dateinamen  \texttt{sdtio.h} zu laden. 
Ich konnte keine Datei mit diesem Namen finden, ich habe aber eine
Datei mit dem Namen \texttt{stdio.h} gefunden. \\ W�re es m�glich, dass
sie diese Datei gemeint haben?''
\end{quote}

Leider sind die wenigsten Compiler so nett zu Anf�ngern und so geschw�tzig.
Der Compiler ist noch dazu nicht besonders schlau. In den meisten F�llen
gibt uns die Fehlermeldung nur einen ersten Hinweis auf das m�gliche 
Problem und manchmal liegt der Compiler mit seinem Hinweis auch g�nzlich
daneben und der Fehler steckt  an einer ganz anderen Stelle. Vorangegangene
Fehler k�nnen Folgefehler produzieren. Es ist deshalb angeraten, die Fehler in 
der Reihenfolge ihres Auftretens zu beheben.

Dennoch kann der Compiler auch ein n�tzliches Werkzeug f�r das
Erlernen der Syntax einer Programmiersprache sein.
Wir beginnen mit einem funktionsf�higen Programm (wie \texttt{hello\_world.c}) 
und modifizieren dieses auf verschiedene Art und Weise.
Sollten wir bei unseren Versuchen eine Fehlermeldung  
erhalten, so pr�gen wir uns die Nachricht des Compilers und die
dazugeh�rige Ursache ein und falls derselbe Fehler wieder
auftritt, wissen wir, was wir ver�ndern m�ssen. Compiler kennen die Syntax einer
Sprache sehr genau. 
Es wird sicher einige Zeit brauchen, bevor Sie die Nachrichten 
des Compilers richtig interpretieren k�nnen -- es lohnt sich aber.



\section{Glossar}

\begin{description}

\item[Algorithmus (engl.: \emph{algorithm}):]  Eine detaillierte und explizite Vorschrift zur 
schrittweisen L�sung eines Problems oder einer Klasse von Problemen mit folgenden Eigenschaften:
\begin{itemize}
\item besteht aus einzelnen Schritten
\item jeder Schritt besteht aus einer einfachen und offensichtlichen Aktion
\item zu jedem Zeitpunkt ist klar, welcher Schritt als n�chstes ausgef�hrt wird
\end{itemize}


\item[Anweisung (engl.: \emph{statement}):] 
Befehl oder Befehlsfolge zur Steuerung eines Computers (einzelner Schritt in einem
Programm). In C werden Anweisungen durch das Semikolon (\texttt{;}) gekennzeichnet.

%A part of a program that specifies an action
%that will be performed when the program runs.  A print statement
%causes output to be displayed on the screen.

\item[Bug (engl.: \emph{bug}):]  Ein Fehler in einem Programm. Man unterscheidet Syntaxfehler, 
Laufzeitfehler und logische Fehler.

\item[Debugging (engl.: \emph{debugging}):]  Der Prozess der Fehlersuche und der Fehlerkorrektur
in einem Programm.

%\item[Probleml�sen (engl.: \emph{problem-solving}):]  The process of formulating a problem, finding
%a solution, and expressing the solution.

\item[High-level Sprache (engl.: \emph{high-level language}):]  Eine Programmiersprache wie C, welche 
f�r Menschen einfach zu lesen und zu schreiben ist. Hochsprachen m�ssen vor der Ausf�hrung
in Maschinensprache �bersetzt werden und k�nnen �blicherweise auf verschiedenen Computersystemen
laufen.

\item[Maschinennahe Sprache (engl.: \emph{low-level language}):]  Eine Programmier\-sprache welche
sich an den Befehlen und F�higkeiten eines bestimmten Prozessors orientiert und vom Programmierer
ein tiefes Verst�ndnis des Aufbaus eines Computers verlangt (z.B. Assembler).  
Maschinennahe Sprachen m�ssen ebenfalls noch in Maschinensprache �bersetzt werden. Diese
�bersetzung ist aber sehr einfach und direkt. Programme in Maschinensprache laufen nur auf einem
bestimmten Computersystem.

\item[Formale Sprache (engl.: \emph{formal language}):]  Eine k�nstliche Sprache, die von Menschen 
 f�r spezielle Zwecke entworfen wurde. Mit formalen Sprachen lassen sich mathematische Ideen  oder
Computerprogramme beschreiben.  Alle Programmiersprachen sind formale Sprachen.

\item[Nat�rliche Sprache (engl.: \emph{natural language}):]  Eine der Sprachen die von Menschen gesprochen
wird und die sich auf evolution�re Weise entwickelt hat.

\item[portabel (engl.: \emph{portability}):]  Der Begriff \emph{portabel} kennzeichnet die Eigenschaft eines Computerprogramms auf unterschiedlichen Computersystemen einsetzbar zu sein (�bertragbarkeit).

\item[interpretieren (engl.: \emph{interpret}):]  Ein Programm ein einer Hochsprache ausf�hren, indem
jede einzelne Programmzeile nacheinander durch einen \emph{Interpreter} �bersetzt und ausgef�hrt wird.

\item[compilieren (engl.: \emph{compile}):]  Ein Programm in einer Hochsprache mit Hilfe eines 
\emph{Compilers} in ein Programm in Maschinensprache zu �bersetzen.
Es wird dabei das komplette Programm �bersetzt und eine neue, ausf�hrbare Datei erzeugt, die
auf einem bestimmten Computersystem ausgef�hrt werden kann.

\item[Quelltext (engl.: \emph{source code}):]  Ein Programm  in einer Hochsprache, bevor
es compiliert wurde.

\item[Objektcode (engl.: \emph{object code}):]  Das Produkt eines Compilers nach der �bersetzung des Quelltextes (Maschinencode).

\item[Ausf�hrbare Datei (engl.: \emph{executable}):]  Der Maschinencode inklusive aller Bibliotheken. Die Datei
kann durch das Betriebssystem gestartet werden und f�hrt dann das Programm aus.

\item[Kommentar (engl.: \emph{comment}):] Ein Teil eines Programms, welcher Informationen f�r den
Programmierer �ber das Programm beinhaltet. Kommentare haben keinen Einfluss auf die Ausf�hrung
des Programms.

\item[Syntax (engl.: \emph{syntax}):]  Die Struktur eines Programms.

\item[Syntaxfehler (engl.: syntax error):]  Ein Fehler in einem Programm, welches es dem Compiler unm�glich
macht das Programm zu parsen (und somit zu �bersetzen).

\item[Semantik (engl.: \emph{semantics}):]  Die Bedeutung eines Programms.


\item[Parsen (engl.: \emph{parse}):]  Ein Programm zu untersuchen und die syntaktische Struktur
zu analysieren.



\item[Logischer Fehler (engl.: \emph{logical error}):]  Ein Fehler in einem Programm welcher dazu f�hrt, dass
das Programm zwar abl�uft, aber etwas anderes macht, als das was der Programmierer beabsichtigt hatte.


%\index{Probleml�sen}
\index{High-level Sprache}
\index{Low-level Sprache}
\index{Formale Sprache}
\index{Nat�rliche Sprache}
\index{Interpretieren}
\index{Kompilieren}
\index{Syntax}
\index{Semantik}
\index{Parsen}
\index{Error}
\index{Fehler}
\index{Programmfehler}
\index{Debugging}
\index{Anweisung}
\index{Kommentar}
\index{Logischer Fehler}
\index{Syntaxfehler}

\end{description}

\section{�bungsaufgaben}

\ifthenelse {\boolean{German}}{ % !TEX root = �bung1.tex

\begin{exercise}

Informatiker haben die �rgerliche Angewohnheit normale Worte 
einer Sprache zu benutzen und ihnen eine ganz eigene Bedeutung
zu geben, die von ihrer normalen Verwendung abweicht.
So haben zum Beispiel die Worte \emph{Anweisung (statement)} und 
\emph{Kommentar (comment)} normalerweise eine sehr �hnliche Bedeutung (zumindest auf Englisch).
In einer Programmiersprache sind sie aber sehr unterschiedlich.
Es ist wichtig, die Bedeutung der Elemente der Programmiersprache genau zu 
kennen und sie richtig einzusetzen, anderenfalls k�nnen sie keine korrekten
Programme schreiben.

Das Glossar am Ende eines jeden Kapitels dient dazu, wichtige Begriffe
und Phrasen zu rekapitulieren und die besondere Bedeutung dieser
Begriffe klar zu machen.

Achten Sie darauf, dass ein Begriff, den Sie aus der Umgangssprache
kennen, eine ganz eigene Bedeutung haben kann, wenn er in der Programmierung
verwendet wird.

\begin{enumerate}

\item Was ist der Unterschied zwischen einer \emph{Anweisung (statement)} und 
einem \emph{Kommentar (comment)} in einer Programmiersprache?

\item Was bedeutet es, wenn man sagt ein Programm sei \emph{portabel}?

\item Was bedeutet es, wenn man sagt ein Programm sei \emph{ausf�hrbar (executable)}?

\end{enumerate}

\end{exercise}

\begin{exercise}

Bevor wir uns eingehender mit der Programmiersprache besch�ftigen, ist es wichtig
herauszufinden, wie sich ein C-Programm auf unserem Computer kompilieren und
ausf�hren l�sst.
Die erforderlichen Schritte k�nnen je nach verwendetem Betriebssystem und
eingesetztem Compiler sehr unterschiedlich sein. 


\begin{enumerate}

\item Geben Sie das  ``Hello World''-Programm aus Abschnitt
%~\ref{hello}
1.5 
des Vorlesungs-Scripts (siehe Moodle) in den Computer ein, kompilieren Sie es und f�hren Sie es aus.

\item F�gen Sie eine zweite \texttt{printf()}-Anweisung hinzu, welche eine weitere 
Nachricht ausgibt.  Irgendeine kurze Bemerkung, 
wie zum Beispiel, ``How are you?''
Speichern, kompilieren und f�hren Sie das Programm erneut aus.

\item F�gen Sie eine Kommentarzeile zu ihrem Programm hinzu 
(wo immer sie wollen), und kompilieren Sie das Programm erneut.

F�hren Sie das Programm aus. Wie hat sich der Kommentar auf den Ablauf
des Programms ausgewirkt?

\end{enumerate}

Diese �bung mag Ihnen trivial erscheinen, aber sie ist der Grundstein
f�r all die vielen Programme, die wir in der n�chsten Zeit entwickeln werden.

Um mit Vertrauen und Zuversicht die Eigenarten und Besonderheiten
einer Programmiersprache zu entdecken, ist es erforderlich, dass man
Vertrauen in die Programmierumgebung hat.
Es ist n�mlich zum Teil sehr einfach die �bersicht dar�ber zu verlieren,
welches Programm jetzt gerade bearbeitet, �bersetzt und ausgef�hrt wird.
Und es kann leicht vorkommen, dass Sie versuchen den Fehler in einem
Programm zu finden, w�hrend Sie versehentlich ein anderes Programm ausf�hren,
oder �nderungen in einem Programm noch gar nicht gespeichert wurden.

Das Hinzuf�gen und �ndern von Ausgabeanweisungen (\texttt{printf()})
ist ein einfacher Weg um herauszufinden, ob das Programm, das sie
�ndern, auch das Programm ist, dass sie ausf�hren.

\end{exercise}

%\newpage

\begin{exercise}

Es ist eine gute Idee, sich mit einer Programmiersprache vertraut
zu machen, indem man viele Sachen ausprobiert.

Wir k�nnen zum Beispiel in unser Programm ganz bewusst Fehler
einbauen und beobachten, ob der Compiler diese Fehler findet
und wie er sie uns anzeigt.
Manchmal wird der Compiler uns genau sagen, was falsch gelaufen
ist und wie wir den Fehler beheben k�nnen. Manchmal bekommen
wir nur eine unverst�ndliche Meldung. 

Durch einfaches Ausprobieren k�nnen wir uns einen �berblick
verschaffen, wann wir dem Compiler trauen k�nnen und wann
wir selbst herausfinden m�ssen, was falsch gelaufen ist.

Nehmen Sie ein lauff�higes Programm und probieren Sie nacheinander die
folgenden Ver�nderung. 

\textbf{Achtung:} Ver�ndern Sie immer nur eine Stelle in ihrem Programm und 
lassen Sie es danach ausf�hren. Machen Sie die �nderung r�ckg�ngig, bevor Sie die
n�chste �nderung vornehmen.

\begin {enumerate}

\item Entfernen Sie die schlie�ende, geschweifte Klammer (\}).

\item Entfernen Sie die �ffnende, geschweifte Klammer (\{).

\item Entfernen Sie das {\tt int} vor {\tt main}.

\item Anstelle von {\tt main} schreiben Sie {\tt mian}.

\item Entfernen Sie das schlie�ende {\tt */} von einem Kommentar.

\item Ersetzen Sie {\tt printf} durch {\tt pintf} im Quelltext des Programms.

% Example of a logical error:
%
%\item Replace {\tt printf} with {\tt print}.  This one is
%tricky because it is a logical error, not a syntax error.
%The statement {\tt System.out.print} is legal, but it may or may
%not do what you expect.

\item L�schen Sie  eine der Klammern:  {\tt (} oder  {\tt )}  

\item F�gen Sie eine weitere Klammer hinzu.

\item L�schen Sie  das Semikolon nach der {\tt return} Anweisung.

\item L�schen Sie das Semikolon nach der {\tt printf()}-Anweisung.

\item Entfernen Sie das hintere Anf�hrungszeichen in den Klammern von {\tt printf()}.

\item Entfernen Sie das vordere Anf�hrungszeichen in den Klammern von {\tt printf()}.

\item Schreiben Sie {\tt <sdtio.h>} statt {\tt <stdio.h>} .

\item Lassen Sie das {\tt \#} vor der Pr�prozessoranweisung {\tt include} weg.

\item Kopieren Sie den gesamten Code und f�gen Sie ihn noch einmal am Ende ein.

\item Schreiben Sie die Zeile {\tt printf(...);} einmal vor {\tt int main(void)}.

\end {enumerate}

\vskip 1em

\begin{enumerate}[1)]
\item Nennen Sie jeweils die erste Fehlermeldung, die der Compiler ausgibt (die weiteren sind �blicherweise Folgefehler).
\item Versuchen Sie jede Fehlermeldung zu erkl�ren. Was \emph{'denkt'} der Compiler, was hier passiert? Warum gibt er genau diesen Fehlertext aus? Wird in der Meldung tats�chlich der von Ihnen (im Augenblick) bewusst eingebaute Fehler bem�ngelt? Oder st�rt sich der Compiler eigentlich an etwas ganz anderem?
\end{enumerate}


\end{exercise}

\begin{exercise}

Die \texttt{printf()}-Funktion ist eine der wichtigsten Funktionen f�r das Erlernen der Programmiersprache C.
Mit Hilfe der Funktion lassen sich interne Zust�nde im Programm auf dem Bildschirm ausgeben. 
Leider ist die Funktion nicht besonders einfach zu benutzen und unterst�tzt eine Vielzahl von Optionen und 
Umwandlungszeichen \texttt{(\%d, \%f, ...)}.

Probieren Sie das folgende Progamm aus und erkl�ren Sie die beobachteten Ausgaben:

\begin{verbatim}
  #include <stdio.h>
  #include <stdlib.h>
  
  int main (void)
  {
     printf("%d\n",5/2);
     printf("%f\n",5.0/2);
     printf("%s\n","5/2");
     printf("5/2\n");
     return EXIT_SUCCESS;
  }
  
\end{verbatim}
\end{exercise}
}
{\input{exercises/Exercise_1_english}}


%!TEX root = Main_german.tex

% LaTeX source for textbook ``How to think like a computer scientist''
% Copyright (C) 1999  Allen B. Downey

% This LaTeX source is free software; you can redistribute it and/or
% modify it under the terms of the GNU General Public License as
% published by the Free Software Foundation (version 2).

% This LaTeX source is distributed in the hope that it will be useful,
% but WITHOUT ANY WARRANTY; without even the implied warranty of
% MERCHANTABILITY or FITNESS FOR A PARTICULAR PURPOSE.  See the GNU
% General Public License for more details.

% Compiling this LaTeX source has the effect of generating
% a device-independent representation of a textbook, which
% can be converted to other formats and printed.  All intermediate
% representations (including DVI and Postscript), and all printed
% copies of the textbook are also covered by the GNU General
% Public License.

% This distribution includes a file named COPYING that contains the text
% of the GNU General Public License.  If it is missing, you can obtain
% it from www.gnu.org or by writing to the Free Software Foundation,
% Inc., 59 Temple Place - Suite 330, Boston, MA 02111-1307, USA.

\setcounter{chapter}{1}
\chapter{Variablen und Typen}

\section{Noch mehr Bildschirmausgaben}
\index{Ausgabe}
\index{Anweisung!Ausgabe}

Wie ich bereits im letzten Kapitel erw�hnte, k�nnen wir so viele
Anweisungen in {\tt main()} aufnehmen, wie wir wollen. 
So k�nnen wir zum Beispiel in unserem Programm auch mehr als
eine Zeile ausgeben lassen:

\begin{verbatim}

  #include <stdio.h>
  #include <stdlib.h>

  /* main: generate some simple output */

  int main (void)
  {
        printf ("Hello World!\n");		    /* output one line */
        printf ("How are you?\n");		    /* output another line */       
        return EXIT_SUCCESS;
  }

\end{verbatim}
%
Wie man sehen kann, ist es erlaubt, Kommentare auch 
direkt in eine Programmzeile
zu schreiben und nicht nur in eine separate Zeile.

\index{String}
\index{Zeichenketten|see{String}}
\index{Typ!String}

Die Ausdr�cke innerhalb der Anf�hrungszeichen werden {\bf Strings} 
oder {\bf Zeichenketten} genannt, weil sie aus einer Folge von Buchstaben
bestehen. Strings k�nnen jede beliebige Kombination von Buchstaben,
Ziffern, Satzzeichen und anderen speziellen Zeichen enthalten. Probleme
bereiten uns nur die deutschen Umlaute und das '�'. 

\index{newline}

Manchmal ist es sinnvoll, den Text mehrerer Ausgabeanweisungen
zusammen in einer Bildschirmzeile anzuzeigen. 
Wir k�nnen das ganz einfach umsetzen, in dem wir das {\tt $\backslash$n} 
Zeichen aus der ersten {\tt printf()} Anweisung entfernen:

\begin{verbatim}

    int main (void)
    {
        printf ("Goodbye, ");
        printf ("cruel world!\n");	     
        return EXIT_SUCCESS;
    }

\end{verbatim}
%
In diesen Fall erscheint die Ausgabe auf einer 
einzelnen Zeile wie folgt:

\begin{verbatim}
     Goodbye, cruel world!
\end{verbatim}

Im Programm f�llt auf, dass sich zwischen 
dem \texttt{Goodbye,} und dem 
Anf�hrungszeichen noch ein Leerzeichen befindet. 
Dieses Leerzeichen finden wir auch 
in dem angezeigten Text auf dem Bildschirm wieder, 
das hei�t es beeinflusst das 
Verhalten unseres Programms.

Leerzeichen ausserhalb von Anf�hrungszeichen, irgendwo 
im Quelltext des Programms, haben �blicherweise 
keinen Einfluss auf das Verhalten unseres Programms.
Ich h�tte den Quelltext auch in der folgenden Form aufschreiben k�nnen:

\begin{verbatim}

    int main(void)
    {
    printf("Goodbye, ");
    printf("cruel world!\n");	     
    return EXIT_SUCCESS;
    }

\end{verbatim}
%
Dieses Programm ist genauso gut kompilier- und ausf�hrbar wie das 
Original. Ebenso haben die Zeilenumbr�che an den Zeilenenden
keine Bedeutung. Ich h�tte also schreiben k�nnen:

\begin{verbatim}

    int main(void){printf("Goodbye, ");printf("cruel world!\n");
    return EXIT_SUCCESS;}

\end{verbatim}
%
Das funktioniert auch! Allerdings f�llt auf, dass es
schwerer und schwerer wird das Programm zu lesen.
Zeilenumbr�che und Leerzeichen im Quelltext 
sind ein sinnvolles Mittel um ein Programm visuell zu strukturieren.
Es wird dadurch f�r uns einfacher das Programm zu lesen und
m�gliche Fehler im Programm zu finden, beziehungsweise das
Programm sp�ter zu �ndern und anzupassen.
Moderne Entwicklungsumgebungen k�nnen dabei die
Arbeit erleichtern. Sie bieten die M�glichkeit den
Quelltext automatisch formatieren zu lassen.

\section{Bits und Bytes}
\index{Bit}
\index{Byte}
\index{Computerspeicher}
\index{Digitale Daten}

Computer sind digital, das wei� heute jedes Kind.
Was aber genau bedeutet das eigentlich?

Mein Computer, auf dem ich dieses Buch tippe, kann
T�ne, Bilder, Texte und Videos erzeugen, abspielen und
ver�ndern. Wie k�nnen so unterschiedliche Aufgaben von einer
Maschine geleistet werden?

Das R�tsels L�sung ist darin zu finden wie diese Sachen
gespeichert werden: als \textbf{digitale Daten}!

Computer speichern Daten in Form von Bits. 
Ein Bit ist die kleinste vorstellbare Informationsmenge.
Es kennt nur 2 Zust�nde: \emph{an} oder \emph{aus}, \emph{0} oder \emph{1}, \emph{high} oder \emph{low}.
Das ist scheinbar wenig, aber trotzdem schon recht n�tzlich: leuchtet die Fahrradlampe oder leuchtet sie nicht?
Diese Tatsache l�sst sich mit einem einzelnen Bit beschreiben.

F�r das Speichern von Fotos, Musik und Abschlussarbeiten ben�tigen wir nat�rlich sehr viel mehr Bits,
aber das ist f�r moderne Computer schon lange kein Problem mehr. Das Grundprinzip jedenfalls ist immer
noch g�ltig.

Da es sehr unhandlich ist, mit einzelnen Bits zu hantieren, speichern Computer ihre Daten in 
Gruppen von Bits. 
Wie wir gesehen haben, trifft ein einzelnes Bit eine einfache Fallunterscheidung:  \emph{0} oder \emph{1}, \emph{an} oder \emph{aus}. Gruppieren wir die Bits, so lassen sich mehr F�lle unterscheiden. 
Mit  8 Bit k�nnen wir bereits 256 F�lle unterscheiden -- mehr als genug f�r die Zeichen des lateinischen 
Alphabets, der Ziffern und Satzzeichen. 
%F�r die Speicherung von \emph{Zeichen} reichen also �blicherweise 7 oder 8 Bit.

Eine Gruppe von 8 Bit nennt man Byte und jedes Byte hat im Computer eine 
eindeutige Adresse. Diese ist n�tig, damit keine Daten verloren gehen und unser Programm immer ganz
genau wei�, wo sich welche Daten befinden. 
\index{Byte}
\index{Adresse}

Abbildung   \ref{Chapt2_Memory} zeigt uns einen Ausschnitt aus dem Speicher eines
Computers. Wir sehen eine Matrix von Bits, die entweder den Wert 0 oder 1 besitzen und
sind erst einmal verwirrt. Wie gelingt es hier die �bersicht zu behalten?

F�r die Interpretation der Bitfolgen kommen �blicherweise Codes zum Einsatz, wie zum Beispiel der
im Anhang \ref{ASCII-Table} beschriebene ASCII Code. Dort ist festgelegt, dass die Bitfolge \texttt{00110101}
das \textbf{Zeichen} \emph{5} beschreibt. Nicht zu verwechseln mit dem \textbf{Dezimalwert} \emph{5}, welche durch
die Bitfolge \texttt{00000101} dargestellt wird.
\index{ASCII}

\def\bitfeldIntern(#1,#2,#3,#4,#5,#6,#7,#8) {
\setcounter{collumn}{25}
\put(\thecollumn,\therow){\framebox(\theboxsize,\theboxsize){\textbf{\textsf{#1}}}}
\addtocounter{collumn}{\theboxsize} 
\put(\thecollumn,\therow){\framebox(\theboxsize,\theboxsize){\textbf{\textsf{#2}}}}
\addtocounter{collumn}{\theboxsize} 
\put(\thecollumn,\therow){\framebox(\theboxsize,\theboxsize){\textbf{\textsf{#3}}}}
\addtocounter{collumn}{\theboxsize} 
\put(\thecollumn,\therow){\framebox(\theboxsize,\theboxsize){\textbf{\textsf{#4}}}}
\addtocounter{collumn}{\theboxsize} 
\put(\thecollumn,\therow){\framebox(\theboxsize,\theboxsize){\textbf{\textsf{#5}}}}
\addtocounter{collumn}{\theboxsize} 
\put(\thecollumn,\therow){\framebox(\theboxsize,\theboxsize){\textbf{\textsf{#6}}}}
\addtocounter{collumn}{\theboxsize} 
\put(\thecollumn,\therow){\framebox(\theboxsize,\theboxsize){\textbf{\textsf{#7}}}}
\addtocounter{collumn}{\theboxsize} 
\put(\thecollumn,\therow){\framebox(\theboxsize,\theboxsize){\textbf{\textsf{#8}}}}
}
\def\bitfeld#1{\bitfeldIntern(#1)}

\newcommand{\initBitArray}[1]{
% Counter for drawing the bit-boxes

	\newcounter{boxsize}
	\setcounter{boxsize}{#1}

	\newcounter{row}
	\setcounter{row}{0}

	\newcounter{collumn}
% Counter for drawing left and right annotations
	\newlength{\txtoffset} %vertikaler Offset f�r Beschriftung (mittig zur Box)
	\setlength{\txtoffset}{0.25cm} 

	\newlength{\yspace}
	\yspace=\theboxsize cm
	\divide\yspace by 10

	\newcounter{before}
	\setcounter{before}{0}
	\newcounter{after}
	\setcounter{after}{85}
}

\newcommand{\addRow}{
\addtocounter{row}{\theboxsize}
}

\def\bitHeaderIntern(#1,#2,#3,#4,#5,#6,#7,#8) {
\setcounter{collumn}{25} 
\addtocounter{collumn}{3} % zentrieren


\addtocounter{row}{\theboxsize}
\newlength{\mylen}
%\setlength{\mylen}{36.5cm} 
\mylen=\therow cm
\divide\mylen by 10
\addtolength{\mylen}{0.15cm}

\put(\thecollumn,\LenToUnit{\mylen}){{\scriptsize \texttt{$#8$}}}
\addtocounter{collumn}{\theboxsize} 
\put(\thecollumn,\LenToUnit{\mylen}){{\scriptsize \texttt{$#7$}}}
\addtocounter{collumn}{\theboxsize} 
\put(\thecollumn,\LenToUnit{\mylen}){{\scriptsize \texttt{$#6$}}}
\addtocounter{collumn}{\theboxsize} 
\put(\thecollumn,\LenToUnit{\mylen}){{\scriptsize \texttt{$#5$}}}
\addtocounter{collumn}{\theboxsize} 
\put(\thecollumn,\LenToUnit{\mylen}){{\scriptsize \texttt{$#4$}}}
\addtocounter{collumn}{\theboxsize} 
\put(\thecollumn,\LenToUnit{\mylen}){{\scriptsize \texttt{$#3$}}}
\addtocounter{collumn}{\theboxsize} 
\put(\thecollumn,\LenToUnit{\mylen}){{\scriptsize \texttt{$#2$}}}
\addtocounter{collumn}{\theboxsize} 
\put(\thecollumn,\LenToUnit{\mylen}){{\scriptsize \texttt{$#1$}}}
}
\def\bitHeader#1{\bitHeaderIntern(#1)}

\unitlength0.1cm

\begin{figure}[H]
  \centering

\begin{picture}(120,45)%(0,-40)%shift origin eg: (0,-40)

\initBitArray{7} %7 is the size of the box in mm

\bitfeld{0,1,0,1,0,1,0,1} \addRow % wir zeichnen von unten nach oben
\bitfeld{0,1,0,1,0,1,0,1} \addRow % ToDo: kann man das �ndern?
\bitfeld{0,1,0,1,0,1,0,1} \addRow
\bitfeld{0,0,1,1,0,1,0,1} \addRow
\bitfeld{0,0,0,0,0,1,0,1}

\bitHeader{2^0,2^1,2^2,2^3,2^4,2^5,2^6,2^7}

\put(0,\LenToUnit{\txtoffset}){{\large \texttt{Byte \textbf{...}}}}
\addtolength{\txtoffset}{\yspace}
\put(0,\LenToUnit{\txtoffset}){{\large \texttt{Byte \textbf{6683}}}}
\addtolength{\txtoffset}{\yspace}
\put(0,\LenToUnit{\txtoffset}){{\large \texttt{Byte \textbf{6682}}}}
\addtolength{\txtoffset}{\yspace}
\put(0,\LenToUnit{\txtoffset}){{\large \texttt{Byte \textbf{6681}}}}

\put(\theafter,\LenToUnit{\txtoffset}){{\large \texttt{\textbf{ASCII-Zeichen '5'}}}}
\addtolength{\txtoffset}{\yspace}
\put(\thebefore,\LenToUnit{\txtoffset}){{\large \texttt{Byte \textbf{6680}}}}
\put(\theafter,\LenToUnit{\txtoffset}){{\large \texttt{\textbf{Dezimalwert 5}}}}

\end{picture}
  \caption{Ein hypothetischer Ausschnitt aus dem Speicher eines Computers}
  \label{Chapt2_Memory}
\end{figure}




Als Programmierer einer Hochsprache m�ssen sie die genaue numerische Adresse
der Daten sowie die verwendete Code-Tabelle nicht auswendig lernen. 
Da wir uns als Menschen besser Namen 
als lange Zahlenfolgen merken, werden wir in den n�chsten Abschnitten kennenlernen, wie 
wir \emph{Datentypen} und \emph{Variablen} benutzen um Daten zu speichern und zu 
bearbeiten. Die �bersetzung in die richtige Bitfolge �bernimmt dann der Compiler f�r uns.


\section{Werte und Datentypen}
\index{Wert}
\index{Daten}
\index{Datentyp}
\index{Typ|see{Datentyp}}

Computerprogramme arbeiten mit Daten, die im 
Speicher des Computers abgelegt sind. 
Daten besitzen einen \textbf{Wert}  -- das hei�t sie repr�sentieren eine konkrete Zahl oder einen Buchstaben -- 
und sind eines der fundamentalen Dinge mit denen ein Computerprogramm
arbeiten kann.  






%
%\begin{center}
%\newcommand{\bitlabel}[2]{%
%        \bitbox[]{#1}{%
%          %\raisebox{0pt}[0pt][0pt]{%
%            \fontsize{7}{7}\selectfont#2}}%
%	 
%%}
%
%\begin{bytefield}{8}
% \bitlabel{1}{$2^7$} & \bitlabel{1}{$2^6$} &
%   \bitlabel{1}{$2^5$} & \bitlabel{1}{$2^4$} &
%   \bitlabel{1}{$2^3$} &  \bitlabel{1}{$2^2$}
%   \bitlabel{1}{$2^1$} & \bitlabel{1}{$2^0$} \\
%%\bitheader{2,7} \\
%\bitbox[]{1}{\texttt{0x003FFFFF}0} & 
%\bitbox{1}{0} & 
%\bitbox{1}{0} &
%\bitbox{1}{0} &
%\bitbox{1}{0} &
%\bitbox{1}{1} &
%\bitbox{1}{0} &
%\bitbox{1}{1} \\
%\end{bytefield}
%\end{center}


Daten repr�sentieren so unterschiedliche Dinge wie die 
ganzen Zahlen, die reellen Zahlen und Buchstaben. Man sagt die Daten
haben einen bestimmten \textbf{Typ}.

Es ist wichtig, dass ein Programm ganz genau wei�, um welche Art von
Daten es sich handelt, da unterschiedlichen Anforderungen f�r die Speicherung der Daten
im Computer existieren. 

Wie wir gesehen haben, werden Daten als Bitfolgen gespeichert. 
F�r die Speicherung von  \emph{Buchstaben} des lateinischen Alphabets reichen �blicherweise 7 oder 8 Bit.
F�r die Speicherung von  \emph{Zahlen} werden dagegen wesentlich mehr Bit ben�tigt. Wir wollen ja nicht nur
von 0 bis 255 z�hlen k�nnen. 

Es macht f�r den Computer also einen Unterschied, ob wir den Zahlenraum der \emph{ganzen Zahlen} oder der \emph{reelle Zahlen} 
nutzen. Denn obwohl  Computer mit hoher Geschwindigkeit rechnen 
gibt es ein Problem: Zahlenbereiche in der Mathematik sind unendlich. Unser Computerspeicher
aber ist es nicht. Daher m�ssen wir uns bewusst machen, dass es zu Einschr�nkungen
im Wertebereich und der Genauigkeit kommen kann.

Ein weiterer Grund f�r die Unterscheidung der Datentypen liegt darin begr�ndet, dass
nicht alle Operationen f�r jeden Datentyp sinnvoll sind.
Wir k�nnen zwei Zahlen addieren, aber die Addition von Buchstaben \texttt{'a' + 'b'}
ist nicht definiert. 
Der Datentyp legt daher fest, welche Bedeutung die Bitfolgen im Speicher des Computers haben: \emph{Buchstabe}, \emph{ganze Zahlen}, usw. 

%Deshalb spricht man von elektronischer Datenverarbeitung
Die einzigen Daten, mit denen wir bisher gearbeitet haben, waren 
Folgen von Buchstaben, auch Zeichenketten oder Strings genannt.
\index{String}
Wir haben zum Beispiel {\tt ''Hello, world!''} auf dem Bildschirm ausgegeben.  
Wir  (und der Compiler) k�nnen diese Zeichenketten
anhand der umschlie�enden Anf�hrungszeichen erkennen.
%constant values

Die \emph{ganzen Zahlen} (beispielsweise 1 oder 17) werden in C als \emph{integer} 
bezeichnet.  
Unser Programm kann nicht nur Zeichenketten, sondern auch 
ganze Zahlen auf dem Bildschirm ausgeben:
\index{int|see{Ganze Zahlen}}
\index{integer|see{Ganze Zahlen}}
\index{Ganze Zahlen}

\vskip 0.7em
\begin{verbatim}
   printf("%i\n", 16);
\end{verbatim}
\vskip 0.5em

Die Ausgabe sieht auf den ersten Blick auch nicht anders aus, als wenn wir uns
eine Zeichenkette ausgeben lassen:
% Aufgabe: Vergleichen Sie die Ausgaben von 
\vskip 0.7em
\begin{verbatim}
    printf("16\n");
\end{verbatim}
\vskip 0.5em

Zahlen werden vom Computer aber anders behandelt als Zeichenketten, so kann man
zum Beispiel mit Zahlen rechnen:

\vskip 0.7em
\begin{verbatim}
   printf("%i\n", 16 + 1);
\end{verbatim}
\vskip 0.5em

Schauen wir uns die  \texttt{printf()} Anweisung genau an, so f�llt
ein \texttt{\%i} zwischen den Anf�hrungszeichen auf. Dabei handelt es sich um ein 
Platzhalterzeichen, welches angibt das eine ganze Zahl ausgegeben werden
soll. Die auszugebene Zahl folgt erst hinter den Ausf�hrungszeichen, durch Komma getrennt.
Es gibt eine Reihe solcher Platzhalter f�r unterschiedliche Datentypen.
Den N�chsten werden wir gleich kennenlernen.

Der Datentyp \emph{character} repr�sentiert einen Buchstaben, eine Ziffer
oder ein Satzzeichen. 
C benutzt f�r die Speicherung der Werte vom Typ \emph{character} den
ASCII-Code (siehe Anhang \ref{ASCII-Table}). In diesem Code ist leider nur 
das englische Alphabet definiert. Landestypische Erweiterungen des 
Zeichensatzes werden  nicht ber�cksichtigt. Aus diesem Grund
ist es am Anfang einfacher erst einmal komplett auf \textit{�}, 
\textit{�}, \textit{�} und \textit{�} zu verzichten
und statt dessen \textit{ae}, \textit{oe}, \textit{ue} und \textit{ss} zu schreiben. 
Ein \emph{character}-Wert in unserem Programm wird durch einfache 
Anf�hrungsstriche kenntlich gemacht, wie zum Beispiel {\tt 'a'} oder {\tt '5'}:


\vskip 0.7em
\begin{verbatim}
   printf("%c\n", '$');
\end{verbatim}
\vskip 0.5em
%
F�r die Ausgabe von Daten vom Typ \emph{character} ben�tigen wir das Platzhalterzeichen \texttt{\%c}.
Unser Beispiel gibt ein einzelnes Dollarzeichen in einer eigenen
Bildschirmzeile aus. 

%\index{{\tt char}|see{Zeichen}}
\index{char|see{Zeichen}}
\index{character|see{Zeichen}}
\index{Zeichen}

Am Anfang ist es schwer, die einzelnen Typen der Werte {\tt ''5''}, {\tt
'5'} und {\tt 5} zu unterscheiden. Man muss sehr genau auf die
Zeichensetzung achten, dann wird klar, dass der erste Wert
ein String, der zweite Wert ein Buchstabe und der dritte eine
ganze Zahl darstellt.
Der Grund f�r diese Unterscheidung wird uns im Laufe des Kurses
noch klarer werden.

\section {Variablen}
\index{Variablen}
\index{Wert}

Eine der m�chtigsten F�higkeiten einer Programmiersprache
ist die M�glichkeit digitale Daten zu speichern, wieder abzurufen und zu
ver�ndern.  
In unseren bisherigen Versuchen waren alle verwendeten Werte
durch die Angaben im Quelltext des Programms statisch festgelegt. 
Ab jetzt werden wir oft \textbf{Variablen} benutzen um Werte dynamisch zu speichern und zu
ver�ndern. Variablen k�nnen wir uns wie die Memory-Taste an
einem Taschenrechner vorstellen, nur etwa 1000x flexibler und m�chtiger, weil unser 
Programm beliebig viele Variablen nutzen und nicht nur Zahlen speichern kann.  

Eine Variable ist aber gar nichts sonderlich Geheimnisvolles. Ich hatte
ja bereits erw�hnt, dass die Bytes im Speicher unseres Computers Adressen
besitzen. Da wir als Menschen nicht sehr gut darin sind uns lange Zahlenfolgen
zu merken, verwenden wir daf�r besser einen sinnvollen, selbstgew�hlten Namen,
den \textbf{Variablennamen}.
\index{Variablenname} 

Variablen sollen verschiedene Arten von Daten speichern k�nnen und
m�ssen daher auch verschiedene Datentypen unterst�tzten.
Es gibt einige Programmiersprachen, bei denen der Computer 
selbstst�ndig den Typ der Variable anhand des zu speichernden
Werts erkennt. In C muss der Typ immer angeben werden.


Wollen wir eine neue Variable verwenden, so m�ssen wir sie erst einmal \emph{deklarieren}, das hei�t in
unserem Programm bekannt machen.
Um eine Variable zu deklarieren die ein Zeichen speichert, muss der Typ \emph{character} als dem Namen vorangestelltes {\tt char} angegeben werden.  
Die folgende Anweisung, die man auch als {\bf Deklaration} bezeichnet, erzeugt eine neue Variable 
mit dem Namen {\tt fred} vom  Typ \textit{character}:
\index{Deklaration}
\index{Anweisung!Deklaration}

\vskip 0.7em
\begin{verbatim}
    char fred;    /* creates a new character variable */
\end{verbatim}
\vskip 0.5em
%
\index{Typ!char}
%Diese Art von Anweisungen hei�t {\bf Deklaration}.

Der Typ einer Variable bestimmt, welche Werte gespeichert werden k�nnen.
Eine {\tt char} Variable kann genau ein Zeichen speichern. Ganze 
Zahlen k�nnen als  {\tt int} Variablen gespeichert werden.
Um eine einfache Variable vom Typ \emph {integer} anzulegen, verwenden wir folgende
Syntax:

\vskip 0.7em
\begin{verbatim}
    int bob;
\end{verbatim}
\vskip 0.5em
\index{Typ!int}
%
Dabei ist {\tt bob} ein beliebiger Name, den wir ausw�hlen k�nnen um
die Variable zu identifizieren. Es ist im Allgemeinen eine gute
Idee  Namen zu w�hlen, welche die Daten beschreiben,
die in ihnen gespeichert werden sollen. Das erleichtert  den
Umgang mit Variablen und macht ein Programm
leichter lesbar. Weiterhin sollten Sie sich bereits am Anfang mit den Regeln f�r
die Namensverwendungen vertraut machen (siehe Anhang \ref{Conventions for names}).


Schauen wir uns zum Beispiel die folgenden Variablendeklarationen
an:

\begin{verbatim}
    char first_letter;
    char last_letter;
    int hour, minute;
\end{verbatim}
%
Wir k�nnen wahrscheinlich eine erste, zutreffende Vermutung
�u�ern, welche Werte in diesen Variablen gespeichert werden.
Dieses Beispiel zeigt auch, wie wir einfach mehrere Variablen des
gleichen Typs deklarieren k�nnen: {\tt hour} und {\tt minute}
sind beides Variablen f�r ganze Zahlen ({\tt int} Typ).

F�r sehr gro�e und komplexe Programme ist auch diese
Form der Variablenbezeichnung noch zu un�bersichtlich.
Deswegen hat der aus Ungarn stammende Programmierer 
Charles Simonyi ein System der Variablenbezeichnung 
entworfen, indem dem Namen einer Variablen noch weitere
Informationen hinzugef�gt werden k�nnen.\footnote{\url{http://de.wikipedia.org/wiki/Ungarische_Notation}}
Wir werden dieses System in diesem Buch aber nicht anwenden,
da unsere Programme noch sehr klein und �bersichtlich sind.

Im Gegensatz zu anderen Programmiersprachen gibt es in C keinen
eigenen Datentyp, um die Werte von Zeichenketten in einer 
Variable zu speichern. Das ist schade, aber wir werden lernen
damit umzugehen. Leider brauchen wir daf�r noch ein tieferes 
Verst�ndnis der Programmiersprache, so dass ich erst sp�ter im
Kapitel~\ref{strings} darauf zur�ckkommen werde. 
F�r den Anfang beschr�nken wir uns also auf Zahlen und einzelne Zeichen.
%but we
%are going to skip that for now (see Chapter~\ref{strings}).


ACHTUNG: Der �ltere C89 Standard erlaubt die Deklaration von
Variablen nur am Anfang eines neuen Abschnitts (Block) im Quelltext. 
Es ist deshalb sinnvoll, alle in einer Funktion ben�tigten Variablen
gleich am Anfang der Funktion zu deklarieren -- selbst wenn wir
diese Variable erst viel sp�ter in unserem Programm benutzen wollen. \hint


\section{Zuweisung}
\label{sec:assignment}
\index{Zuweisung}
\index{Anweisung!Zuweisung}

Nachdem wir jetzt einige Variablen erzeugt haben, m�chten wir
gern Werte in ihnen speichern. Dazu benutzen wir eine
Anweisung, die eine {\bf Zuweisung} vornimmt:

\vskip 0.7em
\begin{verbatim}
    first_letter = 'a';   /* give first_letter the value 'a' */
    hour = 11;            /* assign the value 11 to hour */
    minute = 59;          /* set minute to 59 */
\end{verbatim}
\vskip 0.5em
%
Dieses Beispiel zeigt drei Zuweisungen und die Kommentare
geben uns drei Beispiele, wie Programmierer �ber den
Vorgang des Speicherns eines Wertes in einer Variable sprechen.
Das Vokabular ist vielleicht etwas verwirrend, aber die
Idee ist eigentlich ziemlich einfach zu beschreiben:

\begin{enumerate}

\item Wenn wir eine Variable deklarieren, erschaffen wir eine benannte 
Speicherstelle.

\item Wenn wir eine Zuweisung zu dieser Variable vornehmen,
speichern wir einen Wert in dieser Speicherstelle.

\end{enumerate}

Wir k�nnen dabei den zweiten Schritt nicht vor dem ersten Schritt
tun. Sollten wir den zweiten Schritt vergessen, l�sst sich die
Variable durchaus verwenden (zum Beispiel in einer 
Ausgabeanweisung), der Wert der Variable ist aber nicht bestimmt.
Man sagt dazu, die Variable ist nicht initialisiert. Der Compiler legt 
in der Regel beim Erstellen einer Variable keinen Anfangswert fest.
Es ist deshalb eine gute Idee, einer Variablen einen definierten 
Anfangswert zuzuweisen -- anderenfalls kann die Verwendung der
Variable leicht zu schwer zu findenden Programmfehlern f�hren. 

Eine weit verbreitete Methode Variablen auf Papier darzustellen,
besteht darin einen Kasten mit dem Variablennamen zu zeichnen
und den Wert der Variable hier einzutragen. 
Das folgende Diagramm zeigt den Effekt der 
drei Zuweisungsanweisungen:

%\vspace{0.1in}
%\centerline{\epsfig{figure=figs/assign.eps}}
%\vspace{0.1in}

\setlength{\unitlength}{1mm}
\begin{picture}(20,17)
\put(7,12){\large \texttt{first\_letter}}
\put(46,12){\large \texttt{hour}}
\put(74,12){\large \texttt{minute}}
\put(10,0){\framebox(20,10){{\large \textsf{a}}}}
\put(40,0){\framebox(20,10){{\large \textsf{11}}}}
\put(70,0){\framebox(20,10){{\large \textsf{59}}}}
\end{picture}


%\begin{picture}(50,10)
%\put(7,7){\large \texttt{Text}}
%\put(10,0){\framebox(5,5){{\large \texttt{0}}}}
%\put(15,0){\framebox(5,5){\texttt{1}}}
%\put(20,0){\framebox(5,5){{\large \textsf{1}}}}
%\thicklines
%\put(30,2.5){\vector(1,0){5}}
%\put(35,0){\framebox(5,5){ \textsf{1}}}
%\end{picture}


Wir k�nnen auf diese Weise den aktuellen Zustand einer Variablen 
darstellen.
Dieser Zustand ist abh�ngig von der Ausf�hrung von 
Anweisungen in unserem Programm und kann sich w�hrend
des Programmablaufs �ndern. Solche Ver�nderungen
kann man in einem {\bf Zustandsdiagramm} darstellen. 

%I sometimes use different shapes to indicate different
%variable types.  These shapes should help remind you that one of the
%rules in C  is that a variable has to have the same type as the
%value you assign it.  

Bei der Zuweisung von Werten an Variablen m�ssen
wir aufpassen, dass der Typ des Werts mit dem Typ der
Variable �bereinstimmt.
So k�nnen wir zum Beispiel keine Zeichenkette in einer
{\tt int} Variable speichern.  Die folgende Anweisung f�hrt zu einer
Warnmeldung des Compilers:

\vskip 0.7em
\begin{verbatim}
    int hour;
    hour = "Hello.";       /* WRONG !! */
\end{verbatim}
\vskip 0.5em
%
Diese Regel f�hrt manchmal zu Verwirrungen, weil es viele M�glichkeiten
gibt, Werte von einem Typ in einen anderen Typ zu konvertieren.
Manchmal nimmt C diese Typumwandlung auch automatisch vor.
Es hilft aber sich einzupr�gen, dass Variablen
und Werte den gleichen Typ haben m�ssen. Wir werden uns sp�ter
um die Spezialf�lle k�mmern.

Eine weitere Quelle f�r Verwechslungen besteht darin, 
dass einige Zeichenketten wie Zahlen \emph{aussehen}, aber keine
sind.
So ist zum Beispiel die Zeichenkette {\tt ''123''}, aus
den Zeichen {\tt 1}, {\tt 2} und {\tt 3} zusammengesetzt.
Der \emph{String} {\tt ''123''} unterscheidet sich f�r den Computer grundlegend
von der {\em Zahl} {\tt 123}.
Die folgende Zuweisung ist illegal:

\vskip 0.7em
\begin{verbatim}
   minute = "59";         /* WRONG!! */
\end{verbatim}
%
\section{Variablen ausgeben}
\label{output variables}

Wir k�nnen die Werte von Variablen mit den selben Kommandos ausgeben, die
wir auch f�r die Ausgabe von einfachen Werten genutzt haben:

\begin{verbatim}

    int hour, minute;
    char colon;

    hour = 11;
    minute = 59;
    colon = ':';

    printf ("The current time is ");
    printf ("%i", hour);
    printf ("%c", colon);
    printf ("%i", minute);
    printf ("\n"); 

\end{verbatim}
%
Dieses Programmfragment erzeugt zwei \emph{integer} Variablen mit Namen {\tt hour} und {\tt
minute}, und die \emph{character} Variable {\tt colon}.  
Den Variablen werden geeignete Werte zugewiesen, um danach mit einer
Reihe von Ausgabeanweisungen die folgende Nachricht auf dem Bildschirm auszugeben:


\begin{verbatim}
    The current time is 11:59
\end{verbatim}

Wenn wir davon sprechen eine Variable ``auszugeben'', meinen wir,
dass wir den {\em Wert} der Variable ausgeben.  
Der Name einer Variable ist nur f�r den Programmierer wichtig.
Wir erinnern uns, es ist ein Name f�r eine Speicherstelle.
Das kompilierte Programm enth�lt diese f�r Menschen
lesbare Referenzen nicht mehr. Den Benutzer interessiert nur noch
der dort gespeicherte Wert. 


%The name of a variable only has significance for
%the programmer. The compiled program no longer contains a human readable
%reference to the variable name in your program. 
%If you need to output the {\em name} of a variable,
%you have to print  it in quotes.  For example: {\tt cout << "hour";}


Die \texttt{ printf()} Ausgabeanweisung kann mehr als einen Wert in einer
einzigen Anweisung ausgeben. Daf�r m�ssen wir so viele Platzhalterzeichen
wie auszugebende Werte in die Anweisung einf�gen und danach
die auszugebenden Werte mit Komma getrennt anf�gen. Wichtig ist es
dabei, auf die richtige Reihenfolge und den Typ der Werte zu achten. 
Damit k�nnen wir unser Programm folgenderma�en zusammenfassen:

\begin{verbatim}

    int hour, minute;
    char colon;

    hour = 11;
    minute = 59;
    colon = ':';

    printf ("The current time is %i%c%i\n", hour, colon, minute);

\end{verbatim}
%
In einer Programmzeile k�nnen wir jetzt einen \emph{string}, zwei \emph{integer} und 
einen \emph{character} Wert ausgeben.  Sehr eindrucksvoll!

\section{Schl�sselw�rter}
\index{Schl�sselw�rter}

Einige Abschnitte zuvor sagte ich, dass wir f�r unsere Variablen
beliebige Namen verwenden d�rfen.
Das war leider nicht ganz richtig. 

Es gibt in C einige Namen, die reserviert sind,
weil sie bereits vom Compiler genutzt werden,
um die Struktur eines C Programms zu parsen.
Wenn wir diese W�rter als Variablennamen verwenden,
w�rden Mehrdeutigkeiten entstehen und der Compiler k�nnte
nicht mehr auseinanderhalten, ob es sich dabei um den 
Variablennamen oder das reservierte Wort der Sprache handelt. 

Die komplette Liste der Schl�sselw�rter ist im jeweiligen C Standard festgelegt.
Die hier aufgef�hrten Schl�sselw�rter entsprechen der Sprachdefinition,
wie sie die Internationale Organisation f�r Normung (ISO) am 1. September 1998 
festgelegt hat.  

Die reservierten W�rter werden {\bf Schl�sselw�rter} genannt. 
In der folgenden Tabelle sind alle derzeit definierten Schl�sselw�rter
der Sprache aufgef�hrt. 

\vskip 1em

\setlength{\fboxsep}{6pt} 
\begin{center}
\begin{boxedminipage}[c]{0.9\linewidth}
\begin{center}
\begin{multicols}{5}[\underline{Reservierte Schl�sselw�rter der Sprache C}]
\begin{verbatim}
auto 
break 
case 
char 
const 
continue 
default 
do 
double 
else 
enum 
extern 
float 
for 
goto 
if 
inline 
int 
long 
register 
restrict 
return 
short 
signed 
sizeof 
static 
struct 
switch 
typedef 
union 
unsigned 
void 
volatile 
while 
_Bool 
_Complex 
_Imaginary 
\end{verbatim}
\end{multicols}
\end{center}
\end{boxedminipage}
\end{center}

\vskip 1em


%You can download a copy electronically from
%
%\begin{verbatim}
%    http://www.ansi.org/
%\end{verbatim}
%
Anstatt diese Liste jetzt auswendig zu lernen, empfehle ich einen
der Vorteile moderner Entwicklungsumgebungen zu nutzen: Syntaxhervorhebungen.
Wenn wir den Quelltext in einer Entwicklungsumgebung
wie Code::Blocks\footnote{http://www.codeblocks.org/} eingeben, werden
unterschiedliche Teile unseres Programms unterschiedlich farblich
eingef�rbt. So erscheinen beispielsweise Schl�sselw�rter in der
Farbe blau, Zeichenketten rot und alle anderen Befehle schwarz.
Wenn wir jetzt einen Variablennamen eingeben und dieser in der
Farbe blau erscheint, sollten wir aufpassen! Der Compiler 
wird wahrscheinlich ein seltsames Verhalten zeigen.
 
%
%Rather than memorize the list, I would suggest that you
%take advantage of a feature provided in many development
%environments: code highlighting.  As you type, different
%parts of your program should appear in different colors.  For
%example, keywords might be blue, strings red, and other code
%black.  If you type a variable name and it turns blue, watch
%out!  You might get some strange behavior from the compiler.

\section{Mathematische Operatoren}
\label{operators}
\index{Operatoren!mathematische}


Mathematische {\bf Operatoren} sind spezielle Symbole, die dazu benutzt werden,
einfache Berechnungen wie Addition und Multiplikation darzustellen. Viele
der mathematischen Operatoren in C verhalten sich genauso, wie wir das von den
gebr�uchlichen mathematischen Symbolen kennen. So wird zum
Beispiel das Zeichen {\tt +}  f�r die Addition von zwei Zahlen benutzt.
F�r die Multiplikation wird das Zeichen {\tt *} und f�r die Division
das Zeichen {\tt /} verwendet.

\index{Ausdruck}

Wenn wir Operatoren mit Operanden kombinieren, entsteht ein
sogenannter {\bf Ausdruck} der einen Wert repr�sentiert. Ausdr�cke
k�nnen Variablen, Werte und Operatoren enthalten. 
In jedem Fall werden immer die Namen der Variablen durch 
die Werte der Variablen ersetzt, bevor die Berechnung (Auswertung)
des Ausdrucks vorgenommen wird.
 
Die folgenden Ausdr�cke der Sprache C sind legal und ihre 
Bedeutung erschlie�t sich praktisch von selbst:

\begin{verbatim}
  1+1        hour-1       hour*60+minute     minute/60
\end{verbatim}
%
Addition, Subtraktion und Multiplikation funktionieren in C so, wie
wir das erwarten w�rden. �berrascht werden wir vom Ergebnis
der Division. Schauen wir uns das folgende Programm an:

\begin{verbatim}

   int hour, minute;
   hour = 11;
   minute = 59;
   printf ("Minutes since midnight: %i\n", hour*60 + minute);
   printf ("Fraction of the hour that has passed: %i\n", minute/60);

\end{verbatim}
%
Es erzeugt die folgende Ausgabe: 

\begin{verbatim}
    Number of minutes since midnight: 719
    Fraction of the hour that has passed: 0
\end{verbatim}
%
Die erste Zeile der Ausgabe ist korrekt, aber die zweite Zeile ist
seltsam. Der Wert der Variablen {\tt minute} ist 59 und
59 dividiert durch 60 ergibt 0,98333, nicht~0.  Der Grund f�r dieses
Verhalten liegt darin, dass C f�r ganze Zahlen eine 
{\bf ganzzahlige Division} durchf�hrt.

\index{Datentyp!int}
\index{Ganzzahldivision}
\index{arithmetic!integer}
\index{Division!ganze Zahlen}
\index{Operand}

Wenn beide {\bf Operanden} ganzzahlige Werte sind,
%(operands are the things operators operate on)
dann ist das Resultat der Berechnung ebenfalls ein ganzzahliger Wert.
Nach der Definition wird dabei eine Division mit Rest durchgef�hrt.
Das Ergebnis ist ein Ganzzahlquotient und ein Divisionsrest und
hierbei wird niemals aufgerundet, selbst dann, wenn  
die n�chste Ganzzahl noch so nah ist.

Eine m�gliche Alternative w�re in unserem Fall statt des gebrochenen
Anteils den Prozentwert auszurechnen:

\begin{verbatim}
    printf ("Percentage of the hour that has passed: ");
    printf ("%i\n", minute*100/60);
\end{verbatim}
%
Das Resultat ist:

\begin{verbatim}
    Percentage of the hour that has passed: 98
\end{verbatim}
%
Auch hier fehlen wieder die Nachkommastellen, aber jetzt ist die
Antwort wenigstens ann�hernd korrekt.  
Um eine noch genauere Antwort zu erhalten, m�ssen wir einen
ganz neuen Typ von Variablen f�r die Berechnung benutzen: Flie�kommazahlen.
Mit diesem Variablentyp k�nne auch die Resultate von Br�chen und
reellen Zahlen gespeichert werden. 
%In order to get an even more accurate
%answer, we could use a different type of variable, called
%floating-point, that is capable of storing fractional values.
Wir werden darauf im n�chsten Kapitel zur�ckkommen.

\section{Rangfolge der Operatoren}
%\index{precedence}
\index{Rangfolge der Operatoren}
\index{Vorrang}
\index{Operatoren!Rangfolge}


Wenn mehr als ein Operator in einem Ausdruck enthalten ist,
wird die Reihenfolge der Auswertung von  {\bf Vorrangregeln} 
zwischen den Operatoren bestimmt.
Eine komplette Erkl�rung dieser Regeln w�re sehr umfangreich,
so dass ich hier nur die Wichtigsten erw�hnen m�chte:

\begin{itemize}

\item Multiplikation und Division werden vor
Addition und Subtraktion ausgef�hrt (Punktrechnung geht vor Strichrechnung).  
{\tt 2*3-1} ergibt 5, nicht 4 und {\tt 2/3-1} ergibt -1, nicht 1 
%(remember that in integer division {\tt 2/3} is 0).

\item Wenn die Operatoren den selben Rang aufweisen, werden sie von
links nach rechts ausgewertet. So wird in dem Ausdruck {\tt minute*100/60},
zuerst die Multiplikation ausgef�hrt, was {\tt 5900/60} ergibt, die weitere
Auswertung ergibt {\tt 98}. 
%W�re die Berechnung von rechts nach links
%durchgef�hrt worden , 
%the result would be {\tt 59*1} which is {\tt 59}, which
%is wrong.

\item Immer dann, wenn wir diese Vorrangregeln �ndern wollen (oder
wenn wir uns nicht komplett sicher sind), k�nnen wir Klammern 
setzen. Klammern setzen die automatischen Vorrangregeln au�er Kraft.
Ausdr�cke in Klammern werden zuerst ausgewertet, so wird 
{\tt 2*(3-1)} zu 4 ausgewertet.
Weiterhin kann man Klammern dazu verwenden einen Ausdruck
einfacher lesbar zu machen, wie in {\tt (minute*100)/60}, 
obwohl das nat�rlich keine Auswirkungen auf das Resultat hat.

\end{itemize}

\section{Operationen �ber Buchstaben}
%\index{character operator}
\index{Operator!Zeichen}

Interessanterweise k�nnen wir die gleichen mathematischen Operationen
f�r \emph{integer} Werte auch mit  \emph{character} Werten
benutzen. 
So gibt zum Beispiel das folgende Programm,

\begin{verbatim}

    char letter;
    letter = 'a' + 1;
    printf ("%c\n", letter);

\end{verbatim}
%
den Buchstabe {\tt b} auf dem Bildschirm aus. Der Grund daf�r besteht
in der Art und Weise, wie Buchstaben im Computer gespeichert werden.
So ist es erlaubt Buchstaben auch zu multiplizieren,
aber es ist sehr selten sinnvoll das zu tun.

Ich habe im Abschnitt \ref{sec:assignment} davon gesprochen,
dass wir \emph{integer} Variablen nur \emph{integer} Werte zuweisen 
k�nnen und \emph{character} Variablen nur \emph{character} Werte.
Ich sagte aber auch, dass es viele Ausnahmen von dieser Regel
gibt. So ist das folgende Beispiel v�llig legal: 


\begin{verbatim}
    int number;
    number = 'a';
    printf ("%i\n", number);
\end{verbatim}
%
Das Resultat unseres Programms ist 97. Dabei handelt es sich um
die Zahl die von C dazu benutzt wird um den Buchstaben {\tt 'a'}
darzustellen (siehe Anhang~\nameref{ASCII-Table}). 
Wir m�ssen uns immer wieder klar machen, der Computer 
speichert alle Daten (Zahlen, Buchstaben, T�ne, Bilder) als 
Folgen von Bits.    
Diese Bitfolgen k�nnen unterschiedlich \emph{interpretiert} werden.
Es ist allerdings eine gute Idee, Zahlen als Zahlen
zu behandeln und Buchstaben als Buchstaben und 
nur dann eine Darstellung in die andere umzuwandeln, wenn
daf�r ein guter Grund besteht.

Die automatische Umwandlung eines Datentyps in einen anderen
(engl.: \emph{automatic type conversion}) ist ein Beispiel f�r ein 
h�ufiges Designproblem bei der Erschaffung einer Programmiersprache.
Dabei besteht der Konflikt zwischen {\bf Formalismus} und 
{\bf Bequemlichkeit}.
Der Formalismus gebietet, dass eine Programmiersprache
einfache Regeln mit wenigen Ausnahmen aufweist.
Die Bequemlichkeit fordert, dass eine Programmiersprache
einfach benutzbar ist.

Im Falle von C hat die Bequemlichkeit gewonnen.
Das ist gut f�r den erfahrenen Programmierer, er wird vor rigorosen,
aber manchmal unhandlichen Formalismen verschont. 
F�r den Programmieranf�nger ist das allerdings schlecht, 
weil er oft von der Komplexit�t der Regeln und der Vielzahl
von Ausnahmef�llen verwirrt ist.
In diesem Buch habe ich versucht den Einstieg in die Programmierung
mit C dadurch zu vereinfachen, dass ich die generellen Regeln
hervorhebe und viele der bestehenden Ausnahmen weglasse.


%formalism}, which is the requirement that formal languages should have
%simple rules with few exceptions, and {\bf convenience}, which is the
%requirement that programming languages be easy to use in practice.


\section{Komposition}
\index{Komposition}
\index{Ausdruck}

Bisher haben wir uns die Elemente einer Programmiersprache
 -- Variablen, Ausdr�cke und Anweisungen -- jeweils einzeln
 betrachtet, ohne dar�ber nachzudenken, wie wir sie miteinander
 kombinieren k�nnen.
 
Eine der n�tzlichsten Eigenschaften einer Programmiersprache
besteht darin, einzelne kleine Bausteine zu nehmen und
diese zu einer m�chtigeren Konstruktion zusammenzusetzen.
So wissen wir zum Beispiel, wie man ganzzahlige Werte
multipliziert, und wir wissen, wie man Variablen ausgibt.
Es stellt sich  heraus, dass wir beides zur gleichen Zeit
tun k�nnen:

\begin{verbatim}
    printf ("%i\n", 17 * 3);
\end{verbatim}
%
Also eigentlich sollte ich nicht sagen ``zur gleichen Zeit'', weil die
Multiplikation bereits ausgef�hrt sein muss, bevor
das Ergebnis auf dem Bildschirm angezeigt wird.

Worauf ich hinaus will, ist das folgende: 
Es ist m�glich, beliebige Ausdr�cke (bestehend aus
Zahlen, Buchstaben, Variablen und Operatoren) in einer
Anweisung zu benutzen. Der Ausdruck wird vom Computer
ausgewertet und mit dem aktuell ermittelten Wert weitergearbeitet. 
Wir haben daf�r bereits ein Beispiel gesehen:

\begin{verbatim}
    printf ("%i\n", hour * 60 + minute);
\end{verbatim}
%
Wir k�nnen auch beliebige Ausdr�cke auf die rechte Seite einer
Zuweisungsanweisung schreiben:

\begin{verbatim}
    int percentage;
    percentage = (minute * 100) / 60;
\end{verbatim}
%
Diese F�higkeit ist momentan noch nicht sehr eindrucksvoll, aber
wir werden weitere Beispiele sehen, wo wir mit Hilfe der 
F�higkeit zur Komposition komplexe Berechnungen einfach
und pr�zise ausdr�cken k�nnen.

ACHTUNG: Es existieren ein paar Einschr�nkungen, 
wo wir bestimmte Ausdr�cke verwenden k�nnen. 
So d�rfen sich auf der linken Seite einer
Zuweisungsanweisung nur \emph{Variablen} und keine
zusammengesetzten Ausdr�cke befinden.
Der Grund daf�r liegt darin, dass die linke Seite einer 
Zuweisung eine Speicherstelle darstellt, der ein Wert zugewiesen
wird. Ausdr�cke repr�sentieren keine Speicherstelle, nur Werte.
Somit w�re die folgende Schreibweise illegal: 
\begin{verbatim}
     minute + 1 = hour;      /*WRONG!! */
\end{verbatim}
 
%There are limits on where you can use certain
%expressions; most notably, the left-hand side of an assignment
%statement has to be a {\em variable} name, not an expression.
%That's because the left side indicates the storage location
%where the result will go.  Expressions
%do not represent storage locations, only values.  
\section{Glossar}

\begin{description}

\item[Variable (engl.: \emph{variable}):] Eine benannte Stelle im Hauptspeicher. Dort k�nnen Werte 
abgelegt, ver�ndert und wieder gefunden werden.  Alle Variablen haben einen Typ. Dieser legt fest,
welche Werte gespeichert werden k�nnen.

\item[Wert (engl.: \emph{value}):] Ein anderer Begriff f�r digitale Daten. Das k�nnen Buchstaben, Zahlen oder
andere Repr�sentation von Information sein. Diese lassen sich verarbeiten und in Variablen speichern.

\item[Typ (engl.: \emph{type}):] Die Kategorie der Daten. Datentypen mit denen wir bisher gearbeitet haben
sind die ganzen Zahlen ({\tt int} in C) und Buchstaben ({\tt char} in C).

\item [Schl�sselwort (engl.: \emph{keyword}):] Ein Wort, welches
in dieser Programmiersprache eine bestimmte Bedeutung hat, und nicht als Name von 
Variablen oder Funktionen verwendet werden darf.
Bisher haben wir die Schl�sselw�rter {\tt int}, {\tt void} und {\tt char} verwendet.

\item[Anweisung (engl.: \emph{statement}):] 
Eine Befehlszeile die einen abgeschlossenen Schritt (Kommando, Aktion)
in einem Programm darstellt. In C werden Anweisungen mit einem Semikolon beendet.
%Bisher kennen wir Anweisungen die  declarations,
%assignments, and output statements.

\item[Deklaration (engl.: \emph{declaration}):] Eine Anweisung welche den Typ
und den Namen einer Variable festlegt.
\label{Glossary:Declaration}

\item[Zuweisung (engl.: \emph{assignment}):] Eine Anweisung welche einer Variablen (einer
Speicherstelle) einen Wert zuweist.

\item[Ausdruck (engl.: \emph{expression}):] Eine Kombination von Variablen, Operatoren und
Werten, welche zu einem Resultat ausgewertet werden, also wiederum einen Wert darstellen. 
Ausdr�cke haben einen bestimmten Typ, der durch die verwendeten Operatoren und Operanden
bestimmt wird.

\item[Operator (engl.: \emph{operator}):] Ein spezielles Symbol, welches eine einfache 
Verkn�pfung oder Berechnung  darstellt (z.B. Multiplikation oder Addition).
%that represents a simple computation like addition or multiplication.

\item[Operand (engl.: \emph{operand}):] Einer der Werte mit denen ein Operator eine Operation durchf�hrt. 

\item[Vorrang (engl.: \emph{precedence}):] Die Reihenfolge in der die einzelnen Teilschritte einer
Operation mit mehreren Operatoren ausgewertet wird. % order in which operations are evaluated.

\item[Komposition (engl.: \emph{composition}):] 
Die F�higkeit einfache Ausdr�cke und Anweisungen zu komplexeren Ausdr�cken und
Anweisungen zusammenzufassen. Komplexe Probleme und Sachverhalte lassen sich 
durch die geeignete, schrittweise Ausf�hrung einfacher Anweisungen beschreiben und l�sen.
%
%The ability to combine simple
%expressions and statements into compound statements and expressions
%in order to represent complex computations concisely.

\index{Variablen}
\index{Wert}
\index{Typ}
\index{Schl�sselwort}
\index{Anweisung}
\index{Zuweisung}
\index{Ausdruck}
\index{Operator}
\index{Operand}
\index{Vorrang}
\index{Komposition}

\end{description}


\section{�bungsaufgaben}
\setcounter{exercisenum}{0}

\ifthenelse {\boolean{German}}{ \begin{exercise}
\label{ex.date}


%\paragraph*{Deutsche �bersetzung der Aufgabe:}

\begin{enumerate}

\item 
Erstellen Sie ein neues Programm mit dem Namen {\tt MyDate.c}.
Kopieren Sie dazu die Struktur des "Hello, World"-Programms 
und stellen Sie sicher, dass Sie dieses kompilieren und ausf�hren k�nnen. 

\item
Folgen Sie dem Beispiel in Abschnitt ~\ref{output variables} und definieren 
Sie in dem Programm die folgenden Variablen: {\tt day}, {\tt month}
und {\tt year}.
{\tt day} enth�lt den Tag des Monats, {\tt month} den Monat und {\tt year} das Jahr.
Von welchem Typ sind diese Variablen? 

Weisen Sie den Variablen
Werte zu, welche dem heutigen Datum entsprechen. 

\item
Geben Sie die Werte auf dem Bildschirm aus. Stellen Sie jeden Wert auf einer
eigenen Bildschirmzeile dar. Das ist ein Zwischenschritt, der ihnen dabei hilft
zu �berpr�fen, ob das Programm funktionsf�hig ist.

\item
Modifizieren Sie das Programm dahingehend, dass es das Datum im amerikanischen
Standardformat darstellt: {\tt mm/dd/yyyy}.

\item
Modifizieren Sie das Programm erneut, um eine Ausgabe nach folgendem Muster 
zu erzeugen:
\begin{verbatim}
American format:
3/18/2009
European format:
18.3.2009
\end{verbatim}

\end{enumerate}

Diese �bung soll Ihnen dabei helfen, formatierte Ausgaben von 
Werten unterschiedlicher Datentypen mittels der  {\tt printf} Funktion zu
erzeugen. Weiterhin sollen Sie die kontinuierliche Entwicklung von komplexen
Programmen durch das schrittweise Hinzuf�gen von einigen, wenigen 
Anweisungen erlernen.

\end{exercise}


\begin{exercise}


%\paragraph*{Deutsche �bersetzung der Aufgabe:}
\begin{enumerate}

\item 
Erstellen Sie ein neues Programm mit dem Namen {\tt MyTime.c}. 
In den nachfolgenden Aufgaben werde ich Sie nicht mehr daran
erinnern mit einem kleinen, funktionsf�higen Programm zu beginnen.
Allerdings sollten Sie dieses auch weiterhin tun.

\item 
Folgen Sie dem Beispiel im Abschnitt~\ref{operators} und erstellen
Sie Variablen mit dem Namen {\tt hour}, {\tt minute} und {\tt second}.
Weisen Sie den Variablen Werte zu, welche in etwa der 
aktuellen Zeit entsprechen. 
Benutzen Sie dazu das 24-Stunden Zeitformat.

\item 
Das Programm soll die Anzahl der Sekunden seit Mitternacht berechnen.

\item 
Das Programm soll die Anzahl der noch verbleibenden Sekunden
des Tages berechnen und ausgeben.

\item 
Das Programm soll berechnen, wieviel Prozent des Tages bereits verstrichen
sind und diesen Wert ausgeben.

\item 
Ver�ndern Sie die Werte von  {\tt hour}, {\tt minute} und {\tt second},
um die aktuelle Zeit wiederzugeben. 
�berpr�fen Sie, ob das Programm mit unterschiedlichen Werten korrekt
arbeitet.

\end{enumerate}

In dieser �bung f�hren Sie arithmetische Operationen
durch und beginnen dar�ber nachzudenken, wie 
komplexere Datenobjekte, wie z.B. die Uhrzeit,
als Zusammensetzung von mehreren Werten dargestellt werden
k�nnen.
Weiterhin entdecken Sie m�glicherweise Probleme, die sich aus der Darstellung
und Berechnung mit dem ganzzahligen Datentypen {\tt int} ergeben (Prozentberechnung).
Diese Probleme k�nnen mit der Verwendung von Flie�kommazahlen
umgangen werden (siehe n�chstes Kapitel).  


HINWEIS: Sie k�nnen weitere Variablen benutzen, um
Zwischenergebnisse der Berechnung abzulegen.
Diese Variablen, welche in einer Berechnung genutzt, aber niemals
ausgegeben werden, bezeichnet man auch als tempor�re Variablen.

\end{exercise}
}
{\input{exercises/Exercise_2_english}}





%!TEX root = Main_german.tex


\selectlanguage{ngerman}
\setcounter{chapter}{2}
\chapter{Funktionen}

\section{Flie�kommazahlen}
\index{Flie�kommazahl}
\index{Gleitkommazahl}
\index{Typ!double}
\index{Typ!float}
\index{double (floating-point)}

Im letzten Kapitel hatten wir Problem mit der der Darstellung von
Ergebnissen die keine ganzen Zahlen waren.
Wir haben uns dadurch geholfen, dass wir keine Dezimalbr�che
sondern Prozentwerte benutzt haben. 
Eine bessere L�sung stellt die Verwendung des Datentyps
der Flie�kommazahlen dar. Eine Flie�kommazahl, auch als Gleitkommazahl bezeichnet, kann 
auch gebrochene Anteile reeller Zahlen darstellen. 
% more general solution is
%to use floating-point numbers, which can represent fractions
%as well as integers.  

In C existieren zwei Arten der Darstellung von Flie�kommazahlen,
genannt {\tt float} und {\tt double}.  
F�r Computer ist es schwierig das mathematische Konzept der
Unendlichkeit zu realisieren, da der verf�gbare Speicher begrenzt ist. 
So weist das Ergebnis der folgenden einfachen Division, $10 : 3 = 3,\overline{3}$ unendlich viele Nachkommastellen auf.
\todo{Wertebreich}
Die Datentypen {\tt float} und {\tt double} unterscheiden sich darin, wie
genau das Ergebnis dargestellt wird.
In diesem Buch verwenden wir ausschlie�lich 
{\tt double}s.

Wir k�nnen Flie�komma-Variablen erzeugen und ihnen Werte zuweisen
mit der gleichen Syntax die wir f�r andere
Datentypen benutzen. Ein Beispiel: 

\begin{verbatim}
    double pi;
    pi = 3.14159;
\end{verbatim}
%


Es ist auch erlaubt eine Variable zu deklarieren und ihr zur gleichen Zeit einen
Wert zuzuweisen:

\begin{verbatim}
    int x = 1;
    char first_char = 'a';
    double pi = 3.14159;
\end{verbatim}
%
Diese Syntax ist sogar ziemlich weit verbreitet.
Die kombinierte Deklaration
und Zuweisung wird auch als  {\bf Initialisierung} bezeichnet.

\index{Initialisierung}

WICHTIG:  F�r die Darstellung von Dezimalbr�chen wird im englischen Sprachraum
ein Punkt statt eines Kommas verwendet. C verwendet ebenfalls diese Schreibweise.
Die Verwendung des Dezimalkommas stellt einen Syntaxfehler dar. 

\pagebreak

Obwohl Flie�kommazahlen n�tzlich sind, f�hren Sie doch
auch zu Verwirrungen, weil es eine scheinbare
�berschneidung zwischen ganzen Zahlen und 
Flie�kommazahlen gibt.
Wenn wir uns zum Beispiel den Wert~{\tt 1} anschauen, ist 
das eine ganze Zahl, eine Flie�kommazahl, oder beides?

Genau genommen unterscheidet C zwischen dem Wert
der ganzen Zahl {\tt 1} und dem Wert der Flie�kommazahl {\tt 1.0}, 
obwohl beide die gleiche Zahl repr�sentieren. 
Diese Werte geh�ren zu unterschiedlichen Datentypen 
und wir hatten die generelle Regel aufgestellt, dass es
nicht erlaubt ist Zuweisungen 
zwischen unterschiedlichen Datentypen durchzuf�hren.
Im folgenden Beispiel,


\begin{verbatim}
    double y = 1;
\end{verbatim}
%
ist die Variable auf der linken Seite ein {\tt double} und der
Wert auf der rechten Seite ein {\tt int}.  
Allerdings f�hrt die Zuweisung  nicht zu einer Fehlermeldung
des Compilers. C nimmt an dieser Stelle eine automatische
Typumwandlung vor. 
\index{Typumwandlung!automatisch}
Die automatische Umwandlung ist f�r den Programmierer bequem, kann aber zu 
allerlei Problemen f�hren.
So gehen uns im folgenden Beispiel alle Nachkommastellen verloren:

\begin{verbatim}
    int x = 1.98;    	             /* Ergebnis: x = 1 */
\end{verbatim}
%
Das Ergebnis des folgenden Programmcodes ist f�r Anf�nger besonders verwunderlich:

\begin{verbatim}
    double y = 1 / 3;    	         /* Ergebnis: y = 0.0 */
\end{verbatim}
%
Es w�re zu erwarten, dass der Variablen {\tt y} der Wert 
{\tt 0.333333} zugewiesen wird, 
%which is a legal floating-point value, 
allerdings hat sie den Wert  {\tt 0.0}.\\
Der Grund liegt in der Art und Weise, wie der Compiler
die Anweisung auswertet und ausf�hrt.
Der Compiler untersucht zuerst den Ausdruck auf der rechten Seite der
Zuweisung (rechts vom \texttt{=}).
Dieser Ausdruck beschreibt ein Verh�ltnis zwischen
zwei ganzen Zahlen. %und daher wird eine Ganzzahldivision durchgef�hrt.
\index{Ganzzahldivision}
\index{Division mit Rest|see{Ganzzahldivision}}
Bei der Division ganzer Zahlen wird eine \emph{Division mit Rest} durchgef�hrt, welche den ganzzahligen Wert {\tt 0} zum Ergebnis hat.  
Erst bei der darauf folgenden Zuweisung an die Variable, wird dieser Wert in eine Flie�kommazahl
konvertiert und das Resultat betr�gt jetzt {\tt 0.0}.


%
% Gemischte Ausdr�cke!
%

Eine M�glichkeit dieses Problem zu l�sen (nachdem 
wir die Ursache herausgefunden haben) besteht darin
den Ausdruck auf der rechten Seite als Flie�kommazahl
darzustellen:

\begin{verbatim}
    double y = 1.0 / 3.0;	
\end{verbatim}
%
Dadurch wird {\tt y} der erwartete Wert {\tt 0.333333} zugewiesen.
Wir m�ssen also beachten, dass immer erst der rechte Ausdruck
einer Zuweisung komplett ausgewertet  und anschlie�end die
Zuweisung durchgef�hrt wird.

\index{arithmetic!floating-point}

Alle Berechnungen die wir bisher gesehen haben -- Addition, Subtraktion,
Multiplikation und Division -- funktionieren sowohl f�r Flie�kommazahlen als auch f�r
ganze Zahlen.
Allerdings m�ssen wir wissen, dass die grundlegenden Mechanismen
der Speicherung und Verarbeitung dieser Zahlen komplett verschieden
realisiert sind. Die meisten modernen Prozessoren haben spezielle 
Hardwareerweiterungen f�r die Verarbeitung von Flie�kommazahlen.
Es ist ebenfalls wichtig zu beachten, dass Flie�kommazahlen nur
mit eingeschr�nkter Genauigkeit dargestellt werden k�nnen, was zu
Rundungsverlusten bei Berechnungen f�hren kann.
 

% double y =123456789987654321;
% 123456789987654320.000000 

\section{Konstanten}
\label{sec:Constants}
\index{Konstanten}
\index{Konstante Werte}

Im letzten Abschnitt haben wir den Wert 3.14159 einer Variable 
vom Typ {\tt double} zugewiesen. 
Variablen haben die wichtige Eigenschaft, dass unser Programm in ihnen
unterschiedliche Werte zu unterschiedlichen Zeiten speichern kann.

%An important thing to remember about variables
%is, that they can hold -- as their name implies -- 
%different values at different points in your program. 
So k�nnen wir zum Beispiel der Variablen {\tt pi}
aktuell den Wert 3.14159 zuweisen und ihr sp�ter erneut
einen anderen Wert geben:

\begin{verbatim}
    double pi = 3.14159;
    ...
    pi = 10.999;  /* wahrscheinlich ein logischer Fehler */
\end{verbatim}
%

Der zweite Wert hat nichts mehr mit dem urspr�nglich
 {\tt pi} genannten Wert in unserem Programm zu tun.
 Der Wert von $\pi$ ist konstant und �ndert sich nicht im Laufe der Zeit.
 Wenn wir die Speicherstelle  {\tt pi} dazu benutzen auch beliebige andere
 Werte zu speichern, kann das zu schwer zu findenden Fehlern in unserem
 Programm f�hren.

Wir ben�tigen eine M�glichkeit um festzulegen, dass es sich bei einer
Speicherstelle um einen konstanten Wert handeln soll, der w�hrend
des Programmablaufs nicht ver�ndert werden darf.
In C m�ssen wir dazu zus�tzlich das Schl�sselwort  {\bf {\tt const}}
bei der Deklaration der Speicherstelle angeben. Weiterhin
ist es nat�rlich erforderlich auch den Datentyp der Konstanten 
anzugeben.
%Definition/Deklaration/Initialisierung

\todo{Konstanten in C besser �ber \#define}
 
Der Wert der Konstanten muss zum Zeitpunkt der Deklaration
festgelegt werden (Initialisierung). Im weiteren Ablauf des
Programms kann dieser Wert nicht mehr ver�ndert werden.  
Um Konstanten
visuell von Variablen zu unterscheiden, verwenden wir f�r die 
Namen von Konstanten ausschlie�lich Gro�buchstaben.

\begin{verbatim}
    const double PI = 3.14159;
    printf ("Pi: %f\n", PI);
     ...
    PI = 10.999;  /* falsch, Fehler wird vom Compiler gefunden */
\end{verbatim}
%

Es ist nicht l�nger m�glich den Wert von {\tt PI}  nachtr�glich zu ver�ndern. 
Unabh�ngig davon k�nnen wir aber Konstanten genau so in Berechnungen
zu verwenden wie Variablen.





\section{Explizite Umwandlung von Datentypen}
\label{rounding}
\label{typecasting} 
\index{rounding}
\index{Typumwandlung!cast}

Wie wir gesehen haben wandelt C automatisch
zwischen den numerischen Datentypen um, wenn n�tig.
%As I mentioned, C converts {\tt int}s
%to {\tt double}s automatically if necessary, because no
%information is lost in the translation.  On the other hand,
%going from a {\tt double} to an {\tt int} requires rounding
%off.  C doesn't perform this operation automatically, in
%order to make sure that you, as the programmer, are aware
%of the loss of the fractional part of the number.
Manchmal ist es jedoch sinnvoll die Konvertierung nicht dem
Kompiler zu �berlassen, sondern selbst zu bestimmen
wann und wie eine Typumwandlung durchgef�hrt werden soll.

\pagebreak[3]
In dem folgenden Beispiel soll das Ergebnis der Berechnung
als Dezimalbruch ermittelt werden:
\begin{verbatim}
   int x = 9;
   int y = 2;
   double result = x / y;      /* result = 4.0 */
\end{verbatim}

Wir erinnern uns, bei Operanden
vom Typ  {\tt int} f�hrt C automatisch eine Ganzzahldivision durch
und das Ergebnis entspricht nicht unserer Anforderung.
Wir m�ssten die Werte in dem Ausdruck auf der rechten Seite von 
{\tt int} nach  {\tt double} wandeln, damit die Berechnung das
korrekte Ergebnis liefert.

Um eine ganze Zahl in eine Flie�kommazahl zu wandeln, k�nnen 
wir einen {\bf Cast} oder {\bf Typecast} benutzen.  Typecasting erlaubt es
uns in einem Ausdruck so zu tun, als h�tten wir einen anderen Datentyp vorliegen.
Der Wert selbst wird dabei nicht ver�ndert und auch der Typ
der urspr�nglichen Variablen oder Konstanten �ndert sich nicht.

Die Syntax f�r die Typumwandlung erfordert die explizite Angabe 
des Zieltyps  {\tt (type)} in Klammern vor dem umzuwandelnden Ausdruck.
Zum Beispiel:

\begin{verbatim}
   int x = 9;
   int y = 2;
   double result = (double) x / (double) y;     /* result = 4.5 */
\end{verbatim}
%
Der {\tt (double)} Operator interpretiert {\tt x} und {\tt y} als Flie�kommatypen
und das Ergebnis der Berechnung liefert uns den erwarteten Dezimalbruch.

%so {\tt x} gets the value
%3.  Converting to an integer always rounds down, even if the fraction
%part is 0.99999999.

F�r jeden Datentyp in C, gibt es einen entsprechenden Operator welcher
das Argument in den entsprechenden Typ umwandelt.

\section{Mathematische Funktionen}
\index{Mathematische Funktion}
\index{Funktion!mathematische}
\index{Ausdruck}
\index{Argument}

Aus dem Bereich der Mathematik kennen wir Funktionen wie $\sin$ und
$\log$ und wir haben gelernt mathematische Ausdr�cke wie $\sin(\pi/2)$ 
und $\log(1/x)$ auszuwerten.  
Dazu wird zuerst der Ausdruck in Klammern, das {\bf Argument} der Funktion,
ausgewertet. So ist zum Beispiel, $\pi/2$ n�herungsweise 1.571 und $1/x$ ergibt
0.1 (wenn wir f�r $x$ den Wert 10 annehmen).
Danach wird die Funktion selbst berechnet, entweder durch Nachschlagen
in einer Tabelle oder �ber die Durchf�hrung verschiedener Berechnungen.
Der Sinus ($\sin$) von 1.571 ist
1 und der Logarithmus ($\log$) von 0.1 ist  -1 (unter der Annahme, dass $\log$ 
den Logarithmus zur Basis 10 darstellt).

Dieser Prozess kann mehrfach wiederholt werden, um immer kompliziertere
Ausdr�cke, wie  $\log(1/\sin(\pi/2))$ auszuwerten.  Zuerst werten wir
die innersten Funktionen aus, danach die umgebenden Funktionen und
immer so weiter.

C bietet uns eine Reihe von eingebauten Funktionen, welche die meisten
bekannten mathematischen Operationen unterst�tzen.
Die mathematischen Funktionen werden in einer der mathematischen
Schreibweise �hnelnden Form aufgerufen:

\begin{verbatim}
    double my_log = log (17.0);
    double angle = 1.5;
    double height = sin (angle);
\end{verbatim}
%
Das erste Beispiel weist der Variable {\tt my\_log} den Logarithmus von 17 zur Basis
$e$ zu. Es existiert auch eine Funktion {\tt log10} welche den Logarithmus zur 
Basis 10 ermittelt.

Das zweite Beispiel findet den Sinus des Werts der Variablen 
{\tt angle}.  C nimmt an, dass die Werte die wir mit {\tt sin()} und 
anderen trigonometrischen Funktionen  
({\tt cos}, {\tt tan}) benutzen in {\em Radian} (rad) vorliegen.  Um
von Grad in Radiant umzurechnen m�ssen wir den Wert durch
360 dividieren und mit $2 \pi$ multiplizieren.  

Wenn uns momentan nicht einf�llt wie die ersten 15 Ziffern von $\pi$ lauten, so k�nnen wir
diese mit der {\tt acos()} Funktion berechnen.  Der Arkuskosinus 
(oder invertierter Kosinus) von -1 ist $\pi$ (da der Kosinus von 
$\pi$ als Ergebnis -1 liefert).

\begin{verbatim}
    const double PI = acos(-1.0);
    double degrees = 90;
    double angle = degrees * 2 * PI / 360.0;
    
\end{verbatim}
%
WICHTIG: Bevor wir mathematische Funktionen in unserem 
Programm verwenden k�nnen, m�ssen wir die mathematische
Bibliothek in unser Programm einbinden. 
Daf�r reicht es die entsprechende {\bf Header-Datei}
in unser Programm einzuf�gen.

Header-Dateien enthalten Informationen die der Kompiler
ben�tigt um Funktionen nutzen zu k�nnen, die au�erhalb unseres
Progamms definiert wurden.
So haben wir zum Beispiel in unserem ``Hello, world!''-
Programm  eine Header-Datei mit dem Namen {\tt stdio.h} eingef�gt.
Zum Einf�gen nutzen wir den {\bf include} Befehl:

\begin{verbatim}
    #include <stdio.h>
\end{verbatim}
%

\index{Header-Datei}
\index{<stdio.h>}
\index{<math.h>}
\index{Header-Datei!stdio.h}
\index{Header-Datei!math.h}
\index{Bibliothek!stdio.h}
\index{Bibliothek!math.h}

Die Header-Datei {\tt stdio.h} enth�lt Informationen �ber die in C verf�gbaren �ber Ein- und 
Ausgabefunktionen  (Input und Output -- I/O).
Gleicherma�en enth�lt die Bibliothek  {\tt math.h}  Informationen
�ber mathematische Funktionen in C. \\
Wir k�nnen diese Datei zusammen mit
 {\tt stdio.h} in unser Programm einbinden:

% you can use any of the math functions, you have to
%include the math {\bf header file}. 

\begin{verbatim}
    #include <stdio.h>
    #include <math.h>
\end{verbatim}

\section {Komposition}
\label{composition}
\index{Komposition}
\index{Ausdruck}

So wie in der Mathematik, k�nnen wir in C Funktionen
zusammensetzten. 
%Just as with mathematical functions, C functions can be {\bf
%composed}, 
Das hei�t, wir k�nnen einen Ausdruck als Teil eines anderen 
verwenden. So l�sst sich zum Beispiel ein beliebiger Ausdruck als
Argument einer Funktion verwenden:

\begin{verbatim}
    double x = cos (angle + PI/2);
\end{verbatim}
%
Diese Anweisung nimmt den Wert von {\tt PI}, teilt ihn durch zwei und
addiert das Ergebnis zum Wert von {\tt angle}.  Die Summe wird dann
als Argument der {\tt cos()} Funktion benutzt.

Wir k�nnen auch das Resultat einer Funktion nehmen und es als
Argument einer weiteren Funktion benutzen:

\begin{verbatim}
    double x = exp (log (10.0));
\end{verbatim}
%
Diese Anweisung ermittelt zuerst den Logarithmus zur Basis $e$ von 10 
um $e$ anschlie�end mit diesem Ergebnis zu potenzieren. 
Das Resultat wird der Variablen {\tt x} zugewiesen. Ich hoffe
Ihnen ist klar, was das Ergebnis ist.

\section{Hinzuf�gen neuer Funktionen}
\index{Funktion!Definition}
\index{main()}
\index{Funktion!main()}

Bisher haben wir ausschlie�lich Funktionen benutzt die bereits
Teil der Programmiersprache sind. Es ist aber genauso gut m�glich
neue Funktionen hinzuzuf�gen.
Eigentlich haben wir das auch bereits getan, ohne es zu bemerken.
Wir haben unserem Programm bereits eine Funktion hinzugef�gt: {\tt main}. 
 %Actually, we have already seen one function definition: {\tt main}.  
Die Funktion mit Namen {\tt main} ist ganz besonders wichtig, weil
sie angibt, wo in einem Programm die Ausf�hrung der Befehle beginnt.
Die Syntax f�r die Definition von {\tt main()} ist aber die gleiche wie f�r
jede andere Funktion auch:

\begin{verbatim}
    DATENTYP FUNKTIONSNAME ( LISTE DER PARAMETER ) 
    {
        ANWEISUNGEN;
    }
\end{verbatim}
%
Wir k�nnen einer neuen Funktion einen beliebigen Namen 
geben, wir d�rfen sie aber nicht {\tt main} oder nach einem Schl�sselwort
der Programmiersprache benennen.
Die Liste der Parameter gibt an, welche Informationen 
bereitgestellt werden m�ssen, um die neue Funktion zu nutzen (oder {\bf aufzurufen}).
Diese Liste kann auch leer sein. 
In diesem Fall machen wir mit dem Schl�sselwort {\tt void}
klar, dass es keine Parameter gibt, hier also nichts vergessen wurde. 

Die {\tt main()}-Funktion hat in unseren Beispielen keine Parameter, 
wie wir aus dem Schl�sselwort {\tt (void)} zwischen den Klammern
der Funktionsdefinition entnehmen k�nnen.
Die ersten Funktionen die wir schreiben, kommen ebenfalls ohne
Parameter aus, sie haben auch keinen festgelegten Datentyp, so 
das die Syntax wie folgt aussieht:

\begin{verbatim}
    void PrintNewLine (void) 
    {
        printf ("\n");
    }
\end{verbatim}
%
Diese Funktion mit Namen {\tt PrintNewLine()} enth�lt nur eine einzige
Anweisung, welche eine Leerzeile auf dem Bildschirm darstellt.

Der Name unserer neuen Funktion beginnt mit einem Gro�buchstaben.
Die folgenden Worte des Funktionsnamens haben wir ebenfalls gro�
geschrieben. Dabei handelt es sich um eine oft genutzte Konvention,
der wir in diesem Buch folgen werden.
Die Gro�schreibung verhindert es, dass wir zuf�llig den gleichen
Namen wie eine bereits existierende Funktion w�hlen. Der Name der
Funktion selbst sollte eine treffende Beschreibung der Aufgabe dieser
Funktion sein.


%Notice that we start
%the function name with an uppercase letter. The following words of the
%function name are also capitalized. We will use this convention for the naming 
%of functions consistently throughout the book.

In unser {\tt main()}-Funktion k�nnen wir die neue Funktion jetzt aufrufen.
Dazu benutzen wir eine Syntax die vergleichbar ist mit dem Aufruf der
standardm��ig in C vorhandenen Funktionen:

\begin{verbatim}
    int main (void)
    {
       printf ("First Line.\n");
       PrintNewLine ();
       printf ("Second Line.\n");
       return EXIT_SUCCESS;
    }
\end{verbatim}
%
Die Ausgabe des Programms sieht dann folgenderma�en aus:

\begin{verbatim}
    First line.

    Second line.

\end{verbatim}
%
Haben Sie den zus�tzlichen Abstand zwischen den beiden Zeilen bemerkt?\\
Wie k�nnen wir noch mehr Abstand zwischen den Texten erzeugen?
Wir rufen die Funktion einfach mehrfach nacheinander auf:

\begin{verbatim}
  int main (void)
  {
      printf ("First Line.\n");
      NewLine ();
      NewLine ();
      NewLine ();
      printf ("Second Line.\n");
      return EXIT_SUCCESS;
  }
\end{verbatim}
%
Oder wir k�nnten eine neue Funktion schreiben die wir {\tt PrintThreeLines()} nennen und diese
Funktion gibt drei Leerzeilen auf dem Bildschirm aus: 

\begin{verbatim}
  void PrintThreeLines (void)
  {
    PrintNewLine ();  PrintNewLine ();  PrintNewLine ();
  }

  int main (void)
  {
      printf ("First Line.\n");
      PrintThreeLines ();
      printf ("Second Line.\n");
      return EXIT_SUCCESS;
  }
\end{verbatim}
%
Ein paar Dinge in diesem Programm sollten besonders beachtet werden:

\begin{itemize}

\item Wir k�nnen die selbe Funktion mehrfach aufrufen.  Das kommt in der
Tat in den meisten Programmen ziemlich oft vor und ist ein sinnvoller Umgang
mit Funktionen. 

\item Wir k�nnen aus einer Funktion eine andere Funktion aufrufen.  In unserem
Fall ruf {\tt main()} die Funktion {\tt PrintThreeLines()} und {\tt PrintThreeLines()}
die Funktion {\tt PrintNewLine()} auf.  Auch das ist n�tzlich und verbreitet der Fall.

\item In {\tt PrintThreeLines()} habe ich drei Anweisungen in die gleiche Zeile
geschrieben. Einerseits ist das syntaktisch erlaubt (Leerzeichen und Zeilenumbr�che
�ndern typischerweise die Bedeutung unseres Programm nicht). 
Andererseits ist es besser jede Anweisung in eine eigene
Programmzeile zu schreiben -- das Programm wird dadurch einfacher lesbar.
Ich weiche manchmal von dieser Regel ab, um in diesem Buch etwas Platz zu
sparen.
\end{itemize}

Bisher d�rfte vielleicht noch nicht klar geworden sein, warum wir uns
die M�he mit der Erstellung dieser neuen Funktionen machen.
Es gibt daf�r eine ganze Reihe von Gr�nden, ich m�chte an dieser Stelle
aber nur auf zwei von ihnen eingehen: 

%So far, it may not be clear why it is worth the trouble to
%create all these new functions.  Actually, there are a lot
%of reasons, but this example only demonstrates two:

\begin{enumerate}

\item In dem wir einen neue Funktion erstellen, erhalten wir die M�glichkeit 
einer Gruppe von Anweisungen einen Namen zu geben.
Funktionen k�nnen die Struktur eines Programms vereinfachen. Wir verbergen
komplexe Berechnungen hinter einem einfachen Befehl. 
Wir k�nnen f�r diesen Befehl  'sprechende' Bezeichnungen w�hlen, statt 
direkt obskuren Programmcode lesen zu m�ssen. Was ist klarer {\tt PrintNewLine()} 
oder {\tt printf("$\backslash$n")}?

\index{Sprechende Bezeichner}

\item Durch die Verwendung von eigenen Funktionen k�nnen wir
ein Programm k�rzer machen, indem wir wiederholende Anweisungen eliminieren.
So k�nnen wir auf einfache Weise neun aufeinanderfolgende
Lehrzeilen dadurch erzeugen, dass wir {\tt PrintThreeLines()} drei mal nacheinander
aufrufen. Wie gehen wir vor, wenn wir 27 Lehrzeilen ausgeben m�ssen?

\end{enumerate}

\section {Definitionen und Aufrufe}

Wenn wir alle Code-Fragmente der vorigen Abschnitte zusammenf�gen, sieht
das komplette Programm wie folgt aus:

\begin{verbatim}
  #include <stdio.h>
  #include <stdlib.h>

  void PrintNewLine (void) 
  {
      printf ("\n");
  }

  void PrintThreeLines (void)
  {
      PrintNewLine ();  PrintNewLine ();  PrintNewLine ();
  }

  int main (void)
  {
     printf ("First Line.\n");
     PrintThreeLines ();
     printf ("Second Line.\n");
     return EXIT_SUCCESS;
  }
\end{verbatim}

Das Programm enth�lt drei Funktionsdefinitionen: {\tt PrintNewLine()},
{\tt PrintThreeLine()} und {\tt main()}.

Innerhalb der Definition von {\tt main()} befindet sich eine Anweisung
welche die Funktion {\tt PrintThreeLine()} aufruft. Auf die gleiche
Art ruft  {\tt PrintThreeLine()} die Funktion {\tt PrintNewLine()} drei mal auf.  
Auffallend ist, dass die Definition jeder Funktion vor der Stelle erfolgt,
an der die Funktion aufgerufen wird.

Diese Reihenfolge ist in C notwendig. Die Deklaration einer Funktion muss
erfolgt sein, bevor die Funktion verwendet werden kann. Der Grund daf�r liegt
in der Art und Weise wie der Compiler versucht unser Programm zu �bersetzen.
Er geht dabei den Quellcode nur ein einziges mal durch und muss daher bereits
alle Funktionen kennen, bevor sie verwendet werden.  
\hint

Sie k�nnen ja einmal versuchen die Reihenfolge der Funktionen zu vertauschen
und das Programm zu kompilieren. Dies wird in den meisten F�llen zu einer
Fehlermeldung des Kompilers f�hren.


\todo{ Impliziter Typ int!}
\todo{Funktionsprotoyp, Index}
\todo{Kapitel FAQ einf�hren: Fehlermeldung}

Wenn uns diese Verhalten nicht gef�llt, k�nnen wir es durch das Hinzuf�gen eines 
\textbf{Funktionsprototyp} �ndern.
\index{Funktionsprototyp}
\index{Prototyp (Funktion)}
Ein Funktionsprototyp erlaub es dem Compiler zu erkennen, welche Funktionsnamen,
Typen und Parameter in unserem Programm verwendet werden, bevor wir die Funktion
�berhaupt aufgeschrieben haben. 
Dazu m�ssen wir vor \texttt{main()}  die folgenden Zeilen einf�gen. Danach ist die Reihenfolge
der Funktionsdefinition nicht mehr wichtig.\\
ACHTUNG: Funktionsprotoypen m�ssen mit einem Semikolon (;) abgeschlossen werden:
\hint
\begin{verbatim}
  #include <stdio.h>
  #include <stdlib.h>

  void PrintNewLine (void);
  void PrintThreeLines (void);

  int main (void)
  ...
\end{verbatim}

\section {Programme mit mehreren Funktionen}

Wenn wir uns den Quelltext eines C Programms mit mehreren
Funktionen anschauen, so ist es verf�hrerisch, diesen von
oben nach unten durchzulesen (als Menschen lesen wir Texte so).

Das kann allerdings schnell zur Verwirrung f�hren, da die einzelnen
Funktionen in der Regel unabh�ngig voneinander sind. F�r das 
Verst�ndnis ist es deshalb
besser den Quelltext in der {\bf Reihenfolge der Programmausf�hrung}
zu lesen.

Die Ausf�hrung beginnt immer mit der ersten Anweisung in {\tt main()},
unabh�ngig, davon, wo sich die  {\tt main()}-Funktion im Programm befindet
(sehr oft ist dies am Ende des Programms).
Anweisungen werden der Reihe nach, eine nach der anderen ausgef�hrt,
bis wir einen Funktionsaufruf erreichen.
Funktionsaufrufe k�nnen wir uns wie eine Umleitung in der Programmausf�hrung
vorstellen. Anstatt die n�chstfolgende Anweisung zu bearbeiten wird die
Ausf�hrung des Programms mit der ersten Anweisung der aufgerufenen
Funktion fortgesetzt. Es werden dann alle Anweisungen der Funktion
ausgef�hrt und erst danach wird die Programmausf�hrung an die Stelle
im Programm zur�ckgekehrt von der wir die Funktion aufgerufen haben.

Das h�rt sich einfach genug an, allerdings m�ssen wir bedenken, dass
eine Funktion wiederum selbst andere Funktionen aufrufen kann.
So kommt es, dass w�hrend wir uns mitten in der  {\tt main()}-Funktion befinden
die Ausf�hrung pl�tzlich in  {\tt PrintThreeLines()} fortgesetzt wird und die  Anweisungen
dort ausgef�hrt werden. Bei der Ausf�hrung von {\tt PrintThreeLines()} werden wir drei 
mal unterbrochen um Anweisungen in der Funktion {\tt PrintNewLine()} auszuf�hren.

Gl�cklicherweise ist C geschickt darin die �bersicht �ber die Programmausf�hrung zu behalten. 
Jedes mal, wenn {\tt PrintNewLine()} abgearbeitet wurde,  setzt das Programm
genau an der Stelle fort, wo {\tt PrintThreeLine()} verlassen wurde und 
kehrt schlie�lich zur�ck zu {\tt main()} um das Programm erfolgreich
zu beenden.

Was ist die Moral dieser Geschichte?  Wenn wir ein Programm verstehen wollen
sollten wir besser nicht von oben nach unten lesen, sondern dem
Fluss der Programmausf�hrung folgen.

\section {Parameter und Argumente}
\index{Parameter}
\index{Argumente}

Einige der eingebauten Funktionen die wir bisher benutzt haben 
hatten {\bf Parameter}. Ein Parameter ist ein Objekt welches
die Funktion ben�tigt um ihre Aufgabe zu erf�llen.
Wenn wir zum Beispiel den Sinus einer Zahl bestimmen wollen,
so m�ssen wir angeben um welche Zahl es sich dabei handelt.
Demzufolge hat die {\tt sin()}-Funktion einen {\tt double} Wert als Parameter.
% of the built-in functions we have used have {\bf parameters},
%which are values that you provide to let the function do its job.  


Einige Funktionen besitzen mehr als einen Parameter, wie zum 
Beispiel die {\tt pow()}-Funktion zu Potenzberechnung,  
welche zwei {\tt double}s, Basis und Exponent, f�r die Berechnung 
ben�tigt.

Wir m�ssen weiterhin beachten, dass in allen diesen F�llen nicht 
nur die Anzahl der Parameter eine Rolle spielt, sondern auch die
Beachtung des Datentyps wichtig ist.
Es sollte also nicht �berraschen, dass wir in der Funktionsdefinition
auch den jeweiligen Typ eines Parameters in der Parameterliste
mit angeben m�ssen. Ein Beispiel: 
%Notice that in each of these cases we have to specify not
%only how many parameters there are, but also what type they
%are.  So it shouldn't surprise you that when you write a
%function definition, the parameter list indicates the type of
%each parameter.  For example:

\begin{verbatim}
  void PrintTwice (char phil) 
  {
      printf("%c%c\n", phil, phil);
  }
\end{verbatim}
%
Diese Funktion hat einen Parameter, mit Namen {\tt phil}, dieser
ist vom Typ {\tt char}.  Je nachdem welchen Wert der Parameter 
w�hren des Funktionsaufrufs hat (und zu diesem Zeitpunkt
haben wir keine Idee welcher Wert das ist), dieser Wert wird 
zwei mal ausgegeben, gefolgt von einem Zeilenumbruch.
Ich habe den Namen {\tt phil} f�r den Parameter gew�hlt um zu
zeigen, dass der Name des Parameters v�llig nebens�chlich ist.
Wir k�nnen den Namen frei w�hlen, meistens ist es sinnvoll  
anstelle von sinnfreien Namen wie {\tt phil} 'sprechende' Bezeichner zu verwenden.


Um diese Funktion in unserem Programm aufzurufen, m�ssen wir
einen {\tt char} Wert angeben.
Unsere {\tt main()}-Funktion k�nnte zum Beispiel folgenderma�en aussehen:


\begin{verbatim}
  int main (void) 
  {
      PrintTwice ('a');
      return EXIT_SUCCESS;
  }
\end{verbatim}
%
Der bereitgestellte {\tt char} Wert wird {\bf Argument} genannt und 
man sagt das Argument wird an die Funktion {\bf �bergeben}.  In unserem
Fall wird der Wert {\tt 'a'} als Argument an die Funktion
{\tt PrintTwice()} �bergeben, wo er zwei mal ausgegeben wird.

Alternativ, h�tten wir auch eine {\tt char} Variable deklarieren k�nnen und
diese als Argument benutzen k�nnen:

\begin{verbatim}
  int main (void) 
  {
      char argument = 'b';
      PrintTwice (argument);
      return EXIT_SUCCESS;
  }
\end{verbatim}
%
WICHTIG: Wir sollten uns das Folgende gut einpr�gen: 
der Name der Variablen ({\tt argument}) die wir als Argument an die 
Funktion �bergeben, hat nichts mit dem Namen des Parameters
der Funktion ({\tt phil}) zu tun.  Ich m�chte das gleich noch einmal wiederholen:
\hint

\begin{quote}

{\bf Der Name der Variablen die wir als Argument �bergeben hat nichts mit dem
Namen des Parameters zu tun!}

\end{quote}

Die Namen k�nnen gleich sein, oder sie k�nnen sich unterscheiden.
Wichtig ist, sie stellen unterschiedliche Objekte (Speicherstellen im
Hauptspeicher des Rechners) dar und haben nur eine Gemeinsamkeit:
sie besitzen den gleichen Wert (in unserem Fall das Zeichen {\tt 'b'}).

Die Werte die wir als Argumente �bergeben, m�ssen den gleichen Typ
haben wie die Parameter der aufgerufenen Funktion.
Diese Regel ist wichtig, obwohl auch hier C manchmal Argumente automatisch
von einen Typ in einen anderen umwandelt. 
Momentan ist es aber wichtig die generelle Regel zu kennen und
sich sp�ter um die Ausnahmen zu k�mmern.
%important, but it is sometimes confusing because C sometimes
%converts arguments from one type to another automatically.  For
%now you should learn the general rule, and we will deal with
%exceptions later.

\section {Parameter und Variablen sind lokal g�ltig}
\index{Variablen!lokale}
\index{Lokale Variablen}


Parameter und Variablen existieren nur innerhalb ihrer
eigenen Funktionen in denen sie definiert wurden.  
Innerhalb von {\tt main()} gibt es kein {\tt phil}.
Wenn wir dort versuchen  {\tt phil} zu verwenden wird sich 
der Kompiler beschweren.
Gleichfalls gilt, innerhalb von {\tt PrintTwice()} k�nnen wir nicht
auf das Objekt {\tt argument} zugreifen.

Variablen dieser Art nennt man {\bf lokale Variablen}.  
Um die �bersicht �ber alle Funktionsparameter und lokalen
Variablen zu behalten ist es n�tzlich ein so genanntes 
{\bf Stack Diagramm} zu zeichnen.  
Stack Diagramme zeigen �hnlich wie Zustandsdiagramme
den Wert jeder Variablen an. Der Unterschied besteht darin,
dass in Stack Diagrammen die Variablen innerhalb gr��erer
K�stchen gezeichnet werden. Diese K�stchen stellen die
Funktionen dar, zu denen diese Variablen geh�ren.
So sieht das Stack Diagram f�r {\tt PrintTwice()} 
folgenderma�en aus:

\index{Diagramm!Stack}
\index{Stackdiagramm}
\todo{Stackdiagramm umdrehen, auch an anderen Stellen}
\vspace{0.1in}
\centerline{\epsfig{figure=figs/stack.pdf,width=6cm}}
\vspace{0.1in}
%
\todo{hei�t das bei C Instanz? Kopie?}

Jedes mal, wenn eine Funktion aufgerufen wird, wird die Programmausf�hrung
in der aufrufenden Funktion unterbrochen und zur aufgerufenen
Funktion gesprungen. Die aufgerufene Funktion k�nnen wir uns als eigenst�ndiges Exemplar
des Programmcodes der Funktion im Arbeitsspeicher des Computers vorstellen.
Jede Funktion enth�lt ihre eigenen Parameter und lokalen Variablen und ist unabh�ngig
von allen anderen Funktionen des Programms. Wird die aufgerufene Funktion beendet,
kehrt das Programm zur aufrufenden Funktion zur�ck.

Im Diagramm wird jede aufgerufene Funktion durch einen
Kasten mit dem Funktionsnamen an der Au�enseite und den
Variablen und Parametern im Inneren dargestellt.

In unserem Beispiel hat {\tt main()} eine lokale Variable {\tt argument}, und
keine Parameter.  {\tt PrintTwice()} hat keine lokalen Variablen aber 
einen Parameter mit Namen {\tt phil}.

Ein Stack Diagramm ist ein dynamische Diagramm. Im Programmablauf
werden Funktionen aufgerufen und wieder beendet.
Ein Stack Diagramm stellt also einen bestimmten Zeitpunkt in der 
Programmabarbeitung dar. 


\section {Funktionen mit mehreren Parametern}
\index{Parameter!mehrere}
\index{Funktion!mehrere Parameter}
%\index{class!Time}

Die Syntax f�r die Deklaration und den Aufruf von Funktionen mit
mehreren Parametern ist die Quelle vieler Programmierfehler. Zuerst
m�ssen wir daran denken, dass wir bei der Definition eine Funktion 
den Typ jedes Parameters der Funktion mit angeben m�ssen. Zum Beispiel:

\begin{verbatim}
  void PrintTime (int hour, int minute) 
  {
    printf ("%i", hour);
    printf (":");
    printf ("%i", minute);
  }
\end{verbatim}
%
Es erscheint verlockend hier einfach die zweite Typangabe 
wegzulassen und {\tt (int hour, minute)} zu schreiben.
Diese Schreibweise ist leider nur bei der Deklaration von Variablen
erlaubt und nicht f�r Parameter.

Eine weiteres Missverst�ndnis besteht darin, dass wir 
den Typ der Parameter angeben,
den Typ der Argumente aber nicht angeben m�ssen, ja sogar
einen Fehler machen, wenn wir den Typ der Argumente 
mit aufschreiben.
Der folgende Code ist falsch!

\begin{verbatim}
    int hour = 11;
    int minute = 59;
    PrintTime (int hour, int minute);   /* WRONG! */
\end{verbatim}
%
Der Kompiler kennt den Typ von {\tt hour} und {\tt minute}. 
Wir haben ihn bei der Deklaration angegeben. 
Es ist daher nicht n�tig und erlaubt den Typ mit anzugeben,
wenn wir Variablen als Argumente einer Funktion verwenden.
Die korrekte Syntax lautet: {\tt PrintTime (hour, minute);}

\section {Funktionen mit Ergebnissen}
\index{Funktionen mit Ergebnissen}
\index{Funktionen!Ergebnisse}

Es sollte mittlerweile aufgefallen sein, dass einige Funktionen die wir
benutzen (wie zum Beispiel die mathematischen Funktionen) Ergebnisse
liefern. Andere Funktionen wie {\tt PrintNewLine()} f�hren bestimmte
Aktionen durch, liefern aber kein Ergebnis an die aufrufende Funktion zur�ck.
Dieses Verhalten l�sst einigen Fragen offen:

\begin{itemize}

\item Was passiert, wenn wir eine Funktion aufrufen und
danach nichts mit dem Resultat der Funktion machen
(wir weisen es keiner Variablen zu und wir nutzen
es auch nicht als Teil eines gr��eren Ausdrucks)?
%do anything with the result (i.e. you don't assign it to
%a variable or use it as part of a larger expression)?

\item Was passiert, wenn wir eine Funktion die kein Resultat
liefert, als Teil eines Ausdrucks verwenden, wie zum Beispiel
{\tt PrintNewLine() + 7} ?

\item K�nnen wir auch Funktionen schreiben, die uns Resultate
liefern, oder m�ssen wir uns mit Funktionen vom Typ 
{\tt PrintNewLine()} und {\tt PrintTwice()} begn�gen?

\end{itemize}

Die Antwort auf die dritte Frage ist ``ja, wir k�nnen Funktionen 
schreiben welche ein Resultat zur�ckgeben,'' und wir werden
das in einigen Kapiteln auch tun. 
Ich �berlasse es ihnen die Antworten auf die ersten beiden 
Fragen durch Probieren herauszufinden. Jedes mal, wenn die Frage auftaucht
was in der Programmiersprache C erlaubt und m�glich ist,
ist es eine gute Ideen den C-Compiler zu fragen. 


\section{Glossar}

\begin{description}

\item[Konstante (engl: \emph{constant}):] Eine bennante Speicherstelle, �hnlich einer Variable.
Im Interschied zu Variaben k�nnen Konstanten nicht mehr ver�ndert werden, nachdem ihr Wert
festgelegt wurde: \\z.B. \texttt{\#define PI 3.141592}

\item[Flie�komma (engl: \emph{floating-point}):] Der Typ einer Variable (oder eines Werts) welche
reelle Zahlen speichern kann.  Es existieren mehrere Flie�kommatypen 
in C; in diesem Buch verwenden wir meistens {\tt double}.

\item[Initialisierung (engl: \emph{initialization}):]  Eine Anweisung welche eine neue Variable
definiert und gleichzeitig dieser Variable einen Wert zuweist.

\item[Definition (engl: \emph{definition}):] Eine Deklaration (siehe \ref{Glossary:Declaration}) gibt 
nur bekannt, das eine Variable oder eine Funktion existiert (Funktionsprototyp). Eine Definition 
spezifiziert die Variable und Funktion und legt sie im Speicher an. Es darf dabei mehrere Deklarationen aber
nur eine Definition geben.

%\item[Instanz (engl: \emph{instance}):] Eine Instanz ist ein eigenst�ndiges Exemplar des Programmcodes einer 
%Funktion im Arbeitsspeicher des Computers.
%Jede Instanz einer Funktion enth�lt ihre eigenen Parameter und lokalen Variablen.
%Instanzen werden beim Funktionsaufruf erzeugt und existieren nur w�hren der Abarbeitung
%der Funktion. 

\item[Funktion (engl: \emph{function}):]  Eine eigenst�ndige Folge von Anweisungen die �ber einen
Funktionsnamen aufgerufen werden kann. Funktionen k�nnen Parameters besitzen und einen Wert
an die aufrufenden Funktion zur�ckgeben - m�ssen aber nicht.

\item[Parameter (engl: \emph{parameter}):]  Eine Variable in einer Funktion, deren Wert 
beim Funktionsaufruf durch die aufrufende Funktion bestimmt wird.\\
Der Name und Wert des Parameters ist nur innerhalb der Funktion g�ltig!

\item[Argument (engl: \emph{argument}):]  Der Ausdruck (Wert), mit dem die Funktion aufgerufen wird.
Argumente m�ssen in Typ und Reihenfolge mit den Parametern der Funktion �bereinstimmen.



\item[Aufruf (engl: \emph{call}):]  f�hrt dazu, dass eine Funktion ausgef�hrt wird.

\index{Funktion}
\index{Parameter}
\index{Argument}
\index{Aufruf}
\index{Initialisierung}
\index{Instanz}


\end{description}


\section{�bungsaufgaben}
\setcounter{exercisenum}{0}

\ifthenelse {\boolean{German}}{ \begin{exercise}

%\subsubsection*{Deutsche �bersetzung der Aufgabe}
In dieser �bung sollen Sie das Lesen von Programmcode
praktizieren. Sie sollen den Ablauf der Ausf�hrung von Programmen
mit mehreren Funktionen verstehen und nachvollziehen lernen.


\begin{enumerate}

\item 
Was gibt dieses Programm auf dem Bildschirm aus?
Geben Sie pr�zise an wo sich Leerzeichen und Zeilenumbr�che
befinden. 

HINWEIS: Beginnen Sie mit einer verbalen Beschreibung dessen
was die Funktionen {\tt Ping} und {\tt Baffle} tun, wenn sie aufgerufen
werden.

\begin{verbatim}
#include <stdio.h>
#include <stdlib.h>

  void Ping (void) 
  {
    printf (".\n");
  }

  void Baffle (void) 
  {
    printf ("wug");
    Ping ();
  }

 void Zoop (void) 
 {
    Baffle ();    
    printf ("You wugga ");
    Baffle ();
  }

  int main (void) 
  {
    printf ("No, I ");
    Zoop ();
    printf ("I ");
    Baffle ();
    return EXIT_SUCCESS;
  }
\end{verbatim}


\item 
Zeichnen Sie ein Stackdiagram welches den Status des Programms
wiedergibt wenn {\tt Ping} zum ersten Mal aufgerufen wird.

\end{enumerate}
\end{exercise}

%%%%%%%%%%%%%%%%%%%%%%%%%%%%%%

\begin{exercise}

%\subsubsection*{Deutsche �bersetzung der Aufgabe}

In dieser �bung lernen Sie wie man Funktionen mit Parametern 
schreibt und aufruft.

\begin{enumerate}

\item 
Schreiben Sie die erste Zeile einer Funktion mit dem Namen {\tt Zool}.
Die Funktion hat drei Parameter: ein {\tt int} und zwei {\tt char}.

\item 
Schreiben Sie eine Code-Zeile in der Sie {\tt Zool} aufrufen und
die folgenden Werte als Argumente �bergeben: {\tt 11}, den Buchstaben {\tt a}, und 
den Buchstaben {\tt z}.
\end{enumerate}
\end{exercise}


%%%%%%%%%%%%%%%%%%%%%%%%%%%%%%%


\begin{exercise}

%\subsubsection*{Deutsche �bersetzung der Aufgabe}

In dieser �bung werden wir ein Programm aus einer vorigen �bung anpassen
und ver�ndern, so dass eine Funktion mit Parametern zum Einsatz kommt. Starten
mit einer funktionsf�higen Programmversion.
%~\ref{ex.date}.

\begin{enumerate}

\item 
Schreiben Sie eine Funktion mit dem Namen {\tt PrintDateAmerican}
diese hat die folgenden Parameter day, month und year 
und gibt das Datum im amerikanischen Standardformat aus.

\item 
Testen Sie die Funktion indem Sie diese aus {\tt main} heraus aufrufen
und die entsprechenden Parameter als Argumente �bergeben.
Das Ergebnis sollte folgendem Muster entsprechen:
%
\begin{verbatim}
3/29/2009
\end{verbatim}
%
\item 
Nachdem Sie die Funktion {\tt PrintDateAmerican} erfolgreich erstellt und
ausgef�hrt haben, schreiben Sie eine weitere Funktion 
{\tt PrintDateEuropean} welche das Datum im europ�ischen Format
ausgibt.

\end{enumerate}

\end{exercise}

%%%%%%%%%%%%%%%%%%%%%%%%%%%%%%%

\begin{exercise}
\label{ex.multadd}


%\subsubsection*{Deutsche �bersetzung der Aufgabe}

Viele Berechnungen lassen sich �bersichtlich als  ``multadd''
Operation ausf�hren, dazu wird mit drei Operanden folgende Berechnung
durchgef�hrt {\tt a*b + c}.  Einige Prozessoren bieten f�r diesen 
Befehl sogar eine Hardwareimplementierung f�r Gleitkommazahlen.

\begin{enumerate}

\item 
Erstellen Sie ein neues Programm mit dem Namen {\tt Multadd.c}.

\item 
Schreiben Sie eine Funktion {\tt Multadd} welche drei  {\tt doubles}
als Parameter besitzt und  welche das Ergebnis der Multaddition ausgibt.

\item 
Schreiben Sie eine {\tt main} Funktion welche {\tt Multadd} 
durch den Aufruf mit einigen einfachen Parametern testet
und das Ergebnis ausgibt. 
So sollte zum Beispiel f�r die Parameter {\tt 1.0, 2.0, 3.0} als 
Ergebnis {\tt 5.0} ausgegeben werden.

\item 
Benutzen Sie {\tt Multadd} in der {\tt main} Funktion um den folgenden
Wert zu berechnen:
%
\begin{eqnarray*}
& \sin \frac{\pi}{4} + \frac{\cos \frac{\pi}{4}}{2} & 
%\\
%\\
%& \log 10 + \log 20 &
\end{eqnarray*}
%
\item 
Schreiben Sie eine Funktion {\tt Yikes} welche ein {\tt double} als 
Parameter �bernimmt und {\tt Multadd} f�r die Berechnung und
Ausgabe benutzt:
%
\begin{eqnarray*}
x e^{-x} + \sqrt{1 - e^{-x}}
\end{eqnarray*}
%
HINWEIS: Die mathematische Funktion f�r die Berechnung von  $e^x$ lautet {\tt double exp(double x);}.

\end{enumerate}

In der letzten Aufgabe sollen Sie eine Funktion schreiben, welche ihrerseits
eine selbst erstellte Funktion aufruft. Dabei sollten Sie stets daran denken
die erste Funktion ausgiebig zu testen bevor Sie mit der Arbeit an der 
zweiten Funktion beginnen.
Ansonsten kann es vorkommen, dass Sie gleichzeitig zwei Methoden
debuggen m�ssen - ein sehr m�hsames Unterfangen.

Ein weiteres Ziel dieser �bung ist es ein spezielles Problem als Teil einer
allgemeineren Klasse von Problemen zu erkennen. Wenn immer m�glich sollten
Sie versuchen Programme zu entwickeln, die allgemeine Probleme l�sen.


\end{exercise}}
{\input{exercises/Exercise_3_english}}



%!TEX root = Main_german.tex

% LaTeX source for textbook ``How to think like a computer scientist''
% Copyright (C) 1999  Allen B. Downey

% This LaTeX source is free software; you can redistribute it and/or
% modify it under the terms of the GNU General Public License as
% published by the Free Software Foundation (version 2).

% This LaTeX source is distributed in the hope that it will be useful,
% but WITHOUT ANY WARRANTY; without even the implied warranty of
% MERCHANTABILITY or FITNESS FOR A PARTICULAR PURPOSE.  See the GNU
% General Public License for more details.

% Compiling this LaTeX source has the effect of generating
% a device-independent representation of a textbook, which
% can be converted to other formats and printed.  All intermediate
% representations (including DVI and Postscript), and all printed
% copies of the textbook are also covered by the GNU General
% Public License.

% This distribution includes a file named COPYING that contains the text
% of the GNU General Public License.  If it is missing, you can obtain
% it from www.gnu.org or by writing to the Free Software Foundation,
% Inc., 59 Temple Place - Suite 330, Boston, MA 02111-1307, USA.


\chapter{Abh�ngigkeiten und Rekursion}
\label{condrecursion}


\section{Bedingte Abarbeitung}
\index{Abh�ngigkeit}
\index{Anweisung!Verzweigung}

Unsere bisherigen Programme hatten eine Gemeinsamkeit.
Es wurden alle Befehle in strenger Reihenfolge nacheinander abgearbeitet.
Solche Programme haben die Eigenschaft, dass sie jedes mal genau
die gleichen Aktionen durchf�hren und die gleichen Ergebnisse produzieren.

F�r die meisten realen Anwendungen brauchen wir aber auch die M�glichkeit
das Vorliegen bestimmter Bedingungen �berpr�fen zu k�nnen und
die Abarbeitungsreihenfolge der Befehle (das Verhalten unseres Programms) 
an die jeweiligen �u�eren und inneren Bedingungen anpassen zu k�nnen.
  
%In order to write useful programs, we almost always need the ability
%to check certain conditions and change the behavior of the program
%accordingly.  
{\bf Bedingte Anweisungen} geben uns diese M�glichkeit.  Die einfachste 
Form ist dabei die {\tt if}-Anweisung:

\begin{verbatim}
    if (x > 0) 
    {
        printf ("x ist positiv\n");
    }
\end{verbatim}
%
Der angegebene Ausdruck in Klammern ist die Bedingung.
Wenn er \emph{wahr} ist,  werden die Anweisungen zwischen den den geschweiften
Klammern ausgef�hrt. Die �ffnende und schlie�ende geschweifte Klammer bildet
einen so genannten {\bf Anweisungsblock} oder einfach {\bf Block}. Was das genau ist 
erkl�re ich im n�chsten Abschnitt genauer. Einfach gesagt handelt es sich dabei um
eine Gruppierung von Anweisungen. 

Wenn die Bedingung \emph{falsch} ist, wird das Programm mit dem n�chsten
Befehl nach dem Block fortgesetzt. In unserem Beispiel passiert gar nichts.


\index{Operator!Vergleich}
\index{Vergleichsoperator}

Die Bedingung  kann die folgenden {\bf Vergleichsoperatoren} enthalten:

\begin{verbatim}
    x == y               /* x ist gleich y */
    x != y               /* x ist nicht gleich (ungleich) y */
    x > y                /* x ist gr��er als y */
    x < y                /* x ist kleiner als y */
    x >= y               /* x ist gr��er als oder gleich y */
    x <= y               /* x ist kleiner als oder gleich y */
\end{verbatim}
%
Diese Operationen sollten aus dem Bereich der Mathematik bereits bekannt sein.
Allerdings benutzt C eine Syntax die von den gebr�uchlichen mathematischen
Symbolen abweicht, wo wir $=$, $\neq$, $\ge$ und $\le$ verwenden. Ein oft gemachter
Fehler ist die Verwendung eines einzelnen {\tt =} anstelle des doppelten {\tt ==} 
f�r den Ausdruck der Gleichheit zweier Operanden.  
Erinnern wir uns, {\tt =} ist der Zuweisungsoperator und {\tt ==} ist
ein Vergleichsoperator.  Bitte auch beachten: {\tt =<} or {\tt =>} sind keine g�ltigen
Operatoren.

WICHTIG: Die Ausdr�cke auf den beiden Seiten des Vergleichsoperators sollten
vom gleichen Typ sein. Wir k�nnen zwar {\tt int}s mit {\tt double}s  vergleichen,
es wird eine automatische Typumwandlung durchgef�hrt. Man sollte aber darauf achten,
nicht auf Gleichheit zu testen ({\tt ==}) sobald ein Operator als Flie�kommazahl dargestellt wird.
Das gilt auch, wenn {\tt double}s mit {\tt double}s verglichen werden.

Ungl�cklicherweise, kennen wir derzeit noch keine Methode um Strings (Zeichenketten) zu 
vergleichen!  Es gibt einen Weg das zu tun, aber wir m�ssen leider noch einige Kapitel
darauf warten.

\section{Anweisungsbl�cke}
\index{Anweisungsbl�cke}
\index{Block}

Die Anweisung stellt einen einzelnen Befehl oder Abarbeitungsschritt in einem Programm
dar. Anstelle einer einfachen Anweisung kann in einem C Programm aber immer auch eine 
zusammengesetzte Anweisung, ein
so genannter {\bf Block}, geschrieben werden.

Einen Anweisungsblock wird mit Hilfe der geschweiften Klammern {\tt \{} und {\tt \}} gebildet.
C behandelt einen Anweisungsblock so, als w�re es eine einzelne Anweisung. Das ist
besonders wichtig bei der bedingten und wiederholten Abarbeitung von Programmteilen.
Wir k�nnen damit mehrere Anweisungen zusammenfassen und diese in Abh�ngigkeit von
der zu �berpr�fenden Bedingungen ausf�hren lassen. Im folgenden Beispiel werden alle
Anweisungen zwischen den geschweiften Klammern ausgef�hrt, wenn die Variable  {\tt x}
gr��er als Null ist. Wenn  {\tt x} kleiner oder gleich Null ist, wird der komplette Block nicht
ausgef�hrt:

\begin{verbatim}
    if (x > 0) 
    {
        printf ("x hat den Wert %i\n", x);
        printf ("x ist positiv\n");
    }
\end{verbatim}

Einzelne Anweisungen werden mit einen Semikolon ({\tt ;}) abgeschlossen. 
Eine Blockanweisung wird durch schlie�ende Klammer  {\tt \}} beendet. 
Wir m�ssen daher am Ende eines Blocks kein zus�tzliches Semikolon anf�gen.

%Eigene Variablen definieren. 

\section{Der Modulo-Operator}
\index{Modulorechnung}
\index{Operator!Modulo}

Der Modulo-Operator kann auf ganzzahlige Werte (und ganzzahlige Ausdr�cke)
angewendete werden und liefert uns den {\em ganzzahligen  Divisionsrest} wenn
wir den ersten Operanden durch den zweiten Operanden teilen.
In C wird der Modulo-Operator durch das Prozentzeichen
{\tt \%} dargestellt.  Die Syntax ist genau die gleiche wie bei anderen mathematischen 
Operatoren mit zwei Operanden:

\begin{verbatim}
    int quotient = 11 / 4;
    int rest     = 11 % 4;
\end{verbatim}
%
Das Ergebnis der ersten Berechnung (Ganzzahldivision!) ist 2.  Die zweite
Berechnung liefert das Ergebnis 3, denn 11 geteilt durch 4 ergibt 2 mit dem Rest~3.

Die Modulorechnung kann erstaunlich n�tzlich sein. So kann man zum Beispiel 
auf einfache Weise �berpr�fen, ob eine Zahl durch eine andere Zahl teilbar ist.
Wenn {\tt x \% y} den Rest Null liefert, dann ist {\tt x} durch {\tt y} teilbar.

Weiterhin k�nnen wir die Modulorechnung daf�r verwenden um die 
rechts stehenden Ziffern einer Zahl zu extrahieren. 
So liefert uns die Operation {\tt x \% 10} die Einerstelle des in der Variable {x} gespeicherten Werts im Dezimalsystem.
Mit der Operation {\tt x \% 100} lassen sich die letzten zwei Ziffern extrahieren.

% you can use the modulus operator to extract the rightmost
%digit or digits from a number.  For example, {\tt x \% 10} yields
%the rightmost digit of {\tt x} (in base 10).  Similarly
%{\tt x \% 100} yields the last two digits.

\section {Alternative Ausf�hrung}
\label{alternative}
\index{Bedingung!alternative}

Eine zweite Form der bedingten Ausf�hrung ist die alternative Ausf�hrung
bei der zwei M�glichkeiten existieren und die Bedingung festlegt, welche
davon zur Ausf�hrung kommt. Die Syntax sieht folgenderma�en aus:

\begin{verbatim}
    if (x%2 == 0)
    {
        printf ("x ist gerade\n");
    } 
    else 
    {
        printf ("x ist ungerade\n");
    }
\end{verbatim}
%
Wenn das Ergebnis der Division von {\tt x} durch 2 Null ergibt, dann wissen
wir das {\tt x} gerade ist. 
Unser Programm gibt dann die Nachricht  {\tt x ist gerade} auf dem Bildschirm aus.

Wenn die Bedingung falsch ist, wird der zweite Anweisungsblock ausgef�hrt. 
Da die Bedingung nur wahr oder falsch sein kann, wird genau eine der 
beiden Alternativen ausgef�hrt. Es ist nicht notwendig im zweiten Anweisungsblock die
Bedingung erneut zu �berpr�fen.

So nebenbei, wenn wir den Eindruck haben, dass wir in unserem Programm
die Parit�t (Geradzahligkeit, Ungeradzahligkeit) von Zahlen �fters �berpr�fen m�ssen,
dann kann es sinnvoll sein diesen Code in eine Funktion einzubetten:

\begin{verbatim}
    void PrintParity (int x) 
    {
        if (x%2 == 0) 
        {
            printf ("x ist gerade\n");
        } 
        else 
        {
            printf ("x ist ungerade\n");
        }
    }
\end{verbatim}
%
Wir haben jetzt eine Funktion {\tt PrintParity()} die uns auf dem Bildschirm 
anzeigt, ob ein �bergebener Integerwert gerade oder ungerade ist.
Wir k�nnen die Funktion in {\tt main()} wie folgt aufrufen:

\begin{verbatim}
     PrintParity (17);
\end{verbatim}
%
Wir m�ssen immer daran denken, dass, wenn wir eine Funktion {\em aufrufen}
der Typ der Argumente nicht mit angeben wird.
C kennt den Typ der verwendeten Variablen und Konstanten. Wir machen einen
Fehler, wenn wir die Funktion in der folgenden Weise aufrufen:

\begin{verbatim}
    int number = 17;
    PrintParity (int number);         /* WRONG!!! */
\end{verbatim}

\section {Mehrfache Verzweigung}
\index{Verzweigung!mehrfache}

Manchmal kommt es vor, dass wir eine ganze Reihe von
zusammengeh�rigen Bedingungen �berpr�fen und eine
Auswahl aus mehreren m�glichen Aktionen treffen wollen.

Es gibt verschiedene M�glichkeiten das zu erreichen. Eine davon
ist die {\bf Verkettung} einer Serie von {\tt if}s und {\tt else}s:

\begin{verbatim}
    if (x > 0) 
    {
        printf ("x is positive\n");
    } 
    else if (x < 0) 
    {
        printf ("x is negative\n");
    } 
    else 
    {
        printf ("x is zero\n");
    }
\end{verbatim}
%
Diese Kette kann so lang werden wie wir wollen, allerdings
wird es immer schwerer die �bersicht zu behalten, je l�nger
sie wird. Eine M�glichkeit die Lesbarkeit zu erh�hen, besteht
darin, mit Formatierungen durch Einr�ckungen zu arbeiten.
Wenn wir alle Anweisungen und geschweiften Klammern
an der gleichen Stelle untereinander schreiben, kommt
es viel seltener vor, dass wir einen Syntaxfehler machen und
falls es doch passiert, wir w�rden wir ihn viel schneller finden. 

\section{Verschachtelte Abh�ngigkeiten}
\index{Abh�ngigkeiten!verschachtelte}

Zus�tzlich zur Verkettung k�nnen wir  %Bedingungen
Abh�ngigkeiten auch ineinander verschachteln.  Wir h�tten
das vorige Beispiel auch in dieser Form schreiben k�nnen:

\begin{verbatim}
    if (x == 0) 
    {
        printf ("x is zero\n");
    } 
    else 
    {
        if (x > 0) 
        {
            printf ("x is positive\n");
        }
        else 
        {
            printf ("x is negative\n");
        }
    }
\end{verbatim}
%
Es gibt jetzt eine �u�ere Bedingung die eine Verzweigungen erzeugt.  
Der erste Zweig enth�lt eine einfache Ausgabeanweisung.
Der zweite Zweig enth�lt  eine weitere {\tt if}-Anweisung, welche wiederum
eine Verzweigung erzeugt.  Gl�cklicherweise sind deren Zweige 
Ausgabeanweisungen, es h�tten aber durchaus auch weitere bedingte Anweisungen
folgen k�nnen.

Auch hier ist es wichtig, dass wir die Struktur des Programms durch
Einr�ckungen sichtbar machen. Allerdings ist es prinzipiell so, dass ab
einer bestimmten Verzweigungstiefe der �berblick verloren geht. 
Man sollte sich deshalb gut �berlegen, ob die tiefe Verschachtelung 
von bedingten Anweisungen im eigenen Programm unbedingt n�tig ist.
\index{Verschachtelte Struktur}
Auf der anderen Seite finden wir ineinander {\bf verschachtelte Strukturen} 
in einer Reihe von Programmen wieder, so dass es eine gute Idee ist, sich
an dieses Programmierkonzept zu gew�hnen.

%Pr�fung: optimieren von if/else if /else Strukturen (nicht immer wieder neu pr�fen!)

\section{Die {\tt return}-Anweisung}
\index{return}
\index{Anweisung!return}

Die {\tt return} Anweisung erlaubt es uns die Ausf�hrung einer Funktion zu
beenden, ohne dass alle Anweisungen der Funktion bis zum Ende ausgef�hrt
werden m�ssen. Der Einsatz der  {\tt return}-Anweisung ist zum Beispiel dann
sinnvoll, wenn unser Programm eine Fehlerbedingung erkannt hat:

\index{Mathematische Funktion!log()}
\begin{verbatim}
    #include <math.h>

    void PrintLogarithm (double x) 
    {
        if (x <= 0.0) 
        {
            printf ("Positive numbers only, please.\n");
            return;
        }

       double result = log (x);
       printf ("The log of x is %f\n", result);
    }
\end{verbatim}
%
Wir definieren eine Funktion namens {\tt PrintLogarithm()} mit dem
Parameter~{\tt x} vom Typ {\tt double}. 
Die Funktion �berpr�ft zu allererst, ob  {\tt x} kleiner oder gleich Null ist.
In diesem Falle w�rde die Funktion eine Fehlermeldung anzeigen und
die Funktion mittels {\tt return} beenden. 
Die Abarbeitung des Programms wird in der aufrufenden Funktion fortgesetzt.
Die �brigen Programmzeilen der Funktion werden in diesem Fall nicht ausgef�hrt.

Ich habe einen Flie�kommawert auf der rechten Seite der Bedingung 
benutzt, weil  die Variable auf der linken Seite vom Typ {\tt  double} ist.
%used a floating-point value on the right side of the condition
%because there is a floating-point variable on the left.

\index{Mathematische Bibliothek}
WICHTIG: Jedes mal wenn wir eine Funktion aus der mathematischen Bibliothek 
benutzen, m�ssen wir die Headerdatei {\tt math.h} in unser Programm einbinden.
\hint

\section{Rekursion}
\label{recursion}
\index{Rekursion}

Ich habe im letzten Kapitel erw�hnt, dass es v�llig legal ist, dass eine Funktion
 eine andere Funktion aufzurufen kann und wir haben bereits einige 
Beispiele f�r dieses Verhalten gesehen.
Ich habe nicht erw�hnt, dass eine Funktion sich auch selbst aufrufen kann -- das m�chte ich
jetzt nachholen.
 
In der Tat ist es m�glich und erlaubt, dass eine Funktion sich selbst aufruft. 
Es mag nicht offensichtlich sein, wof�r das gut sein sollte, aber dabei handelt
es sich  um eines der merkw�rdigsten und interessantesten Verhalten
die ein Programm besitzen kann.
%It may not be obvious why that is a good thing, but it turns out to be
%one of the most magical and interesting things a program can do.

%Original/Kopie: Code / Instanz

Schauen wir uns die folgende Funktion an:

\begin{verbatim}
    void Countdown (int n) 
    {
        if (n == 0) 
        {
            printf ("Start!");
        }
        else
        {
            printf ("%i", n);
            Countdown (n-1);
        }
    }
\end{verbatim}
%
Der Name der Funktion ist {\tt Countdown()} und sie besitzt einen
einzelnen Parameter vom Typ {\tt  int}.  Wenn der Parameter den Wert Null hat, 
wird das Wort ``Start!'' ausgegeben. Anderenfalls wird der Wert des Parameters 
auf dem Bildschirm ausgegeben und eine Funktion mit dem Namen 
{\tt Countdown()} --die gleiche Funktion-- aufgerufen und {\tt n-1} als Argument �bergeben.

Was passiert wenn wir diese Funktion in der folgenden Weise aufrufen?

\begin{verbatim}
    int main (void)
    {
         Countdown (3);
         return EXIT_SUCCESS;
    }
\end{verbatim}
%
Die Ausf�hrung von {\tt Countdown()} beginnt mit {\tt n=3} und
weil {\tt n} nicht Null ist, wird der Wert 3 ausgegeben und dann ruft die
Funktion {\tt Countdown()} erneut auf...

\begin{quote}
Die Ausf�hrung von {\tt Countdown()} beginnt mit {\tt n=2} und
weil {\tt n} nicht Null ist, wird der Wert 2 ausgegeben und dann ruft die
Funktion {\tt Countdown()} erneut auf...

\begin{quote}
Die Ausf�hrung von {\tt Countdown()} beginnt mit {\tt n=1} und
weil {\tt n} nicht Null ist, wird der Wert 1 ausgegeben und dann ruft die
Funktion {\tt Countdown()} erneut auf...

\begin{quote}
Die Ausf�hrung von {\tt Countdown()} beginnt mit {\tt n=0} und
weil {\tt n} jetzt den Wert Null hat, wird das Wort ``Start!''
ausgegeben. Danach ist die Funktion beendet und die 
Abarbeitung des Programms wird in der aufrufenden Funktion fortgesetzt.
\end{quote}

Die Funktion {\tt Countdown()}, welche {\tt n=1} als Argument erhalten hat,
ist beendet und die Abarbeitung des Programms wird in der 
aufrufenden Funktion fortgesetzt.

\end{quote}

Die Funktion {\tt Countdown()}, welche {\tt n=2} als Argument erhalten hat,
ist beendet und die Abarbeitung des Programms wird in der 
aufrufenden Funktion fortgesetzt.

\end{quote}

Die Funktion {\tt Countdown()}, welche {\tt n=3} als Argument erhalten hat,
ist beendet und die Abarbeitung des Programms wird in der 
aufrufenden Funktion fortgesetzt.

\noindent Und dann ist unser Programm zur�ck in {\tt main()} (was f�r eine Reise).  
Die gesamte Ausgabe sieht folgenderma�en aus:

\begin{verbatim}
    3
    2
    1
    Start!
\end{verbatim}
%

\index{Instanz}
In der Umgangssprache der Programmierer sagt man \emph{eine Funktion ruf sich selbst
auf}. Viele Programmieranf�nger (und nicht nur die) haben Probleme damit sich das vorstellen und erkl�ren
zu k�nnen. 

Erinnern wir uns. Wir haben gelernt, dass, wenn eine Funktion 
aufgerufen wird, eine neue Instanz des Programmcodes erzeugt wird. Eine Instanz 
ist also eine Kopie des Programmcodes. 
Unsere {\tt Countdown()} Funktion erzeugt mehrere Kopien des gleichen
Programmcodes. Allerdings sind diese Instanzen nicht komplett gleich, sondern 
unterscheiden sich in ihren Parameterwerten. Genau genommen ist es also falsch 
und irref�hrend zu sagen, dass eine Funktion sich selbst aufruft. Sie ruft nur den gleichen Code auf.


Schauen wir uns als zweites Beispiel erneut die Funktionen
{\tt PrintNewLine()} und {\tt PrintThreeLines()} an:

\begin{verbatim}
    void PrintNewLine () 
    {
        printf ("\n");
    }

    void PrintThreeLines () 
    {
        PrintNewLine ();  PrintNewLine ();  PrintNewLine ();
    }
\end{verbatim}
%
Wir k�nnen die Funktionen mit unserem neuen Wissen verbessern, 
so dass wir so viele Zeilen ausgeben lassen k�nnen wie wir wollen,
seien es 2 oder 106:

\begin{verbatim}
    void PrintLines (int n) 
    {
        if (n > 0) 
        {
            printf ("\n");
            PrintLines (n-1);
        }
    }
\end{verbatim}
%
Das Programm ist �hnlich wie {\tt Countdown()} aufgebaut. So lange
{\tt n} gr��er als Null ist wird eine Leerzeile ausgegeben und dann
wird die gleiche Funktion aufgerufen um weitere {\tt n-1} Leerzeilen auszugeben. 
Somit ergibt sich die Gesamtzahl der auszugebenden Leerzeilen
aus {\tt 1 + (n-1)}, was n�herungsweise {\tt
n} entspricht.

\index{Rekursion}
\index{newline}

Der Vorgang einer sich selbst aufrufenden Funktion wird als {\bf Rekursion} bezeichnet -- 
Funktionen welche diese Eigenschaft aufweisen sind {\bf rekursive} Funktionen.

\section {Unendliche Rekursion}

In unseren Beispielen im letzten Abschnitt f�llt auf, dass bei jedem 
rekursiven Funktionsaufruf das Funktionsargument um den Wert 1 verringert
wird, bis am Ende die Funktion mit dem Wert Null aufgerufen wird.
In diesem Fall wird die Funktion beendet, ohne einen weiteren 
Funktionsaufruf durchzuf�hren.
Dieser Fall --wenn die Funktion beendet wird ohne einen weiteren rekursiven
Funktionsaufruf zu machen-- nennt man die {\bf Abbruchbedingung}.

Wenn eine Rekursion niemals die Abbruchbedingung erreicht, w�rde 
sich die rekursive Funktion (theoretisch) unendlich oft selbst aufrufen und
das Programm w�rde nie beendet. Diesen Fall bezeichnet man auch
als {\bf unendliche Rekursion} und ist generell keine gute Idee.
Dieser Fehler kann auch auftreten wenn eine Abbruchbedingung zwar
vorhanden ist, aber w�hrend der Abarbeitung des Programms nicht erreicht 
werden kann. 

\index{Rekursion!Abbruchbedingung}
\index{Rekursion!unendliche}
\index{Unendliche Rekursion}
\index{Run-time error}


In den meisten F�llen wird ein Programm mit unendlich rekursiven 
Funktionen nicht wirklich f�r immer laufen. Die Ressourcen
eines Computers sind endlich und fr�her oder sp�ter wird unser
Progamm mit einer Fehlermeldung beendet werden.
Einen Programmfehler dieser Art bezeichnet man als \emph{Run-time error} oder 
\emph{Laufzeitfehler}
(ein Fehler der erst in Erscheinung tritt, wenn das Programm ausgef�hrt wird --
zur 'Laufzeit' eines Programms)
Die Nichtexistenz einer Abbruchbedingung ist ein Beispiel f�r diese Art von Fehlern.

\begin{description}
\item [AUFGABE 1:]  Schreiben Sie ein kleines Programm, welches immer wieder
die gleiche Funktion aufruft, ohne abzubrechen und beobachten Sie das
Verhalten dieses Programms.

\item [AUFGABE 2:]
Was passiert wenn wir die Funktion {\tt Countdown()} mit dem Argument -1 aufrufen?
Wie k�nnen wir diesen Fehler vermeiden?
\end{description}




\section {Stack Diagramme f�r rekursive Funktionen}
\index{Stack}
\index{Diagramm!Zustand}
\index{Diagramm!Stack}

Im vorigen Kapitel haben wir ein Stackdiagramm verwendet um
den Zustand eines Programms und seiner Funktionen w�hrend 
eines Funktionsaufrufs darzustellen.

Wir k�nnen ein Stackdiagramm auch dazu verwenden um uns
noch einmal verst�ndlich zu machen, was w�hrend eines rekursiven
Funktionsaufrufs in unserem Programm vorgeht.

Es ist f�r das Verst�ndnis wichtig, dass wir uns erinnern, dass jedes
Mal wenn eine Funktion aufgerufen wird eine neue Instanz des 
Programmcodes und der lokalen Variablen und Parameter dieser
Funktion erzeugt wird.

Die Abbildung zeigt und das Stackdiagramm f�r die Funktion 
{\tt Countdown()}, wenn sie in der Hauptfunktion {\tt main()}mit dem 
Argument {\tt n = 3} wie folgt aufgerufen wird:

\index{Diagramm!Stack}
\index{Stackdiagramm}

\begin{verbatim}
    Countdown(3);
\end{verbatim}

\vspace{0.1in}
\centerline{\epsfig{figure=figs/stack2.pdf,width=6cm}}
\vspace{0.1in}

Es existiert genau eine Instanz von {\tt main()} und vier Instanzen von
{\tt Countdown}, jede mit einem unterschiedlichen Parameterwert f�r 
{\tt n}.  Das unterste Element des Stapels, {\tt Countdown()} mit {\tt n=0}
ist die Abbruchbedingung. An dieser Stelle wird kein weiterer rekursiver
Funktionsaufruf durchgef�hrt, so dass keine weiteren Instanzen von 
{\tt Countdown()} erzeugt werden.
%todo: Stack ist von oben nach unten gezeichnet (entgegen intuition)
Die Instanz von {\tt main()} ist leer, weil in {\tt main()} keine Parameter
oder lokalen Variablen definiert sind.

Versuche Sie doch einmal  ein Stackdiagramm f�r {\tt PrintLines()} 
zu zeichnen, wenn es mit dem Parameter {\tt n=4} aufgerufen wird!


\section{Glossar}

\begin{description}

\item[Anweisungsblock (engl: \emph{block}):] Eine Gruppe von Anweisungen die durch eine
�ffnende und schlie�ende geschweifte Klammer gebildet wird. Die Anweisungen werden
gemeinsam ausgef�hrt und vom Compiler wie eine einzige Anweisung behandelt.

\item[Modulo (engl: \emph{modulus}):]  Der Modulo-Operator berechnet den
ganzzahligen Divisionsrest bei einer Division von ganzen Zahlen.  In C
wird der Operator durch das Prozentzeichen dargestellt ({\tt \%}).

\item[Bedingte Anweisung (engl: \emph{conditional}):]  Ein Anweisungsblock der nur dann ausgef�hrt wird,
wenn eine bestimmte Bedingung erf�llt ist.

\item[Mehrfache Verzweigung (engl: \emph{chaining}):]  Durch die verbundene  Abfrage mehrer
Bedingungen lassen sich Fallunterscheidungen durchf�hren. Es wird nur ein Programmabschnitt
von mehreren Alternativen ausgef�hrt. 

\item[Verschachtelte Abh�ngigkeiten (engl: \emph{nesting}):] werden erzeugt in dem man eine
weitere Fallunterscheidung in einem Zweig einer bedingten Anweisung durchf�hrt.

\item[Rekursion (engl: \emph{recursion}):]  Wird aus einer Funktion heraus die gleiche Funktion
erneut aufgerufen bezeichnet man dies als Rekursion. Der Aufruf erfolgt �blicherweise mit
ge�nderten Argumenten und sollte zu einem definierten Ende f�hren. Anderenfalls droht die

\item[Unendliche Rekursion (engl: \emph{infinite recursion}):]  Eine Funktion ruft sich wiederholt
selbst auf, ohne die Abbruchbedingung zu erreichen. Diese Verhalten belegt den kompletten Stack-Speicher
und f�hrt zu einem Laufzeitfehler.

\index{Modulo}
\index{Bedingte Anweisung}
\index{Verzweigung!mehrfache}
\index{Mehrfache Verzweigung}
\index{Bedingung!verschachtelt}
\index{Rekursion}
\index{Rekursion!endlose}
\index{Unendliche Rekursion}

\end{description}


\section{�bungsaufgaben}
\setcounter{exercisenum}{0}

\ifthenelse {\boolean{German}}{ 
\begin{exercise}
Der erste Vers des Lieds ``99 Bottles of Beer'' lautet:

\begin{quote}
99 bottles of beer on the wall,
99 bottles of beer,
ya' take one down, ya' pass it around,
98 bottles of beer on the wall.
\end{quote}

Die nachfolgenden Verse sind identisch bis auf die Anzahl
der Flaschen. Deren Anzahl nimmt in jedem Vers um eine Flasche
ab, bis schlie�lich der letzte Vers lautet:

\begin{quote}
No bottles of beer on the wall,
no bottles of beer,
ya' can't take one down, ya' can't pass it around,
'cause there are no more bottles of beer on the wall!
\end{quote}
%
Und dann ist diese Lied schlie�lich zu Ende.

Schreiben Sie ein Programm, welches den gesamten Text 
des Lieds ``99 Bottles of Beer'' ausgibt.
Ihr Programm sollte eine rekursive Funktion f�r die Ausgabe
des Liedtextes verwenden.
Sie k�nnen weitere Funktionen verwenden um ihr Programm
zu strukturieren.

W�hrend Sie den Programmcode schreiben und testen sollten
Sie mit einer kleineren Anzahl von Versen beginnen, z.B. 
``3 Bottles of Beer.''

Der Sinn dieser �bung besteht darin ein Problem zu analysieren und
in kleinere, l�sbare Bestandteile zu zerlegen.
Diese kleineren Einheiten lassen sich unabh�ngig und nacheinander
entwickeln und testen und f�hren im Ergebnis zu einer schnelleren
und robusteren L�sung.
\end{exercise}

\begin{exercise}
In C k�nnen Sie die {\tt getchar()} Funktion benutzen um Zeichen von
der Tastatur einzulesen. Diese Funktion stoppt die Ausf�hrung des
Programms und wartet auf eine Eingabe des Benutzers. Die 
{\tt getchar()} Funktion ist vom Typ {\tt int} und
erfordert kein Argument. Sie liefert den ASCII-Code des eingegeben
Zeichens von der Tastatur zur�ck.

Schreiben Sie ein Programm, welches den Benutzer auffordert eine 
Ziffer von 0-9 einzugeben. 

�berpr�fen Sie die Eingabe des Benutzers und geben Sie 
einen Hinweis aus, falls es sich bei dem eingegeben Wert nicht
um eine Zahl handeln sollte. Geben Sie nach erfolgreicher Pr�fung die
Zahl aus.

% Kapitel 5 (Return)
%Schreiben Sie dazu eine Funktion {\tt AsciiToNumber()} welche ein
%{\tt int} als Typ und als Argument besitzt. �bergeben Sie der Funktion
%den eingelesenen Wert und 

\end{exercise}




\begin{exercise}
Fermat's "Letzter Satz" besagt, dass es keine ganzen Zahlen
$a$, $b$ und $c$ gibt, f�r die gilt

\[a^n + b^n = c^n \]
%
au�er f�r den Fall, dass $n=2$.

Schreiben Sie eine Funktion mit dem Namen {\tt CheckFermat()} 
welche vier {\tt int} als Parameter hat ---{\tt a}, {\tt b}, {\tt c} and {\tt n}--- und
welche �berpr�ft, ob Fermats Satz Bestand hat. Sollte sich f�r
$n$ gr��er als 2 herausstellen, dass $a^n + b^n = c^n$,
dann sollte ihr Programm ausgeben: ``Holy smokes, Fermat was wrong!''
In allen anderen F�llen sollte das Programm ausgeben: ``No, that doesn't work.''

Verwenden Sie f�r die Berechnung der Potenzen die Funktion {\tt pow()} aus der 
mathematischen Bibliothek. Diese Funktion �bernimmt zwei {\tt double} als 
Argument. Das erste Argument stellt dabei die Basis und das zweite Argument den
Exponenten der Potenz dar. Die Funktion liefert als Ergebnis wiederum ein {\tt double}.

Um die Funktion in unserem Programm nutzen zu k�nnen m�ssen die Datentypen
angepasst werden (siehe Abschnitt \ref{typecasting}). Dabei wandelt C den Datentyp
 {\tt int}  automatisch in  {\tt double} um. Um einen  {\tt double}  Wert in  {\tt int} zu wandeln
 muss  der Typecast-Operator  {\tt (int)} verwendet werden.  

Zum Beispiel:

\begin{verbatim}
    int x = (int) pow(2, 3);
\end{verbatim}
%
weist  {\tt x} den Wert {\tt 8} zu, weil $2^3 = 8$.
\end{exercise}

}
{\input{exercises/Exercise_4_english}}


%!TEX root = Main_german.tex

% LaTeX source for textbook ``How to think like a computer scientist''
% Copyright (C) 1999  Allen B. Downey

% This LaTeX source is free software; you can redistribute it and/or
% modify it under the terms of the GNU General Public License as
% published by the Free Software Foundation (version 2).

% This LaTeX source is distributed in the hope that it will be useful,
% but WITHOUT ANY WARRANTY; without even the implied warranty of
% MERCHANTABILITY or FITNESS FOR A PARTICULAR PURPOSE.  See the GNU
% General Public License for more details.

% Compiling this LaTeX source has the effect of generating
% a device-independent representation of a textbook, which
% can be converted to other formats and printed.  All intermediate
% representations (including DVI and Postscript), and all printed
% copies of the textbook are also covered by the GNU General
% Public License.

% This distribution includes a file named COPYING that contains the text
% of the GNU General Public License.  If it is missing, you can obtain
% it from www.gnu.org or by writing to the Free Software Foundation,
% Inc., 59 Temple Place - Suite 330, Boston, MA 02111-1307, USA.


\chapter{Funktionen mit Ergebnissen}

\section{Return-Werte}
\index{return}
\index{Anweisung!return}
\index{Funktionen mit Ergebnissen}
\index{return value}
\index{Funktion!R�ckgabewert}
\index{R�ckgabewert}
\index{void}
\index{Funktion!void}

Wir haben ja bereits einige Erfahrung bei der Verwendung von 
Funktionen in C. Bei einigen der Standardfunktionen, wie
zum Beispiel den mathematischen Funktionen ist uns aufgefallen,
das die Funktion einen Wert berechnet -- die Funktion hat ein 
Resultat produziert.

Mit diesem Wert kann unser Programm weiterhin arbeiten. Wir 
k�nnen ihn in einer Variable speichern, auf dem 
Bildschirm ausgeben oder als Teil eines Ausdrucks verwenden.  
Zum Beispiel

\index{Mathematische Funktion!exp()}
\index{Mathematische Funktion!sin()}

\begin{verbatim}
    double e = exp (1.0);
    double height = radius * sin (angle);
\end{verbatim}
%
Allerdings waren alle unsere Funktionen, die wir bisher selbst
geschrieben haben {\bf void} Funktionen, das hei�t sie haben
den Datentyp  {\bf void} und liefern kein Ergebnis zur�ck.

Wenn wir  {\bf void} Funktionen aufrufen, steht der Funktionsaufruf
�blicherweise f�r sich allein als Anweisung in einer Programmierzeile
ohne dass dabei ein Wert zugewiesen oder erwartet wird:


\begin{verbatim}
    PrintLines (3);
    Countdown (n-1);
\end{verbatim}
%
In diesem Kapitel werden wir herausfinden, wie wir Funktionen 
schreiben, welche eine R�ckgabe an die aufrufende Funktion erzeugt.
Weil mir ein guter Name daf�r fehlt werde ich sie {\bf Funktionen mit Ergebnissen} 
nennen. 

Das erste Beispiel ist die Funktion {\tt CalculateCircleArea()}. Sie hat einen  {\tt
double} Wert als Parameter und liefert die berechnete Fl�che eines Kreises in Abh�ngigkeit vom
gegebenen Radius:

\index{Mathematische Funktion!acos()}
\index{pi}

\begin{verbatim}
    double CalculateCircleArea (double radius) 
    {
        double pi = acos (-1.0);
        double area = pi * radius * radius;
        return area;
    }
\end{verbatim}
%
Beim Betrachten der Funktionsdefinition f�llt  als erstes auf, dass die Funktion
anders beginnt, als alle andern Funktionen, die wir bisher geschrieben haben.
Der erste Begriff in einer Funktionsdefinition gibt den Datentyp der Funktion an.
Anstelle von {\tt void} steht hier {\tt double}. Damit wird angezeigt, dass 
der R�ckgabewert der Funktion vom Typ {\tt double} ist.

Immer, wenn wir mit Daten arbeiten, neue Werte berechnen, Funktionen aufrufen,
oder Ein- undAusgaben erzeugen, m�ssen wir exakt angeben um welchen 
Datentyp es sich dabei handelt. Nur so kann der Compiler pr�fen, ob die
tats�chlich verwendeten Daten dem richtigen Typ entsprechen und uns vor gr��eren
Problemen bewahren. Das erscheint am Anfang vielleicht etwas ungewohnt und
l�stig, wird uns aber bald ganz selbstverst�ndlich von der Hand gehen.

Wenn wir uns die letzte Zeile anschauen, dann f�llt auf, dass die 
{\tt return}-Anweisung jetzt auch einen Wert enth�lt.
Die Bedeutung dieser Anweisung ist die folgende:  
``kehre unmittelbar zur aufrufenden Funktion zur�ck und verwende den 
Wert des folgenden Ausdrucks als  R�ckgabewert.'' 
Der angegebene Ausdruck kann dabei von beliebiger Komplexit�t sein.
Wir h�tten also die Funktion auch sehr viel knapper zusammenfassen k�nnen:

\begin{verbatim}
    double Area (double radius) 
    {
        return acos(-1.0) * radius * radius;
    }
\end{verbatim}
%
Auf der anderen Seite erleichtert uns die Verwendung von {\bf tempor�re} 
Variablen wie {\tt area} die Suche nach Programmfehlern.
Wichtig ist in jedem Fall, dass der Typ des Ausdrucks in der 
{\tt return}-Anweisung mit dem angegebenen Typ der Funktion �bereinstimmt.

In anderen Worten, wenn wir in einer Funktionsdeklaration angeben
das der R�ckgabewert vom Typ {\tt double} ist, geben wir ein Versprechen
die Funktion schlie�lich ein Ergebnis vom Typ {\tt double} produziert.
Wenn wir keinen Wert zur�ckgeben ({\tt return} ohne Ausdruck benutzen)
oder den falschen Typ zur�ckgeben ist das fast immer ein Fehler und der
Compiler wird uns daf�r zur Rede stellen. Allerdings gelten auch hier die
Regeln der automatischen Typumwandlung.

\index{Tempor�re Variable}
\index{Variable!tempr�re}

Manchmal ist es n�tzlich mehrere  {\tt return}-Anweisung in einer
Funktion zu haben. Zum Beispiel eine f�r jede Programmverzweigung:

\begin{verbatim}
  double AbsoluteValue (double x) 
    {
        if (x < 0) 
        {
            return -x;
        } 
        else 
        {
            return x;
        }
    }
\end{verbatim}
%
Da sich die {\tt return} -Anweisungen in alternativen Zweigen unseres 
Programms befinden wird jeweils nur eine von ihnen auch ausgef�hrt.
Obwohl es legal ist mehrere {\tt return}-Anweisungen in einer Funktion
zu haben, ist es wichtig daran zu denken, dass eine davon ausgef�hrt wird,
die Funktion beendet ist, ohne noch irgendwelche anderen Anweisungen
auszuf�hren.

Programmcode, welcher hinter einer {\tt return}-Anweisung steht, wird 
nicht mehr ausgef�hrt und wird deshalb {\bf unereichbarer Code} genannt.
Sollte eine Funktion also nicht das erwartete Ergebnis produzieren, so sollten
Sie pr�fen ob die Anweisungen auch wirklich ausgef�hrt werden. 
Manche Compilers geben eine Warnung aus, wenn in einem Programm
solche Codezeilen existieren.

\index{Unerreichbarer Code}

Wenn wir  {\tt return}-Anweisungen in Programmverzweigungen 
benutzen, m�ssen wir garantieren, das  {\em jeder m�gliche Pfad} durch
das Programm auf eine  {\tt return}-Anweisung trifft.
Zum Beispiel gibt es ein Problem im folgenden Programm:

\begin{verbatim}
  double AbsoluteValue (double x) 
    {
        if (x < 0) 
        {
            return -x;
        } 
        else if (x > 0) 
        {
            return x;
        }                 /* Fehlendes return f�r x==0!! */
    }
\end{verbatim}
%
Dieses Programm ist nicht korrekt, weil im Fall, dass {\tt x} den Wert  0 hat, 
keine der beiden Bedingungen zutrifft und die Funktion beendet wird,
ohne auf eine  {\tt return}-Anweisung zu treffen.
Ungl�cklicherweise k�nnen viele C Compilers diesen Fehler nicht
finden. Es ist also h�ufig der Fall, dass sich das Programm kompilieren und
ausf�hren l�sst, aber der R�ckgabewert f�r den Fall {\tt x==0} 
nicht definiert ist. Wir k�nnen nicht voraussagen, welcher Wert letztendlich 
zur�ckgegeben wird und es ist wahrscheinlich, dass es unterschiedliche
Werte f�r unterschiedliche Umgebungen sein werden.

\index{Absolutwert}
\index{Error!compile-time}
\index{Compile-time error}

Mittlerweile haben Sie bestimmt schon die Nase voll davon Compiler-Fehler zu sehen.
Allerdings kann ich versichern, dass es nur eine Sache gibt die schlimmer ist als 
Compiler-Fehler zu erhalten -- und das ist {\em keine} Compiler-Fehler zu erhalten,
wenn das Programm falsch ist.

Ich beschreibe mal kurz was wahrscheinlich passieren wird: Sie testen {\tt
AbsoluteValue()} mit mehreren verschiedenen Werten f�r {\tt x} und die
Funktion scheint korrekt zu arbeiten.
Dann geben Sie ihr Programm an jemand anderen weiter und er oder sie versucht es 
in einer ge�nderten Umgebung (anderer Compiler oder Rechnerarchitektur) laufen zu lassen.
Das Programm produziert pl�tzlich auf mysteri�se Art und Weise Fehler.\\
Es wird mehrere Tage und viel Debugging-Aufwand kosten herauszufinden,
dass die Implementierung von  {\tt AbsoluteValue()} fehlerhaft war - wie froh
w�ren Sie gewesen, wenn Sie der Compiler doch nur gewarnt h�tte!

\index{Compile-time error}
\index{Error!compile-time}
\index{Debugging}

Von jetzt an sollten wir nicht dem Compiler die Schuld geben, wenn er wieder
auf einen Fehler in unserem Programm hinweist. 
Im Gegenteil, wir sollten ihm danken, dass er einen Fehler so einfach gefunden
hat und uns viel Zeit und Aufwand erspart hat den Fehler selbst aufsp�ren
zu m�ssen. 
Die meisten Compilers verf�gen �ber eine Option mit der wir dem Compiler
mitteilen k�nnen unser Programm besonders strikt und sorgf�ltig zu pr�fen und
alle Fehler zu melden die er nur finden kann. Sie sollten diese Option w�hrend
der gesamten weiteren Programmentwicklung nutzen.

\index{Mathematische Funktion!fabs()}

Ach �brigens wollte ich nur kurz noch erw�hnen, es gibt in der 
mathematischen Bibliothek eine Funktion namens {\tt fabs()}.
Sie berechnet den Absolutwert eines {\tt double} -- korrekt und einwandfrei.

\section{Programmentwicklung}
\label{distance}
\index{Programmentwicklung}

An diesem Punkt sollten wir in der Lage sein komplette C Funktionen
lesen und erkl�ren zu k�nnen.
Es ist aber sicher noch nicht so klar, wie man vorgeht um eigene
Funktionen zu entwerfen und aufzuschreiben.
Ich m�chte deshalb an dieser Stelle eine Technik vorstellen, die ich 
{\bf inkrementelle Entwicklung} nenne.

\index{Inkrementelle Entwicklung}
\index{Programmentwicklung}

Stellen wir uns folgende Beispielaufgabe vor: Wir wollen den Abstand
zwischen zwei Punkten herausfinden, deren Position jeweils durch x- und y-Koordinaten
bestimmt ist. Ein Punkt hat die Koordinaten $(x_1, y_1)$, der andere $(x_2, y_2)$.  
Wir k�nnen den Abstand mit Hilfe der folgenden mathematischen Funktion ermitteln:
\begin{equation}
distance = \sqrt{(x_2 - x_1)^2 + (y_2 - y_1)^2}
\end{equation}
%
Wenn wir jetzt eine passende  {\tt Distance} Funktion in C
entwerfen wollen, m�ssen wir im ersten Schritt �berlegen,
welche Eingaben (Parameter) und welche Ausgaben (R�ckgabewerte)
unsere Funktion f�r die Berechnung ben�tigt.

In unserem Fall sind die Koordinaten der zwei Punkte die Parameter.
Wir m�ssen einen Datentyp festlegen und es ist nur nat�rlich hierf�r
reelle Zahlen vorzusehen, wir verwenden also vier {\tt double}s.  
Der R�ckgabewert unserer Funktion ist die Entfernung zwischen den
Punkten und ist vom gleichen Typ {\tt double}.

Damit k�nnen wir bereits die Grundz�ge unsere Funktion in C aufschreiben:
%Already we can write an outline of the function:

\begin{verbatim}
    double Distance (double x1, double y1, double x2, double y2) 
    {
        return 0.0;
    }
\end{verbatim}
%
Die {\tt return}-Anweisung ist ein Platzhalter, so das sich die Funktion 
kompilieren l�sst und einen Wert zur�ckgibt, obwohl das nat�rlich
nicht die richtige Antwort ist.

An diesem Punkt tut die Funktion noch nichts sinnvolles, 
aber es ist trotzdem eine gute Idee sie bereits einmal zu kompilieren
um eventuell vorhandene Syntaxfehler zu finden, bevor wir
weitere Anweisungen hinzuf�gen.

Um die Funktion in einem Programm zu testen m�ssen wir sie mit
Beispielwerten aufrufen. Irgendwo in {\tt main()} k�nnten folgende
Anweisungen hinzuf�gen:

\begin{verbatim}
    double dist = Distance (1.0, 2.0, 4.0, 6.0);
    printf  ("%f\n" dist);
\end{verbatim}
%
Ich habe die Werte speziell ausgew�hlt, so dass der horizontale
Abstand 3 und  der vertikale Abstand 4 ist; auf diese Weise ergibt
der korrekte Abstand den Wert 5 ( die Hypotenuse eines 3-4-5 Dreiecks).
Wenn wir die R�ckgabewerte einer Funktion testen wollen, ist es eine
gute Idee vorher die richtigen Antworten zu kennen.

Nachdem wir �berpr�ft haben, dass die Syntax der Funktionsdefinition 
korrekt ist k�nnen wir anfangen weitere Codezeilen f�r die Berechnung
hinzuzuf�gen.
Nach jeder gr��eren �nderung kompilieren wir unser Programm 
und f�hren es erneut aus. Auf diese Weise ist es sehr einfach 
Fehler zu finden, die beim letzten Kompilieren noch nicht da waren.
Sie m�ssen in den neu hinzugef�gten Programmzeilen stecken!

Der n�chste Schritt der Berechnung ermittelt die Differenz zwischen
$x_2 - x_1$ und $y_2 - y_1$.  Ich  werde diese Werte in 
tempor�ren Variablen mit Namen {\tt dx} und {\tt dy} speichern.

\begin{verbatim}
    double Distance (double x1, double y1, double x2, double y2) 
    {
        double dx = x2 - x1;
        double dy = y2 - y1;
        printf  ("dx is %f\n",  dx);
        printf  ("dy is %f\n", dy;
        return 0.0;
    }
\end{verbatim}
%
In der Funktion habe ich noch zwei Ausgabeanweisungen hinzugef�gt,
so dass ich erst einmal die Zwischenergebnisse �berpr�fen kann, bevor
ich weitermache.
Ich habe es bereits erw�hnt, ich erwarte an
dieser Stelle die Werte 3.0 und 4.0.

\index{Debug-Code}

Wenn die Funktion fertig ist werde ich die Ausgabeanweisungen wieder
entfernen. Programmcode, der zu einem Computerprogramm hinzugef�gt
wird um bei der Programmentwicklung zu helfen, wird auch als {\bf Debug-Code}
bezeichnet.
Dieser Programmcode sollte in der Endversion unseres Programms nicht
mehr enthalten sein. 
Man kann Debug-Anweisungen auch im Quelltext eines Programms belassen und
ihn zum Beispiel nur auskommentieren, dass hei�t als Kommentar kennzeichnen.
Auf diese Weise ist es einfach ihn sp�ter wieder zu aktivieren, wenn er
gebraucht werden sollte. 

\index{Mathematische Funktion!pow()}
\index{Mathematische Funktion!sqrt()}
Der n�chste Schritt in unserer Berechnung ist die Quadrierung von {\tt dx} und {\tt dy}.
Wir k�nnten daf�r die {\tt pow()} Funktion von C benutzen, es ist aber an dieser
Stelle einfacher und schneller jeden Term einfach mit sich selbst zu multiplizieren:

\begin{verbatim}
    double Distance (double x1, double y1, double x2, double y2)
    {
        double dx = x2 - x1;
        double dy = y2 - y1;
        double dsquared = dx*dx + dy*dy;
        printf  ("d_squared is %f\n", dsquared);
        return 0.0;
    }
\end{verbatim}
%
Es ist ratsam, an dieser Stelle das Programm erneut zu kompilieren und 
auszuf�hren. Dabei sollten wir den Wert des Zwischenergebnis kontrollieren
-- dieser sollte den Wert 25.0 haben.

Zum Schluss benutzen wir die {\tt sqrt()} Funktion um das Endergebnis
zu berechnen und geben dieses Resultat an die aufrufende Funktion
zur�ck:

\begin{verbatim}
    double Distance (double x1, double y1, double x2, double y2) 
    {
        double dx = x2 - x1;
        double dy = y2 - y1;
        double dsquared = dx*dx + dy*dy;
        double result = sqrt (dsquared);
        return result;
    }
\end{verbatim}
%
In {\tt main()} sollten wir uns diesen Wert ausgeben lassen und erneut
�berpr�fen, ob das Resultat mit unseren Erwartungen �bereinstimmt.

Im Laufe der Zeit, wenn wir unsere Programmiererfahrung verbessert haben,
werden wir mehr und mehr Programmzeilen schreiben, bevor wir 
�berpr�fen, ob unser Programm fehlerfrei l�uft. Trotzdem ist der 
inkrementelle Entwicklungsprozess auch dann noch sinnvoll und 
kann Ihnen helfen eine Menge Zeit bei der Fehlersuche zu sparen.

Die Schl�sselaspekte des Prozesses sind:

\begin{itemize}

\item Beginne die Programmentwicklung mit einem funktionsf�higen Programm
und mache kleine, inkrementelle �nderungen. Kompiliere den Programmcode
nach jeder �nderung. Jedes Mal wenn ein Fehler auftaucht ist sofort klar, wo
dieser Fehler zu suchen ist.

\item Benutze tempor�re Variablen um Zwischenergebnisse zu speichern. 
Nutze diese Variablen um sie auf dem Bildschirm auszugeben und ihre Werte zu �berpr�fen.

\item Nachdem das komplette Programm funktioniert, sollten Debug-Anweisungen
entfernt (auskommentiert, nicht mit �bersetzt) werden und einzelne Anweisungen
k�nnen zu komplexen Ausdr�cken zusammengefasst werden. 
Es sollte aber darauf geachtet werden, dass dabei das Programm leicht lesbar
bleibt. Es ist generell zu empfehlen einfachere Ausdr�cke zu bevorzugen
und dem Compiler die Optimierung des Programms zu �berlassen.

\end{itemize}

\section{Komposition}
\index{Komposition}

Wie wir bereits schon vermutet haben k�nnen wir, nachdem wir eine
neue Funktion definiert haben, diese Funktion auch als Teil eines 
Ausdrucks verwenden. Ebenso k�nnen wir neue Funktionen mit Hilfe
bereits existierender Funktionen erstellen.

Nehmen wir zum Beispiel an, dass uns jemand zwei Punkte nennt.
Einer dieser Punkte sei der Mittelpunkt und der andere
Punkt befindet sich auf dem Umkreis eines den Mittelpunkt umgebenden Kreises.
Sie haben die Aufgabe aus diesen Angaben die Fl�che des Kreises zu ermitteln.
 
Die Koordinaten des  Mittelpunkts sollen in den Variablen {\tt xc}
und {\tt yc} und die Koordinaten des Punkts auf dem Umkreis in {\tt xp} und
{\tt yp} gespeichert sein.  
Der erste Schritt der Fl�chenberechnung besteht darin den Radius des Kreises
zu ermitteln, welcher sich aus dem Abstand der beiden Punkte ergibt.
Gl�cklicherweise haben wir bereits eine Funktion  {\tt Distance()}, die genau das
f�r uns tut:

\begin{verbatim}
    double radius = Distance (xc, yc, xp, yp);
\end{verbatim}
%
Der zweite Schritt besteht darin den Kreisfl�cheninhalt auf der Basis des Radius zu
berechnen und zur�ckzugeben (die Funktion  {\tt  AreaCircle()} m�ssen wir noch schreiben!):

\begin{verbatim}
    double result = AreaCircle (radius);
    return result;
\end{verbatim}
%
Wir k�nnen diese Schritte in einer neuen Funktion  {\tt  AreaFromPoints()} zusammenfassen:

\begin{verbatim}
    double AreaFromPoints (double xc, double yc, double xp, double yp) 
    {
        double radius = Distance (xc, yc, xp, yp);
        double result = AreaCircle (radius);
        return result;
    } 
\end{verbatim}
%

Die tempor�ren Variablen {\tt radius} und {\tt area} sind n�tzlich f�r
die Programmentwicklung und die Fehlersuche, aber nachdem unser
Programm funktioniert k�nnen wir den Programmcode knapp und pr�zise 
darstellen, indem wir die Funktionsaufrufe zu einem Ausdruck zusammenfassen:

\begin{verbatim}
    double AreaFromPoints (double xc, double yc, double xp, double yp) 
    {
        return AreaCircle (Distance (xc, yc, xp, yp));
    } 
\end{verbatim}

%Overloading not supported in C!!!

%\section{Overloading}
%\label{overloading}
%\index{overloading}

%In the previous section you might have noticed that {\tt Fred}
%and {\tt Area} perform similar functions---finding
%the area of a circle---but take different parameters.  For
%{\tt Area}, we have to provide the radius; for {\tt Fred}
%we provide two points.

%If two functions do the same thing, it is natural to give them
%the same name.  In other words, it would make more sense if
%{\tt Fred} were called {\tt Area}.

%Having more than one function with the same name, which is called {\bf
%overloading}, is legal in C {\em as long as each version takes
%different parameters}.  So we can go ahead and rename {\tt Fred}:

%\begin{verbatim}
%  double Area (double xc, double yc, double xp, double yp) 
%  {
%      return Area (Distance (xc, yc, xp, yp));
%  } 
%\end{verbatim}
%%
%This looks like a recursive function, but it is not.  Actually,
%this version of {\tt area} is calling the other version.
%When you call an overloaded function, C knows which version you
%want by looking at the arguments that you provide.  If you write:

%\begin{verbatim}
%    double x = Area (3.0);
%\end{verbatim}
%%
%C goes looking for a function named {\tt area} that
%takes a {\tt double} as an argument, and so it uses the
%first version.  If you write

%\begin{verbatim}
%    double x = Area (1.0, 2.0, 4.0, 6.0);
%\end{verbatim}
%%
%C uses the second version of {\tt area}.  

%Many of the built-in C commands are overloaded, meaning that there
%are different versions that accept different numbers or types of
%parameters.

%Although overloading is a useful feature, it should be used with
%caution.  You might get yourself nicely confused if you are trying to
%debug one version of a function while accidently calling a different
%one.

%Actually, that reminds me of one of the cardinal rules of debugging:
%{\bf make sure that the version of the program you are looking at is
%the version of the program that is running!}  Some time you may find
%yourself making one change after another in your program, and seeing
%the same thing every time you run it.  This is a warning sign that for
%one reason or another you are not running the version of the program
%you think you are.  To check, stick in an output statement (it
%doesn't matter what it says) and make sure the behavior of the
%program changes accordingly.

\section{Boolesche Werte}
\index{\_Bool}
\index{Werte!boolesche}

Die numerischen Datentypen die wir bisher kennengelernt haben k�nnen
Werte in einem sehr gro�en Wertebereiche speichern. Wir k�nnen sehr viele
ganze Zahlen und noch mehr Flie�kommazahlen darstellen. Das ist auch
notwendig da die Zahlenbereiche in der Mathematik unendlich sind.
%todo: Wertebereiche / Datentypen / Darstellung der Speicherung im Computer
Im Vergleich dazu ist die Menge der darstellbaren Zeichen vergleichsweise klein.
Das hat Auswirkungen darauf, wie viel Speicherplatz ein Computer f�r die 
Speicherung dieser Werte ben�tigt. So ben�tigt ein Wert vom Typ {\tt int} 2 bis 4 Byte,
ein Wert vom Typ {\tt double} 8 Byte und ein Wert vom Typ {\tt char} 1 Byte Speicherplatz.
%todo: Tabelle?

Viele Programmiersprachen implementieren noch einen weiteren fundamentalen 
Datentyp, der kleinste Informationseinheiten speichern kann und der noch viel kleiner
ist. Es handelt sich dabei um so genannte {\bf boolesche Werte} f�r deren 
Speicherung ein einzelnes Bit ausreicht.  Boolesche Werte k�nnen nur zwei
Zust�nde unterscheiden und werden �blicherweise f�r die Darstellung der Wahrheitswerte
\emph{true} und \emph{false} genutzt.


Ungl�cklicherweise haben fr�he Versionen des C Standards boolesche Werte nicht
als separaten Datentyp implementiert. Sie benutzten statt dessen die ganzzahligen 
(integer) Werte 0 und 1 f�r die Darstellung der Wahrheitswerte. Dabei steht die 0 f�r den
Wert {\tt false} und die 1 f�r den Wert {\tt true}. 
Genaugenommen interpretiert C jeden ganzzahligen Wert ungleich 0 als {\tt true}. 
Das m�ssen wir beachten, wenn einen Wert auf  {\tt true} testen wollen. Wir d�rfen
ihn nicht mit {\tt 1} vergleichen sondern m�ssen �berpr�fen, ob er ungleich  {\tt !=} 0 ist.

%todo: C99  {\tt \_Bool}, and

%

Ohne dar�ber nachzudenken, haben wir bereits im letzten Kapitel boolesche Werte 
benutzt. Die Bedingung innerhalb einer {\tt if}-Anweisung ist ein boolescher Ausdruck.
Die Vergleichsoperatoren liefern uns einen booleschen Wert als Resultat:

\begin{verbatim}
    if (x == 5) 
    {
        /* do something*/
    }
\end{verbatim}
%
Der Operator {\tt ==} vergleicht zwei ganze Zahlen und erzeugt einen
Wahrheitswert (boolescher Wert). 

\index{Operator!Vergleich}
\index{Vergleichsoperator}
\index{Pr�prozessor!\#define}
\index{\#define}

%todo Pre C99 has no keywords for the expression of {\tt true} or {\tt false}.
Da fr�here C Standards auch keine Schl�sselw�rter f�r die Angabe von 
{\tt true} oder {\tt false} kennen, verwenden viele Programme den 
C Pr�prozessor um sich selbst entsprechende Konstanten zu definieren.
Diese k�nnen dann �berall verwendet werden, wo ein boolescher Ausdruck
gefordert ist.

Zum Beispiel:

\begin{verbatim}
    #define FALSE 0
    #define TRUE 1
     ...
    if (x != FALSE) 
    {
        /* wird ausgef�hrt, wenn x ungleich 0 ist  */
    }
\end{verbatim}
%
%todo: is a standard idiom for a loop that should run forever (or
%until it reaches a {\tt return} or {\tt break} statement).

\section{Boolesche Variablen}
\index{Typ!{\tt \_Bool}}
\index{Typ!{\tt short}}

Boolesche Werte werden in vielen C Versionen nicht direkt unterst�tzt. Mit dem C99 Standard wurde
das ge�ndert und der Datentyp {\tt \_Bool} eingef�hrt. 
Es gibt aber weiterhin viele Programme, deren Programmcode viel �lter ist und auch nicht alle Compiler
unterst�tzen den C99 Standard komplett. 
%Wenn sich also  keine Variablen vom Datentyp {\tt boole} deklarieren k�nnen, 
Viele Programmierer benutzen statt dessen den Datentyp {\tt short}  in Kombination mit
den bereits erw�hnten Pr�prozessordefinitionen um Wahrheitswerte zu speichern.
In Variablen vom Datentyp {\tt short} k�nnen wie bei {\tt int} Ganze Zahlen gespeichert
werden. Es aber weniger Bit zur Speicherung der Daten benutzt, das hei�t, der Wertebereich
von {\tt short} ist kleiner als der von {\tt int}. 

\begin{verbatim}
    #define FALSE 0
    #define TRUE 1
     ...
    short fred;
    fred = TRUE;
    short testResult = FALSE;
\end{verbatim}
%
Die erste Anweisung des Programms ist eine einfache Variablendeklaration. Wir benutzen den
Datentyp {\tt short} um Speicherplatz zu sparen, wir h�tten auch {\tt int} verwenden k�nnen.
Danach folgt eine Zuweisung, gefolgt von einer Kombination aus Deklaration und 
Zuweisung -- eine so genannte Initialisierung.

\index{Initialisierung}
\index{Anweisung!Initialisierung}

Wie ich bereits erw�hnte, liefern uns die Vergleichsoperatoren einen Wahrheitswert als
Ergebnis. Das Resultat eines Vergleichs l�sst sich in einer Variable speichern:


\begin{verbatim}
  short evenFlag = (n%2 == 0);     /* true if n is even */
  short positiveFlag = (x > 0);    /* true if x is positive */
\end{verbatim}
%
So dass wir es sp�ter als Teil einer bedingten Anweisung nutzen k�nnen:
\begin{verbatim}
  if (evenFlag) 
  {
      printf("n was even when I checked it");
  }
\end{verbatim}
%
Eine Variable die wir in dieser Art nutzen wird als {\bf Flag} bezeichnet,
weil sie uns die Anwesenheit oder Abwesenheit einer Bedingung markiert (\emph{engl.:} to flag).

\index{Flag}

\section{Logische Operatoren}
\label{Logical Operators}
\index{Logischer Operator}
\index{Operator!logischer}
\index{Rangfolge der Operatoren}

Es existieren drei {\bf logische Operatoren} in C: AND, OR und NOT,
welche durch die Symbol {\tt \&\&}, {\tt ||} und
{\tt !} dargestellt werden.  
Die Bedeutung (Semantik) dieser Operatoren leitet sich aus der Booleschen
Algebra ab. Die logischen Operatoren haben eine geringere Priorit�t 
als  arithmetischen Operatoren und Vergleichsoperatoren, dass hei�t, sie werden erst
nach der Auswertung von Vergleichen und Berechnungen angewendet.

\begin{description}
\item[AND-Operator] Im folgenden Ausdruck werden zuerst die Vergleiche durchgef�hrt
und danach werden die Wahrheitswerte der Vergleiche durch den  {\tt AND}-Operator miteinander
verkn�pft:  

\begin{verbatim}
x > 0 && x < 10
\end{verbatim}

Der gesamte Ausdruck ist dann \emph{true}, wenn {\tt x} gr��er als Null und 
(AND) kleiner als 10 ist.

\index{Semantik}

\item[OR-Operator] 
Der {\tt OR}-Operator  wird folgenderma�en verwendet: 
\begin{verbatim}
evenFlag || number%3 == 0
\end{verbatim}
 
Der Ausdruck ist \emph{true}, wenn {\em entweder}
das {\tt evenFlag} einen Wert ungleich Null hat oder (OR) die Variable
 {\tt number} ohne Rest durch 3 teilbar ist. Aus Gr�nden der besseren Lesbarkeit empfiehlt es sich
 auch hier Klammern zu verwenden. Wir h�tten den Ausdruck auch folgenderma�en aufschreiben
 k�nnen: 
\begin{verbatim}
 evenFlag || (number%3 == 0)
\end{verbatim}


\item[NOT-Operator] 
Der  {\tt NOT}-Operator kehrt den Wahrheitswert eines booleschen Ausdrucks in 
sein Gegenteil um (er negiert den Ausdruck).
Damit l�sst sich in unserem vorigen Beispiel die Bedingung umkehren:
\begin{verbatim}
!(number%3 == 0)
\end{verbatim}

Der Ausdruck w�re dann wahr, wenn  {\tt number} nicht  (NOT) durch 3 teilbar ist. 

\end{description}

\index{Verschachtelte Bedingungen}

Logische Operatoren werden oft dazu verwendet verschachtelte Programmverzweigungen
zu vereinfachen. So k�nnte man in dem folgenden Beispiel das Programm
so vereinfachen, dass nur eine einzige  {\tt if}-Anweisung benutzt wird. K�nnten
Sie dazu die Bedingung aufschreiben?

\begin{verbatim}
    if (x > 0) 
    {
        if (x < 10) 
        {
            printf ("x is a positive single digit.\n");
        }
    }
\end{verbatim}

\section{Boolesche Funktionen}
\label{bool}
\index{bool}
\index{Funktion!boolesch}

Es ist manchmal angebracht, dass eine Funktion einen Wahrheitswert
an die aufrufende Funktion zur�ckgibt, so wie wir das bei anderen Datentypen 
auch tun.
Das ist insbesondere dann vorteilhaft, wenn in der Funktion komplexe
Tests durchgef�hrt werden und der aufrufenden Funktion nur mitgeteilt
werden soll, ob der Test erfolgreich war oder nicht.

Zum Beispiel:
\begin{verbatim}
    int IsSingleDigit (int x)
    {
        if (x >= 0 && x < 10) 
        {
            return TRUE;
        } 
        else 
        {
            return FALSE;
        }
    }
\end{verbatim}
%
Der Name der Funktion lautet {\tt IsSingleDigit()}. Es ist �blich 
solchen Testfunktionen einen Namen zu geben, der wie eine
Ja/Nein Frage formuliert ist.
Der R�ckgabewert der Funktion ist {\tt int}, das bedeutet, dass
wir erneut der �bereinkunft folgen, dass  0 -- {\tt false} und 1 -- {\tt true} 
darstellt. Jede {\tt return}-Anweisung muss diese
Konvention befolgen und wir verwenden dazu wieder die bereits
bekannten Pr�prozessordefinitionen.

Der Programmcode ist unkompliziert, wenngleich etwas l�nger als
eigentlich n�tig. Wir  k�nnen versuchen ihn noch weiter zusammenfassen.
Erinnern wir uns, der Ausdruck {\tt x >= 0 \&\& x < 10}
wird zu einem booleschen Wert ausgewertet. Es ist daher ohne weiteres m�glich 
den Wert des Ausdrucks direkt an die aufrufende Funktion zur�ckzugeben
und auf die {\tt if}- Anweisungen komplett zu verzichten:

\begin{verbatim}
    int IsSingleDigit (int x)
    {
        return (x >= 0 && x < 10);
    }
\end{verbatim}
%
In {\tt main} k�nnen wir die Funktion in der �blichen Weise aufrufen:

\begin{verbatim}

    printf("%i\n", IsSingleDigit (2));
    short bigFlag = !IsSingleDigit (17);

\end{verbatim}
%
Die erste Zeile gibt den Wert \emph{true} aus, weil 2 eine einstellige positive Zahl ist.
Ungl�cklicherweise sehen wir bei der Ausgabe von Wahrheitswerten in C
nicht die Worte {\tt TRUE} und {\tt FALSE} auf dem Bildschirm, sondern
die Zahlen  {\tt 1} und {\tt 0}.

%Unfortunately, when C outputs {\tt boolean} values, it
%does not display the words {\tt TRUE} and {\tt FALSE}, but rather the
%integers {\tt 1} and {\tt 0}.
%\footnote{There is a way to fix that
%using the {\tt boolalpha} flag, but it is too hideous to mention.}

Die zweite Zeile ist eine Zuweisung. Der Variablen  {\tt bigFlag} wird
der Wert {\tt true} zugewiesen, wenn das Argument der Funktion 
keine positive einstellige Zahl ist.

Boolesche Funktionen werden sehr h�ufig f�r die Auswertung der
Bedingungen in Programmverzweigungen genutzt:

\begin{verbatim}
    if (IsSingleDigit (x)) 
    {
        printf("x is little\n");
    } 
    else 
    {
        printf("x is big\n");
    }
\end{verbatim}

\section {R�ckgabewerte in der {\tt main()}-Funktion}

Nachdem wir jetzt wissen, dass Funktionen Werte zur�ckgeben
k�nnen, ist es an der Zeit, dass wir uns etwas genauer mit
der Funktion der {\tt return}-Anweisung in der {\tt main}-Funktion
besch�ftigen. 
Wenn wir uns die Definition der Funktion anschauen stellen wir fest, dass sie einen
ganzzahligen Wert (integer) zur�ckgeben sollte :

\begin{verbatim}
    int main (void)
\end{verbatim}

Der �bliche R�ckgabewert aus {\tt main} ist 0. Damit wird angezeigt,
dass das Programm in seiner Ausf�hrung erfolgreich war und genau das
getan hat wozu es programmiert wurde.
Wenn w�hrend der Ausf�hrung des Programms irgend ein Fehler auftritt
ist es �blich -1 oder einen anderen Wert, der angibt um welchen Fehler
es sich handelt, zur�ckzugeben. 

\index{Bibliothek!stdlib.h}
\index{<stdlib.h>}
\index{Header-Datei!stdlib.h}
\index{Bibliothek!stdlib.h}
C stellt in der Standardbibliothek zwei Konstanten zur Verf�gung
 {\tt EXIT\_SUCCESS} und {\tt EXIT\_FAILURE}, die wir in der R�ckgabeanweisung
 nutzen k�nnen. Daf�r m�ssen  {\tt stdlib.h} in unser Programm einbinden:


\begin{verbatim}
    #include <stdlib.h>

    int main (void)
    {
        return EXIT_SUCCESS;   /*program terminated successfully*/
    }  
\end{verbatim}
%
\index{Betriebssystem}

Nat�rlich werden Sie sich fragen wer diese R�ckgabewerte empf�ngt,
weil wir die {\tt main}-Funktion niemals selbst irgendwo aufrufen.
Es stellt sich heraus, dass daf�r das Betriebssystem unseres Rechners
verantwortlich ist. Jedes Mal, wenn das Betriebssystem ein Programm
startet, ruft es {\tt main} auf, so wie wir selbst in unserem Programm 
eigene Funktionen aufrufen. Ist das Programm beendet, erh�lt das 
Betriebssystem eine Mitteilung, ob das Programm erfolgreich war und
k�nnte darauf reagieren (zum Beispiel einen Fehlerbericht verfassen).

Es ist sogar m�glich in der {\tt main}-Funktion Parameter zu verwenden,
um beim Programmstart bereits Daten an das Programm zu �bergeben.
Leider k�nnen wir darauf an dieser Stelle noch nicht genauer eingehen.


%\section {More recursion}
%\index{recursion}
%\index{language!complete}

%So far we have only learned a small subset of C, but you
%might be interested to know that this subset is now
%a {\bf complete} programming language, by which I
%mean that anything that can be computed can be expressed in this
%language.  Any program ever written could be rewritten
%using only the language features we have used so far (actually, we
%would need a few commands to control devices like the keyboard, mouse,
%disks, etc., but that's all).

%\index{Turing, Alan}

%Proving that claim is a non-trivial exercise first
%accomplished by Alan Turing, one of the first computer scientists
%(well, some would argue that he was a mathematician, but a lot of the
%early computer scientists started as mathematicians).  Accordingly, it
%is known as the Turing thesis.  If you take a course on the Theory of
%Computation, you will have a chance to see the proof.

%To give you an idea of what you can do with the tools we have learned
%so far, we'll evaluate a few recursively-defined
%mathematical functions.  A recursive definition is similar to a
%circular definition, in the sense that the definition contains a
%reference to the thing being defined.  A truly circular definition is
%typically not very useful:

%\begin{description}

%\item[frabjuous:] an adjective used to describe
%something that is frabjuous.

%\index{frabjuous}

%\end{description}

%If you saw that definition in the dictionary, you might be
%annoyed.  On the other hand, if you looked up the definition
%of the mathematical function {\bf factorial}, you might
%get something like:

%\begin{eqnarray*}
%&&  0! = 1 \\
%&&  n! = n \cdot (n-1)!
%\end{eqnarray*}

%(Factorial is usually denoted with the symbol $!$, which is
%not to be confused with the C logical operator {\tt !} which
%means NOT.)  This definition says that the factorial of 0 is 1,
%and the factorial of any other value, $n$, is $n$ multiplied
%by the factorial of $n-1$.  So $3!$ is 3 times $2!$, which is 2 times
%$1!$, which is 1 times 0!.  Putting it all together, we get
%$3!$ equal to 3 times 2 times 1 times 1, which is 6.

%If you can write a recursive definition of something, you can usually
%write a C program to evaluate it.  The first step is to decide what
%the parameters are for this function, and what the return type is.
%With a little thought, you should conclude that factorial takes an
%integer as a parameter and returns an integer:

%\begin{verbatim}
%  int factorial (int n)
%  {
%  }
%\end{verbatim}
%%
%If the argument happens to be zero, all we have to do is
%return 1:

%\begin{verbatim}
%  int Factorial (int n)
%  {
%      if (n == 0) 
%      {
%          return 1;
%      }
%  }
%\end{verbatim}
%%
%Otherwise, and this is the interesting part, we have to make
%a recursive call to find the factorial of $n-1$, and then
%multiply it by $n$.

%\begin{verbatim}
%  int Factorial (int n)
%  {
%      if (n == 0) 
%      {
%          return 1;
%      } 
%      else 
%      {
%          int recurse = Factorial (n-1);
%          int result = n * recurse;
%          return result;
%      }
%  }
%\end{verbatim}
%%
%If we look at the flow of execution for this program,
%it is similar to {\tt nLines} from the previous chapter.
%If we call {\tt factorial} with the value 3:

%Since 3 is not zero, we take the second branch and calculate
%the factorial of $n-1$...

%\begin{quote}
%Since 2 is not zero, we take the second branch and calculate
%the factorial of $n-1$...

%\begin{quote}
%Since 1 is not zero, we take the second branch and calculate
%the factorial of $n-1$...

%\begin{quote}
%Since 0 {\em is} zero, we take the first branch and return
%the value 1 immediately without making any more recursive
%calls.

%\end{quote}

%The return value (1) gets multiplied by {\tt n}, which is 1,
%and the result is returned.

%\end{quote}

%The return value (1) gets multiplied by {\tt n}, which is 2,
%and the result is returned.

%\end{quote}

%\noindent The return value (2) gets multiplied by {\tt n}, which is 3,
%and the result, 6, is returned to {\tt main}, or whoever
%called {\tt Factorial (3)}.

%\index{stack}
%\index{diagram!state}
%\index{diagram!stack}

%Here is what the stack diagram looks like for this sequence of
%function calls:

%\vspace{0.1in}
%\centerline{\epsfig{figure=figs/stack3.eps}}
%\vspace{0.1in}
%%
%The return values are shown being passed back up the stack.

%Notice that in the last instance of {\tt factorial}, the local
%variables {\tt recurse} and {\tt result} do not exist because
%when {\tt n=0} the branch that creates them does not execute.

%\section{Leap of faith}
%\index{leap of faith}

%Following the flow of execution is one way to read programs, but as
%you saw in the previous section, it can quickly become labarynthine.
%An alternative is what I call the ``leap of faith.''  When you come to
%a function call, instead of following the flow of execution, you
%{\em assume} that the function works correctly and returns the
%appropriate value.

%In fact, you are already practicing this leap of faith
%when you use built-in functions.  When you call {\tt cos}
%or {\tt exp}, you don't examine the implementations of 
%those functions.  You just assume that they work, because the people
%who wrote the built-in libraries were good programmers.

%Well, the same is true when you call one of your own functions.
%For example, in Section~\ref{bool} we wrote a function called
%{\tt IsSingleDigit} that determines whether a number is between
%0 and 9.  Once we have convinced ourselves that this function
%is correct---by testing and examination of the code---we can
%use the function without ever looking at the code again.

%The same is true of recursive programs.  When you get to the recursive
%call, instead of following the flow of execution, you should {\em
%assume} that the recursive call works (yields the correct result), and
%then ask yourself, ``Assuming that I can find the factorial of $n-1$,
%can I compute the factorial of $n$?''  In this case, it is clear that
%you can, by multiplying by $n$.

%Of course, it is a bit strange to assume that the function works
%correctly when you have not even finished writing it, but that's why
%it's called a leap of faith!

%\section{One more example}
%\index{factorial}

%In the previous example I used temporary variables to spell out the
%steps, and to make the code easier to debug, but I could have saved a
%few lines:

%\begin{verbatim}
%  int Factorial (int n) 
%  {
%      if (n == 0) 
%      {
%          return 1;
%      } 
%      else 
%      {
%          return n * Factorial (n-1);
%      }
%  }
%\end{verbatim}
%%
%From now on I will tend to use the more concise version, but
%I recommend that you use the more explicit version while you
%are developing code.   When you have it working, you can
%tighten it up, if you are feeling inspired.

%After {\tt Factorial}, the classic example of a recursively-defined
%mathematical function is {\tt Fibonacci}, which has the
%following definition:

%\begin{eqnarray*}
%&& fibonacci(0) = 1 \\
%&& fibonacci(1) = 1 \\
%&& fibonacci(n) = fibonacci(n-1) + fibonacci(n-2);
%\end{eqnarray*}
%%
%Translated into C, this is

%\begin{verbatim}
%  int Fibonacci (int n) 
%  {
%      if (n == 0 || n == 1) 
%      {
%          return 1;
%      } 
%      else 
%      {
%          return Fibonacci (n-1) + Fibonacci (n-2);
%      }
%  }
%\end{verbatim}
%%
%If you try to follow the flow of execution here, even for fairly small
%values of {\tt n}, your head explodes.  But according to the leap of
%faith, if we assume that the two recursive calls (yes, you can make
%two recursive calls) work correctly, then it is clear that we get the
%right result by adding them together.

\section{Glossar}
%(engl: \emph{})

\begin{description}


\item[R�ckgabewert (engl: \emph{return value}):]  Der Wert der bei der R�ckkehr aus einer
Funktion an die aufrufende Funktion zur�ckgegeben wird.

\item[R�ckgabetyp (engl: \emph{return type}):]  Der Typ des Werts der bei der R�ckkehr aus einer
Funktion an die aufrufende Funktion zur�ckgegeben wird.

\item[Unerreichbarer Code (engl: \emph{dead code}):]  Der Teil des Programmcodes der
niemals ausgef�hrt wird, zum Beispiel weil sich der Code hinter einer {\tt return} Anweisung
befindet.

\item[Debug Code (engl: \emph{debug code}):]  Anweisungen innerhalb eines Programms,
welche dazu genutzt werden das Programm w�hrend der Entwicklung zu testen.
Diese Anweisungen sollten im fertigen Programm deaktiviert werden.

\item[void (engl: \emph{void}):]  Ein spezieller Typ, der verwendet wird um zu kennzeichnen,
dass eine Funktion keinen Wert zur�ckliefert, und/oder keine Parameter besitzt.

%NO overloading in C!

%\item[overloading:]  Having more than one function with the same name
%but different parameters.  When you call an overloaded function,
%C knows which version to use by looking at the arguments you
%provide.

\item[Boolesche Variable (engl: \emph{boolean}):]  Eine Variable, welche einen von
zwei m�glichen Zust�nden annehmen kann, oft mit $true$ und $false$ bezeichnet. 
In C werden boolsche Werte �berwiegend in Variablen vom Type {\tt int oder \tt short} 
gespeichert und mit Hilfe des Pr�prozessors definiert. \\(z.B. \texttt{\#define TRUE 1} )

\item[Flag (engl: \emph{flag}):]  Eine Variable, welche eine Bedingung oder einen
Statuscode speichert (meistens ein boolscher Wert).

\item[Vergleichsoperator (engl: \emph{comparison operator}):]  Ein Operator, welcher 
zwei Werte vergleicht und als Ergebnis einen Wahrheitswert liefert, welcher die Beziehung
der Operanden des Ausdrucks charakterisiert.

\item[Logischer Operator (engl: \emph{logical operator}):]  Ein Operator, der die Operanden
auf Basis einer logischen Verkn�fung (UND, ODER, NICHT) auswertet und einen Wahrheitswert
zur�ckliefert. 

\index{R�ckgabewert}
\index{R�ckgabetyp}
\index{dead code}
\index{Unerreichbarer Code}
\index{Debug Code}
\index{void}
\index{boolean}
\index{Boolsche Variable}
\index{Operator!vergleichender}
\index{Operator!logischer}
\index{Vergleichsoperator}
\index{Logischer Operator}

\end{description}

\section{�bungsaufgaben}
\setcounter{exercisenum}{0}

\ifthenelse {\boolean{German}}{ \begin{exercise}

Sie haben 3 St�cke erhalten und stehen vor der Aufgabe daraus ein
Dreieck zu formen. Diese Aufgabe kann l�sbar oder unl�sbar sein,
je nachdem wie lang die zur Verf�gung stehenden St�cke sind.

Wenn zum Beispiel einer der St�cke 12cm lang ist und die anderen 
Beiden je nur 2cm, so ist klar, dass diese sich nicht in der Mitte treffen
werden.
Es gibt einen einfachen Test, der f�r drei beliebige L�ngen ermittelt,
ob sich ein Dreieck formen l�sst oder nicht: 


\begin{quotation}
``Wenn eine der drei L�ngen gr��er ist als die Summe der
anderen beiden, dann l�sst sich kein Dreieck formen. Ansonsten
ist es m�glich ein Dreieck zu formen.''
\end{quotation}

Schreiben Sie eine Funktion mit dem Namen {\tt IsTriangle}, welche
drei {\tt integer} als Argumente hat und entweder {\tt TRUE} or {\tt FALSE}
zur�ckgibt, abh�ngig davon, ob sich aus St�cken mit der gegebenen L�nge
ein Dreieck formen l�sst oder nicht. 

Der Sinn dieser �bung bestehet darin eine Funktion mit bedingten Abfragen
zu schreiben, welche als Ergebnis einen Wert zur�ckgibt. 
\end{exercise}
%%%%%%%%%%%%%%%%%%%%%%%%%%%%%%%%%%%%%%%%
\begin{exercise}
\label{ex.isdiv}
Schreiben Sie eine Funktion {\tt IsDivisible} welche zwei {\tt integer} Werte,  {\tt n} and {\tt m} als 
Argumente hat und {\tt TRUE} zur�ckgibt, wenn {\tt n} durch {\tt m} teilbar ist. 
Ansonsten soll die Funktion {\tt FALSE} zur�ckgeben.
\end{exercise}
%%%%%%%%%%%%%%%%%%%%%%%%%%%%%%%%%%%%%%%%
\begin{exercise}
Der Sinn der folgenden �bung besteht darin
das Verst�ndnis f�r die Ausf�hrung logischer Operatoren zu sch�rfen und
den Programmablauf in Funktionen mit R�ckgabewerten nachvollziehbar zu machen.
Wie lautet die Ausgabe des folgenden Programms?

\begin{verbatim}
#define TRUE 1
#define FALSE 0

  short IsHoopy (int x)
   {
      short hoopyFlag;
      if (x%2 == 0) 
      {
          hoopyFlag = TRUE;
      } 
      else 
      {
          hoopyFlag = FALSE;
      }
      return hoopyFlag;
  }

  short IsFrabjuous (int x) 
  {
      short frabjuousFlag;
      if (x > 0) 
      {
          frabjuousFlag = TRUE;
      }
      else 
      {
          frabjuousFlag = FALSE;
      }
      return frabjuousFlag;
  }

  int main (void) 
  {
      short flag1 = IsHoopy (202);
      short flag2 = IsFrabjuous (202);
      printf ("%i\n", flag1);
      printf ("%i\n", flag2);
      if (flag1 && flag2) 
      {
          printf ("ping!\n");
      }
      if (flag1 || flag2) 
      {
          printf ("pong!\n");
      }
      return EXIT_SUCCESS;
  }
\end{verbatim}
\end{exercise}
%%%%%%%%%%%%%%%%%%%%%%%%%%%%%%%%%%%%%%%%

\begin{exercise}
Die Entfernung zwischen zwei Punkten  $(x_1, y_1)$ und $(x_2, y_2)$
ist

\[Distance = \sqrt{(x_2 - x_1)^2 + (y_2 - y_1)^2} \]

Schreiben Sie bitte eine Funktion {\tt Distance} welche vier
{\tt double} als Argumente erh�lt---{\tt x1}, {\tt y1}, {\tt x2} und {\tt
y2}---und welche die Entfernung zwischen den Punkten  $(x_1, y_1)$ und $(x_2, y_2)$ zur�ckgibt.

Sie sollen annehmen, dass bereits eine Funktion mit dem Namen {\tt SumSquares}
existiert, welche die Quadrate der Summen berechnet und zur�ckgibt.

Zum Beispiel:

\begin{verbatim}
    double x = SumSquares (3.0, 4.0);
\end{verbatim}
%
w�rde {\tt x} den Wert {\tt 25.0} zuweisen.

Der Sinn dieser �bung besteht darin eine neue Funktion zu schreiben,
welche eine bereits bestehende Funktion aufruft.
Sie sollen nur die eine Funktion {\tt Distance} schreiben.
Lassen Sie die Funktionen {\tt SumSquares} und {\tt main} weg und rufen
Sie {\tt Distance} auch nicht auf!
\end{exercise}

%\begin{exercise}
%The point of this exercise is to practice the syntax of fruitful
%functions.

%\begin{enumerate}

%\item Use your existing solution to Exercise~\ref{ex.multadd} and make sure
%you can still compile and run it.

%\item Transform {\tt multadd} into a fruitful function, so
%that instead of printing a result, it returns it.

%\item Everywhere in the program that {\tt multadd} gets
%invoked, change the invocation so that it stores the
%result in a variable and/or prints the result.

%\item Transform {\tt yikes} in the same way.

%\end{enumerate}
%\end{exercise}


%\begin{exercise}
%The point of this exercise is to use a stack diagram to understand
%the execution of a recursive program.

%\begin{verbatim}

%    int main (void) 
%    {
%        printf (Prod (1, 4));
%    }

%    int Prod (int m, int n) 
%    {
%        if (m == n) 
%        {
%            return n;
%        } 
%        else 
%        {
%            int recurse = Prod (m, n-1);
%            int result = n * recurse;
%            return result;
%        }
%    }

%\end{verbatim}
%%
%\begin{enumerate}

%\item Draw a stack diagram showing the state of the program just
%before the last instance of {\tt prod} completes.
%What is the output of this program?

%\item Explain in a few words what {\tt Prod} does.

%\item Rewrite {\tt prod} without using the temporary variables
%{\tt recurse} and {\tt result}.

%\end{enumerate}
%\end{exercise}


%\begin{exercise}
%The purpose of this exercise is to translate a recursive definition
%into a C function.  The Ackerman function is defined for non-negative
%integers as follows:

%\begin{eqnarray*}
%A(m,n) = \left\{
%\begin{array}{l@{\quad \mathbf{if} \quad }l}
%n+1 & m=0 \\
%A(m-1, 1) & m>0, n=0 \\
%A(m-1, A(m, n-1)) & m>0, n>0 \\
%\end{array} \right.
%\end{eqnarray*}
%%
%Write a function called {\tt ack} that takes two {\tt int}s as
%parameters and that computes and returns the value
%of the Ackerman function.

%Test your implementation of Ackerman by invoking it
%from {\tt main} and printing the return value.  

%WARNING: the return value gets very big very quickly.  You should try it
%only for small values of $m$ and $n$ (not bigger than 2).

%\end{exercise}

%
%\begin{exercise}
%\begin{enumerate}

%\item Create a program called {\tt Recurse.c} and
%type in the following functions:

%\begin{verbatim}
%    /* First: returns the first character of the given String */
%    char First (String s) 
%    {
%        return s.charAt (0);
%    }

%    /* Rest: returns a new String that contains all but the */
%    /* first letter of the given String */
%    String Rest (String s) 
%    {
%        return s.substring (1, s.length());
%    }

%    /* Length: returns the length of the given String */
%    int Length (String s) 
%    {
%        return s.length();
%    }
%\end{verbatim}
%
%\item Write some code in {\tt main} that tests each of these
%functions.  Make sure they work, and make sure you understand
%what they do.

%\item Write a function called {\tt printString} that takes a
%String as a parameter and that prints the letters of the
%String, one on each line.  It should be a {\tt void} function.

%\item Write a function called {\tt printBackward} that does
%the same thing as {\tt printString} but that prints the String
%backwards (one character per line).

%\item Write a function called {\tt reverseString} that takes
%a String as a parameter and that returns a new String as a
%return value.  The new String should contain the same letters
%as the parameter, but in reverse order.  For example, the
%output of the following code

%\begin{verbatim}
%	String backwards = reverseString ("Allen Downey");
%	System.out.println (backwards);
%\end{verbatim}
%%
%should be

%\begin{verbatim}
%yenwoD nellA
%\end{verbatim}

%
%\end{enumerate}
%\end{exercise}

%%%%%%%%%%%%%%%%%%%%%%%%%%%%%%%%%%%%%%%%
\vskip 1em
\begin{exercise}
%\textbf{!! 1. ZUSATZAUFGABE !!}

Erstellen Sie eine neue Programmdatei mit dem Namen {\tt Sum.c},
und geben Sie die folgenden zwei Funktionen ein:

\begin{verbatim}
  int FunctionOne (int m, int n) 
  {
      if (m == n) 
      {
          return n;
      } 
      else 
      {
          return m + FunctionOne (m+1, n);
      }
  }

  int FunctionTwo (int m, int n) 
  {
      if (m == n) 
      {
          return n;
      } 
      else 
      {
          return n * FunctionTwo (m, n-1);
      }
  }
\end{verbatim}
%
\begin{enumerate}

\item F�gen Sie in den Funktionen eine {\tt prinf()}-Anweisung 
hinzu, mit der Sie sofort nach dem Funktionsaufruf den aktuellen
Wert der Funktionsparameter ausgeben. 
Das ist eine n�tzliche Technik um rekursive Programme zu debuggen,
da sich auf diese Weise die Abarbeitung eines Programms besser nachvollziehen
l�sst.
%
%functions so that they print their arguments each time they are
%invoked.  This is a useful technique for debugging recursive
%programs, since it demonstrates the flow of execution.


\item Schreiben Sie in der {\tt main}-Funktion ihres Programms einige
Zeilen um diese Funktionen zu testen (rufen Sie die Funktionen einige
Male mit unterschiedlichen Argumenten auf und lassen Sie sich die R�ckgabewerte
ausgeben, um zu sehen, was die Funktionen machen.

Nutzen Sie eine Kombination aus gezieltem Testen und der Inspektion
des Quellcodes um herauszufinden, was diese Funktionen machen.
Geben Sie den Funktionen einen neuen Namen, aus dem besser hervorgeht, was
die Funktionen machen. F�gen Sie Kommentare zu den Funktionen hinzu um
ihre Funktion allgemeinverst�ndlich zu beschreiben.


\end{enumerate}
\end{exercise}

%\begin{exercise}
%\label{ex.power}
%Write a recursive function called {\tt Power} that
%takes a double {\tt x} and an integer {\tt n} and that
%returns $x^n$.  Hint: a recursive definition of this
%operation is {\tt Power (x, n) = x * Power (x, n-1)}.
%Also, remember that anything raised to the zeroeth power
%is 1.
%\end{exercise}

%%%%%%%%%%%%%%%%%%%%%%%%%%%%%%%%%%%%%%%%
\begin{exercise}
%\textbf{!! 2. ZUSATZAUFGABE !!}

\label{gcd}
(This exercise is based on page 44 of Ableson and Sussman's
{\em Structure and Interpretation of Computer Programs}.)

The following algorithm is known as Euclid's Algorithm because
it appears in Euclid's {\em Elements} (Book 7, ca. 300 B.C.).
It may be the oldest nontrivial algorithm.

The algorithm is based on the observation that, if $r$ is the
remainder when $a$ is divided by $b$, then the common divisors
of $a$ and $b$ are the same as the common divisors of $b$ and $r$.
Thus we can use the equation

\[ gcd (a, b) = gcd (b, r) \]

to successively reduce the problem of computing a GCD to the
problem of computing the GCD of smaller and smaller pairs of integers.
For example,

\[ gcd (36, 20) = gcd (20, 16) = gcd (16, 4) = gcd (4, 0) = 4\]

implies that the GCD of 36 and 20 is 4.  It can be shown
that for any two starting numbers, this repeated reduction eventually
produces a pair where the second number is 0.  Then the GCD is the
other number in the pair.

Write a function called {\tt gcd} that takes two integer parameters and
that uses Euclid's algorithm to compute and return the greatest
common divisor of the two numbers.
\end{exercise}


%\begin{exercise}
%If you are given three sticks, you may or may not be able to arrange
%them in a triangle.  For example, if one of the sticks is 12 inches
%long and the other two are one inch long, it is clear that you will
%not be able to get the short sticks to meet in the middle.  For any
%three lengths, there is a simple test to see if it is possible to form
%a triangle:

%\begin{quotation}
%``If any of the three lengths is greater than the sum of the other two,
%then you cannot form a triangle.  Otherwise, you can.''
%\end{quotation}

%Write a function named {\tt IsTriangle} that it takes three integers as
%arguments, and that returns either {\tt true} or {\tt false},
%depending on whether you can or cannot form a triangle from sticks
%with the given lengths.

%%return values only at next chapter!!!
%The point of this exercise is to use conditional statements to
%write a function that returns a value.
%\end{exercise}

%\begin{exercise}
%\label{ex.isdiv}
%Write a function named {\tt isDivisible} that takes
%two integers, {\tt n} and {\tt m} and that returns {\tt true}
%if {\tt n} is divisible by {\tt m} and {\tt false} otherwise.
%\end{exercise}

%\begin{exercise}
%What is the output of the following program?  The purpose of
%this exercise is to make sure you understand logical operators
%and the flow of execution through fruitful functions.

%\begin{verbatim}
%  int main (void) 
%  {
%      short flag1 = isHoopy (202);
%      short flag2 = isFrabjuous (202);
%      printf ("%i\n", flag1);
%      printf ("%i\n", flag2);
%      if (flag1 && flag2) {
%          printf ("ping!");
%      }
%      if (flag1 || flag2) {
%          printf ("pong!");
%      }
%  }

%  public static boolean isHoopy (int x) {
%      boolean hoopyFlag;
%      if (x%2 == 0) {
%          hoopyFlag = true;
%      } else {
%          hoopyFlag = false;
%      }
%      return hoopyFlag;
%  }

%  public static boolean isFrabjuous (int x) {
%      boolean frabjuousFlag;
%      if (x > 0) {
%          frabjuousFlag = true;
%      } else {
%          frabjuousFlag = false;
%      }
%      return frabjuousFlag;
%  }
%\end{verbatim}

%\end{exercise}

%

%\begin{exercise}
%The distance between two points $(x_1, y_1)$ and $(x_2, y_2)$
%is

%\[Distance = \sqrt{(x_2 - x_1)^2 + (y_2 - y_1)^2} \]

%Please write a function named {\tt Distance} that takes four
%doubles as parameters---{\tt x1}, {\tt y1}, {\tt x2} and {\tt
%y2}---and that prints the distance between the points.

%You should assume that there is a function named {\tt sumSquares}
%that calculates and returns the sum of the squares of its arguments.
%For example:

%\begin{verbatim}
%    double x = SumSquares (3.0, 4.0);
%\end{verbatim}
%%
%would assign the value {\tt 25.0} to {\tt x}.

%The point of this exercise is to write a new function that uses an
%existing one.  You should write only one function: {\tt Distance}.  You
%should not write {\tt SumSquares} or {\tt main} and you should not
%invoke {\tt Distance}.
%\end{exercise}

%\begin{exercise}
%The point of this exercise is to practice the syntax of fruitful
%functions.

%\begin{enumerate}

%\item Use your existing solution to Exercise~\ref{ex.multadd} and make sure
%you can still compile and run it.

%\item Transform {\tt multadd} into a fruitful function, so
%that instead of printing a result, it returns it.

%\item Everywhere in the program that {\tt multadd} gets
%invoked, change the invocation so that it stores the
%result in a variable and/or prints the result.

%\item Transform {\tt yikes} in the same way.

%\end{enumerate}
%\end{exercise}

%
%\begin{exercise}
%The point of this exercise is to use a stack diagram to understand
%the execution of a recursive program.

%\begin{verbatim}

%    int main (void) 
%    {
%        printf (Prod (1, 4));
%    }

%    int Prod (int m, int n) 
%    {
%        if (m == n) 
%        {
%            return n;
%        } 
%        else 
%        {
%            int recurse = Prod (m, n-1);
%            int result = n * recurse;
%            return result;
%        }
%    }

%\end{verbatim}
%%
%\begin{enumerate}

%\item Draw a stack diagram showing the state of the program just
%before the last instance of {\tt prod} completes.
%What is the output of this program?

%\item Explain in a few words what {\tt Prod} does.

%\item Rewrite {\tt prod} without using the temporary variables
%{\tt recurse} and {\tt result}.

%\end{enumerate}
%\end{exercise}

%
%\begin{exercise}
%The purpose of this exercise is to translate a recursive definition
%into a C function.  The Ackerman function is defined for non-negative
%integers as follows:

%\begin{eqnarray*}
%A(m,n) = \left\{
%\begin{array}{l@{\quad \mathbf{if} \quad }l}
%n+1 & m=0 \\
%A(m-1, 1) & m>0, n=0 \\
%A(m-1, A(m, n-1)) & m>0, n>0 \\
%\end{array} \right.
%\end{eqnarray*}
%%
%Write a function called {\tt ack} that takes two {\tt int}s as
%parameters and that computes and returns the value
%of the Ackerman function.

%Test your implementation of Ackerman by invoking it
%from {\tt main} and printing the return value.  

%WARNING: the return value gets very big very quickly.  You should try it
%only for small values of $m$ and $n$ (not bigger than 2).

%\end{exercise}

%
%\begin{exercise}
%\begin{enumerate}

%\item Create a program called {\tt Recurse.c} and
%type in the following functions:

%\begin{verbatim}
%    /* First: returns the first character of the given String */
%    char First (String s) 
%    {
%        return s.charAt (0);
%    }

%    /* Rest: returns a new String that contains all but the */
%    /* first letter of the given String */
%    String Rest (String s) 
%    {
%        return s.substring (1, s.length());
%    }

%    /* Length: returns the length of the given String */
%    int Length (String s) 
%    {
%        return s.length();
%    }
%\end{verbatim}

%\item Write some code in {\tt main} that tests each of these
%functions.  Make sure they work, and make sure you understand
%what they do.

%\item Write a function called {\tt printString} that takes a
%String as a parameter and that prints the letters of the
%String, one on each line.  It should be a {\tt void} function.

%\item Write a function called {\tt printBackward} that does
%the same thing as {\tt printString} but that prints the String
%backwards (one character per line).

%\item Write a function called {\tt reverseString} that takes
%a String as a parameter and that returns a new String as a
%return value.  The new String should contain the same letters
%as the parameter, but in reverse order.  For example, the
%output of the following code

%\begin{verbatim}
%	String backwards = reverseString ("Allen Downey");
%	System.out.println (backwards);
%\end{verbatim}
%%
%should be

%\begin{verbatim}
%yenwoD nellA
%\end{verbatim}

%
%\end{enumerate}
%\end{exercise}

%\begin{exercise}
%\begin{enumerate}

%\item Create a new program called {\tt Sum.c},
%and type in the following two functions.

%\begin{verbatim}
%  int FunctionOne (int m, int n) 
%  {
%      if (m == n) 
%      {
%          return n;
%      } 
%      else 
%      {
%          return m + FunctionOne (m+1, n);
%      }
%  }

%  int FunctionTwo (int m, int n) 
%  {
%      if (m == n) 
%      {
%          return n;
%      } 
%      else 
%      {
%          return n * FunctionTwo (m, n-1);
%      }
%  }
%\end{verbatim}
%%
%\item Write a few lines in {\tt main} to test these functions.
%Invoke them a couple of times, with a few different values,
%and see what you get.  By some combination of testing and
%examination of the code, figure out what these functions do,
%and give them more meaningful names.  Add comments that
%describe their function abstractly.

%\item Add a {\tt prinf} statement to the beginning of both
%functions so that they print their arguments each time they are
%invoked.  This is a useful technique for debugging recursive
%programs, since it demonstrates the flow of execution.

%\end{enumerate}
%\end{exercise}

%\begin{exercise}
%\label{ex.power}
%Write a recursive function called {\tt Power} that
%takes a double {\tt x} and an integer {\tt n} and that
%returns $x^n$.  Hint: a recursive definition of this
%operation is {\tt Power (x, n) = x * Power (x, n-1)}.
%Also, remember that anything raised to the zeroeth power
%is 1.
%\end{exercise}

%
%\begin{exercise}
%\label{gcd}
%(This exercise is based on page 44 of Ableson and Sussman's
%{\em Structure and Interpretation of Computer Programs}.)

%The following algorithm is known as Euclid's Algorithm because
%it appears in Euclid's {\em Elements} (Book 7, ca. 300 B.C.).
%It may be the oldest nontrivial algorithm.

%The algorithm is based on the observation that, if $r$ is the
%remainder when $a$ is divided by $b$, then the common divisors
%of $a$ and $b$ are the same as the common divisors of $b$ and $r$.
%Thus we can use the equation

%\[ gcd (a, b) = gcd (b, r) \]

%to successively reduce the problem of computing a GCD to the
%problem of computing the GCD of smaller and smaller pairs of integers.
%For example,

%\[ gcd (36, 20) = gcd (20, 16) = gcd (16, 4) = gcd (4, 0) = 4\]

%implies that the GCD of 36 and 20 is 4.  It can be shown
%that for any two starting numbers, this repeated reduction eventually
%produces a pair where the second number is 0.  Then the GCD is the
%other number in the pair.

%Write a function called {\tt gcd} that takes two integer parameters and
%that uses Euclid's algorithm to compute and return the greatest
%common divisor of the two numbers.
%\end{exercise}
}
{\input{exercises/Exercise_5_english}}


%\include{Chapter6_Iteration}
%\include{Chapter7_Array}
%\include{Chapter8_String}
%\include{Chapter9_Struct}
%!TEX root = Main_german.tex

% LaTeX source for textbook ``How to think like a computer scientist''
% Copyright (C) 1999  Allen B. Downey

% This LaTeX source is free software; you can redistribute it and/or
% modify it under the terms of the GNU General Public License as
% published by the Free Software Foundation (version 2).

% This LaTeX source is distributed in the hope that it will be useful,
% but WITHOUT ANY WARRANTY; without even the implied warranty of
% MERCHANTABILITY or FITNESS FOR A PARTICULAR PURPOSE.  See the GNU
% General Public License for more details.

% Compiling this LaTeX source has the effect of generating
% a device-independent representation of a textbook, which
% can be converted to other formats and printed.  All intermediate
% representations (including DVI and Postscript), and all printed
% copies of the textbook are also covered by the GNU General
% Public License.

% This distribution includes a file named COPYING that contains the text
% of the GNU General Public License.  If it is missing, you can obtain
% it from www.gnu.org or by writing to the Free Software Foundation,
% Inc., 59 Temple Place - Suite 330, Boston, MA 02111-1307, USA.

\selectlanguage{ngerman}
\chapter{Iteration}

\section{Zuweisung unterschiedlicher Werte}
\index{Zuweisung}
\index{Anweisung!Zuweisung}
\index{Zuweisung unterschiedlicher Werte}


Ich habe noch nicht dar�ber gesprochen, aber es ist durchaus
erlaubt einer Variablen mehr als einmal einen Wert zuzuweisen.
Der Effekt der zweiten Zuweisung besteht darin, dass der
\emph{alte} Wert der Variablen durch einen \emph{neuen} Wert ersetzt wird:
%
% haven't said much about it, but it is legal in C to
%make more than one assignment to the same variable.  The
%effect of the second assignment is to replace the old value
%of the variable with a new value.

\begin{verbatim}
    int fred = 5;
    printf ("%i", fred);
    fred = 7;
    printf ("%i", fred);
\end{verbatim}
%
Die Ausgabe dieses Programms ist {\tt 57}. \\
Wenn {\tt fred} zum ersten Mal ausgegeben wird, hat die Variable den Wert 5.
Zum Zeitpunkt der zweiten Ausgabe hat die Variable den Wert 7.

Diese Art von {\bf aufeinanderfolgenden Zuweisungen} ist der Grund warum
ich Variablen als ein {\em Container} f�r Werte bezeichnet habe.  
Wenn wir einer Variablen einen Wert zuweisen, wird der Inhalt
dieses Containers ver�ndert, wie in der folgenden Grafik dargestellt:

\vspace{0.1in}
\centerline{\epsfig{figure=figs/assign2.eps}}
\vspace{0.1in}

Wenn wir mehrere Zuweisungen zu einer Variable vornehmen, ist es besonders wichtig,
dass wir zwischen der Zuweisungsanweisung und der Anweisung, welche
die Gleichheit von Werten testet unterscheiden.

Die Programmiersprache C benutzt das Symbol {\tt =} f�r die Zuweisung von Werten.
Es ist daher verf�hrerisch, die Anweisung {\tt a = b}
als eine �berpr�fung der Gleichheit der Variablen {\tt a} und {\tt b} 
zu interpretieren. Was nicht der Fall ist! 

Zuerst einmal k�nnen wir feststellen, dass die Gleichheitsoperation 
kommutativ ist und eine Zuweisungsoperation nicht. In der 
Mathematik gilt:

\vskip -1em
\begin{displaymath}
\mathrm{Wenn}\  a = 7\mathrm{ ,}\  \mathrm{dann }\ 7 = a
\end{displaymath}

In C ist  {\tt a = 7;} eine g�ltige Anweisung. Wenn wir aber
die Anweisung {\tt 7 = a;} in unser Programm schreiben
erhalten wir einen Fehler. Bei der linken Seite einer Zuweisung
muss es sich um einen Ort im Speicher des Computers  
handeln.

Weiterhin ist in der Mathematik ein Ausdruck der Gleichheit
zu jeder Zeit wahr. Wenn  $a = b$ ist, dann wird $a$ \textbf{immer} 
den gleichen Wert besitzen wie $b$.
In C, kann eine Zuweisung zwei Variablen den gleichen Wert
geben, aber die Werte der Variablen sind damit nicht f�r alle
Zeit festgelegt und k�nnen sich �ndern!

\begin{verbatim}
    int a = 5;
    int b = a;     /* a und b haben jetzt den gleichen Wert */
    a = 3;         /* a und b sind nicht l�nger gleich */
\end{verbatim}
%
Die dritte Zeile �ndert den Wert von {\tt a}, der Wert der 
Variablen {\tt b} ist davon aber nicht betroffen. Ab diesem 
Zeitpunkt im Programm sind die Variablen nicht l�nger gleich.
In vielen anderen Programmiersprachen wird deshalb f�r 
die Wertzuweisung an eine Variable ein anderes Symbol (\,{\tt :=}\, oder \, {\tt <-}\,) und nicht das
Gleichheitszeichen benutzt.
Damit wird die Verwechslungsgefahr zwischen den Operationen
verringert.

Obwohl die mehrfache Zuweisung von unterschiedlichen Werten 
an eine Variable oft sehr n�tzlich sein kann, sollten wir diese
Art der Wertzuweisung mit Vorsicht benutzen.
Wenn sich der Wert einer Variablen st�ndig an unterschiedlichen
Stellen in einem Programm ver�ndert, so wird das Lesen und
die Fehlersuche in dem Programm deutlich erschwert. \hint

\section{Iteration - Wiederholungen im Programm}
\index{Iteration}

Eine der wichtigsten Aufgaben die durch Computer �bernommen 
werden ist die Automatisierung st�ndig wiederkehrender Aufgaben.
Die fehlerfreie Wiederholung identischer oder sehr �hnlicher 
Aufgaben ist ein Gebiet auf dem  Computer den menschlichen
Benutzern deutlich �berlegen sind.

Wir haben im Kapitel  \ref{recursion} bereits Funktionen 
wie  {\tt PrintLines()} und {\tt Countdown()} kennengelernt, die mit 
Hilfe der Rekursion wiederkehrende Aufgaben bew�ltigt haben.
Dabei wurden Wiederholungen durch die ineinander geschachtelte
Ausf�hrung von Funktionen erreicht.

Wir werden jetzt eine neue Art der Ausf�hrung von Wiederholungen
kennenlernen bei der mit Hilfe von Kontrollstrukturen die Ausf�hrung
gesteuert werden kann. 
Diese Art der wiederholten Ausf�hrung bezeichnet man auch als
{\bf Iteration}. 
Mit Hilfe der  {\tt while} und der {\tt for}-Anweisung
k�nnen wir die Wiederholung von Anweisungsbl�cken genau steuern.


\section{Die {\tt while}-Anweisung}
\index{Anweisung!while}
\index{while Anweisung}

Ich m�chte kurz an einem Beispiel zeigen, wie wir die
bereits bekannte {\tt Countdown()} Funktion mittels einer  {\tt while}-Anweisung umschreiben k�nnen:

\begin{verbatim}
  void Countdown (int n) 
  {
      while (n > 0) 
      {
          printf ("%i\n", n);
          n = n-1;
      }
      printf ("Blastoff!\n");
  }
\end{verbatim}
%
Was auff�llt ist die gute Lesbarkeit des Quelltextes der
{\tt while}-Anweisung, welcher sich fast von selbst erkl�rt. 
Die Anweisung hat folgende Bedeutung:  
``Solange (engl: \textit{while}) {\tt n} gr��er als Null ist, 
geben wir den aktuellen Wert von {\tt n} auf
dem Bildschirm aus. Danach verringern wir den 
Wert von  {\tt n} um 1.  Wenn  {\tt n} den Wert Null erreicht hat,
wird die Schleife verlassen und das Wort ``Blastoff!''
auf dem Bildschirm ausgegeben.

Etwas formeller k�nnen wir die Arbeitsweise der {\tt while}-Anweisung
folgenderma�en beschreiben:

\begin{enumerate}

\item  Die in Klammern angegebene Bedingung wird ausgewertet und
der Wahrheitswert {\tt true} oder {\tt false} ermittelt.

\item Wenn die Bedingung falsch ist, wird die {\tt while}-Anweisung verlassen
und die Ausf�hrung des Programms mit der n�chstfolgenden Anweisung fortgesetzt.

\item Wenn die Bedingung wahr ist, werden alle Anweisungen im Anweisungsblock
der {\tt while}-Anweisung nacheinander ausgef�hrt und am Ende des Blocks wird zu Schritt
1 zur�ckgekehrt.

\end{enumerate}

\index{Schleifen}
\index{Schleife!K�rper}
\index{K�rper!Schleife}


Diese Art des Programmablaufs wird auch als eine {\bf Schleife} bezeichnet,
weil der dritte Schritt im Ablauf wieder auf den ersten Schritt zur�ckf�hrt. 

Die Anweisungen im Inneren der Schleife bezeichnet man
als den {\bf Schleifenk�rper}.
Sollte es der Fall sein, dass gleich beim ersten �berpr�fen der Bedingung der
Wert  {\tt false} ermittelt wird, so werden die Anweisungen im Schleifenk�rper 
�berhaupt nicht ausgef�hrt.

\index{Schleifen!endlos}
\index{Endlosschleifen}

�blicherweise wird die Ausf�hrungsh�ufigkeit der Schleife durch eine
Kontrollvariable gesteuert.
Der K�rper einer Schleife sollte den Wert der 
Kontrollvariablen so �ndern, dass schlie�lich irgendwann
die Bedingung den Wert  {\tt false} erh�lt und die Schleife
beendet wird. Anderenfalls w�rde die Schleife unendlich oft
wiederholt werden. Eine derartige Schleife bezeichnet man dann
als  {\bf Endlosschleife}. 
Informatiker finden daher  zum Beispiel den Aufdruck 
auf einigen Shampoo-Flaschen sehr am�sant:  ``Einseifen, Aussp�hlen, Wiederholen'' 
ist eine Endlosschleife.

Im Fall von {\tt Countdown()} k�nnen wir beweisen, dass die
Schleife irgendwann beendet sein muss, weil wir
wissen dass der Wert von {\tt n} endlich ist und wir sehen
k�nnen, dass mit jedem Schleifendurchlauf (jeder {\bf Iteration}) der 
Wert von {\tt n} stetig kleiner wird.
Irgendwann wird dieser Wert also Null sein und die Schleife endet. 

Nicht in allen F�llen l�sst sich das so einfach ermitteln:

\begin{verbatim}
  void Sequence (int n) 
  {
      while (n != 1) 
      {
          printf ("%i\n", n);
          if (n%2 == 0)       /* n ist gerade */
          {          
              n = n / 2;
          } 
          else                /* n ist ungerade */
          {                  
              n = n*3 + 1;
          }
      }
  }
\end{verbatim}
%
Die Bedingung dieser Schleife ist {\tt n != 1}, dass hei�t die 
Schleife wird fortgesetzt bis  {\tt n} den Wert 1 erh�lt, was dazu
f�hrt, dass die Bedingung falsch wird.\\
Bei jeder Iteration gibt das Programm den Wert von {\tt n} aus und
pr�ft dann, ob {\tt n} gerade oder ungerade ist.
Ist es gerade, so wird der Wert von 
{\tt n} durch Zwei geteilt.  Ist {\tt n} ungerade ermittelt sich der neue Wert
von {\tt n} aus der  Formel $3n+1$.  

Nehmen wir an, der Startwert der Funktion  (das Argument welches der
Funktion  {\tt Sequence()} �bergeben wurde) ist 3. Damit ergibt sich die
folgende Sequenz:\\
3, 10, 5, 16, 8, 4, 2, 1.

Weil sich nun der Wert von {\tt n} manchmal vergr��ert und manchmal 
verringert, gibt es keinen naheliegenden Beweis, dass {\tt n} 
�berhaupt jemals den Wert  1 erreichen wird (und damit das 
Programm beendet w�rde).
F�r einige bestimmte Werte von {\tt n} l�sst sich beweisen, dass
die Schleife endlich ist. Ist zum Beispiel der Startwert von {\tt n}  
eine Zweierpotenz, so ist jedes Zwischenresultat gerade und
die auszuf�hrende Division f�hrt geradewegs zum Ergebnis 1. 
In unserem vorigen Beispiel stellen die letzten 5 Ziffern eine solche Sequenz
dar, die mit dem Wert 16 beginnt.

Wenn wir aber herausfinden wollen, ob diese Schleife f�r alle nur 
denkbaren Werte von {\tt n} endlich ist, dann stehen wir vor einer
sehr gro�en Herausforderung.
Bisher ist es jedenfalls noch niemandem gelungen dies zu 
Beweisen  {\em oder} den Gegenbeweis anzutreten!

\section{Tabellen}
\index{Tabellen}
\index{Logarithmus}
\index{Sinus}
\index{Kosinus}

Eine Sache die sich mit Hilfe von Schleifen leicht umsetzen l�sst, ist 
die Erzeugung von tabellarischen Daten. 

Bevor Menschen Computer zur Verf�gung hatten,
mussten Logarithmen, Sinus, Kosinus und andere
mathematische Funktionen von Hand berechnet werden.
Um diese Berechnungen zu vereinfachen wurden B�cher mit
Tabellen der Funktionswerte von h�ufig genutzten Funktionen
gedruckt.
Die Erstellung dieser Tabellen war langwierig und langsam
und die Resultate waren oftmals fehlerhaft.

Als schlie�lich Computer verf�gbar wurden, war die erste
Reaktion der Mathematiker, 
 ``Das ist gro�artig!  Von jetzt ab werden wir die Tabellen
 mit Hilfe des Computer berechnen und dann gibt es keine Fehler
 mehr.''  Diese Voraussage stellte sich als
richtig heraus, war aber nicht sehr vision�r, denn
kurze Zeit sp�ter waren Computer und Taschenrechner 
so weit verbreitet, dass niemand mehr die Tabellen benutzt
um Funktionswerte zu ermitteln.

Na ja, jedenfalls meistens.  
F�r einige Berechnungen benutzen selbst Computer Tabellen,
um N�herungswerte zu bestimmen. Mit diesen N�herungswerten
f�hren sie dann weitere Berechnungen durch, um das Ergebnis
zu verbessern.
Leider ist es aber auch schon vorgekommen, dass in diesen 
internen Tabellen Fehler enthalten waren. Ein bekanntes
Beispiel daf�r war der Fehler in den Tabellen des ersten Intel Pentium
Prozessors, welcher dazu f�hrte, dass manche Ergebnisse 
der Division von Flie�kommazahlen nicht korrekt waren.

\index{Division!Flie�kommazahlen}

Obwohl eine  ``Logarithmentafel'' heutzutage nicht mehr so n�tzlich
ist wie fr�her, stellt ihre Berechnung immer noch ein gutes Beispiel f�r
die Anwendung iterativer Algorithmen dar.
Das folgende Programm stellt in der linken Spalte eine Folge von
Werten und in der rechten Spalte die dazugeh�rigen Logarithmen dar.
  

\begin{verbatim}
    double x = 1.0;
    while (x < 10.0) 
    {
        printf ("%.0f\t%f\n", x ,log(x));
        x = x + 1.0;
    }
\end{verbatim}
%
Die Zeichenfolge \verb+\t+ steht dabei f�r das {\bf Tab}-Zeichen.
Die Zeichenfolge \verb+\n+ repr�sentiert das Zeichen f�r den Zeilenumbruch (engl: \textit{newline}).
Diese Zeichenfolgen sind sogenannte Ersetzungszeichen f�r 
Zeichen aus dem ASCII-Zeichensatz, die sich nicht direkt darstellen lassen.
Sie k�nnen an jeder beliebigen Stelle in einem String stehen -- 
in unserem Beispiel sind sie die einzigen Zeichen im Formatierungsstring.

Das {\bf Tab}-Zeichen veranlasst den Cursor um eine bestimmte Anzahl von
Zeichen nach rechts zu r�cken, bis der n�chste {\bf Tab Stop} erreicht ist.
Normalerweise betr�gt dieser Abstand 8 Zeichen. Wie wir gleich sehen werden,
sind Tabulatoren sehr n�tzlich um die Spalten einer Tabelle gleichm��ig auszurichten.
Ein Zeilenumbruch f�hrt dazu, dass der Cursor auf die n�chste Bildschirmzeile
bewegt wird.

Die Ausgabe des Programms sieht folgenderma�en aus:

\begin{verbatim}
    1      0.000000
    2      0.693147
    3      1.098612
    4      1.386294
    5      1.609438
    6      1.791759
    7      1.945910
    8      2.079442
    9      2.197225
\end{verbatim}
%
Wenn Ihnen diese Werte seltsam vorkommen, so erinnern Sie sich bitte
daran, dass die {\tt log}-Funktion die Basis $e$ benutzt. 
Da Zweierpotenzen in der Informatik so eine gro�e Rolle spielen,
kommt es oft vor, dass wir Logarithmen zur Basis 2 finden wollen.
Wir k�nnen dazu folgende Formel benutzen:

\[ \log_2 x = \frac {log_e x}{log_e 2} \]
%
Wenn wir die Ausgabeanweisung wie folgt �ndern,

\begin{verbatim}
      printf ("%.0f\t%f\n", x, log(x) / log(2.0));
\end{verbatim}
%
ergibt sich:

\begin{verbatim}
    1      0.000000
    2      1.000000
    3      1.584963
    4      2.000000
    5      2.321928
    6      2.584963
    7      2.807355
    8      3.000000
    9      3.169925
\end{verbatim}
%
Wir k�nnen erkennen, dass 1, 2, 4 und 8 Zweierpotenzen sind, weil
ihre Logarithmen zur Basis 2 runde Zahlen sind.  Wenn wir 
die Logarithmen weiterer Zweierpotenzen ermitteln wollen, k�nnen
wir das Programm folgenderma�en ver�ndern:

\begin{verbatim}
    double x = 1.0;
    while (x < 100.0) 
    {
        printf ("%.0f\t%.0f\n", x, log(x) / log(2.0));
        x = x * 2.0;
    }
\end{verbatim}
%
Statt bei jedem Schleifendurchlauf einen festen Betrag zu {\tt x} 
hinzuzuaddieren, was zu einer {\bf arithmetischen Folge} f�hrt,
multiplizieren wir {\tt x} mit einem Wert, woraus eine
 {\bf geometrischen Folge} resultiert.
 
 \newpage
Das Resultat ist:

\begin{verbatim}
    1      0
    2      1
    4      2
    8      3
    16     4
    32     5
    64     6
\end{verbatim}
%
Auffallend ist die exakte Ausrichtung der Spalten auch bei gr��eren Werten.
Da wir zwischen den Spalten tab-Zeichen verwenden, h�ngt
die Position der zweiten Spalte nicht von der Anzahl der Ziffern
in der ersten Zeile ab.

Logarithmentafeln m�gen heutzutage nicht mehr sehr n�tzlich sein.
Die Kenntnis der Zweierpotenzen ist allerdings f�r Informatiker und
Elektroniker nach wie vor extrem wichtig!
Modifizieren Sie daher das Programm, so dass es alle Zweierpotenzen
bis zum Wert 65536 (das ist $2^{16}$) ausgibt. 
Drucken Sie das Ergebnis aus und pr�gen Sie sich die Werte ein.

\section{Zweidimensionale Tabellen}
\label{Two-dimensional tables}
\index{Tabellen!Zweidimensional}

Eine zweidimensionale Tabelle ist eine Tabelle, bei der man
die Zeile und Spalte ausw�hlt und den Wert am Kreuzungspunkt
ausliest. Ein gutes Beispiel daf�r ist eine Multiplikationstafel.

Angenommen, wir wollen eine Multiplikationstafel f�r
alle Werte von 1 bis 6 erstellen.
Ein guter Anfang k�nnte darin bestehen, dass
wir eine einfache Schleife schreiben, welche
alle Vielfachen von 2 in einer Zeile ausgibt:

\begin{verbatim}
    int i = 1;
    while (i <= 6) 
    {
        printf("%i   ", i*2);
        i = i + 1;
    }
    printf("\n");
\end{verbatim}
%
In der ersten Zeile wird eine Variable namens {\tt i} initialisiert.
Diese Variable ist die {\bf Schleifenvariable}, die uns als
Z�hler dient. W�hrend der Ausf�hrung der Schleife erh�ht
sich der Wert von  {\tt i} von 1 bis 6. Wenn {\tt i} 
den Wert 7 erreicht, wird die Schleife abgebrochen.
Bei jedem Schleifendurchlauf geben wir den Wert von
{\tt i*2}, gefolgt von drei Leerzeichen aus. 
Indem wir in der ersten Ausgabeanweisung einfach die 
Zeichenfolge \verb+\n+  weg lassen, werden alle auszugebenden
Werte nacheinander in einer Bildschirmzeile ausgegeben.

\index{Schleifenvariable}
\index{Variable!Schleife}

Das Programm erzeugt folgende Ausgabe:

\begin{verbatim}
    2   4   6   8   10   12
\end{verbatim}
%
So weit, so gut. Der n�chste Schritt besteht darin die Funktionalit�t 
des Programms zu {\bf kapseln} und 
zu {\bf verallgemeinern}.

\section {Modularisierung und Verallgemeinerung}
%Daten-kapselung und Verallgemeinerung

\emph{Modularisierung} bedeutet, dass wir unseren
Quellcode in mehrere separate Programmabschnitte (Module) aufteilen.
In C werden diese Module als Funktionen realisiert. Indem wir unseren
Code nehmen und in einer Funktion verpacken, k�nnen wir von allen 
Vorteilen profitieren, die uns Funktionen bei der Programmentwicklung bieten. 
Unsere Programme werden �bersichtlicher und im Laufe der Zeit entsteht
unsere eigene kleine Funktionsbibliothek die wir immer wieder benutzen und
erweitern k�nnen.

Wir haben mit  {\tt PrintParity()} in Section~\ref{alternative} und {\tt
IsSingleDigit()} in Section~\ref{bool}  bereits zwei Beispiele f�r modularisierte Programme
kennengelernt.

\emph{Verallgemeinerung} bedeutet, dass wir aus einer spezifischen, auf ein bestimmtes
Problem bezogenen L�sung, eine allgemeinere L�sung entwickeln, die
es uns erlaubt mehrere Probleme der gleichen Klasse zu l�sen.
Zum Beispiel kann unser Programm derzeit Vielfache von 2 ausgeben.
Wir wollen unser Programm verallgemeinern, so dass wir
die Vielfachen einer beliebigen ganzen Zahl ausgeben k�nnen.

\index{Modularisierung}
\index{Verallgemeinerung}

Ich habe also unsere Schleife als eine Funktion umgeschrieben. 
Gleichzeitig haben wir die Eigenschaft der Funktion verallgemeinert,
so dass es jetzt m�glich ist, die Vielfachen von {\tt n} auszugeben:

\begin{verbatim}
    void PrintMultiples (int n)
    {
        int i = 1;
        while (i <= 6) 
        {
            printf("%i   ", i*n);
            i = i + 1;
        }
        printf("\n");
    }
\end{verbatim}
%
Um die Schleife zu modularisieren, habe ich
einfach die erste Zeile hinzugef�gt. Diese deklariert den Funktionsnamen 
und den R�ckgabewert der Funktion. Der Rest der Schleife wird
in eine Blockanweisung geschrieben.

Um die Funktion zu verallgemeinern ist es notwendig
einen Funktionsparameter hinzuzuf�gen. Dieser gibt 
den Wert an, der in der Schleife vervielfacht wird.
Im Schleifenk�rper ersetzen wir dann einfach den
Wert 2 mit dem Parameter {\tt n}.

Wenn wir diese Funktion mit dem Argument 2 aufrufen, erhalten wir
die gleiche Ausgabe wie zuvor. Mit Argument 3, sieht die Ausgabe folgenderma�en
aus:

\begin{verbatim}
    3   6   9   12   15   18
\end{verbatim}
%
und Argument 4, ergibt die folgende Ausgabe:

\begin{verbatim}
    4   8   12   16   20   24 
\end{verbatim}
%
Mittlerweile sollte klar werden, wie wir weiter vorgehen m�ssen,
um eine komplette Multiplikationstafel zu drucken.
Wir rufen einfach {\tt PrintMultiples()} mehrfach mit unterschiedlichen
Argumenten auf. Am einfachsten ist es, wenn wir dazu wieder
eine Schleife benutzen.

\begin{verbatim}
    int i = 1;
    while (i <= 6) 
    {
        PrintMultiples (i);
        i = i + 1;
    }    
\end{verbatim}
%
Es ist auffallend, wie sehr die �u�ere Schleife der 
inneren Schleife von {\tt PrintMultiples()} �hnelt.  
Ich habe nichts weiter gemacht, als den Aufruf der \texttt{printf()}-Funktion
durch den Funktionsaufruf von {\tt PrintMultiples()} zu ersetzen.

Die Ausgabe des Programms sieht folgenderma�en aus:

\begin{verbatim}
    1   2   3   4   5   6   
    2   4   6   8   10   12   
    3   6   9   12   15   18   
    4   8   12   16   20   24   
    5   10   15   20   25   30   
    6   12   18   24   30   36   
\end{verbatim}
%
Wir sehen eine (leicht unordentliche) Multiplikationstafel.  
Wenn Sie die Unordnung st�rt, dann k�nnen Sie versuchen die Leerzeichen
zwischen den Spalten der Ausgabe durch \textbf{Tab}-Zeichen zu ersetzen
und damit die Anordnung zu verbessern.

\section{Funktionen}
\index{Funktionen}

Im letzte Abschnitt habe ich von  den 
\emph{Vorteilen, die uns Funktionen bei der Programmentwicklung bieten} gesprochen.
Wahrscheinlich haben Sie sich schon gefragt, was genau ich wohl damit gemeint haben
k�nnte. 
Ich m�chte deshalb noch einmal genauer auf den Vorteil des Einsatzes von Funktionen bei
der Programmentwicklung eingehen.

Die Aufteilung eines Programms in Funktionen hat den Vorteil, dass wir jedes Modul unabh�ngig vom
restlichen Quellcode unseres Programms entwickeln und testen k�nnen.
M�glicherweise k�nnen wir diese Funktionen auch sp�ter in anderen
Projekten einfach weiterverwenden. Au�erdem sind modularisierte Programme
viel leichter zu durchschauen, als wenn wir alle Anweisungen einfach hintereinander
in die {\tt main()}-Funktion schreiben w�rden:

\begin{itemize}

\item Indem wir einer konzeptionell zusammengeh�renden Folge von Anweisungen einen 
Namen geben, machen wir unser Programm einfacher lesbar. Gleichzeitig wird die 
Fehlersuche einfacher.

\item Wenn wir ein langes Programm in Teile zerlegen,
k�nnen wir diese Teile einzeln entwickeln und testen. Anschlie�end
k�nnen wir die Funktionen wieder zu einem funktionsf�higen Programm
zusammensetzen.

\item Funktionen erlauben den einfachen Einsatz von rekursiver und iterativer Programmierung.

\item Gut konstruierte Funktionen k�nnen in vielen k�nftigen Programmen
weiter verwendet werden (die Bibliotheksfunktionen von C sind solche Funktionen).
Nachdem wir eine Funktion geschrieben und vorhandene Fehler
entfernt haben, k�nnen wir sie einfach immer wieder verwenden.

\end{itemize}

\section{Noch mehr Modularisierung}
\index{Modularisierung}
\index{Programmentwicklung!Modularisierung}

%To demonstrate encapsulation again, I'll take the code
%from the previous section and wrap it up in a function:
Um ein weiteres Beispiel f�r die Modularisierung von 
Programmen zu geben, werde ich jetzt den Programmcode
aus dem letzten Beispiel nehmen und in einer Funktion
kapseln:

\begin{verbatim}
    void PrintMultTable (void) 
    {
        int i = 1;
        while (i <= 6) 
        {
            PrintMultiples (i);
            i = i + 1;
        }
    }
\end{verbatim}
%
Der Prozess, den ich hier demonstriere, ist eine ziemlich 
verbreitete Entwicklungsstrategie. 
Wir entwickeln unser Programm schrittweise, indem wir
Progammzeilen zu unserer {\tt main()}-Funktion oder einer anderen
Funktion hinzuf�gen. Wenn wir ein lauff�higes Programm 
erstellt haben, versuchen wir den Code zu extrahieren und
in einer eigenen Funktion unterzubringen.

Nicht immer wissen wir schon vor der Erstellung unseres
Programms genau, wie wir dieses in einzelne Module
strukturieren k�nnen. 
Mit dem gerade vorgestellten Ansatz k�nnen wir die 
Struktur unseres Programms entwerfen, w�hrend
wir programmieren.
Nat�rlich k�nnen wir die Funktionen auch schon vorher 
festlegen, aber manchmal ist das einfach nicht m�glich.

\section{Lokale Variablen}

Haben Sie sich schon gefragt, wie es m�glich ist, dass ich die 
gleiche Variable {\tt i} in beiden Funktionen {\tt PrintMultiples()} und {\tt
PrintMultTable()} verwenden kann? \\
Hatte ich nicht gesagt, das man eine Variable nur einmal deklarieren
darf?\\
Und f�hrt es nicht zu Problemen, wenn eine der Funktionen
den Wert der Variablen ver�ndert?

Die Antwort der letzten beiden Fragen lautet  ``nein,'' weil das {\tt i} in {\tt
PrintMultiples()} und das {\tt i} in {\tt PrintMultTable()} 
{\em nicht die selbe Variable sind}.  Sie haben den selben Namen, aber
sie verweisen nicht auf die gleiche Speicherstelle. Somit ist klar, dass
wenn wir den Wert der einen Variable �ndern, hat das keine Auswirkung
auf den Wert der anderen Variablen.

\index{Lokale Variablen}
\index{Variablen!lokale}

Erinnern wir uns: Variablen, die innerhalb einer Funktion
deklariert werden, sind so genannte \emph{lokale Variablen}.  
Es ist nicht m�glich auf eine lokale Variable von au�erhalb ihrer
``Heimatfunktion'' zuzugreifen und wir k�nnen mehrere
Variablen mit dem gleichen Namen haben, solange sie sich 
in unterschiedlichen Funktionen befinden.
% cannot access a local variable from outside its
%``home'' function, and you are free to have multiple variables with
%the same name, as long as they are not in the same function.

Das Stackdiagramm f�r dieses Programm macht es v�llig klar,
dass  die zwei Variablen mit Namen {\tt i} an unterschiedlichen
Stellen im Speicher liegen.
Sie k�nnen unterschiedliche Werte besitzen und wenn wir eine
der Variablen �ndern hat das keine Auswirkungen auf die andere.

\index{Stackdiagramm}
\index{Diagramm!Stack}

\vspace{0.1in}
\centerline{\epsfig{figure=figs/stack4.pdf,width=6.5cm}}
\vspace{0.1in}
%
Der Wert des Parameters {\tt n} in der Funktion
{\tt PrintMultiples()} ist dabei identisch mit dem Wert 
von {\tt i} in {\tt PrintMultTable()}. 
Der Wert von {\tt i} in {\tt PrintMultiples()} l�uft von
1 bis 6.  In unserem Diagramm, steht der Wert derzeit bei 3.
Beim n�chsten Durchlauf der Schleife wird der Wert 4 sein.

Es ist oft besser f�r unterschiedlichen Funktionen  
unterschiedliche Variablennamen zu verwenden, um Verwechslungen
zu vermeiden.
Wenn wir uns an die Richtlinien f�r die Verwendung von Variablen- 
und Funktionsnamen halten (siehe Anhang \ref{Conventions for names})
und \emph{sprechende Bezeichner} w�hlen, sollten wir damit
keine Probleme haben.

\index{Sprechende Bezeichner}

Es existieren aber auch gute Gr�nde daf�r stets die gleichen
Namen zu nutzen. So hat es sich zum Beispiel eingeb�rgert {\tt i}, {\tt j} 
und {\tt k} f�r die Bezeichnung von Schleifenvariablen zu nutzen.
Wenn jetzt  unseren Funktionen andere Namen f�r
Schleifenvariablen verwenden, kann es dazu f�hren, dass unsere
Programme schwerer lesbar werden.

\index{Schleifenvariable}
\index{Variable!Schleife}

%%
\section{Noch mehr Verallgemeinerung}
\label{More generalization}
\index{Verallgemeinerung}

Wir k�nnen unser Programm aber noch weiter verallgemeinern.
Stellen wir uns vor, wir brauchen ein Programm, welches eine
Multiplikationstafel von beliebiger Gr��e und nicht nur 
im Format 6x6 ausgibt.  

Um die L�nge der Tabelle anzugeben, k�nnen wir einen Parameter
zur Funktion {\tt PrintMultTable()} hinzuf�gen:

\begin{verbatim}
    void PrintMultTable (int high) 
    {
        int i = 1;
        while (i <= high) 
        {
            PrintMultiples (i);
            i = i + 1;
        }
    }
\end{verbatim}
%
Ich haben den Wert 6 mit dem Parameter {\tt high} ersetzt.  
Wenn ich jetzt {\tt PrintMultTable()} mit dem Argument 7 aufrufe, 
erhalte ich:

\begin{verbatim}
    1   2   3   4   5   6   
    2   4   6   8   10   12   
    3   6   9   12   15   18   
    4   8   12   16   20   24   
    5   10   15   20   25   30   
    6   12   18   24   30   36   
    7   14   21   28   35   42   
\end{verbatim}
%
Das sieht auf dem ersten Blick schon ganz gut aus,
allerdings m�chte ich, dass meine Multiplikationstafel
ausbalanciert ist, die Anzahl der Spalten und Zeilen �bereinstimmt.
%which is fine, except that I probably want the table to
%be square (same number of rows and columns), which means
%I have to add another parameter to {\tt PrintMultiples()},
%to specify how many columns the table should have.
%allerdings sollte es auch m�glich sein
%die Anzahl der Vielfachen in der Multiplikationstafel 
%festlegen zu k�nnen.
Das kann ich erreichen, indem ich einen weiteren Parameter 
zu {\tt PrintMultiples()} hinzuf�ge, welcher angibt wie
viele Spalten unsere Tabelle haben sollte.

%Just to be annoying, I will also call this parameter {\tt high},
Nur um l�stig zu sein m�chte ich noch einmal demonstrieren,
dass auch die Parameter unterschiedlicher Funktionen den gleichen
Namen besitzen d�rfen (so wie lokale Variablen auch):

\begin{verbatim}
    void PrintMultiples (int n, int high) 
    {
        int i = 1;
        while (i <= high) 
        {
            printf ("%i    ", n*i);
            i = i + 1;
        }    
        printf ("\n");
    }

    void PrintMultTable (int high) 
    {
        int i = 1;
        while (i <= high) 
        {
            PrintMultiples (i, high);
            i = i + 1;
        }
    }
\end{verbatim}
%
Wenn ich einen neuen Parameter zu einer Funktion hinzuf�ge, muss
ich die erste Zeile der Funktion anpassen und 
die Stelle in der Funktion �ndern, wo dieser Parameter verwendet werden 
soll (in der Schleifenbedingung wird der Wert 6  durch {\tt high} ersetzt). 
Weiterhin muss ich nat�rlich auch den Funktionsaufruf in {\tt PrintMultTable()}
anpassen.

Wie erwartet erzeugt unser Programm jetzt eine 7x7 Tabelle:

\begin{verbatim}
    1   2   3   4   5   6   7   
    2   4   6   8   10   12   14   
    3   6   9   12   15   18   21   
    4   8   12   16   20   24   28   
    5   10   15   20   25   30   35   
    6   12   18   24   30   36   42   
    7   14   21   28   35   42   49
\end{verbatim}
%
Wenn wir eine Funktion verallgemeinern, finden wir oft, dass die
neue Funktion Eigenschaften aufweist, die wir so nicht unbedingt beabsichtigt
haben.

Zum Beispiel ist unsere Multiplikationstafel symmetrisch, weil
 $ab = ba$. Daraus folgt, dass fast alle Eintr�ge in der Tabelle
doppelt auftreten.
Wir k�nnten jetzt auf die Idee kommen, Druckkosten zu sparen,
indem wir nur eine H�lfte der Tabelle drucken.
Um das zu tun brauchen wir nur eine Zeile in {\tt PrintMultTable()} zu 
�ndern.
Anstatt 

\begin{verbatim}
      PrintMultiples (i, high);
\end{verbatim}
%
k�nnen wir

\begin{verbatim}
      PrintMultiples (i, i);
\end{verbatim}
%
schreiben und erhalten das folgende Ergebnis:

\begin{verbatim}
    1   
    2   4   
    3   6   9   
    4   8   12   16   
    5   10   15   20   25   
    6   12   18   24   30   36   
    7   14   21   28   35   42   49  
\end{verbatim}
%
\index{Law of Unintended Consequences}
Ich �berlasse es jetzt jedem selbst herauszufinden, wie diese �nderung funktioniert. \\
Allerdings geht uns durch die �nderung die urspr�ngliche Definition der zweidimensionalen 
Tabelle verloren (siehe Abschnitt \ref{Two-dimensional tables}).
Solche unbeabsichtigten Beeinflussungen kommen gar nicht so selten vor und man bezeichnet 
sie auch als das \emph{Gesetz der unbeabsichtigten Folgen} (engl: \textit{Law of Unintended Consequences}).
Es ist daher wichtig vor der Programmierung klare Anforderungen an das Programm zu definieren
und diese durch Tests sicher nachzuweisen.

\section{Glossar}
%(engl: \emph{})
\begin{description}

\item[Schleife (engl: \emph{loop}):]  Eine Anweisung oder Anweisungsblock, der mehrfach
ausgef�hrt wird solange eine Bedingung \emph{wahr} ist, oder bis irgendeine andere Bedingung 
erf�llt ist.


\item[Endlosschleife (engl: \emph{infinite loop}):]  Eine Schleife, deren Bedingung immer \emph{wahr} ist.

\item[Schleifenk�rper (engl: \emph{body}):]  Alle Anweisungen die innerhalb einer Schleife ausgef�hrt werden.

\item[Schleifendurchlauf (engl: \emph{iteration}):]  Ein Durchlauf %(Ausf�hrung) 
der Anweisungen des Schleifenk�rpers.
Dies schlie�t die Auswertung der Bedingung der Schleife mit ein.

\item[Tab (engl: \emph{tab}):] Ein spezielles Zeichen aus dem ASCII-Zeichensatz, geschriebne als \verb+\t+ in C,
welches den Cursor an die n�chste Tab-Stop Position auf der aktuellen Zeile bewegt.

\item[Modularisierung (engl: \emph{modularisation/encapsulation}):]  Die Zerlegung eines gro�en, komplexen
Programms in einzelne, unabh�ngige Komponenten (wie z.B. Funktionen). Die Komponenten k�nnen durch
die Verwendung lokaler Variablen voneinander isoliert werden. 

\item[Lokale Variable (engl: \emph{local variable}):]  Eine Variable, welche innerhalb einer Funktion
deklariert wird und die nur innerhalb dieser Funktion existiert. Auf lokale Variablen kann 
durch andere Funktionen nicht zugegriffen werden. Gleichnamige lokale Variablen in unterschiedlichen
Funktionen sind voneinander unabh�ngig.

\item[Verallgemeinern (engl: \emph{generalize}):]  Beschreibt den Vorgang in dem
wir eine spezifische L�sung die f�r ein eng umrissenes Problem zutrifft %(zum Beispiel beschrieben durch einen konstanten Wert)
durch ein allgemeineres Konzept ersetzt (durch den Einsatz von Variablen oder Parameter), um damit eine ganze Klasse von Aufgabenstellungen zu bearbeiten. 
Die Verallgemeinerung erh�ht die N�tzlichkeit unseres Programms, weil es sich leichter weiterverwenden 
l�sst und f�r einen breiteren Einsatzbereich verwendet werden kann.

\item[Softwareentwicklungsprozess (engl: \emph{software development process}):]  
Die Vorgehensweise die es uns erm�glicht von einer Idee zu einem funktionsf�higen
Programm zu gelangen. In diesem Kapitel habe ich einen Ansatz vorgestellt,
bei dem man ausgehend von einfachen Programmen zur L�sung spezieller
Problem durch Verallgemeinerung und Modularisierung zu umfassenden 
L�sungen gelangt. \\
Ein Teilbereich der Informatik, das Software Engineering befasst
sich mit verschiedenen Methoden der Programmentwicklung.

\index{Schleife}
\index{Endlosschleife}
\index{Schleifenk�rper}
\index{Tab}
\index{Schleife!Endlos}
\index{Iteration}
\index{Modularisierung}
\index{generalization}
\index{Lokale Variable}
\index{Variable!Lokal}
\index{Programmntwicklung}

\end{description}

\section{�bungsaufgaben}
\setcounter{exercisenum}{0}

\ifthenelse {\boolean{German}}{ 
\begin{exercise}\label{infloop}
%changed the condition on the loop so that it will terminate
%(was this *supposed* to be an infinite loop?)
\begin{verbatim}
    void Loop(int n) 
    {
        int i = n;
        while (i > 1) 
        {
            printf ("%i\n",i);
            if (i%2 == 0) 
            {
                i = i/2;
            } 
            else 
            {
                i = i+1;
            }
        }
    }

    int main (void) 
    {
        Loop(10);
        return EXIT_SUCCESS;
    }
\end{verbatim}
%
\begin{enumerate}

\item Zeichnen Sie eine Tabelle welche die Werte der Variablen {\tt i}
und {\tt n} w�hrend der Ausf�hrung der Funktion {\tt Loop()} zeigen. 
Die Tabelle sollte eine Spalte f�r jede Variable und eine Zeile f�r
jede Iteration der {\tt while}-Schleife enthalten.

\item Was gibt dieses Programm aus?  

\end{enumerate}
\end{exercise}

%String Beispiel

%\begin{exercise}
%\begin{enumerate}

%\item Encapsulate the following code fragment, transforming it
%into a function that takes a String as an argument and that
%returns (and doesn't print) the final value of {\tt count}.

%\item In a sentence, describe abstractly what the resulting
%function does.

%\item Assuming that you have already generalized this
%function so that it works on any String, what else could you do to
%generalize it more?

%\end{enumerate}

%\begin{verbatim}
%    char s[] = "((3 + 7) * 2)";
%    int len = strlen(s);

%    int i = 0;
%    int count = 0;

%    while (i < len) 
%    {
%        char c = s[i]);

%        if (c == '(') 
%        {
%           count = count + 1;
%        } 
%        else if (c == ')') 
%        {
%           count = count - 1;
%        }
%        i = i + 1;
%    }

%    printf ("%i\n", count);
%\end{verbatim}
%\end{exercise}


\begin{exercise}

C stellt in der mathematischen Bibliothek die Funktion {\tt pow()} 
zur Verf�gung, welche die Potenz einer reellen Zahl berechnet. 

Schreiben Sie Ihre eigene Version {\tt Power()} dieser Funktion welche zwei
Parameter:  double {\tt x} und integer {\tt n} �bernimmt und das
Resultat der Berechnung $x^n$ zur�ckliefert.  
Ihre Funktion soll die Berechnung iterativ (mit Hilfe einer Schleife) durchf�hren.
\end{exercise}

\begin{exercise} 
%\textbf{Zusatzaufgabe!}

Angenommen Sie haben eine Zahl  $a$, und Sie wollen die 
Quadratwurzel dieser Zahl ermitteln. 

Eine m�gliche Vorgehensweise besteht darin, dass Sie mit
einer ersten groben Sch�tzung, $x_0$, der Antwort beginnen
und diese Sch�tzung mit Hilfe der folgenden Formel verbessern:

\begin{equation}
x_1 = (x_0 + a/x_0) / 2
\end{equation}

Zum Beispiel, suchen wir die Quadratwurzel von 9. Wir beginnen
mit $x_0 = 6$, dann ergibt sich f�r $x_1 = (6 + 9/6) /2 = 15/4 = 3.75$,
welches n�her an der gesuchten L�sung liegt.

Wir k�nnen das Verfahren wiederholen indem wir $x_1$ benutzen um $x_2$
zu berechnen und so weiter...
In diesem Fall ergibt sich $x_2 = 3.075$ und $x_3 = 3.00091$.
Unsere Berechnung konvergiert sehr schnell hin zu der richtigen Antwort (3).

Schreiben Sie eine Funktion {\tt SquareRoot} welche ein {\tt double}
als Parameter �bernimmt und eine N�herung der Quadratwurzel des 
Parameters zur�ckliefert. 
Die Funktion soll dabei den oben beschriebenen Algorithmus benutzen 
und darf nicht die {\tt sqrt()} Funktion der {\tt math.h} Bibliothek verwenden.

Als erste, initiale N�herung sollten Sie $a/2$ verwenden.  
Ihre Funktion soll die Berechnung wiederholen, bis Sie zwei aufeinanderfolgende
N�herungen erhalten, welche um weniger als 0.0001 voneinander abweichen:
mit anderen Worten, bis der Absolutbetrag von $x_n - x_{n-1}$ geringer ist als
0.0001.  F�r die Berechnung des Absolutbetrags k�nnen Sie die 
{\tt abs()} Funktion der {\tt math.h} Bibliothek verwenden.
\end{exercise}


%\begin{exercise}
%Section~\ref{factorial} presents a recursive function
%that computes the factorial function.
%Write an iterative version of {\tt factorial}.
%\end{exercise}

%\begin{exercise}
%One way to calculate $e^x$ is to use the infinite series expansion

%\begin{equation}
%e^x = 1 + x + x^2 / 2! + x^3 / 3! + x^4 / 4! + ...
%\end{equation}

%If the loop variable is named {\tt i}, then the $i$th term is equal to
%$x^i / i!$.

%\begin{enumerate}

%\item Write a function called {\tt myexp} that adds up the first {\tt n}
%terms of the series shown above.  You can use the {\tt factorial}
%function from Section~\ref{factorial} or your iterative version.

%\item You can make this function much more efficient if you realize that
%in each iteration the numerator of the term is the same as its
%predecessor multiplied by {\tt x} and the denominator is the same as
%its predecessor multiplied by {\tt i}.  Use this observation to
%eliminate the use of {\tt Math.pow} and {\tt factorial}, and check
%that you still get the same result.

%\item Write a function called {\tt check} that takes a single parameter,
%{\tt x}, and that prints the values of {\tt x}, {\tt Math.exp(x)} and
%{\tt myexp(x)} for various values of {\tt x}.  The output should look
%something like:

%\begin{verbatim}
%1.0     2.708333333333333       2.718281828459045
%\end{verbatim}


%the next line used to use \\ in an attempt to escape the backslash character
%but this doesn't work: \\ produces a newline
%according to the LaTeX symbol list
%http://www.tex.ac.uk/tex-archive/info/symbols/comprehensive/symbols-letter.pdf
%\textbackslash is the actual way to escape a backslash
%(but it seems not to respect the \tt font)


%HINT: you can use the String {\tt "\textbackslash t"} to print a tab character
%between columns of a table.

%\item Vary the number of terms in the series (the second argument
%that {\tt check} sends to {\tt myexp}) and see the effect on
%the accuracy of the result.  Adjust this value until the estimated
%value agrees with the ``correct'' answer when {\tt x} is 1.

%\item Write a loop in {\tt main} that invokes {\tt check} with the
%values 0.1, 1.0, 10.0, and 100.0.  How does the accuracy of the
%result vary as {\tt x} varies?  Compare the number of digits of
%agreement rather than the difference between the actual and
%estimated values.

%\item Add a loop in {\tt main} that checks {\tt myexp} with the values
%-0.1, -1.0, -10.0, and -100.0.  Comment on the accuracy.

%\end{enumerate}
%\end{exercise}

%
%\begin{exercise}
%One way to evaluate $e^{-x^2}$ is to use the infinite series expansion

%\begin{equation}
%e^{-x^2} = 1 - 2x + 3x^2/2! - 4x^3/3! + 5x^4/4! - ...
%\end{equation}

%In other words, we need to add up a series of terms where the $i$th
%term is equal to $(-1)^i(i+1) x^i / i!$.  Write a function named {\tt gauss}
%that takes {\tt x} and {\tt n} as arguments and that returns the sum
%of the first {\tt n} terms of the series.  You should not use {\tt
%factorial} or {\tt pow}.



%%changed the condition on the loop so that it will terminate
%%(was this *supposed* to be an infinite loop?)
%\begin{verbatim}
%    void loop (int n) 
%    {
%        int i = n;
%        while (i > 1) 
%        {
%            printf ("%i\n",i);
%            if (i%2 == 0) 
%            {
%                i = i/2;
%            } 
%            else 
%            {
%                i = i+1;
%            }
%        }
%    }

%    int main (void) 
%    {
%        Loop (10);
%    }
%\end{verbatim}
%%
%\begin{enumerate}

%\item Draw a table that shows the value of the variables {\tt i}
%and {\tt n} during the execution of {\tt loop}.  The table should
%contain one column for each variable and one line for each
%iteration.

%\item What is the output of this program?  

%\end{enumerate}
%\end{exercise}

%
%\begin{exercise}
%\begin{enumerate}

%\item Encapsulate the following code fragment, transforming it
%into a function that takes a String as an argument and that
%returns (and doesn't print) the final value of {\tt count}.

%\item In a sentence, describe abstractly what the resulting
%function does.

%\item Assuming that you have already generalized this
%function so that it works on any String, what else could you do to
%generalize it more?

%\end{enumerate}

%\begin{verbatim}
%    char s[] = "((3 + 7) * 2)";
%    int len = strlen(s);

%    int i = 0;
%    int count = 0;

%    while (i < len) 
%    {
%        char c = s[i]);

%        if (c == '(') 
%        {
%           count = count + 1;
%        } 
%        else if (c == ')') 
%        {
%           count = count - 1;
%        }
%        i = i + 1;
%    }

%    printf ("%i\n", count);
%\end{verbatim}
%\end{exercise}

%
%\begin{exercise}
%Let's say you are given a number, $a$, and you want to find
%its square root.  One way to do that is to start with a very
%rough guess about the answer, $x_0$, and then improve
%the guess using the following formula:

%\begin{equation}
%x_1 = (x_0 + a/x_0) / 2
%\end{equation}

%For example, if we want to find the square root of 9, and
%we start with $x_0 = 6$, then $x_1 = (6 + 9/6) /2 = 15/4 = 3.75$,
%which is closer.

%We can repeat the procedure, using $x_1$ to calculate $x_2$,
%and so on.  In this case, $x_2 = 3.075$ and $x_3 = 3.00091$.
%So that is converging very quickly on the right answer (which
%is 3).

%Write a function called {\tt squareRoot} that takes a {\tt double}
%as a parameter and that returns an approximation of the square
%root of the parameter, using this algorithm.  You may not use
%the {\tt sqrt()} function from the {\tt math.h} library.

%As your initial guess, you should use $a/2$.  Your function should
%iterate until it gets two consecutive estimates that differ by
%less than 0.0001; in other words, until the absolute value of
%$x_n - x_{n-1}$ is less than 0.0001.  You can use the built-in
%{\tt abs()} function to calculate the absolute value.
%\end{exercise}

%
%\begin{exercise}
%In Exercise~\ref{ex.power} we wrote a recursive version of {\tt
%power}, which takes a double {\tt x} and an integer {\tt n} and
%returns $x^n$.  Now write an iterative function to perform the same
%calculation.
%\end{exercise}

%\begin{exercise}
%Section~\ref{factorial} presents a recursive function
%that computes the factorial function.
%Write an iterative version of {\tt factorial}.
%\end{exercise}

%\begin{exercise}
%One way to calculate $e^x$ is to use the infinite series expansion

%\begin{equation}
%e^x = 1 + x + x^2 / 2! + x^3 / 3! + x^4 / 4! + ...
%\end{equation}

%If the loop variable is named {\tt i}, then the $i$th term is equal to
%$x^i / i!$.

%\begin{enumerate}

%\item Write a function called {\tt myexp} that adds up the first {\tt n}
%terms of the series shown above.  You can use the {\tt factorial}
%function from Section~\ref{factorial} or your iterative version.

%\item You can make this function much more efficient if you realize that
%in each iteration the numerator of the term is the same as its
%predecessor multiplied by {\tt x} and the denominator is the same as
%its predecessor multiplied by {\tt i}.  Use this observation to
%eliminate the use of {\tt Math.pow} and {\tt factorial}, and check
%that you still get the same result.

%\item Write a function called {\tt check} that takes a single parameter,
%{\tt x}, and that prints the values of {\tt x}, {\tt Math.exp(x)} and
%{\tt myexp(x)} for various values of {\tt x}.  The output should look
%something like:

%\begin{verbatim}
%1.0     2.708333333333333       2.718281828459045
%\end{verbatim}

%%the next line used to use \\ in an attempt to escape the backslash character
%%but this doesn't work: \\ produces a newline
%%according to the LaTeX symbol list
%%http://www.tex.ac.uk/tex-archive/info/symbols/comprehensive/symbols-letter.pdf
%%\textbackslash is the actual way to escape a backslash
%%(but it seems not to respect the \tt font)
%HINT: you can use the String {\tt "\textbackslash t"} to print a tab character
%between columns of a table.

%\item Vary the number of terms in the series (the second argument
%that {\tt check} sends to {\tt myexp}) and see the effect on
%the accuracy of the result.  Adjust this value until the estimated
%value agrees with the ``correct'' answer when {\tt x} is 1.

%\item Write a loop in {\tt main} that invokes {\tt check} with the
%values 0.1, 1.0, 10.0, and 100.0.  How does the accuracy of the
%result vary as {\tt x} varies?  Compare the number of digits of
%agreement rather than the difference between the actual and
%estimated values.

%\item Add a loop in {\tt main} that checks {\tt myexp} with the values
%-0.1, -1.0, -10.0, and -100.0.  Comment on the accuracy.

%\end{enumerate}
%\end{exercise}

%
%\begin{exercise}
%One way to evaluate $e^{-x^2}$ is to use the infinite series expansion

%\begin{equation}
%e^{-x^2} = 1 - 2x + 3x^2/2! - 4x^3/3! + 5x^4/4! - ...
%\end{equation}

%In other words, we need to add up a series of terms where the $i$th
%term is equal to $(-1)^i(i+1) x^i / i!$.  Write a function named {\tt gauss}
%that takes {\tt x} and {\tt n} as arguments and that returns the sum
%of the first {\tt n} terms of the series.  You should not use {\tt
%factorial} or {\tt pow}.

%\end{exercise}
}
{\input{exercises/Exercise_6_english}}



%!TEX root = Main_german.tex

% LaTeX source for textbook ``How to think like a computer scientist''
% Copyright (C) 1999  Allen B. Downey

% This LaTeX source is free software; you can redistribute it and/or
% modify it under the terms of the GNU General Public License as
% published by the Free Software Foundation (version 2).

% This LaTeX source is distributed in the hope that it will be useful,
% but WITHOUT ANY WARRANTY; without even the implied warranty of
% MERCHANTABILITY or FITNESS FOR A PARTICULAR PURPOSE.  See the GNU
% General Public License for more details.

% Compiling this LaTeX source has the effect of generating
% a device-independent representation of a textbook, which
% can be converted to other formats and printed.  All intermediate
% representations (including DVI and Postscript), and all printed
% copies of the textbook are also covered by the GNU General
% Public License.

% This distribution includes a file named COPYING that contains the text
% of the GNU General Public License.  If it is missing, you can obtain
% it from www.gnu.org or by writing to the Free Software Foundation,
% Inc., 59 Temple Place - Suite 330, Boston, MA 02111-1307, USA.

\chapter{Arrays}
\label{arrays}
\index{Array}
\index{Typ!Array}

Ein {\bf Array} steht f�r eine Menge von Werten, wobei jeder Wert durch eine
Zahl (genannt Index) identifiziert und referenziert wird.
Der Vorteil von Arrays besteht darin, dass wir eine (m�glicherweise)
sehr gro�e Anzahl von Werten unter dem gleichen Namen ansprechen
k�nnen. 

Nehmen wir an, wir wollen in unserem Programm 
die Tagestemperaturen der letzten 10 Jahre auswerten -- wenn wir 
die Werte in einzelnen Variablen speichern wollten, m�ssten wir daf�r
mindestens 3652 Variablen anlegen. Wenn wir Arrays benutzen brauchen wir nur eins.
 

Wenn wir ein Array deklarieren, m�ssen wir angeben wie viele Elemente in dem
Array gespeichert werden sollen. Ansonsten sieht die Deklaration �hnlich wie f�r andere
Variablentypen aus:

\begin{verbatim}
    int c[4];
    double values[10];
\end{verbatim}


%
Syntaktisch sehen Array-Variablen wie andere C Variablen aus, au�er dass ihnen mit {\tt [ANZAHL\_DER\_ELEMENTE]}  
die  Anzahl der Elemente  des Arrays in eckigen Klammern folgt. 
Die erste Zeile in unserem Beispiel, {\tt int c[4];} ist vom Typ  ``Array von Ganzen Zahlen'' und erzeugt ein 
Array mit vier {\tt int} Werten. Das Array hat den Namen {\tt c}.
Die zweite Zeile, {\tt double values[10];} hat den Typ  ``Array von Flie�kommazahlen'' und
erzeugt ein Array mit zehn Werten. 


%The number
%of elements in {\tt values} depends on {\tt size}. You can use any
%integer expression to determine the size of an array.
%!!! Not in C
%this would be dynamic arrays, that can not be initialised at definition time!!!

%

In C k�nnen wir die Elemente eines Array w�hrend der Deklaration 
sofort initialisieren.
Die Werte der einzelnen Elemente werden dabei in geschwungenen Klammern 
{\tt \{\}} und durch Komma getrennt angegeben:

\begin{verbatim}
    int c[4] = {0, 0, 0, 0};
\end{verbatim}

Diese Anweisung deklariert ein Array mit vier Elementen und initialisiert
alle Werte des Arrays mit Null.
Die folgende Abbildung zeigt wie Arrays in Zustandsdiagrammen dargestellt werden:

%\myfig{figure=figs/array.eps}

\unitlength0.1cm

\begin{picture}(40,10)(-30,-5)
%\put(-4,1.5){{\Large \texttt{c}}}
%\put(0,1.5){\framebox(2,2)}
%\thicklines
%\put(2,2.5){\vector(1,0){8}}
%\thinlines
\put(5,1.5){{\Large \texttt{c}}}
\put(10,0){\framebox(7,5){\textbf{\textsf{0}}}}
\put(17,0){\framebox(7,5){\textbf{\textsf{0}}}}
\put(24,0){\framebox(7,5){\textbf{\textsf{0}}}}
\put(31,0){\framebox(7,5){\textbf{\textsf{0}}}}

\put(10.5,-4){{\scriptsize \texttt{c[0]}}}
\put(17.5,-4){{\scriptsize \texttt{c[1]}}}
\put(24.5,-4){{\scriptsize \texttt{c[2]}}}
\put(31.5,-4){{\scriptsize \texttt{c[3]}}}

\end{picture}

\index{Zustandsdiagramm}

Die fett gedruckten Zahlen innerhalb der K�stchen sind die Werte der {\bf Elemente} des
Arrays. Die kleinen Zahlen au�erhalb der K�stchen geben die Indizes 
an, �ber die wir auf die einzelnen Elemente des Arrays zugreifen k�nnen.
Wenn wir ein neues Array deklarieren, ohne es zu
initialisieren, dann enthalten die Elemente des Arrays 
beliebige, nicht vorher bestimmbare Werte. 
Wir m�ssen das Array mit sinnvollen Werten initialisieren,
bevor wir sie verwenden k�nnen.

Wenn wir bei der Deklaration bereits alle Werte angeben, so k�nnen 
wir auf die Angabe der Gr��e des Arrays verzichten. Der Compiler ermittelt
die ben�tigte Anzahl selbst: 

\begin{verbatim}
    int c[] = {0, 0, 0, 0};
\end{verbatim}

Die angegebene Syntax f�r die Zuweisung von Werten an das Array ist nur g�ltig
zum Zeitpunkt der Deklaration. Wenn wir sp�ter in unserem Programm
die Werte des Arrays neu setzen wollen, m�ssen wir das einzeln f�r jedes Element des Arrays
tun.


\textbf{WICHTIG:}  Ein Array kann aus Elementen beliebigen Datentyps bestehen, wie zum
Beipiel {\tt int} und {\tt double}. Es k�nnen  in einem Array aber 
immer nur Elemente gleichen Typs gespeichert werden. Wir k�nnen 
die Typen innerhalb eines Arrays nicht mischen. \hint
%and user-defined types like {\tt Point} and {\tt Time}.


%%
\section{Inkrement und Dekrement-Operatoren}
\index{Operator!inkrement}
\index{Operator!dekrement}
\index{Inkrement}
\index{Dekrement}
\index{Heraufz�hlen|see {Inkrement}}
\index{Herunterz�hlen|see {Dekrement}}

Einen Wert herauf- oder herunterzuz�hlen 
sind solche h�ufig vorkommenden Operationen in Programmen, dass C
daf�r spezielle Operatoren zur Verf�gung stellt.  
Der {\tt ++} Operator z�hlt einen {\tt int}, {\tt char} oder {\tt double} Wert um eins herauf (inkrement).
Der {\tt -}{\tt -} Operator subtrahiert (dekrementiert) den Wert um 1.  
%Neither operator works on {\tt apstring}s,
%and neither {\em should} be used on {\tt bool}s.

Es ist technisch legal eine Variable in einem Ausdruck zu verwenden
und sie gleichzeitig zu inkrementieren.
Ein Beispiel k�nnte folgenderma�en aussehen:

\begin{verbatim}
    printf ("%i\n ", i++);
\end{verbatim}
%
Wenn wir uns den Code anschauen, so ist nicht unmittelbar klar,
ob der Wert um eins erh�ht wird bevor oder nachdem er auf dem Bildschirm 
ausgegeben wird.
Weil solche Ausdr�cke verwirrend sein k�nnen, m�chte ich von ihrer
Verwendung abraten.
Ich m�chte Sie sogar noch mehr entmutigen, indem ich das Resultat nicht verraten
werde.
Wenn Sie es wirklich 
wissen wollen, k�nnen Sie es ja einfach selbst ausprobieren.

Mit Hilfe des Inkrement-Operators k�nnen wir die {\tt PrintMultTable()}-Funktion aus 
Abschnitt Section~\ref{More generalization} folgenderma�en schreiben:

\begin{verbatim}
    void PrintMultTable(int high) 
    { 
        int i = 1; 
        while (i <= high) 
        { 
            PrintMultiples(i); 
            i++; 
        } 
    }
\end{verbatim}
%
Ein weit verbreiteter Anf�ngerfehler besteht darin folgenden
Code zu schreiben:

\begin{verbatim}
    index = index++;             /* FALSCH!! */
\end{verbatim}
%
Ungl�cklicherweise ist diese Anweisung syntaktisch legal, so 
dass der Compiler uns nicht warnen wird.  
Der unangenehme Effekt dieser Anweisung besteht darin, 
dass der Wert von  {\tt index} unver�ndert gelassen wird.  
Das ist ein Fehler der in einem Programm m�glicherweise nur schwer zu finden ist.

\textbf{HINWEIS:} Entweder schreiben wir {\tt index = index + 1;}  oder  {\tt index++;}\\ 
Wir d�rfen beide Ausdr�cke  nicht miteinander mischen! 

\hint

%%
\section{Zugriff auf Elemente eines Arrays}
\index{Element}
\index{Array!Element}
\index{Array!Index}
Der {\tt []} Operator erlaubt es uns einzelne Elemente eines Arrays zu lesen und
zu schreiben.
Dazu wird innerhalb der eckigen Klammern der Index des Elements angegeben.
Hierbei m�ssen wir besonders aufpassen. Der Index z�hlt n�mlich bei Null beginnend,
das hei�t, das erste Element im Array ist {\tt c[0]}.  
Das Element {\tt c[1]} ist bereits das zweite Element des Arrays!

%\renewcommand{\marginnotevadjust}{-3em} 
\hint
Wir k�nnen den 
{\tt []} Operator in jedem beliebigen Ausdruck verwenden:


\begin{verbatim}
    c[0] = 7;
    c[1] = c[0] * 2;
    c[2]++;
    c[3] -= 60;
\end{verbatim}
%
Alle diese Zuweisungen sind erlaubt. Hier ist das Ergebnis dargestellt:


%\myfig{figure=figs/array2.eps}

\unitlength0.1cm

\begin{picture}(40,10)(-30,-5)
%\put(-11,1.5){\texttt{count}}
%\put(0,1.5){\framebox(2,2)}
%\thicklines
%\put(2,2.5){\vector(1,0){8}}
%\thinlines
\put(5,1.5){{\Large \texttt{c}}}

\put(10,0){\framebox(7,5){\textbf{\textsf{7}}}}
\put(17,0){\framebox(7,5){\textbf{\textsf{14}}}}
\put(24,0){\framebox(7,5){\textbf{\textsf{1}}}}
\put(31,0){\framebox(7,5){\textbf{\textsf{-60}}}}

\put(10.5,-4){{\scriptsize \texttt{c[0]}}}
\put(17.5,-4){{\scriptsize \texttt{c[1]}}}
\put(24.5,-4){{\scriptsize \texttt{c[2]}}}
\put(31.5,-4){{\scriptsize \texttt{c[3]}}}

\end{picture}

Aus unserem Beispiel sollte klar geworden sein, dass die vier Elemente
des Arrays �ber einen Indexwert identifiziert werden, der von 0 bis
3 reicht. In unserem Array gibt es kein Element mit dem Index 4.

Es ist trotzdem ein weit verbreiteter und h�ufiger Fehler die Grenzen eines
Arrays zu �berschreiten.
Modernere Programmiersprachen, wie zum Beispiel Java oder C\#, produzieren
eine Fehlermeldung oder brechen das Programm ab, wenn versucht wird
auf Elemente zuzugreifen, die au�erhalb der Grenzen eines Arrays liegen.
C hingegen �berpr�ft die Grenzen eines Arrays nicht, so dass ein Programm
auf Speicherstellen au�erhalb des Arrays zugreifen kann, so als w�ren diese
Elemente des Arrays. Sollte sich ihr Programm so verhalten, so ist das in
den allermeisten F�llen falsch und kann zu folgenschweren Fehlern in dem
Programm f�hren.


\index{Array!Grenzen}
\begin{quote}
{\bf Es ist daher notwendig das Sie, als der Programmierer,
darauf achten und sichergehen, dass der Programmcode ihres Programms die Grenzen der Arrays
korrekt einh�lt!}\hint

\end{quote}



\index{Laufzeitfehler|see{Run-time error}}
\index{Run-time error}
\index{Index}
\index{Ausdruck}

Wir k�nnen beliebige Ausdr�cke als Index benutzen, solange sie vom Typ 
{\tt int} sind. Meistens verwenden wir eine spezielle \emph{Schleifenvariable} um ein
Array zu indizieren:

\begin{verbatim}
    int i = 0;
    while (i < 4) 
    {
        printf ("%i\n", c[i]);
        i++;
    }
\end{verbatim}

%
Wir benutzen die {\tt while}-Schleife um den Wert der Schleifenvariable {\tt i} schrittweise
zu erh�hen.
Wenn die Schleifenvariable {\tt i} den Wert 4 erreicht, schl�gt die
Auswertung der Pr�fbedingung fehl und die Schleife wird
abgebrochen. Der Schleifenk�rper wird also nur dann
ausgef�hrt, wenn {\tt i} den Wert 0, 1, 2 und 3 hat.
\index{Schleife}
\index{Schleifenvariable}
\index{Variable!Schleife}
Bei jedem Schleifendurchlauf benutzen wir 
{\tt i} als Index f�r das Array und geben das 
{\tt i}-te Element auf dem Bildschirm aus.  

Wenn wir Arrays verwenden, werden wir also immer wieder auch
mit Schleifen arbeiten, denn es kommt oft vor, dass ein Array von ersten bis zum letzten
Element durchlaufen werden muss. 



%%
\section{Kopieren von Arrays}
\index{Array!kopieren}

Mit Arrays lassen sich eine Reihe von Programmieraufgaben 
sehr einfach l�sen. Zum Beispiel k�nnen wir auf diese
Weise sehr einfach gro�e Datenmengen speichern und weiterverarbeiten. 

Allerdings m�ssen wir uns daran gew�hnen, dass C sehr wenige Dinge 
automatisch von allein erledigt. Es ist zum Beispiel nicht m�glich allen
Elementen eines Arrays gleichzeitig einen neuen Wert zuzuweisen.
Es ist auch nicht m�glich die Werte eines Array direkt einem anderen
Array zuzuweisen selbst wenn die Anzahl und der Typ der Elemente 
beider Arrays �bereinstimmen:


\begin{verbatim}
    double a[3] = {1.0, 1.0, 1.0};
    double b[3];

    a = 0.0;     /* Wrong! */
    b = a;       /* Wrong! */
\end{verbatim}
%

Um diese Aufgaben trotzdem auszuf�hren, m�ssen wir die Werte eines
Arrays Element f�r Element bearbeiten. 
Das hei�t, wir m�ssen jedem einzelnen Element eines Arrays  einen neuen Wert
zuweisen. Wenn wir den Inhalt eines Arrays in ein anderes Array kopieren wollen,
m�ssen wir wieder elementweise die Werte kopieren und dazu verwenden wir
am Besten eine Schleife:

% 
%In order to set all of the elements of an array to some value, you must do so element by element.
%To copy the contents of one array to another, you must again do so, by copying each element from
%one array to the other.

\begin{verbatim}
    int i = 0;
    while (i < 3) 
    {
        b[i] = a[i];
        i++;
    }
\end{verbatim}

%%
\section{{\tt for} Schleifen}

Die Schleifen, die wir bisher benutzt haben, weisen einige Gemeinsamkeiten 
auf. Vor der Ausf�hrung der Schleife wird eine Schleifenvariable initialisiert, danach
wird eine Schleifenbedingung getestet, welche von der Schleifenvariable abh�ngt.
Im Schleifenk�rper wird dann die Schleifenvariable ver�ndert, indem man sie zum
Beispiel heraufz�hlt (inkrementiert).


%
%The loops we have written so far have a number of elements
%in common.  All of them start by initializing a variable;
%they have a test, or condition, that depends on that variable;
%and inside the loop they do something to that variable,
%like increment it.

\index{Schleife!for}
\index{for}
\index{Anweisung!for}

Dieser Schleifentyp ist so verbreitet dass es daf�r eine eigene
Schleifenanweisung gibt, mit der man diesen Ablauf 
k�rzer und klarer darstellen kann: die \mbox{{\tt for}-Schleife}.\\
Die Syntax der {\tt for}-Schleife sieht folgenderma�en aus:

\begin{verbatim}
    for (INITIALIZER; CONDITION; INCREMENTOR) 
    {
        BODY;
    }
\end{verbatim}
%
Diese Anweisung ist �quivalent zu:

\begin{verbatim}
    INITIALIZER;
    while (CONDITION) 
    {
        BODY;
        INCREMENTOR;
    }
\end{verbatim}
%

Ich finde die {\tt for}-Schleife �bersichtlicher, weil 
jetzt alle Anweisungen, die f�r die Steuerung der Schleife
n�tig sind, in einer Anweisung zusammengefasst sind.
Allerdings haben gerade Anf�nger teilweise Probleme die
Arbeitsweise der {\tt for}-Schleife zu verstehen. Es ist
wichtig zu wissen, wann die einzelnen Ausdr�cke der
Schleifenanweisung ausgef�hrt werden!

Der INITIALIZER Ausdruck wird in der Schleife nur \emph{\textbf{ein
einziges Mal}} ganz am Anfang ausgef�hrt und damit die 
Schleifenvariable initialisiert. 

Die Schleifenbedingung CONDITION
wird \emph{\textbf{vor jedem Durchlauf}} durch die Schleife gepr�ft. Wenn
die Bedingung \emph{wahr} ist, wird die Schleife durchlaufen,
wenn sie \emph{falsch} ist, wird die Schleife abgebrochen.
Da die Bedingung gepr�ft wird, bevor die Schleife ein erstes
Mal ausgef�hrt wird, kann es passieren, dass die Schleife �berhaupt
nicht ausgef�hrt wird, weil bereits vor der ersten Ausf�hrung
der Test der Schleifenbedingung fehlschl�gt.

Der INCREMENTOR Ausdruck wird \emph{\textbf{nach jedem Durchlauf}} 
durch den Schleifenk�rper einmal ausgef�hrt. In den meisten F�llen
wird dann die Schleifenvariable heraufgez�hlt (inkrementiert).
Es ist aber genauso gut m�glich eine Schleife 'r�ckw�rts' zu durchlaufen.
In diesem Fall m�ssten wir die Schleifenvariable herabz�hlen.

So ist zum Beispiel:

\begin{verbatim}
    int i;
    for (i = 0; i < 4; i++) 
    {
        printf("%i\n", c[i]);
    }
\end{verbatim}
%
�quivalent zu:

\begin{verbatim}
    int i = 0;
    while (i < 4) 
    {
        printf("%i\n", c[i]);
        i++;
    }
\end{verbatim}

%%
\section{Die L�nge eines Arrays}
\label{Array length}
\index{L�nge!Array}
\index{Array!L�nge}
\index{sizeof()}
\index{Operator!sizeof}

C bietet uns keinen bequemen Weg an, mit dessen Hilfe wir
die Gr��e eines Arrays in unserem Programm ermitteln k�nnen.
Die L�nge des Arrays ist zum Beispiel dann wichtig, wenn wir
mit Hilfe einer Schleife das Array elementweise durchgehen und 
die Schleife nach dem letzten Element beenden wollen. 

Um die L�nge des Arrays zu ermitteln k�nnten wir den {\tt sizeof()}
Operator benutzen. Dieser liefert uns allerdings die Gr��e der Datentypen in Bytes.
Die meisten Datentypen in C benutzen mehrere Bytes um einen
Wert zu speichern und je nach nach verwendeter Rechnerarchitektur
k�nnen diese Werte sogar f�r den gleichen Datentyp unterschiedlich sein.

Um jetzt die korrekte Gr��e, das hei�t die Anzahl der Elemente, eines
Arrays zu ermitteln m�ssen wir die ermittelte Arraygr��e noch durch die 
Anzahl der Bytes teilen die ein einzelnes Element eines Arrays belegt.
Dazu benutzen wir sinnvollerweise das Element mit dem Index 0.


\begin{verbatim}
    sizeof(ARRAY)/sizeof(ARRAY_ELEMENT)
\end{verbatim}

Es ist eine gute Idee diesen Wert und keine Konstante als die obere Grenze 
f�r die Schleifenvariable zu benutzen. Auf diese Weise 
k�nnen wir sicherstellen, dass wenn sich irgendwann einmal 
die Gr��e eines Arrays �ndern sollte, wir nicht das gesamte Programm
durchforsten m�ssen um alle Schleifen zu �ndern. Die Schleife
f�r jede Arraygr��e korrekt arbeiten:

\begin{verbatim}
    int i, length;
    length = sizeof (c) / sizeof (c[0]);

    for (i = 0; i < length; i++) 
    {
        printf("%i\n", c[i]);
    }
\end{verbatim}
%
Besonders wichtig ist die korrekte Angabe der Schleifenbedingung.
Wir erinnern uns, der Index der Elemente eines Arrays beginnt mit
\texttt{0}.
In unserer Schleife wird die Schleifenvariable {\tt i} so lange erh�ht, bis 
sie den Wert {\tt length - 1} hat. Dies entspricht genau dem 
Index des letzten Element des Arrays. Anschlie�end
wird {\tt i} ein weiteres Mal erh�ht und hat damit den gleichen
Wert wie {\tt length}. Die Bedingung unserer Schleife ist damit \emph{falsch}
und die Schleife wird abgebrochen. W�rde die Schleife an dieser 
Stelle weiter machen, w�rde das Programm auf Speicherbereiche zugegreifen die nicht
mehr Teil des Arrays sind und unser Programm w�re fehlerhaft.
%
%The last time the body of the loop gets executed, the value of
%is , which is the index of the last element.  When
%{\tt i} is equal to, the condition fails and the body
%is not executed, which is a good thing, since it would access a
%memory location that is not part of the array.

\section{Zufallszahlen}
\label{Random numbers}
\label{random}
\label{pseudorandom}
\index{Zufallszahlen}
\index{deterministisch}
\index{pseudozuf�llig}

Die meisten Computerprogramme verhalten sich bei
jeder Ausf�hrung des Programms stets gleich. Man nennt dieses 
Verhalten {\bf deterministisch}.  
Normalerweise ist das deterministische Verhalten eine gute
Sache, denn gleiche Berechnungen sollen auch immer  gleiche
Ergebnisse liefern.
Es gibt aber auch einige Anwendungsgebiete wo sich das 
Verhalten des Computers nicht vorhersagen lassen soll, zum
Beispiel bei Computerspielen.

Es ist ziemlich schwierig ein Computerprogramm dazu zu
bringen \emph{echte} Zufallswerte zu erzeugen. Es gibt aber eine
M�glichkeit  es so aussehen lassen, als w�rde unser Programm 
zuf�llige Werte erzeugen.
Der Computer kann so genannte  \textbf{pseudozuf�llige Zahlen} 
(engl: pseudorandom numbers) erzeugen und im Programmablauf verwenden.
Pseudozuf�llig Zahlen sind im mathematischen Sinne nicht wirklich zuf�llig,
sie haben aber �hnliche Eigenschaften und k�nnen f�r viele 
Anwendungen anstelle von echten Zufallszahlen verwendet werden.

\index{Bibliothek!stdlib.h}
\index{Header-Datei!stdlib.h}
\index{<stdlib.h>}
C stellt eine Funktion mit dem Namen {\tt rand()} bereit, welche pseudozuf�llige
Zahlen erzeugt. Die Funktion ist in der {\tt stdlib.h} -Bibliothek definiert.
Diese Bibliothek stellt eine Vielzahl von Standardfunktionen bereit und 
wird deshalb als \emph{Standardbibliothek} (engl: standard library) bezeichnet.

Der R�ckgabewert der {\tt rand()} -Funktion ist eine ganze Zahl (\texttt{int}) zwischen \texttt{0} und {\tt
RAND\_MAX}, wobei {\tt RAND\_MAX} eine gro�e Zahl ist ($\approx$ 2 Milliarden
auf meinem Computer) welche in der gleichen Headerdatei definiert ist. 
Jedes mal, wenn wir {\tt rand()} aufrufen, berechnet die Funktion eine andere 
pseudozuf�llige Zahl.  Um ein Beispiel zu sehen k�nnen wir die folgende
Schleife ausf�hren:

\begin{verbatim}
    for (i = 0; i < 4; i++) 
    {
        int x = rand();
        printf("%i\n", x);
    }
\end{verbatim}
%
\newpage
Auf meinem Computer wird die folgende Ausgabe erzeugt:

\begin{verbatim}
    1804289383
    846930886
    1681692777
    1714636915
\end{verbatim}
%
Wenn Sie das Programm ausf�hren, sollte das so �hnlich aussehen, es werden ihnen  
aber wahrscheinlich andere Zahlen angezeigt.

Nat�rlich wollen wir nicht immer mit gigantisch gro�en Zahlen arbeiten.
Meistens ben�tigen wir Zufallszahlen zwischen  \texttt{0} und
einer oberen Grenze. 
Eine einfache M�glichkeit solche Zahlen zu erzeugen besteht darin den 
Modulo-Operator zu verwenden:

\begin{verbatim}
    int x = rand ();
    int y = x % upperBound;
\end{verbatim}
%
So ist {\tt y} der ganzzahlige Divisionsrest  wenn {\tt x} durch
{\tt upperBound} geteilt wird. Die m�glichen Werte f�r {\tt y}
liegen damit zwischen 0 und {\tt upperBound - 1}, inklusive beider
Grenzwerte. Wir m�ssen beachten, dass {\tt y} niemals
den Wert von {\tt upperBound} erreichen kann.

Manchmal ben�tigen wir zuf�llige Flie�kommazahlen in unserem
Programm. Wir k�nnen diese erzeugen, indem wir das Ergebnis 
von \texttt{rand()} durch  {\tt RAND\_MAX} teilen und vorher eine 
Typumwandlung (engl: cast) f�r einen der Operanden durchf�hren:

\begin{verbatim}
    int x = rand ();
    double y = (double) x / RAND_MAX;
\end{verbatim}
%
Diese Programmzeilen weisen {\tt y} einen zuf�lligen Wert im Bereich zwischen
0.0 und 1.0 zu. Die Werte 0.0 und 1.0 sind in diesen Bereich eingeschlossen.

\begin{description}
\item [AUFGABE:] 
�berlegen Sie sich eine M�glichkeit, wie man zuf�llige Flie�kommazahlen
in einem beliebigen Bereich erzeugen kann. 
Erstellen Sie ein Programm welches Zufallszahlen zwischen 100.0 und 200.0
erzeugt. 
\end{description}

%%
%\pagebreak
\section{Statistiken}
\index{Statistiken}
\index{Verteilung}
\index{Durchschnitt}

Die Zahlen die wir mit {\tt rand()} erzeugen sollten eigentlich
gleichverteilt sein. Das hei�t, die Wahrscheinlichkeit mit
der ein Wert aus dem Wertebereich gezogen wird, ist f�r alle
Werte gleich.
Wenn wir also das Vorkommen eines jedes m�glichen Wertes z�hlen,
sollten wir ann�hernd die gleiche Anzahl f�r alle Werte erhalten
unter der Voraussetzung, dass wir eine sehr gro�e Anzahl von
Werten untersuchen.

In den n�chsten Abschnitten werden wir ein Programm entwickeln, 
welches eine Folge von Zufallszahlen erzeugt und �berpr�ft, ob
diese Eigenschaft gegeben ist.

%%
\section{Arrays mit Zufallszahlen}
\label{Array of random numbers}

Der erste Schritt besteht darin eine gro�e Anzahl von Zufallszahlen
zu erzeugen und diese in einem Array zu speichern.
Ich werde dazu erst einmal mit der  ``gro�en Zahl'' von 20 Zufallszahlen beginnen.  
Es ist eigentlich immer eine gute Idee mit einer �berschaubaren
Anzahl von Werten zu starten. Das macht es einfacher das Programm zu
durchschauen und m�gliche Fehler zu finden. Wir k�nnen dann sp�ter
die Anzahl sehr einfach weiter erh�hen.

Die folgende Funktion hat die Aufgabe ein Array von {\tt int}s 
mit zuf�lligen Werten im Bereich von 0 bis {\tt upperBound-1} zu f�llen.
Die Funktion besitzt 3 Parameter, ein Array von ganzen Zahlen (int), 
die Gr��e des Arrays und einen oberen Grenzwert (upperBound) f�r den
Wertebereich der Zufallszahlen. 

\begin{verbatim}
    void RandomizeArray (int array[], int length, int upperBound) 
    {
        int i;
        for (i = 0; i < length; i++) 
        {
            array[i] = rand() % upperBound;
        }
    }
\end{verbatim}
%
Der R�ckgabewert der Funktion ist {\tt void}, was bedeutet, dass die
Funktion keinen Wert an die aufrufende Funktion zur�ckgibt.
Um diese Funktion zu testen, ist es bequem eine weitere Funktion 
zu schreiben, welche den Inhalt eines Arrays auf dem Bildschirm ausgibt:

\begin{verbatim}
    void PrintArray (int array[], int length) 
    {
        int i;
        for (i = 0; i < length; i++) 
        {
            printf ("%i ",  array[i]);
        }
        printf ("\n");
    }
\end{verbatim}
%
Die folgenden Programmzeilen erzeugen ein Array welches mit zuf�lligen Werten 
gef�llt wird und geben den Inhalt des Arrays auf dem Bildschirm aus:

\begin{verbatim}
    int r_array[20];
    int upperBound = 10;
    int length = sizeof(r_array) / sizeof(r_array[0]);
  
    RandomizeArray (r_array, length, upperBound);
    PrintArray (r_array, length);
\end{verbatim}

%
Auf meinem Computer sieht die Ausgabe folgenderma�en aus

\begin{verbatim}
    3 6 7 5 3 5 6 2 9 1 2 7 0 9 3 6 0 6 2 6 
\end{verbatim}
\nopagebreak%
und scheint auf den ersten Eindruck schon ziemlich zuf�llig zu sein.

Wenn diese Zahlen wirklich zuf�llig verteilt sind, dann 
erwarten wir, dass jede Zahl gleich h�ufig auftritt. 
In der ermittelten Folge kommt aber die Zahl 6 f�nf mal vor, die Zahl 4
und 8 hingegen �berhaupt nicht.

Hei�t das jetzt, das unsere Zahlenfolge keine wirklich Zufallsfolge ist?
Mit so einer kleinen Stichprobe l�sst sich das schwer sagen.
Nur f�r eine sehr gro�e Anzahl von Versuchen k�nnen wir davon 
ausgehen, die erwartete Gleichverteilung der Werte auch feststellen
zu k�nnen.

Um unsere Theorie zu testen wollen wir daher einige Programme
schreiben, die das Vorkommen jedes Werts z�hlen, so dass 
wir �berpr�fen k�nnen was passiert, wenn wir die Anzahl der Elemente
erh�hen. 
%
%To test this theory, we'll write some programs that count the
%number of times each value appears, and then see what happens
%when we increase the number of elements in our array.

%%
\section{Ein Array an eine Funktion �bergeben}
\label{Passing an array to a function}
\index{call by reference}
\index{call by value}
\index{Arrays als Parameter}
Schauen wir uns zuerst aber die {\tt RandomizeArray()} Funktion noch einmal etwas genauer
an. Etwas an dieser Funktion ist ungew�hnlich:
Wir �bergeben ein Array an die Funktion und auf irgend eine Art und
Weise ist es der Funktion gelungen das Array mit zuf�lligen 
Werten zu f�llen ohne das die Funktion einen Wert zur�ckgibt, der 
R�ckgabewert der Funktion ist n�mlich {\tt void}.

%You probably have noticed that our {\tt RandomizeArray()} function 
%looked a bit unusual. We pass an array to this function and expect 
%to get a a randomized array back. Nevertheless, we have declared it to 
%be a {\tt void} function, and miraculously the function appears to have 
%altered the array.

Dieses Verhalten widerspricht allem, was ich bisher �ber die
Verwendung von Variablen in Funktionen gesagt habe.
C benutzt f�r die Argumente einer Funktion die sogenannte {\bf call-by-value} 
Auswertung des angegebenen Ausdrucks.
Das hei�t, es wird der Wert des Ausdruck in der aufrufenden Funktion ermittelt
und anschlie�end in die Parametervariable der aufgerufenen Funktion kopiert.
Der selbe Vorgang l�uft in der umgekehrten Richtung ab, wenn eine Funktion
einen Wert zur�ckgibt.
Ver�nderungen in den internen Variablen der aufgerufenen Funktion haben
keine Auswirkungen auf die externen Wert der aufrufenden Funktion.

%If you pass a value to a function it gets copied from
%the calling function to a variable in the called function. The same
%is true if the function returns a value.
%Changes to the internal variable in the called function do not affect the external 
%values of the calling function.

Wenn wir allerdings ein Array an eine Funktion �bergeben, so l�sst sich nur
schwer das gesamte Array als ein Wert an die aufgerufene Funktion �bergeben.
Dazu m�sste das gesamte Array aus der aufrufenden Funktion an die
aufgerufene Funktion kopiert und am Ende der Funktion vielleicht auch wieder
zur�ckkopiert werden.

C �bergibt deshalb nur einen Verweis (engl: reference) auf die Speicherstelle des Arrays
an die Funktion. Die aufgerufene Funktion kann dann mit Hilfe des Verweises
auf das originale Array zugreifen und dort direkt alle notwendigen �nderungen vornehmen.
Dieses Verhalten ist auch der Grund daf�r, dass unsere Funktion keinen
R�ckgabewerte hat. Die �nderungen an den Daten 
wurden ja bereits vorgenommen und m�ssen nicht noch einmal gespeichert werden.
Eine solche Art der �bergabe der Funktionsargumente  nennt man {\bf call-by-reference}.

%When we pass an array to a function this behaviour changes to
%something called {\bf call-by-reference} evaluation.
%C does not copy the array to an internal array --  it rather generates a
%reference to the original array and any operations in the called function 
%directly affects the original array.
%This is also the reason why we do not have to return anything from our 
%function. The changes have already taken place. 

Call-by-reference macht es notwendig, dass wir der aufgerufenen Funktion die
L�nge des Arrays explizit mitteilen m�ssen. Wenn wir n�mlich den {\tt sizeof}
Operator in der aufgerufenen Funktion benutzen um die Arraygr��e zu ermitteln
werden wir feststellen, dass dieser uns nur die Gr��e des Verweises liefert.
Die aufgerufene Funktion hat keine Information dar�ber wie unser Array in der
aufrufenden Funktion definiert wurde.

%would determine the size of the reference
%and not the original array.

%!!!Reference to later chapter needed!!!
Wir werden die Aufrufstrategien  {\bf call-by-reference} und {\bf call-by-value} im 
Kapitel~\ref{Pointers and Addresses}, Kapitel~\ref{Call by value} und
\ref{Call by reference} noch genauer diskutieren.

%%
\section{Z�hlen der Elemente eines Arrays}
\label{counting}
\index{Array!durchsuchen}
\index{Schleife!Z�hler}
\index{Z�hler}

In unserem aktuellen Progammbeispiel wollen wir eine m�glicherweise sehr gro�e
Menge von Elementen durchsuchen und dabei z�hlen, wie oft 
ein bestimmter Wert darin vorkommt.
% current example we want to examine a potentially large set
%of elements and count the number of times a certain value appears.
Dieses Programm entspricht einem allgemeinem Muster,
welches man ``Durchsuchen und Z�hlen'' nennen kann. 
Das Musters hat folgende Teilschritte:
 
%You can think of this program as an example of a pattern called ``traverse
%and count.''  The elements of this pattern are:

\begin{itemize}

\item Es existiert eine Menge von Daten, zum Beispiel ein Array, die von Anfang bis Ende durchsucht werden kann. 
%set or container that can be traversed, like a string
%or a array.

\item Es gibt einen Test der auf jedes Element der zu durchsuchenden Menge angewandt wird.
%that you can apply to each element in the container.

\item Ein Z�hler registriert wie viele Elemente den Test bestehen oder nicht bestehen.

\end{itemize}

F�r unseren Anwendungsfall stelle ich mir eine Funktion mit dem Namen {\tt HowMany()} vor,
die die Anzahl der Elemente in einem Array ermittelt die mit einem vorgegebenen Wert �bereinstimmen.
Die Parameter der Funktion sind das Array, die L�nge des Arrays und der gesuchte Wert.
Der R�ckgabewert der Funktion ist die Anzahl des Vorkommens des gesuchten Werts im
Array.
%counts the number of elements in a array that are equal to a given value.
%The parameters are the array, the length of the array and the integer value we are looking
%for.  The return value is the number of times the value appears.

\begin{verbatim}
    int HowMany (int array[], int length, int value) 
    {
        int i; 
        int count = 0;
  
        for (i=0; i < length; i++) 
            {
                if (array[i] == value) count++;
            }
        return count;
    }
\end{verbatim}

Ein praktikables Vorgehen f�r die L�sung von solchen Programmierproblemen
besteht darin sich einfache Funktionen auszudenken die eine 
bestimmte Teilaufgabe erledigen und einfach zu schreiben und zu verstehen 
sind. Anschlie�end nutzen wir dann mehrere dieser Funktionen um ein
komplexeres Problem zu l�sen. Diese Vorgehensweise wird auch 
{\bf Bottom-up Entwurf} genannt.

Nat�rlich ist es nicht einfach immer schon im voraus zu wissen, welche
Art von Funktionen wir f�r die L�sung unseres Problems ben�tigen und
es ist nicht immer offensichtlich welche Teilfunktionen einfach
zu schreiben sind.
Je mehr Erfahrung wir aber in der Programmierung bekommen um so
leichter wird es uns fallen. 
\index{Bottom-up Entwurf}
\index{Software Entwicklung!Bottom-up}
Ein guter Ansatz besteht darin nach Teilproblemen zu suchen die
ein bestimmtes Muster aufweisen, welches auch f�r andere Arten
von Problemen interessant und hilfreich sein kann.
 
%Also, it is not always obvious what sort of things are easy to write,
%but a good approach is to look for subproblems that fit a pattern you
%have seen before.

\index{Entwurfsmuster!Z�hler}

%Back in Section~\ref{loopcount} we looked at a loop that traversed a
%string and counted the number of times a given letter appeared.  


\section{�berpr�fung aller m�glichen Werte}

{\tt HowMany()} z�hlt nur das Auftreten eines bestimmten
Werts in einem Array. Wir wollen aber herausfinden, wie oft
jeder m�gliche Wert in dem Array vorkommt.
Dazu k�nnen wir eine Schleife einsetzen:

\begin{verbatim}
    int i;
    int r_array[20];
    int upperBound = 10;
    int length = sizeof(r_array) / sizeof(r_array[0]);
  
    RandomizeArray(r_array, length, upperBound);

    printf ("value\tHowMany\n");
    for (i = 0; i < upperBound; i++) 
    {
        printf("%i\t%i\n", i, HowMany(r_array, length, i));
    }
\end{verbatim}
%

%! ! ! Applies only to C++! ! !
%Notice that it is legal to declare a variable inside a {\tt for}
%statement.  This syntax is sometimes convenient, but you should
%be aware that a variable declared inside a loop only exists
%inside the loop.  If you try to refer to {\tt i} later, you
%will get a compiler error.

Diese Programmzeilen benutzen die Schleifenvariable als ein 
Argument von {\tt HowMany()} um die Anzahl aller Werte 
zwischen 0 und 9 in dem Array zu ermitteln, mit folgendem Resultat:

\begin{verbatim}
    value   HowMany
    0       2
    1       1
    2       3
    3       3
    4       0
    5       2
    6       5
    7       2
    8       0
    9       2
\end{verbatim}
%
In dem Beispiel ist es immer noch schwierig zu ermitteln, ob die Ziffern
wirklich gleichverteilt sind. Wir erh�hen daher die Anzahl der 
Werte in unserem Array auf  100.000 und f�hren damit den Test
durch:
\begin{verbatim}
    value   HowMany
    0       10130
    1       10072
    2       9990
    3       9842
    4       10174
    5       9930
    6       10059
    7       9954
    8       9891
    9       9958
\end{verbatim}
%
Jetzt l�sst sich feststellen, dass die Schwankungen der H�ufigkeit jede Wert 
circa 1\% des erwarteten Werts (10,000) betragen und
unsere Zufallszahlen sehr wahrscheinlich gleichverteilt sind.

\section {Ein Histogramm}
\index{Histogramm}

Es ist of n�tzlich die Tabellen aus dem vorigen Beispiel nicht
nur auf dem Bildschirm auszugeben sondern diese im Computer
f�r sp�tere Aufgaben vorzuhalten.
Daf�r m�ssen wir also 10 ganzzahlige Werte speichern.

Wir k�nnten jetzt 10 Variablen vom Typ {\tt int} erstellen und
ihnen Namen geben wie  {\tt howManyOnes},
{\tt howManyTwos} und so weiter.  
Das w�rde eine Menge Tipparbeit abgeben und es w�re
ziemlich umst�ndlich wollten wir sp�ter zum Beispiel den
Bereich der zu �berpr�fenden Zahlenwerte �ndern.

Eine viel bessere L�sung besteht darin ein weiteres Array mit
10 Elementen zu benutzen. 
So k�nnen wir alle zehn Speicherpl�tze mit einem Mal erstellen
und wir k�nnen per Index bequem auf die jeweilige Speicherstelle
zugreifen ohne mit zehn verschiedenen Namen hantieren zu m�ssen.
Hier ist das Ergebnis:

\begin{verbatim}
    #define UPPER_BOUND 10
    int i;
    int r_array[100000];
    int histogram[UPPER_BOUND];
    int length = sizeof(r_array) / sizeof(r_array[0]);
  
    RandomizeArray(r_array, length, UPPER_BOUND);

    for (i = 0; i < UPPER_BOUND; i++) 
    {
        int count = HowMany(r_array, length, i);
        histogram[i] = count;
    }  
\end{verbatim}
%
Ich habe das Array {\bf histogram} genannt, weil 
das die Bezeichnung f�r eine statistische H�ufigkeitsverteilung ist.
In einem Histogramm werden Daten in Klassen eingeteilt und
es wird erfasst, wie h�ufig die Ereignisse in jeder Klasse sind.

\index{Histogramm}

In unserem Programm ist ein kleiner Trick enthalten, denn
ich benutze die Schleifenvariable f�r zwei verschiedene Zwecke.
Zuerst wird sie als Argument der {\tt HowMany()} Funktion 
verwendet um den Wert anzugeben f�r den wir uns interessieren.
Weiterhin verwende ich die Schleifenvariable als Index das Element des
Histogramms um anzugeben, wo das Ergebnis der Funktion gespeichert
werden soll.

\section{Eine optimierte L�sung}

Die vorgestellte L�sung funktioniert, allerdings ist sie nicht
besonders effizient.
Jedes Mal, wenn wir die Funktion {\tt HowMany()} aufrufen
durchsucht diese das gesamte Array von Anfang bis zum
Ende. In unserem Beispiel m�ssen wir das Array zehn mal
durchsuchen!

%Although this code works, it is not as efficient as it could
%be.  Every time it calls {\tt HowMany()}, it traverses the
%entire array.  In this example we have to traverse the
%array ten times!

Unser Programm w�rde schneller arbeiten, wenn wir
das Array nur ein einziges Mal durchsuchen m�ssten.
Wir m�ssten nur einfach f�r jeden gefundenen Wert im
Array den entsprechenden Z�hler im Histogramm finden
und inkrementieren.
Da unser Histogramm die Verteilung der Werte von 0 bis 9
darstellt, k�nnen wir die gefundenen Werte unseres Arrays direkt als
Index f�r das Histogramm benutzen, wir m�ssen nur vorher
alle Elemente des Histogramms auf den Wert {\tt 0} setzen:
% 
%It would be better to make a single pass through the array.
%For each value in the array we could find the corresponding
%counter and increment it.  In other words, we can use the
%value from the array as an index into the histogram.  Here's
%what that looks like:

\begin{verbatim}
    #define UPPER_BOUND 10
    int i;
    int r_array[100000];
    int histogram[UPPER_BOUND] = {0};
    int length = sizeof(r_array) / sizeof(r_array[0]);
    
    RandomizeArray(r_array, length, UPPER_BOUND);
    
    for (i = 0; i < length; i++) 
    {
        int index = r_array[i];
        histogram[index]++;
    }
\end{verbatim}
%
Wir initialisieren {\tt histogram}, indem wir dem ersten Element den
Wert {\tt {0}} zuweisen. Alle nachfolgenden Elemente f�r die wir keinen
Wert angegeben haben werden danach automatisch auf {\tt {0}} gesetzt.
Jetzt k�nnen wir den ({\tt ++}) -Operator innerhalb der Schleife 
dazu benutzen, das Vorkommen der Zufallszahlen zu z�hlen.
Es wird oft vergessen, dass ein Z�hler initialisiert werden muss, bevor
er benutzt werden kann.

\begin{description}
\item [AUFGABE:] 
Erstellen Sie aus den Programmzeilen des Beispiels eine eigene Funktion mit dem
Namen {\tt Histogram()}. Die Funktion soll mindestens folgende Parameter besitzen:
ein Array mit den zu durchsuchenden Werten, 
den Suchbereich (in diesem Fall 0 bis 9) als zwei Parameter \texttt{min} und \texttt{max}, und
ein zweites Array, welches gro� genug ist das Histogram der Werte
des ersten Arrays zu speichern.
\end{description}

\section{Zuf�llige Startwerte}
\label{Random seeds}
\index{Zufallszahlen!Startwert}
\index{Zufallszahlen}

Wenn wir die Programme aus diesem Kapitel mehrmals hintereinander
ausf�hren f�llt auf, dass wir bei jedem Programmstart die gleichen
'zuf�lligen' Zahlenwerte angezeigt bekommen. Das ist nicht sehr
zuf�llig!
%If you have run the code in this chapter a few times, you might
%have noticed that you are getting the same ``random'' values
%every time.  That's not very random!

Eine Eigenschaft der pseudozuf�lligen Zahlen besteht darin,
dass der Algorithmus  f�r die Erzeugung der Zufallsfolge immer wieder die 
gleichen Werte generiert, wenn er mit dem gleichen \textbf{Startwert} beginnt.
Da wir keinen besonderen Startwert angegeben haben, erzeugt
der Zufallszahlengenerator bei jedem Programmstart immer 
wieder die gleiche Zahlenfolge.

%One of the properties of pseudorandom number generators is that
%if they start from the same place they will generate
%the same sequence of values.  The starting place is called
%a {\bf seed}; by default, C uses
%the same seed every time you run the program.

W�hrend wir unser Programm entwickeln ist dieses Verhalten oft
hilfreich, weil wir auf diese Weise die Ausgaben des Programms
vor und nach einer �nderung miteinander vergleichen k�nnen. 
%That way, when you make
%a change to the program you can compare the output before and
%after the change.
In vielen Anwendungen, wie zum Beispiel Computerspielen, ist es 
allerdings vorteilhaft, wenn unser Programm bei jedem Programmstart
eine unterschiedliche Zufallssequenz verwendet. 

Um dieses Verhalten zu erreichen, m�ssen wir einen unterschiedlichen
Startwert f�r unsere Zufallsfunktion angeben. 
Wir k�nnen daf�r die {\tt srand()} Funktion benutzen.  Wir m�ssen der
Funktion ein einziges Argument mitgeben, welches eine ganze Zahl
zwischen 0 und {\tt RAND\_MAX} sein muss.

Wo bekommen wir aber diesen einzigartigen Startwert her? 
Ein verbreiteter Weg f�r die Erzeugung eines Startwerts
besteht darin, die Bibliotheksfunktion {\tt time()} zu benutzen.
Die Funktion liefert  einen Wert f�r die aktuelle Zeit auf
dem Computersystem. Die genaue Interpretation dieses Werts ist
f�r uns momentan nicht wichtig. Wir benutzen den Wert nur, um 
unseren Zufallszahlengenerator mit einem ver�nderlichen Wert
zu starten und k�nnen  daf�r den  Befehl
\begin{verbatim}
    srand((unsigned) time(NULL));
\end{verbatim}
verwenden.
%and unrepeatable, like the number of seconds since January
%1970, and use that number as a seed.  The details
%of how to do that depend on your development environment.

\section{Glossar}

\index{Array}
\index{Element}
\index{Index}
\index{deterministisch}
\index{Zufallszahlen}
\index{pseudozuf�llig}
\index{Zufallszahlen!Startwert}
\index{Histogramm}
\index{Inkrement}
\index{Dekrement}
\begin{description}
%(engl: \emph{})
\item[Array (engl: \emph{array}):]  Eine Ansammlung von mehreren Werten gleichen
Typs unter einem gemeinsamen Namen. Die einzelnen Werte in einem Array 
werden als Elemente bezeichnet und lasst sich �ber einen Index identifizieren.
Ein Array ist ein Beispiel f�r eine Datenstruktur.
% 
%A named collection of values, where all the
%values have the same type, and each value is identified by
%an index.

\item[Element (engl: \emph{element}):]  Ein einzelner Wert in einem Array.  Mit Hilfe des {\tt []}
-Operators kann man auf die einzelnen Elemente eines Arrays zugreifen.

\item[Index (engl: \emph{index}):]  Eine ganzzahliger Wert oder Variable die dazu
benutzt wird auf ein bestimmtes Element in einem Array zuzugreifen.

\item[deterministisch (engl: \emph{deterministic}):]  In einem Computerprogramm
folgt auf eine Anweisung unter gleichen Voraussetzungen immer die gleiche
n�chste Anweisung und es wird f�r jeden Durchlauf ein reproduzierbares Ergebnis erzeugt.
% that does the same thing every
%time it is run.

\item[pseudo-zuf�llig (engl: \emph{pseudorandom}):]  Eine Zahlenfolge die 
f�r einen au�enstehenden Betrachter nicht von einer wirklich zuf�lligen
Zahlenfolge unterschieden werden kann, die aber durch deterministische
Berechnungen erzeugt wurde.
%  sequence of numbers that appear to be
%random, but which are actually the product of a deterministic
%computation.

\item[Startwert (engl: \emph{seed}):]  Ein Wert der dazu benutzt wird den Algorithmus f�r
die Erzeugung der Pseudozufallsfolge zu iniitialisieren.
Wenn der gleiche Startwert verwendet wird produziert der Algorithmus die 
gleiche Folge von pseudo-zuf�lligen Werten.

\item[Inkrementieren/Heraufz�hlen (engl: \emph{increment}):]  Den Wert einer Variablen um 1 erh�hen.
In C kann dazu der {\tt ++} Operator benutzt werden. 

\item[Dekrementieren/Herunterz�hlen (engl: \emph{decrement}):]  Den Wert einer Variablen um 1 verringern.
In C kann dazu der {\tt -}{\tt -} Operator benutzt werden.

\item[Bottom-up Entwurf (engl: \emph{bottom-up design}):]  Eine Method der Softwareentwicklung bei
der ausgehend von kleinen, n�tzlichen Funktionen durch deren Kombination ein komplexeres 
Programm mit vergr��ertem Funktionsumfang erstellt wird. Die alternative zum Bottom-up Entwurf ist der
Top-down Entwurf bei dem zuerst ein Gesamtkonzept erstellt wird, ohne die Details der 
programmtechnischen Umsetzung bereits zu kennen.

\item[Histogramm (engl: \emph{histogram}):]  Die Erfassung der H�ufigkeitsverteilung klassifizierbarer
Merkmale. Ein Histogramm kann in C als Array von ganzen Zahlen dargestellt werden, bei der
in den Elementen des Arrays die H�ufigkeit des Auftretens eines bestimmten Merkmals gez�hlt wird.




\end{description}

%%
\section{�bungsaufgaben}
\setcounter{exercisenum}{0}

\ifthenelse {\boolean{German}}{ \begin{exercise}
Schreiben Sie eine  Funktion namens {\tt CheckFactors(int,~int[], int)} mit 3 Parametern.
Die Funktion �bernimmt einen Integerwert {\tt n}, ein Array von Integerwerten
sowie die L�nge des Arrays {\tt len} als drittes Argument. 

Die Funktion  soll {\tt TRUE} zur�ckliefern, falls alle Zahlen in dem �bergebenen Array 
 Faktoren von {\tt n} sind 
(d.h.   {\tt n} durch alle diese Zahlen teilbar ist).
F�r den Fall, dass mindestens eines der Array-Elemente kein Faktor von 
{\tt n} ist soll {\tt FALSE} zur�ckgegeben werden.

HINWEIS: Ermitteln Sie vor dem Aufruf der Funktion {\tt CheckFactors()} die L�nge des Arrays in der {\tt main()} Funktion, siehe dazu \ref{Array length}.
Vergleichen Sie ebenfalls die L�sung der �bungsaufgabe~\ref{ex.isdiv}.
\end{exercise}

%%%%%%%%%%%%%%%%%%%%%%%%%%%%%%%%%%%
\pagebreak
\begin{exercise}
Schreiben Sie eine Funktion  {\tt void SetToZero(int[], int)}
welche ein Array von  {\tt int} und die L�nge dieses Arrays �bernimmt und 
anschlie�end dieses Array f�r alle Elemente auf
den Wert {\tt 0} initialisiert.

F�r die Ermittlung der L�nge des Arrays k�nnen Sie die Funktion aus dem
Abschnitt~\ref{Array length} �bernehmen.
Testen Sie die korrekte Implementierung dieser Funktion mit Hilfe
der {\tt PrintArray()} Funktion aus Abschnitt~\ref{Array of random numbers}.
\end{exercise}

%%%%%%%%%%%%%%%%%%%%%%%%%%%%%%%%%%%
\begin{exercise}
Schreiben Sie eine Funktion welche ein Array von {\tt int}, 
die L�nge des Arrays {\tt len} und 
einen {\tt int}  mit dem Namen
{\tt target}  als Argumente �bernimmt.  Die Funktion soll das
Array durchsuchen und den Index zur�ckliefert an dem
{\tt target} zum ersten Mal in dem Array auftritt. Sollte {\tt target} 
nicht in dem Array enthalten sein soll -1 zur�ckgegeben werden.
\end{exercise}

%%%%%%%%%%%%%%%%%%%%%%%%%%%%%%%%%%%
\begin{exercise}
%\textbf{Zusatzaufgabe!}

Seit der Erfindung der Computer werden diese genutzt um Arrays 
mit Daten zu sortieren. Unz�hlige Algorithmen wurden entworfen und
hinsichtlich ihrer Effizienz verglichen. 

Eine nicht-besonders-effizienter Algorithmus l�uft folgenderma�en
ab: Finde das gr��te Element im Array und tausche es mit dem ersten Element.
Finde das zweit-gr��te Element im Array und tausche es mit dem zweiten Element,
und so weiter...

\begin{enumerate}

\item Schreiben Sie eine Funktion {\tt IndexOfMaxInRange()}, welche  
ein Array von ganzen Zahlen (integers) �bernimmt und das
gr��te Element in einem bestimmten Bereich (range) des Arrays findet und seine Position als {\em index} zur�ckliefert.

\item Schreiben Sie eine Funktion {\tt SwapElement()}, welche ein
Array von ganzen Zahlen und zwei Indexe �bernimmt und anschlie�end die Werte der Elemente an
den gegebenen Indexen vertauscht.

\item Schreiben Sie eine Funktion {\tt SortArray()}, welche ein Array von ganzen Zahlen
�bernimmt und die Funktionen {\tt IndexOfMaxInRange()} und {\tt SwapElement()} benutzt
um das Array vom gr��ten zum kleinsten Wert zu sortieren.

\end{enumerate}
\end{exercise}
}
{\input{exercises/Exercise_7_english}}



%!TEX root = Main_german.tex

% LaTeX source for textbook ``How to think like a computer scientist''
% Copyright (C) 1999  Allen B. Downey

% This LaTeX source is free software; you can redistribute it and/or
% modify it under the terms of the GNU General Public License as
% published by the Free Software Foundation (version 2).

% This LaTeX source is distributed in the hope that it will be useful,
% but WITHOUT ANY WARRANTY; without even the implied warranty of
% MERCHANTABILITY or FITNESS FOR A PARTICULAR PURPOSE.  See the GNU
% General Public License for more details.

% Compiling this LaTeX source has the effect of generating
% a device-independent representation of a textbook, which
% can be converted to other formats and printed.  All intermediate
% representations (including DVI and Postscript), and all printed
% copies of the textbook are also covered by the GNU General
% Public License.

% This distribution includes a file named COPYING that contains the text
% of the GNU General Public License.  If it is missing, you can obtain
% it from www.gnu.org or by writing to the Free Software Foundation,
% Inc., 59 Temple Place - Suite 330, Boston, MA 02111-1307, USA.


\chapter{Strings and things}
\label{strings}

\section{Darstellung von Zeichenketten}

In den vorangegangenen Kapiteln haben wir vier Arten von Daten kennengelernt
--- Zeichen, Ganzzahlen, Gleitkommazahlen und Zeichenketten (Strings) --- aber nur drei Datentypen
f�r Variablen: {\tt char}, {\tt int} und {\tt double}. Bis jetzt haben wir keine
M�glichkeit kennengelernt eine Zeichenkette in einer Variable zu speichern
oder Zeichenketten in unserem Programm zu manipulieren.

In diesem Kapitel soll dieser Misstand behoben werden und 
ich kann jetzt das Geheimnis l�ften was das Besondere der Strings ist.
\index{String}
\index{String!Begrenzungszeichen}
\index{Typ!String}
Strings werden in C  als Arrays von Zeichen gespeichert. Dabei wird das Ende der
Zeichenkette in dem Array durch ein besonderes Zeichen (das Zeichen {\tt \textbackslash 0}) 
markiert, was dazu f�hrt, dass das {\tt char}-Array immer mindestens 1 Zeichen l�nger sein
muss als die zu speichernde Zeichenkette.


Mittlerweile k�nnen wir mit dieser Erkl�rung etwas anfangen und es wird klar, dass wir
erst ein ganze Menge �ber die Funktionsweise der Sprache lernen mussten,
bevor wir unsere Aufmerksamkeit den Strings und Stringvariablen zuwenden konnten.

Im vorigen Kapitel haben wir gesehen, dass Operationen
�ber Arrays nur minimale Unterst�tzung durch die Programmiersprache C 
erhalten und wir die erweiterten Funktionen f�r den Umgang mit Arrays 
selbst programmieren mussten.
Gl�cklicherweise liegen die Dinge etwas besser, wenn es um die Manipulation
dieser speziellen {\tt char}-Arrays geht, die wir zur Darstellung von Zeichenketten nutzen.
Es existiert eine Anzahl von Bibliotheksfunktionen in {\tt string.h} 
welche uns die Handhabung und Verarbeitung von Strings etwas einfacher
machen, als die Verarbeitung von reinen Arrays. 
\index{<string.h>}
\index{Header-Datei!string.h}
\index{Bibliothek!string.h}


Trotzdem muss man feststellen, dass die Stringverarbeitung in 
C um einiges komplizierter und m�hsamer ist als in anderen Programmiersprachen.
Die sorgf�ltige Beachtung dieser Besonderheiten ist notwendig um
die Verarbeitung von Zeichenketten nicht zu einer potentiellen Fehlerquelle in 
unseren Programmen werden zu lassen.

\section{Stringvariablen}

Wir k�nnen eine Stringvariable als ein Array von Zeichen in der folgenden Art erzeugen:

\begin{verbatim}
    char first[] = "Hello, ";
    char second[] = "world.";
\end{verbatim}

Die erste Zeile erzeugt eine Stringvariable {\tt first} und weist ihr den Wert {\tt ''Hello,~''} zu.
In der zweiten Zeile deklarieren wir eine zweite Stringvariable. Erinnern wir uns,
die gleichzeitige Deklaration und Zuweisung eines Wertes zu einer Variablen 
wird als Initialisierung bezeichnet.

Nur zum Zeitpunkt der Initialisierung k�nnen wir dem String einen Wert direkt zuweisen
(so wie bei Arrays im Allgemeinen). Die Initialisierungsparameter werden in der Form
einer Stringkonstanten  in Anf�hrungszeichen ({\tt''}\ldots {\tt''}) angegeben.
Wie wir zu einem sp�teren Zeitpunkt der Stringvariablen einen neuen Wert
zuweisen k�nnen, erfahren wir erst im Abschnitt~\ref{Zuweisung von neuen Werten an Stringvariablen}.


Auffallend ist der Unterschied in der Syntax der Initialisierung von Arrays und Strings. 
Wir k�nnten den String auch in der normalen Arraysyntax initialisieren:
\begin{verbatim}
    char first[] = {'H','e','l','l','o',',',' ','\0'};
\end{verbatim}
Es ist allerdings viel bequemer einen String als Stringkonstante zu schreiben.
Das  sieht nicht nur nat�rlicher aus, sondern ist auch sehr viel einfacher einzugeben.



Wenn wir die Stringvariable direkt zum Zeitpunkt der Deklaration auch gleich initialisieren, ist 
es normalerweise nicht notwendig, die Gr��e des Arrays mit anzugeben.
Der Compiler ermittelt die notwendige Gr��e des Arrays anhand der 
angegebenen Stringkonstante. Wir k�nnen allerdings die Gr��e
der Stringvariable auch selbst definieren. Ich erkl�re gleich, was wir dabei beachten m�ssen. 

\index{String!Begrenzungszeichen}
Erinnern wir uns, was ich �ber den Aufbau eines Strings gesagt habe.
Es ist ein Array vom Typ  {\tt char} in dem ein Begrenzungszeichen das Ende der
Zeichenkette markiert: das Begrenzungszeichen {\tt \textbackslash 0}. Dieses Zeichen
darf nicht mit dem Zeichen {\tt 0} verwechselt werden. Das Zeichen  {\tt 0}  wird in der ASCII-Tabelle
(siehe Anhang~\ref{ASCII-Table}) mit dem Wert 48 kodiert. Bei dem Zeichen {\tt \textbackslash 0}
handelt es sich um den Wert 0 in der Tabelle.

Normalerweise m�ssen wir das Begrenzungszeichen nicht selbst mit angeben.
Der Compiler versteht unseren Programmcode und f�gt diese Zeichen automatisch ein.
Allerdings haben wir in dem vorigen Beispiel einen String genau wie ein Array behandelt
und in diesem Fall m�ssen wir uns selbst um das Einf�gen des Begrenzungszeichen
k�mmern.

Wenn wir eine Stringvariable benutzen, um
w�hrend des Programmablaufs unterschiedlich gro�e Strings zu speichern, 
m�ssen wir bei der Definition des Arrays die Gr��enangabe so w�hlen, dass
auch gen�gend Platz f�r die l�ngste Zeichenkette reserviert wird.
Wir m�ssen weiterhin beachten, dass auch Platz f�r das Begrenzungszeichen 
�brig bleibt, dass hei�t, unser Array muss exakt ein Zeichen l�nger sein, als
die gr��te zu speichernde Zeichenfolge.  

Wir k�nnen Strings auf dem Bildschirm ausgeben, in dem wir den Variablennamen 
an die {\tt printf()} Funktion �bergeben. Dazu verwenden wir den  
Formatierungsparameter {\tt \%s}:

\begin{verbatim}
    printf("%s", first);
\end{verbatim}



%%
\section{Einzelne Zeichen herausl�sen}

Zeichenketten werden \emph{Strings}  genannt, weil sie aus einer Folge (engl: string) von
Zeichen bestehen. 
%Elektronische Datenverarbeitung - Stringmanipulationen.
Die erste Aufgabe die wir l�sen wollen, besteht darin 
aus einer Zeichenkette ein bestimmtes Zeichen herauszul�sen.
Da wir wissen dass die Zeichenkette in C als Array gespeichert ist
k�nnen wir einen Index in eckigen Klammern ({\tt [} und {\tt ]}) f�r diese Operation benutzen:

\begin{verbatim}
    char fruit[] = "banana";
    char letter = fruit[1];
    printf ("%c\n", letter);
\end{verbatim}
%
Der Ausdruck {\tt fruit[1]} gibt an, dass ich das Zeichen mit der Indexnummer~1
aus dem String namens {\tt fruit} ermitteln m�chte.  Das Resultat wird in einer {\tt
char}-Variable namens {\tt letter} gespeichert. Wenn wir uns anschlie�end den Wert 
der Variable {\tt letter} anzeigen lassen, so sollte es nicht �berraschen, dass dabei
der folgende Buchstabe auf dem Bildschirm ausgegeben wird:

\begin{verbatim}
   a
\end{verbatim}
%
{\tt a} ist nicht der erste Buchstabe von {\tt ''banana''}. Wie wir bereits im letzten
Kapitel besprochen haben, nummerieren Informatiker die Elemente eines Arrays
immer beginnend mit 0. Das gilt auch f�r Stringvariablen.
Der erste Buchstabe von {\tt ''banana''} ist {\tt b} und hat den Index 0.  Der zweite Buchstabe
{\tt a} den Index 1 und der Buchstabe mit dem Index 2 ist das {\tt n}.

Wenn wir also den ersten Buchstaben eines Strings ermitteln wollen m�ssen wir 
die  Null als Index in eckige Klammern setzen:

\begin{verbatim}
    char letter = fruit[0];
\end{verbatim}

\section{Die L�nge eines Strings ermitteln}
\index{String!L�nge ermitteln}
\index{L�nge ermitteln!String}
\index{<string.h>}
\index{Bibliothek!string.h}

Um die L�nge eines Strings herauszufinden (die Anzahl der Zeichen die dieser String enth�lt), 
k�nnen wir die Funktion {\tt strlen()} nutzen.  Die Funktion wird aufgerufen indem wir den String
als Argument benutzen:

\begin{verbatim}
    #include <string.h>
    int main(void)
    { 
       int length;
       char fruit[] = "banana"; 

       length = strlen(fruit);
       return EXIT_SUCCESS;
    }   
\end{verbatim}
%
Der R�ckgabewert von {\tt strlen()} ist in diesem Fall 6. Wir weisen diesen Wert der
Variablen {\tt length} zu, um ihn sp�ter weiter zu nutzen.  

Um dieses Programm zu kompilieren, m�ssen wir die {\tt string.h} Bibliothek in unser
Programm einbinden. Diese Bibliothek stellt eine gro�e Anzahl von n�tzlichen
Funktionen f�r den Umgang und die Bearbeitung von Zeichenketten bereit.
Es ist sinnvoll, sich mit diesen Funktionen vertraut zu machen, weil sie uns helfen
k�nnen unsere Programmieraufgaben einfacher und schneller zu l�sen.

%The type of the {\tt strlen()} is {\tt size_t}, an unsigned value large enough to
%enumerate any object that the system can handle (such as a string).
%for our example we can safely assume that the size of the string object 
%in our example will never
%exceed the range of the integer type. 


%Notice
%that it is legal to have a variable with the same name as a function.

Um das letzte Zeichen in einem String zu finden mag es naheliegen  
die folgenden Anweisungen einzugeben:

\begin{verbatim}
    int length = strlen(fruit);
    char last = fruit[length];       /* WRONG!! */
\end{verbatim}
%
Das funktioniert leider nicht. Der Grund daf�r liegt wieder darin, dass {\tt fruit} ein
Array darstellt und einfach kein Buchstabe unter dem 
Index  {\tt fruit[6]} in {\tt ''banana''} gespeichert ist.  
Wir erinnern uns: Der Index eines Arrays wird beginnend von 0 gez�hlt und
die 6 Buchstaben tragen deshalb die Indexnummern 0 bis 5.  Um den letzten Buchstaben zu erhalten
m�ssen wir den Wert von {\tt length} um 1 verringern:

\begin{verbatim}
    int length = strlen(fruit);
    char last = fruit[length-1];
\end{verbatim}

\section{Zeichenweises Durchgehen}
%\index{traverse}

Eine h�ufig vorkommende Aufgabe besteht darin einen 
String zeichenweise durchzugehen. Wir w�hlen das 
ersten Zeichen eines Strings aus, f�hren irgendwelche 
Operationen durch und wiederholen dieses Vorgehen, bis
wir beim letzten Zeichen des Strings angekommen sind.

Wir k�nnen diese Aufgabe sehr einfach mit Hilfe einer {\tt for} Schleife
erledigen:

\begin{verbatim}
    int index;
    for (index = 0; index < strlen(fruit); index++) 
    {
        char letter = fruit[index];
        printf("%i. Buchstabe: %c\n" , index+1, letter);
    }
\end{verbatim}
%
\index{Schleifenvariable}
\index{Variable!Schleife}
\index{Index}

Der Name unserer Schleifenvariablen ist {\tt index}.  Ein {\bf
Index} ist eine Variable oder ein Wert der ein bestimmtes Element 
einer geordneten Menge identifiziert --- in unserem Fall der Menge
von Zeichen in einem String.
Der Index gibt dabei an, um welches Zeichen der Menge es sich dabei
handelt.  Die Menge muss  geordnet sein, so dass 
jedem Buchstaben des Strings ein Index und jedem Index genau
ein Buchstabe eineindeutig zugewiesen ist.


Die Schleife geht den String zeichenweise durch und gibt
jeden Buchstaben auf einer eigenen Bildschirmzeile aus.
Um die nat�rliche Z�hlweise einzuhalten, geben wir den 
Wert des Index erh�ht um 1 aus (das hat keinen Einfluss auf
den Wert von {\tt index}). 

Beachtenswert ist weiterhin die Formulierung der Bedingung unser Schleife
{\tt index < strlen(fruit)}. Wenn 
der Wert von {\tt index} gleich der L�nge des Strings ist, wird
die Bedingung falsch und der Schleifenk�rper verlassen.
Das letzte Zeichen, welches wir ausgeben hat somit den 
Index {\tt strlen(fruit)-1}.



 
\begin{description}
\item[AUFGABE:] Schreiben Sie eine Funktion welche einen {\tt string}
als Argument �bernimmt und dann alle Buchstaben des Strings auf einer
Bildschirmzeile r�ckw�rts ausgibt.

\end{description}

%\section{A run-time error}
%\index{error!run-time}
%\index{run-time error}

%Way back in Section~\ref{run-time} I talked about run-time errors,
%which are errors that don't appear until a program has started
%running.

%So far, you probably haven't seen many run-time errors, because we
%haven't been doing many things that can cause one.  Well, now we are.
%If you use the {\tt []} operator and you provide an index that is
%negative or greater than {\tt length-1}, you will get a run-time
%error and a message something like this:

%\begin{verbatim}
%index out of range: 6, string: banana
%\end{verbatim}
%%
%Try it in your development environment and see how it looks.


%%
\section{Ein Zeichen in einem String finden}
\label{Finding a  character in a string}

Wenn wir in einem String nach einem bestimmten Zeichen
suchen, m�ssen wir die gesamte Zeichenkette durchgehen
und die Position ermitteln, an der sich das Zeichen befindet.

Hier ist eine m�gliche Implementation dieser Funktion:

\begin{verbatim}
    int LocateCharacter(char *s, char c)
    {
        int i = 0;
        while (i < strlen(s)) 
        {
            if (s[i] == c) return i;
            i = i + 1;
        }
        return -1;
    }
\end{verbatim}

Wenn wir diese Funktion aufrufen, �bergeben wir der Funktion den 
zu durchsuchenden String als erstes, und das zu suchende
Zeichen als zweites Argument.

Die Funktion gibt dann die erste ermittelte Position des Zeichens zur�ck.
Es kann  allerdings auch vorkommen, dass das zu suchende
Zeichen gar nicht in dem String enthalten ist. 
F�r diesen Fall ist es sinnvoll einen Wert zur�ckzugeben, der 
normalerweise nicht vorkommt und einen Fehlerfall signalisiert.
Wurde das Zeichen nicht gefunden so gibt die Funktion den Wert
{\tt -1} an die aufrufende Funktion zur�ck.


%%
\section{Pointer und Adressen}
\label{Pointers and Addresses}
\index{Pointer}
\index{Adresse}

Wenn wir uns die Definition der {\tt LocateCharacter()} Funktion
anschauen werden wir feststellen, dass die Angabe des Parameters {\tt char *s} ungew�hnlich aussieht.

Erinnern Sie sich noch daran, wie wir in Kapitel~~\ref{Passing an array to a function}
dar�ber gesprochen haben wie wir ein Array an eine Funktion �bergeben?
Ich sagte, dass anstelle einer Kopie des Arrays ein Verweis auf
das Array an die Funktion �bergeben wird. Ich habe aber nicht genau
erkl�rt, worum es sich bei dem Verweis genau handelt. Das werde ich jetzt nachholen.

%Remember, when we discussed how we had to pass
%an array to a function, back in Section~\ref{Passing an array to a function},
%we said that instead of copying the array, we only pass a reference to the function. 
%Back then, we did not say exactly what this reference was.


C ist eine der wenigen High-level Programmiersprachen welche die direkte
Manipulation von Datenobjekten im Speicher des Computers unterst�tzt.
Um den direkten Zugriff auf Speicherobjekten durchzuf�hren, m�ssen
wir die Position der Objekte im Speicher kennen: ihre Adresse.
Genau wie andere Daten auch, k�nnen wir diese Adressen in Variablen
speichern und an Funktionen �bergeben, denn so eine Adresse stellt ja selbst
wieder einen Wert dar. Adressen werden in Variablen eines speziellen Typs 
gespeichert. Diese Variablen zeigen auf andere Objekte im 
Speicher, wie zum Beispiel Integer-Werte, Arrays, Strings, usw. und werden deshalb
 {\bf Pointer}-Variablen genannt.

%Adresses can be stored in variables of a special type.
%These variables that point to other objects in memory 
%(such as variables, arrays and strings) 
%are therefore called {\bf pointer} variables. 

\index{Pointer}
\index{Zeiger|see {Pointer}}
Ein Pointer (manchmal auch als Zeiger bezeichnet) verweist auf die Stelle im Speicher des Computers, wo sich das
jeweilige Objekt befindet. Wenn wir einen Pointer definieren, m�ssen wir auch
immer angeben welchen Typ das Objekt besitzt, auf das verwiesen wird.
Die Definition einer Pointervariablen auf ein ganzzahliges Speicherobjekt vom Typ
 {\tt int}  kann folgenderma�en vorgenommen werden:

\begin{verbatim}
    int *i_pointer;
\end{verbatim}

Diese Deklaration sieht �hnlich aus wie andere Variablendeklarationen, mit einem
Unterschied - dem Sternchen vor dem Variablennamen. Damit geben wir
dem Compiler bekannt, dass es sich bei {\tt i\_pointer} um eine Pointervariable
handelt.

Anders als bei normalen Variablen bezieht sich der angegebene Typ {\tt int} nicht 
auf den Pointer selbst, sondern definiert auf welche Art von Objekten
dieser Pointer zeigen soll (in diesem Fall auf Objekte vom Typ {\tt integer}).
%. The type specification has nothing to do
%with the pointer itself, but rather defines which object this pointer is
%supposed to reference (in this case an {\tt integer}).
Das ist notwendig, damit mit der Compiler wei�, wie er das Speicherobjekt
behandeln soll, auf das der Pointer verweist. Der Pointer selbst ist normalerweise
nur eine Adresse im Speicher des Computers.
%This allows the compiler to do some type checking on, what would
%otherwise be, an anonymous reference.
  
Ein Pointer nur f�r sich allein hat keine sinnvolle Bedeutung. Wir ben�tigen immer
auch ein Objekt auf das der Pointer verweist: 

\begin{verbatim}
    int number = 5; 
    int *i_pointer;
\end{verbatim}
 
Diese Programmzeilen definieren eine  {\tt int} Variable und einen Pointer. 
Momentan besteht noch keinerlei Verbindung zwischen diesen beiden Variablen.

\section{Adress- und Indirektionsoperator}
\index{Pointer!Indirektionsoperator}
\index{Pointer!Adressoperator}

Um eine Verbindung zwischen einem Speicherobjekt und einem Pointer herzustellen
m�ssen wir den \textbf{Adressoperator}~{\tt \&} benutzen. Damit k�nnen wir 
die Speicherstelle der Variablen  {\tt number} ermitteln und dem Pointer 
zuweisen. Man sagt: ``der Pointer  {\tt i\_pointer} zeigt auf  {\tt number}'':
 
 \begin{verbatim}
    i_pointer = &number;
\end{verbatim}

%{\tt}
%Pointer {\tt i\_p} now references integer variable {\tt number}.
Ab jetzt k�nnen wir auf den Wert der Variable {\tt number} auch �ber den
Pointer zugreifen. Dazu ben�tigen wir den \textbf{Indirektionsoperator}~{\tt *}. 
Damit greifen wir auf das Objekt zu, auf das unser Pointer zeigt:

 \begin{verbatim}
    printf("%i\n", *i_pointer);
\end{verbatim}

Diese Programmzeile gibt {\tt 5} auf dem Bildschirm aus, welches dem
Wert der Variable {\tt number} entspricht. 
Bei der Verwendung des  Indirektionsoperators~{\tt *} m�ssen wir sehr gut
aufpassen. Wir d�rfen niemals den Indirektionsoperator mit einem Pointer verwenden
der noch nicht initialisiert wurde:
\hint

\begin{verbatim}
    int *z_pointer;
    printf("%i\n", *z_pointer);    /*** WRONG ***/
\end{verbatim}

Das Gleiche gilt wenn wir �ber einem Pointer dem Speicherobjekt einen neuen
Wert zuweisen wollen und der Pointer nicht initialisiert wurde. Hierbei kann es 
zu Programmfehlern oder sogar zum Programmabsturz kommen:

\begin{verbatim}
    int *z_pointer;
    *z_pointer = 1;                /*** WRONG ***/
\end{verbatim}


Beim Lesen von Programmen m�ssen wir aufpassen, dass wir
den vorangestellten  Indirektionsoperators~{\tt *} nicht mit der
Deklaration von Pointern verwechseln. Auch bei der Deklaration verwenden
wir einen {\tt *}, der aber nur dazu dient die Variable als Pointervariable kenntlich
zu machen. 
Leider ergibt sich daraus f�r Pointervariablen
der Umstand, dass eine direkte Initialisierung der Pointervariablen eine andere Syntax aufweist, als
eine sp�tere Zuweisung im Programm:\hint

\begin{verbatim}
    int number = 5; 
    int *i_pointer = &number; /* Initialisierung des Pointers */
    i_pointer = $number;      /* gleicher Effekt!  */
\end{verbatim}


Mit Pointern k�nnen wir also direkt auf Speicherstellen zugreifen und diese nat�rlich auch
ver�ndern:

 \begin{verbatim}
    *i_pointer = *i_pointer + 2;
    printf("%i\n", number);
\end{verbatim}

Unsere Variable {\tt number} hat jetzt den Wert {\tt 7} und wir fangen so langsam
an zu verstehen wie es der {\tt LocateCharacter()} Funktion m�glich ist 
auf die Werte unserer Stringvariablen mittels eines {\tt char} -Pointers
zuzugreifen.

Viele C Programme benutzen Pointer und wir haben gerade einmal 
die Oberfl�che des Themas angekratzt. Durch die Verwendung von
Pointern k�nnen bestimmte Probleme sehr effizient gel�st werden, 
allerdings kann der direkte und unkontrollierte Zugriff auf den Computerspeicher
auch zu schwerwiegenden Programmfehlern f�hren, wenn nicht
sehr sorgf�ltig gearbeitet wird, oder die Verwendung von Pointern nicht
komplett verstanden wurde. Der Compiler hat in diesen F�llen keine M�glichkeit
das richtige Programmverhalten zu pr�fen.
Aus diesem Grund untersagen viele andere Programmiersprachen 
die direkte Manipulation von Speicherstellen �ber Pointer.
 

%If we are looking for a letter in an {\tt string}, we may
%not want to start at the beginning of the string.  One way
%to generalize the {\tt find} function is to write a version
%that takes an additional parameter---the index where we should
%start looking.  Here is an implementation of this function.

%\begin{verbatim}
%    int Find (char *s, char c, int i)
%    {
%        while (i < strlen(s)) 
%        {
%            if (s[i] == c) return i;
%            i = i + 1;
%        }
%        return -1;
%    }
%\end{verbatim}
%
%We have to pass the {\tt string}
%as the first argument.  The other arguments are the character
%we are looking for and the index where we should start.



%%
%\section{Looping and counting}
%\label{loopcount}
%\index{traverse!counting}
%\index{loop!counting}

%The following program counts the
%number of times the letter {\tt 'a'} appears in a string:

%\begin{verbatim}
%    char fruit[] = "banana";
%    int length = strlen(fruit);
%    int count = 0;

%    int index = 0;
%    while (index < length) 
%    {
%        if (fruit[index] == 'a') 
%        {
%            count ++;
%        }
%        index++;
%    }
%    printf ("%i\n", count);
%\end{verbatim}
%%
%This program demonstrates a common idiom, called a {\bf counter}.  The
%variable {\tt count} is initialized to zero and then incremented each
%time we find an {\tt 'a'}.  (To {\bf increment} is to increase by one;
%it is the opposite of {\bf decrement}.)  When we exit the loop, {\tt count}
%contains the result: the total number of a's.

%\index{counter}

%As an exercise, encapsulate this code in a function named
%{\tt CountLetters()}, and generalize it so that it accepts the
%string and the letter as arguments.
%% m�ssen wir die L�nge vorher ermitteln und �bergeben?

%\index{encapsulation}
%\index{generalization}

%As a second exercise, rewrite this function so that instead
%of traversing the string, it uses the version of
%{\tt find} we wrote in the previous section.


%\section{The {\tt strchr} function}
%\index{find}

%


% The {\tt strchr} function is like the opposite of the
%{\tt []} operator.  Instead of taking an index and extracting the
%character at that index, {\tt strchr} takes a character and finds the
%index where that character appears.

%\begin{verbatim}
%  char fruit[] = "banana";
%  int index = strchar(fruit,'a'));
%\end{verbatim}
%%
%This example finds the index of the letter {\tt 'a'} in the string.
%In this case, the letter appears three times, so it is not obvious
%what {\tt find} should do.  According to the documentation, it returns
%the index of the {\em first} appearance, so the result is 1.  If the
%given letter does not appear in the string, {\tt find} returns -1.

%In addition, there is a 
%version of {\tt find} that takes another {\tt string} as
%an argument and that finds the index where the substring
%appears in the string.  For example,

%\begin{verbatim}
%  apstring fruit = "banana";
%  int index = fruit.find("nan");
%\end{verbatim}
%%
%This example returns the value 2.




%%
%\pagebreak[4]

\section{Verkettung von Strings}

Im Abschnitt~\ref{Finding a  character in a string} k�nnen wir sehen, wie
man eine Suchfunktion implementiert die ein {\tt character} in einem {\tt string}
findet.

Eine n�tzliche Operation f�r Strings ist deren {\bf Verkettung}.  
Darunter verstehen wir die Verkn�pfung des Endes eines 
Strings mit dem Anfang des n�chsten Strings.  So wird zum Beispiel aus:  {\tt auf}
und {\tt passen} wird {\tt aufpassen}.

\index{<string.h>}
\index{Bibliothek!string.h}
\index{String!Verkettung}
Gl�cklicherweise m�ssen wir nicht alle diese n�tzlichen Funktion selbst schreiben.
Die {\tt string.h} Bibliothek stellt verschiedene Funktionen bereit die wir f�r die
Bearbeitung von Strings nutzen k�nnen. 

Um zwei Strings miteinander zu verketten k�nnen wir die {\tt strncat()} Funktion benutzen:

\begin{verbatim}
    char fruit[20] = "banana";
    char bakedGood[] = " nut bread";
    strncat(fruit, bakedGood, 10);
    printf ("%s\n", fruit);
\end{verbatim}
%
Diese Programmzeilen erzeugen die Ausgabe:  {\tt banana nut bread}.

Wenn wir Bibliotheksfunktionen benutzen ist es sehr wichtig, dass wir 
alle notwendigen Argumente der Funktion kennen und ein Grundverst�ndnis
der Arbeitsweise der Funktion besitzen.
% are using library functions it is important to completely understand
%all the necessary arguments and to have a complete understanding
%of the working of the function. 

Man k�nnte jetzt annehmen, dass die {\tt strncat()} Funktion zwei Strings
nimmt, sie zusammenf�gt und einen neuen, zusammengesetzten String
erzeugt. So ist es aber nicht. Die Funktion kopiert n�mlich den Inhalt des zweiten
Strings an das Ende des ersten Strings. 

Dieser kleine Unterschied hat f�r uns als Programmierer wichtige 
Konsequenzen. Wir selbst m�ssen daf�r sorgen, dass die erste Stringvariable auch
lang genug ist um auch den zweiten String mit aufzunehmen.
Aus diesem Grund habe ich die erste Stringvariable {\tt fruit} als Array von 20 Zeichen definiert
{\tt char fruit[20]}. Dieses Array bietet Platz f�r 19 Zeichen + 1 Begrenzungszeichen. 
Das dritte Argument von {\tt strncat()} gibt an, wie viele Zeichen aus dem
zweiten in den ersten String kopiert werden. Dabei ist darauf zu achten, dass nur
so viele Zeichen in das erste Array kopiert werden wie dort hineinpassen. Weiterhin muss noch
Platz f�r das Begrenzungszeichen bleiben. Werden mehr Zeichen kopiert k�nnen Speicherbereiche
�berschrieben werden.


%It is also possible to concatenate a character onto the
%beginning or end of an {\tt string}.  In the following example, we
%will use concatenation and character arithmetic to output
%an abecedarian series.

%``Abecedarian'' refers to a series or list in which the elements
%appear in alphabetical order.  For example, in Robert McCloskey's book
%{\em Make Way for Ducklings}, the names of the ducklings are Jack,
%Kack, Lack, Mack, Nack, Ouack, Pack and Quack.  Here is a loop that
%outputs these names in order:

%\begin{verbatim}
%    char name[5];
%    char suffix[] = "ack";

%    char letter = 'J';
%    name[0] = letter;
%    name[1] = '\0';
%    
%    while (letter <= 'Q') 
%    {
%	/* Wrong, does not work, string gets longer and longer...*/        
%        printf("%s\n", strncat (name, suffix, 3));
%        letter++;
%        name[0] = letter;
%    }
%\end{verbatim}
%%
%The output of this program is:

%\begin{verbatim}
%Jack
%Kack
%Lack
%Mack
%Nack
%Oack
%Pack
%Qack
%\end{verbatim}
%%
%Of course, that's not quite right because I've misspelled ``Ouack''
%and ``Quack.''  As an exercise, modify the program to correct
%this error.

%Again, be careful to use string concatenation only with {\tt apstring}s
%and not with native C strings.  Unfortunately, an expression like
%{\tt letter + "ack"} is syntactically legal in C++, although it
%produces a very strange result, at least in my development environment.

%%
%\section{{\tt string}s are mutable}
%\index{immutable}
%\index{string}

%You can change the letters in an {\tt string} one at a time
%using the {\tt []} operator on the left side of an assignment.
%For example,

%\begin{verbatim}
%    char greeting[] = "Hello, world!";
%    greeting[0] = 'J';
%    printf ("%s", greeting);
%\end{verbatim}
%
%produces the output {\tt Jello, world!}.

\section{Zuweisung von neuen Werten an Stringvariablen}
\label{Zuweisung von neuen Werten an Stringvariablen}
\index{Zuweisung!String}
\index{String}

Bisher haben wir gesehen, wie eine Stringvariable w�hrend
der Deklaration initialisiert wird.
Da es sich bei Strings im Grunde um ein mehr oder weniger
normales Array handelt gilt auch hier, dass wir nicht den
Zuweisungsoperator benutzen k�nnen um dem String einen
neuen Wert zuzuweisen. 

%As with arrays in general, it is not 
%legal to assign values directly to strings, because it is
%not  possible to assign a value to an entire array.

\begin{verbatim}
    fruit = "apple"; /* Wrong: Direkte Zuweisung nicht m�glich! */
\end{verbatim}

Wir k�nnen allerdings die {\tt strncpy()} Funktion benutzen um
dem String einen neuen Wert zuzuweisen:

\begin{verbatim}
    char greeting[15];
    strncpy (greeting, "Hello, world!", 13);
\end{verbatim}

{\tt strncpy()} kopiert 13 Zeichen aus dem zweiten Argument-String
in den ersten Argument-String und somit haben wir der Stringvariable
einen neuen Wert zugewiesen.

Allerdings m�ssen wir dabei wieder einige Einschr�nkungen beachten. 
Die {\tt strncpy()} Funktion kopiert genau 13 Zeichen aus dem 
zweiten String in den ersten String.  Was passiert dabei aber mit dem
Begrenzungszeichen  {\tt \textbackslash 0}?

%\pagebreak[4]

Die Funktion setzt das Begrenzungszeichen \textbf{nicht} automatisch. 
Wir k�nnten unsere Kopieranweisung so �ndern, dass 14 statt 13 Zeichen 
kopiert werden. In diesem Fall wird das unsichtbare Begrenzungszeichen
einfach mitkopiert und unser String {\tt greeting} w�re wieder korrekt:

\begin{verbatim}
    strncpy (greeting, "Hello, world!", 14);
\end{verbatim}
oder wir benutzen die {\tt strlen()} Funktion um {\tt strlen() + 1} Zeichen zu kopieren:
\begin{verbatim}
    strncpy (greeting, "Hello, world!", strlen("Hello, world!")+1);
\end{verbatim}


Wenn wir allerdings nur einen Teil des zweiten Strings in den ersten
String kopieren wollen, so m�ssen wir selbst daf�r sorgen, dass das
 Zeichen n+1 im String {\tt greeting[15]}
 hinterher auf den Wert {\tt \textbackslash 0} gesetzt wird:

\begin{verbatim}
    strncpy (greeting, "Hello, world!", 5); /*kopiert nur Hello */
    greeting[5] = '\0';         /*das 6. Zeichen hat den Index 5*/
\end{verbatim}

\vskip 1.5em


{\bf Achtung!}  In den letzten beiden Abschnitten haben wir die 
{\tt strncpy()} und die {\tt strncat()} Funktion benutzt, bei der wir
immer genau angeben m�ssen, wie viele Zeichen aus dem zweiten
in den ersten String kopiert werden oder an diesen angehangen werden.\hint

Die {\tt string.h} Bibliothek stellt uns noch weitere Funktionen zur Verf�gung
die so �hnlich arbeiten. Es gibt zum Beispiel auch noch die {\tt strcpy()} und 
die {\tt strcat()} Funktion. Bei diesen beiden Funktionen wird die Anzahl der
zu kopierenden Zeichen nicht mit angegeben. Die Funktionen
kopieren so lange Zeichen aus dem zweiten in den ersten String, bis in
dem zweiten String ein Begrenzungszeichen gefunden wird.

Von der Benutzung dieser Funktionen wird dringend abgeraten! 
Die Benutzung dieser Funktionen hat zu einer gro�en Anzahl von 
Sicherheitsproblemen in C Programmen gef�hrt. C �berpr�ft 
keine Array-Grenzen und wenn in dem zweiten String das Begrenzungszeichen
zum Beispiel wegen einer fehlerhaften Benutzereingabe fehlt, so kann es leicht vorkommen,
dass Zeichen �ber den Speicherbereich der ersten Stringvariable 
hinaus in den Speicher geschrieben werden und dadurch andere Daten �berschrieben 
werden.
%check array boundaries and will continue copying characters
%into computer memory even past the length of the variable.


%%
\section{Strings sind nicht direkt vergleichbar}
\label{incomparable}
\index{String!Vergleichen}
\index{String}

Unsere Vergleichsoperatoren die wir f�r den Vergleich von {\tt int}s und
{\tt double}s benutzt haben k�nnen wir nicht auf Strings anwenden.  So k�nnten
wir auf die Idee kommen die folgenden Programmzeilen zu schreiben um 
zwei Strings miteinander zu vergleichen:

\begin{verbatim}
    if (word == "banana")  /* Wrong! */ 
\end{verbatim}

Dieser Test schl�gt leider jedes Mal fehl.

%
Wir k�nnen aber die {\tt strcmp()} Funktion benutzen um zwei Strings
miteinander zu vergleichen. Die Funktion hat einen R�ckgabewert von {\tt 0} 
wenn die zwei Strings identisch sind, einen negativen Wert, wenn der
erste String 'alphabetisch kleiner' ist als der zweite
(wenn er in einem W�rterbuch vor dem anderen String gelistet w�rde) oder einen
positiven Wert, wenn der zweite String 'gr��er' ist.

Bitte beachten Sie, dass der R�ckgabewert nicht der normalen Interpretation der
Wahrheitswerte normaler Vergleichsoperatoren entspricht, wobei der R�ckgabewert
 {\tt 0}  als 'Falsch' interpretiert wird. \hint


Mit der   {\tt strcmp()} Funktion k�nnen wir sehr einfach beliebige W�rter
in ihre alphabetische Reihenfolge bringen:

\begin{verbatim}
    if (strcmp(word, "banana") < 0) 
    {
        printf( "Your word, %s, comes before banana.\n", word);
    } 
    else if (strcmp(word, "banana") > 0) 
    {
        printf( "Your word, %s, comes after banana.\n", word);
    } 
    else 
    {
        printf ("Yes, we have no bananas!\n");
    }
\end{verbatim}
%
Dabei m�ssen wir aber wieder aufpassen. Die {\tt strcmp()} Funktion 
behandelt Gro�buchstaben und Kleinbuchstaben anders als
wir das normalerweise gew�hnt sind.
Alle Gro�buchstaben kommen bei einem Vergleich vor den 
Kleinbuchstaben. Der Grund daf�r liegt in der Art der verwendenden
Zeichenkodierung (siehe Anhang:~\nameref{ASCII-Table}).
Dort haben Gro�buchstaben einen kleineren Wert als Kleinbuchstaben
und produzieren das folgende Ergebnis:

\begin{verbatim}
    Your word, Zebra, comes before banana.
\end{verbatim}
%
Ein oft genutzter Ausweg aus diesem Dilemma besteht darin
Strings in einer einheitlichen Formatierung zu vergleichen.  
Zuerst werden alle Zeichen der Strings zu Klein- oder 
Gro�buchstaben gewandelt und erst danach wird
der Vergleich durchgef�hrt. Der n�chste Abschnitt zeigt wie
das gemacht wird.
%%
\section{Klassifizierung von Zeichen}

Es ist of n�tzlich den Wert, der in einer \texttt{char} -Variable gespeichert ist, 
im Programm untersuchen zu k�nnen und zu entscheiden, ob es
sich dabei um einen Gro�- oder Kleinbuchstaben, oder 
um eine Ziffer oder ein Sonderzeichen handelt.
C stellt auch daf�r wieder eine Bibliothek von unterschiedlichen
Funktionen bereit, welche eine solche Klassifizierung von Zeichen
erm�glichen. Um diese Bibliotheksfunktionen in unserem Programm nutzen
zu k�nnen m�ssen wir die Header-Datei {\tt ctype.h} in unser Programm
aufnehmen.
\index{<ctype.h>}
\index{Header-Datei!ctype.h}
\index{Bibliothek!ctype.h}
\index{Zeichen!Klassifizierung}
\index{Zeichen!Klassifizierung!islower()}
\index{Zeichen!Klassifizierung!isupper()}
\index{Zeichen!Klassifizierung!isspace()}
\index{Zeichen!Klassifizierung!isalpha()}
\index{Zeichen!Klassifizierung!isdigit()}


\begin{verbatim}
    char letter = 'a';
    if (isalpha(letter)) 
    {
        printf("The character %c is a letter.", letter);
    }
\end{verbatim}
%
Der R�ckgabewert der {\tt isalpha()} Funktion ist ein \texttt{int} 
und besitzt den Wert \texttt{0} wenn das Funktionsargument kein
Buchstabe ist und einen von Null verschiedenen Wert wenn
die Funktion mit einem Buchstabe aufgerufen wurde.

Wir k�nnen daher die Funktion direkt als Bedingung der \texttt{if} -Anweisung
einsetzen, wie in unserem Beispiel zu sehen.
Der R�ckgabewert \texttt{0} wird als {\tt false} interpretiert und l�sst die Bedingung 
fehlschlagen. Alle anderen R�ckgabewerte werden als {\tt true} interpretiert.


Weitere Funktionen zur Klassifizierung von Zeichen sind {\tt isdigit()}, welche
dazu dient die Ziffern 0 bis 9 zu identifizieren, {\tt isspace()} identifiziert
nicht druckbare Zeichen wie zum Beispiel Leerzeichen, Tabulatoren, Zeilenumbr�che.  
Es gibt au�erdem die {\tt isupper()} und {\tt islower()} Funktion, welche 
zwischen Buchstaben in Gro�- und Kleinschreibung unterscheidet.

\index{Zeichen!Umwandlung}
\index{Zeichen!Umwandlung!toupper}
\index{Zeichen!Umwandlung!tolower}
Schlie�lich gibt es auch zwei Funktionen, welche sich zur Umwandlung zwischen
Gro�- und Kleinschreibung nutzen lassen. Sie hei�en {\tt toupper()} und {\tt tolower()}.  
Beide Funktionen erwarten ein einzelnes Zeichen als Argument
und geben ein (m�glicherweise umgewandeltes) Zeichen zur�ck.

\begin{verbatim}
    char letter = 'a';
    letter = toupper (letter);
    printf("%c\n", letter);
\end{verbatim}
%
Die Ausgabe dieser Programmzeilen ist {\tt A}.

\begin{description}
\item[Aufgabe:] Benutzen Sie die Funktionen der \texttt{ctype.h} 
Bibliothek um zwei Funktionen mit den Namen {\tt StringToUpper()} und
{\tt StringToLower()} zu schreiben. Beide Funktionen sollen einen
einzelnen Parameter vom Typ String besitzen und diesen 
String so modifizieren, dass im Anschluss alle Zeichen des Strings
Gro�- oder Kleinbuchstaben sind. Zeichen die keine Buchstaben sind
sollen nicht ver�ndert werden. Der R�ckgabewert der Funktionen soll
{\tt void} sein.
\end{description}


%%%
%\section{Other {\tt string} functions}

%This chapter does not cover all the {\tt apstring} functions.
%Two additional ones, {\tt c\_str} and {\tt substr}, are covered
%in Section~\ref{finput} and Section~\ref{parsing}.

\section{Benutzereingaben im Programm}
\label{input}
\index{Benutzereingabe}
\index{Input!Keyboard}

Die meisten Programme die wir bisher geschrieben haben verhalten
sich sehr berechenbar. Sie f�hren bei jedem Programmstart die selben Anweisungen aus.
Das liegt unter anderem daran, dass wir es bisher vermieden
haben Eingaben vom Benutzer entgegenzunehmen und 
darauf zu reagieren.

Es gibt eine Vielzahl von M�glichkeiten mit einem Computer
in Interaktion zu treten. Dazu z�hlen Tastatureingaben,
Mausbewegungen, Sensoren sowie exotischere Mechanismen wie
Sprachsteuerung und Iris-Scanning. In diesem Buch werden wir
uns ausschlie�lich mit Tastatureingaben besch�ftigen.


\index{scanf()}
\index{printf()}

F�r die Eingabe von Werten �ber die Tastatur stellt
C die Funktion {\tt scanf()} bereit, welche Benutzereingaben �hnlich 
behandelt wie  {\tt printf()} Programmausgaben auf dem Bildschirm darstellt. 
Wir k�nnen die folgenden Programmzeilen dazu benutzen um den 
Benutzer einen ganzzahligen Wert einzugeben:

\begin{verbatim}
    int x;
    printf("Please enter a number: ");
    scanf("%i", &x);
\end{verbatim}
%
Die {\tt scanf()} Funktion h�lt die Ausf�hrung des Programms an 
und wartet darauf, dass der Benutzer eine Eingabe �ber die Tastatur
des Computers macht. Wenn der Benutzer einen g�ltigen 
ganzzahligen Wert eingegeben hat wird dieser von der Funktion in
in der Variable {\tt x} gespeichert.

Was passiert, wenn der Benutzer etwas anderes als eine ganze Zahl �ber die
Tastatur eingibt? C gibt keine Fehlermeldung aus oder macht sonst einen 
Versuch das Problem zu beheben. Die {\tt scanf()} Funktion bricht dann
einfach ab und l�sst den Wert von {\tt x} unver�ndert.

Gl�cklicherweise gibt es einen Weg um herauszufinden, ob die 
Eingabe funktioniert hat, oder nicht. Die {\tt scanf()} Funktion hat einen
R�ckgabewert, der angibt, wie viele Elemente erfolgreich eingelesen
wurden (die Funktion kann auch mehrere Werte gleichzeitig einlesen).
In unserem Beispiel erwarten wir einen R�ckgabewert von {\tt 1} wenn 
erfolgreich eine ganze Zahl eingelesen werden konnte.
Wenn wir einen anderen R�ckgabewert als  {\tt 1} erhalten, so wissen
wir, dass die Eingabeoperation nicht erfolgreich war und wir nicht einfach
so im Programm weitermachen k�nnen.

Das folgende Programm fragt nach einer Benutzereingabe und �berpr�ft,
ob wirklich eine Zahl eingelesen werden konnte:

\begin{verbatim}
    int main (void)
    {
        int success, x;

        /* prompt the user for input */
        printf ("Enter an integer: \n");

        /* get input */
        success = scanf("%i", &x);

        /* check and see if the input statement succeeded */
        if (success == 1) 
        {
            /* print the value we got from the user */
            printf ("Your input: %i\n", x);
            return EXIT_SUCCESS;
        }
        printf("That was not an integer.\n");
        return EXIT_FAILURE;
    }
\end{verbatim}
%
Es gibt noch eine weitere Sache die wir bei der Verwendung der {\tt scanf()} Funktion
beachten m�ssen.
Wenn wir unbedingt eine bestimmte Benutzereingabe ben�tigen k�nnte man
auf die Idee kommen das Programm folgenderma�en zu schreiben, um den Benutzer
immer wieder nach einer Eingabe zu fragen, bis diese vom Programm akzeptiert wird:

 \begin{verbatim}
    if (success != 1) 
    {
          while (success != 1)                                      
          { 
               printf("That was not a number. Please try again:\n");
               success = scanf("%i", &x);
          }  
     }
\end{verbatim}

\index{Input!Eingabepuffer}
\index{Input!Tastaturpuffer}
\index{Input!Eingabepuffer!l�schen}
\index{Input!Tastaturpuffer!l�schen}
\index{Endlosschleife}
Ungl�cklicherweise f�hrt dieses Programm in eine Endlosschleife wenn der Benutzer etwas
anderes als eine Zahl eingibt. Sie fragen sich jetzt sicher, warum?

Die Ein- und Ausgabefunktionen unseres Programms schreiben nicht selbst
Zeichen auf den Bildschirm oder beobachten wie der Benutzer einzelne Tasten dr�ckt.
Sie benutzen dazu Funktionen des jeweiligen Betriebssystems des Computers.
Die Tastatureingaben des Benutzers werden unserem Programm vom Betriebssystem
in einem so genannten Tastatur- oder Eingabepuffer (engl: input buffer) bereitgestellt. 

Die {\tt scanf()} liest dann die Zeichen aus dem Tastaturpuffer und wenn alles gut geht wird
anschlie�end der Puffer gel�scht.
Wenn allerdings, wie in unserem Beispiel die {\tt scanf()} Funktion keinen g�ltigen Wert
lesen kann dann wird der Puffer nicht geleert. Wenn wir jetzt also {\tt scanf()} erneut aufrufen
um einen neuen Wert zu lesen wird sogleich der noch gef�llte Puffer erneut gelesen - 
ist das Problem klar?

Wir m�ssen daher nach einer missgl�ckten Eingabe den Eingabepuffer l�schen um
ihn f�r die erneute Tastatureingabe des Benutzers bereit zu machen.
Leider gibt es genau daf�r keine Standardfunktion in C, das hei�t wir m�ssen uns
selbst eine schreiben. 
Ich benutze dazu die {\tt getchar()} Funktion die einzelne Zeichen aus dem Puffer lesen
kann. Die Funktion wird direkt in der Bedingung einer  {\tt while} -Schleife so oft aufgerufen 
bis keine Zeichen mehr im Eingabepuffer vorhanden sind:

\begin{verbatim}
      while (success != 1)                                      
      { 
          char ch;   /* helper variable stores discarded chars*/
          printf("That isn't a number. Please try again:\n");

          /* now we empty the input buffer*/
          while ((ch = getchar()) != '\n' && ch != EOF);
          success = scanf("%i", &x);
      }    
\end{verbatim}
 

Mit der {\tt scanf()} Funktion k�nnen wir auch Strings einlesen:

\begin{verbatim}
    char name[80] ;

    printf ("What is your name?");
    scanf ("%79s", name);
    printf ("%s", name);
\end{verbatim}
%
Hierbei ist es wieder sehr wichtig, dass wir daf�r sorgen, dass unsere
Stringvariable gro� genug ist um die komplette Benutzereingabe 
aufzunehmen. Da wir nicht wissen wie viele Zeichen der Benutzer wirklich
auf der Tastatur eingibt, beschr�nken wir die Anzahl der Zeichen die
von der Funktion aus dem Eingabepuffer gelesen werden auf 79, so dass
noch Platz f�r das Begrenzungszeichen des Strings bleibt.

Auff�llig ist der Unterschied der Schreibweise des Arguments
der {\tt scanf()} Funktion, je nachdem ob wir eine ganze Zahl oder einen 
String einlesen. Das Funktionsargument ist keine Wert sondern ein Pointer
auf eine Speicherstelle. Schlie�lich soll die Funktion ja die Tastatureingabe
direkt in einer Variablen speichern.
Wenn wir eine ganze Zahl ({\tt int}) einlesen, m�ssen wir deshalb der Funktion
die Adresse der Variable mitteilen, wo der eingelesene Wert gespeichert werden
soll. Dazu benutzen wir den Adressoperator {\tt \&} mit dem Variablenname. 
Wenn wir einen String einlesen, m�ssen wir nur den Namen der Stringvariable
selbst angeben. Wir erinnern uns: Arrays wurden per \emph{call-by-reference}
an Funktionen �bergeben!

Ein letzter Hinweis gilt der Arbeitsweise der {\tt scanf()} Funktion. Sollte sich in der
einzulesenden Zeichenfolge des Eingabepuffers ein oder mehrere Leerzeichen befinden,
so wird an dieser Stelle die Einleseoperation f�r das aktuelle Element beendet. 
Wenn Sie zum Beispiel im vorigen Beispiel Vor- und Nachname eingegeben haben, so
wird nur das erste Wort dem String  {\tt s} zugewiesen. Der Rest der Eingabezeile
wird im Puffer gespeichert und w�rde von der n�chsten Eingabeanweisung ausgewertet werden. 
Wenn Sie also das Programm so ausf�hren, w�rde immer nur ihr Vorname ausgegeben werden.

%Because of these problems (inability to handle errors and
%funny behavior), I avoid using the {\tt >>} operator altogether,
%unless I am reading data from a source that is known to be
%error-free.

%Instead, I use a function in the {\tt apstring} called {\tt getline}.

%\begin{verbatim}
%  apstring name;

%  cout << "What is your name? ";
%  getline (cin, name);
%  cout << name << endl;
%\end{verbatim}
%%
%The first argument to {\tt getline} is {\tt cin}, which is
%where the input is coming from.  The second argument is the
%name of the {\tt apstring} where you want the result to be
%stored.

%{\tt getline} reads the entire line until the user hits
%Return or Enter.  This is useful for inputting strings that
%contain spaces.

%In fact, {\tt getline} is generally useful for getting input
%of any kind.  For example, if you wanted the user to type an
%integer, you could input a string and then check to see if
%it is a valid integer.  If so, you can convert it to an integer
%value.  If not, you can print an error message and ask the user
%to try again.

%To convert a string to an integer you can use the {\tt atoi()}
%function defined in the header file {\tt stdlib.h}.  
%We will get to that in Section~\ref{parsing}.


\section{Glossar}

\begin{description}

\item[String (engl: \emph{string}):]  Eine Variable in der eine Zeichenkette gespeichert ist.
In C werden Strings in Arrays vom Typ \texttt{char} gespeichert. Das Ende der Zeichenkette
wird durch den Wert \texttt{'\textbackslash 0'} markiert.

%\item[traverse (engl: \emph{}):]  To iterate through all the elements of a set
%performing a similar operation on each.

\item[Z�hler (engl: \emph{counter}):]  Eine Variable, die genutzt wird um etwas 
zu z�hlen (z.B. die Anzahl der Schleifendurchl�ufe). Eine Z�hlervariable wird
normalerweise mit 0 initialisiert und dann heraufgez�hlt (inkrementiert).

\item[Verketten (engl: \emph{concatenate}):] Eine Funktion �ber Zeichenketten, bei
der die eine Zeichenkette an eine andere angef�gt wird.

\item[Pointer (Zeiger) (engl: \emph{pointer}):] Ein Verweis auf ein Datenobjekt im Speicher.
Pointer sind Variablen in denen als Wert die Adresse eines anderen Datenobjekts gespeichert ist.

\item[Adresse (engl: \emph{address}):] Die genaue Speicherstelle eines Datenobjekts im Hauptspeicher
des Computers.

\index{String}
%\index{traverse}
\index{Z�hler}
\index{counter|see{Z�hler}}
\index{Inkrement}
\index{Verkettung von Strings}
\index{Pointer}
\index{Zeiger|see {Pointer}}
\index{Adresse}


\end{description}

\section{�bungsaufgaben}
\setcounter{exercisenum}{0}

\ifthenelse {\boolean{German}}{ 

\begin{exercise}

Schreiben Sie eine Funktion {\tt LetterHist()}, welche einen String
als Parameter �bernimmt und Ihnen ein Histogramm der Buchstaben in diesem
String liefert.

Das 'nullte' Element des Histogramms soll die Anzahl der {\tt a}'s 
(gemeinsam f�r Gro�- und Kleinschreibung) in dem String enthalten. Das 25. Element
die Anzahl der {\tt z}'s 

{\bf Zusatzaufgabe:}
Ihre L�sung soll den String nur genau einmal durchsuchen.
\end{exercise}


\begin{exercise}
Es existiert eine bestimmte Anzahl Worte bei denen jeder Buchstabe
genau zwei Mal im Wort vorkommt.

Beispiele aus einem Englisch-W�rterbuch enthalten:
\begin {quote}
Abba, Anna, appall, appearer, appeases, arraigning, beriberi,
bilabial, boob, Caucasus, coco, Dada, deed, Emmett, Hannah,
horseshoer, intestines, Isis, mama, Mimi, murmur, noon, Otto, papa,
peep, reappear, redder, sees, Shanghaiings, Toto
\end{quote}

Schreiben Sie eine Funktion {\tt IsDoubleLetterWord()} welche {\tt TRUE}
zur�ck liefert wenn das �bergebene Wort die oben beschriebene Eigenschaft 
aufweist, ansonsten soll {\tt FALSE} zur�ckgegeben werden.
\end{exercise}

\begin{exercise}

Der R�mische Kaiser Julius C�sar soll seine geheimen Botschaften
mit einem einfachen Verschl�sselungsverfahren gesichert haben.
Dazu hat er in seiner Botschaft jeden Buchstaben durch den Buchstaben
ersetzt, der 3 Positionen weiter hinten im Alphabet zu finden ist.

So wurde zum Beispiel aus {\tt a} ein {\tt d} und aus {\tt b} ein {\tt e}.
Die Buchstaben am Ende des Alphabets werden wieder auf den Anfang
abgebildet. So wird aus {\tt z} dann ein {\tt c}.

\begin{enumerate}
\item Schreiben Sie eine Funktion, welche zwei Strings �bernimmt. Einer
der Strings enth�lt die originale Botschaft, in dem anderen String soll
die verschl�sselte Geheimnachricht gespeichert werden.

Der String kann Gro�- und Kleinschreibung sowie Leerzeichen enthalten.
Andere Satzzeichen (Punkt, Komma, etc.) sollen nicht vorkommen.
Die Funktion soll die Buchstaben vor der Verschl�sselung in eine einheitliche
Darstellung umwandeln (Gro�- oder Kleinschreibung). Leerzeichen werden
nicht verschl�sselt.


\item Generalisieren Sie die Verschl�sselungsfunktion, so dass anstelle
der festen Verschiebung um 3 Positionen, Sie die Verschiebung frei
w�hlen k�nnen.

Sie sollten damit in der Lage sein die Nachrichten auch wieder zu entschl�sseln,
indem Sie z.B. mit dem Wert 13 verschl�sseln und mit -13 wieder entschl�sseln.

\end{enumerate}
\end{exercise}
}
{\input{exercises/Exercise_8_english}}


%!TEX root = Main_german.tex

% LaTeX source for textbook ``How to think like a computer scientist''
% Copyright (C) 1999  Allen B. Downey

% This LaTeX source is free software; you can redistribute it and/or
% modify it under the terms of the GNU General Public License as
% published by the Free Software Foundation (version 2).

% This LaTeX source is distributed in the hope that it will be useful,
% but WITHOUT ANY WARRANTY; without even the implied warranty of
% MERCHANTABILITY or FITNESS FOR A PARTICULAR PURPOSE.  See the GNU
% General Public License for more details.

% Compiling this LaTeX source has the effect of generating
% a device-independent representation of a textbook, which
% can be converted to other formats and printed.  All intermediate
% representations (including DVI and Postscript), and all printed
% copies of the textbook are also covered by the GNU General
% Public License.

% This distribution includes a file named COPYING that contains the text
% of the GNU General Public License.  If it is missing, you can obtain
% it from www.gnu.org or by writing to the Free Software Foundation,
% Inc., 59 Temple Place - Suite 330, Boston, MA 02111-1307, USA.


\chapter{Strukturen}
\label{structs}
\index{struct}

\section{Aggregierte Datentypen}

Die meisten Datentypen mit denen wir bisher gearbeitet haben
repr�sentieren einen einzelnen Wert -- eine ganze Zahl,
eine Flie�kommazahl oder einen Zeichenwert. 
Arrays und Strings dagegen sind aus mehreren Elementen 
aufgebaut, die jedes f�r sich genommen einen Wert darstellen, 
im Falle von Strings aus den einzelnen Zeichen. Diese
Art von Datentypen bezeichnet man als {\bf aggregierte} 
(zusammengesetzte) Datentypen. 

Je nachdem was wir in unserem Programm mit den Daten machen,
k�nnen wir aggregierte Datentypen als ein einzelnes Objekt
betrachten oder auf die einzelnen Elemente des Datentyps zugreifen.
Diese Mehrdeutigkeit ist f�r die Programmierung n�tzlich.
Beim Funktionsaufruf �bergeben wir das aggregierte Objekt als
einzelnes Argument. Innerhalb der Funktion greifen wir dann auf die
einzelnen Elemente des Objekts zu, verbergen diese Komplexit�t aber
vor allen anderen Funktionen in unserem Programm.
%
%member variables).  This ambiguity is useful.

Arrays sind aggregierte Datenobjekte aus Elementen des
gleichen Typs, auf deren Elemente �ber
ihren Index zugegriffen wird. 
In der Programmiersprache C gibt es einen weiteren
Mechanismus um aggregierte Datenobjekte zu bilden die 
auch aus Elementen unterschiedlichen Typs bestehen k�nnen
und weitere interessante Eigenschaften besitzen. Diese
Datenobjekte werden in C als {\tt struct} (Struktur)  bezeichnet
und k�nnen zum
Beispiel benutzt werden um eigene Datentypen zu definieren.  

\section{Das Konzept eines geometrischen Punkts}
\index{Point}
\index{struct!Point}

Um die das Konzept der Strukturen zu erkl�ren, m�chte ich  
ein einfaches Beispiel aus dem Bereich der Geometrie verwenden.
Stellen wir uns die geometrische Beschreibung eines Punkts vor.
In einem zweidimensionalen, kartesischen Koordinatensystem k�nnen wir jeden
Punkt durch die Angabe von 2 Zahlen (Koordinaten) beschreiben.
Diese beiden Koordinaten bilden zusammen ein Objekt, welches
wir f�r weitere Betrachtungen immer gemeinsam verwenden wollen.
In der Geometrie stellen wir deshalb Punkte oft als Zahlenpaar in
Klammern dar, wobei ein Komma die Koordinaten trennt.  So 
stellt zum Beispiel $P(0, 0)$ den Ursprung unseres Koordiatensystems
dar und $P(x, y)$ kennzeichnet den Punkt der $x$ Einheiten auf der 
X-Achse und $y$ Einheiten auf der Y-Achse vom Ursprung des
Koordinatensystems entfernt ist.

Solch einen Punkt k�nnen wir in C als zwei
Werte vom Typ {\tt double}  speichern.
Es bleibt aber die Frage, wie wir diese beiden Werte zu einem
zusammengesetzten Objekt vom Typ {\tt Point\_t} zusammenfassen.
Es kann ja vorkommen, dass wir in unserem Programm eine
Vielzahl von Punkten definieren und bearbeiten wollen, f�r die jeweils
zwei Werte zusammengenommen den Punkt definieren. 

Die Antwort auf diese Frage liefern die folgenden Programmzeilen.
Wir definieren einfach einen neuen  {\tt Point\_t} -Datentyp als {\tt struct}:

\begin{verbatim}
    typedef struct 
    {
        double x;
        double y;
    } Point_t;  
\end{verbatim}
%


Die Definition gibt an, dass es in unserer Struktur zwei Elemente mit
Namen {\tt x} und {\tt y} gibt.  Diese Elemente werden auch als
{\bf Komponenten} der Struktur bezeichet. Die {\tt struct} Definition muss in unserem Programm au�erhalb 
der Funktionsdefinitionen vorgenommen werden, am besten direkt
nach den {\tt \#include} Anweisungen.

Es ist ein h�ufiger Fehler das Semikolon am Ende der Strukturdefinition
zu vergessen. Normalerweise m�ssen wir hinter geschweiften Klammern
kein Semikolon setzen, bei Strukturdefinitionen hingegen schon.

Nachdem wir jetzt einen neuen Struktur-Datentyp definiert haben, 
k�nnen wir ganz einfach neue Variablen dieses Typs erzeugen:

\begin{verbatim}
    Point_t blank;
    blank.x = 3.0;
    blank.y = 4.0;   
\end{verbatim}
%
\index{Punktoperator}
Die erste Zeile ist eine ganz normale Variablendeklaration: die Variable
mit Namen {\tt blank} hat den Datentyp {\tt Point\_t}. 
Die n�chsten zwei Zeilen initialisieren die beiden Komponenten 
des \texttt{struct}.  Der Punkt zwischen dem Namen der Strukturvariable
und den Komponenten ist ein eigener Operator, der so genannte  {\bf Punktoperator}.
Mit dem Punktoperator k�nnen wir �ber den Namen der Strukturvariable auf
die einzelnen Komponenten der Struktur zugreifen.

\index{Deklaration}
\index{Anweisung!Deklaration}
%\index{reference}
\index{Zustandsdiagramm}
\index{Zustand}

Das Resultat der Zuweisung wird in dem folgenden Zustandsdiagramm gezeigt:

\vspace{0.1in}
\centerline{\epsfig{figure=figs/point.pdf, width=3cm}}
\vspace{0.1in}

So wie bei anderen Variablen auch, wird der Variablenname {\tt blank} au�erhalb  
und der Wert innerhalb des Kastens geschrieben.
In unserem Fall setzt sich der innere Teil des Kastens aus den einzelnen Werten der
zwei Komponenten des aggregierten Datenobjekts zusammen.
%and its value appears inside the box.  In this case, that value is
%a compound object with two named member variables.

\section{Zugriff auf die Komponenten von Strukturen}
\index{struct!Komponenten}

Nat�rlich k�nnen wir die Werte der Komponenten einer Struktur nicht nur 
setzen, sondern diese auch wieder auslesen. Auch hier m�ssen wir
wieder den Punktoperator anwenden.:

\begin{verbatim}
    double x = blank.x;
\end{verbatim}
%
Der Ausdruck {\tt blank.x} bedeutet ``gehe zu dem Objekt mit Namen {\tt
blank} und ermittle den Wert von {\tt x}.''  In unserem Falle weisen wir dann den
ermittelten Wert einer lokalen Variable mit Namen {\tt x} zu.  
Dabei ist es wichtig zu verstehen, dass es keinen Konflikt zwischen
der lokalen Variable {\tt x} und der Komponente {\tt x} der Variablen
{\tt blank} gibt. Es handelt sich um zwei voneinander verschiedene 
Speicherstellen und �ber den Punktoperator ist auch sichergestellt,
dass es keinen Namenskonflikt zwischen {\tt x} und {\tt blank.x} gibt.
%Notice that there is no
%conflict between the local variable named  and the member
%variable named {\tt x}.  The purpose of dot notation is to identify
%{\em which} variable you are referring to unambiguously.

Wir k�nnen den Punktoperator in jedem normalen Ausdruck in C verwenden,
wie man an den folgenden Beispielen sehen kann:

\begin{verbatim}
    printf ("%0.1f, %0.1f\n", blank.x, blank.y);
    double distance = sqrt (blank.x * blank.x + blank.y * blank.y);
    blank.y++;
\end{verbatim}
%
Die erste Programmzeile erzeugt die Bildschirmausgabe {\tt 3.0, 4.0}.
Die zweite Zeile errechnet den Wert 5 und speichert diesen in der Variable
{\tt distance}.
In der dritten Zeile wird der Wert von {\tt blank.y} um {\tt 1} erh�ht.

\section{Operatoren und Strukturen}


Die meisten Operatoren die wir bisher f�r einfache Datentypen
wie {\tt int} und {\tt double} benutzt haben lassen sich leider
nicht direkt f�r Strukturvariablen verwenden.
Das betrifft auch die mathematischen Operatoren ( {\tt +}, {\tt \%}, etc.) 
und die Vergleichsoperatoren ({\tt ==}, {\tt >}, etc.).

\index{struct!Zuweisungsoperator}
Der Zuweisungsoperator hingegen kann mit Strukturvariablen verwendet
werden. Es gibt zwei Arten wie wir ihn einsetzen k�nnen. 
Zuerst l�sst sich mit dem Zuweisungsoperator eine Strukturvariable
bei ihrer Definition bereits initialisieren. 
Dar�ber hinaus lassen sich Strukturvariablen gleichen Typs auch direkt kopieren, 
ohne dass wir jedes Element einzeln kopieren m�ssten.
Eine Initialisierung sieht folgenderma�en aus:

\begin{verbatim}
    Point_t blank = { 3.0, 4.0 };
\end{verbatim}
%
Die Werte in den geschweiften Klammern werden der Reihe nach den
einzelnen Komponenten der Strukturvariable zugewiesen. Wir m�ssen
daher genau wissen in welcher Reihenfolge die einzelnen Komponenten
in der Typdefinition angegeben wurden.
In unserem Fall wird der erste Wert der Komponente {\tt x} und der zweite
Wert der Komponenten {\tt y} zugewiesen.

Ungl�cklicherweise k�nnen wir diese Syntax nur w�hrend der 
Initialisierung der Variablen verwenden und nicht dazu benutzen
der Variablen zu einem sp�teren Zeitpunkt neue Werte zuzuweisen.
Die folgenden Programmzeilen sind daher fehlerhaft:

\begin{verbatim}
    Point_t blank;
    blank = { 3.0, 4.0 };       /* WRONG !! */
\end{verbatim}
%
Jetzt habe ich aber behauptet, dass wir Strukturvariablen direkt kopieren k�nnen,
warum ist dann diese Anweisung fehlerhaft?
Es gibt eine kleine Einschr�nkung hinsichtlich der Kopierbarkeit. Strukturen
k�nnen nur dann kopiert werden, wenn sie \emph{kompatibel} sind, das hei�t,
wenn Sie von der gleichen Typdefinition abgeleitet sind. 
I unserem Falle wei� der Kompiler nicht, dass es sich bei der rechten Seite
um den Inhalt einer Struktur handeln soll. Wir m�ssen daher noch den entspechenden
Typ als explitite Typumwandlung (engl: cast) hinzuf�gen:
\index{Typumwandlung}

\begin{verbatim}
    Point_t blank;
    blank = (Point_t){ 3.0, 4.0 };
\end{verbatim}
%
Das funktioniert!

Wenn wir zwei Strukturvariablen vom gleichen Typ definiert haben, k�nnen wir
nat�rlich auch einfach eine Zuweisungsoperation ausf�hren:

\begin{verbatim}
    Point_t p1 = { 3.0, 4.0 };
    Point_t p2 = p1;
    printf ("(%0.1f, %0.1f)\n", p2.x, p2.y);
\end{verbatim}
%
Die Ausgabe dieser Programmzeilen betr�gt {\tt 3.0, 4,0}.

%%
\section{Strukturen als Parameter}
\label{Structures as parameters}
\index{Funktionsparameter}
\index{struct!Funktionsparameter}

Wir k�nnen eine Struktur einfach als Parameter einer Funktion angeben:

\begin{verbatim}
    void PrintPoint (Point_t point) 
    {
        printf ("(%0.1f, %0.1f)\n", point.x, point.y);
    }
\end{verbatim}
%
Die Funktion {\tt PrintPoint()} wird mit einer Struktur vom Typ {\tt Point\_t}
aufgerufen und gibt die Koordinaten des Punkts auf dem Bildschirm  aus.
Wenn wir die Funktion folgenderma�en aufrufen {\tt PrintPoint(blank)},
erzeugt sie die Ausgabe {\tt (3.0, 4.0)}.

In einem zweiten Beispiel k�nnen wir die {\tt ComputeDistance()} Funktion aus
Section~\ref{distance} so umschreiben, dass sie zwei {\tt Point\_t} -Variablen
anstelle von vier {\tt double} -Variablen als Parameter besitzt:

\begin{verbatim}
    double ComputeDistance (Point_t p1, Point_t p2) 
    {
        double dx = p2.x - p1.x;
        double dy = p2.y - p1.y;
        return sqrt (dx*dx + dy*dy);
    }
\end{verbatim}

\section{Call by value}
\label{Call by value}
\index{Parameter�bergabe}
\index{call by value}

Wir erinnern uns,  Parameter und Argument einer Funktion
sind unterschiedliche Dinge. 
Der Parameter ist eine  Variable innerhalb der aufgerufenen Funktion.
Bei dem Argument handelt es sich um den Wert den die aufrufenden
Funktion an die aufgerufene Funktion �bergibt. 
Dieser Wert wird beim Funktionsaufruf in die Parametervariable
der aufgerufenen Funktion kopiert. Diesen Vorgang bezeichnet man
als Parameter�bergabe.
%there are two variables (one in the caller and one in the
%callee) that have the same value, at least initially. 

Wenn wir eine Struktur als Argument der Funktion {\tt PrintPoint()} 
�bergeben, sieht das Stackdiagramm folgenderma�en aus:
%
%you pass a structure as an argument, remember that the
%argument and the parameter are not the same variable. For
%example, when we call {\tt PrintPoint()}, the stack diagram
%looks like this:

\index{Diagramm!Stack}
\index{Stackdiagramm}

\vspace{0.1in}
\centerline{\epsfig{figure=figs/stack_point2.pdf, width=6cm}}
\vspace{0.1in}
%
Wenn {\tt PrintPoint()} jetzt den Wert der Komponenten 
von {\tt point} ver�ndert, so w�rde das keine Auswirkungen auf die
Komponenten von {\tt blank} haben.  Da  in unserem Beispiel 
die Funktion {\tt PrintPoint()}  die Parametervariable nicht
modifiziert, ist diese Form der Parameter�bergabe f�r das Programm angemessen.


Wie bereits erw�hnt, bezeichnen wir diese Art der Parameter�bergabe als \emph{call by value},
weil nur der Wert der Struktur (oder eines anderen Datentyps) an die Funktion �bergeben wird
und nicht die Struktur selbst. 

\section{Call by reference}
\label{Call by reference}
\index{Parameter�bergabe}
\index{call by reference}
\index{Pointer}

Es gibt noch eine andere Art der Parameter�bergabe an die
aufgerufene Funktions, das so genannte \emph{call by reference}.
In diesem Fall wird nicht der Wert des Arguments, sondern ein Verweis 
an die aufgerufene Funktion �bergeben.

Genau genommen gibt es in C kein echtes \emph{call by reference}.
Wir behelfen uns damit, dass wir statt der direkten �bergabe des Objekts
einen Pointer auf ein Speicherobjekt an die Funktion �bergeben.
Dabei wird Wert des Adresse des Speicherobjekts als Argument in den
Pointerparameter der Funktion kopiert. Da sich dabei die Adresse nicht ver�ndert
kann die Funktion direkt auf das Speicherobjekt au�erhalb der Funktion zugreifen.   

%alternative parameter-passing mechanism that is available
%in C is called \emph{call by reference}.  
%By now we already know that C uses pointers as references.
Wir k�nnen mit diesem Mechanismus auch einen Pointer auf eine Struktur an 
eine Funktion �bergeben und die Funktion dazu benutzen um die Struktur
selbst zu ver�ndern.

%This mechanism makes
%it possible to pass a structure to a procedure and modify it directly.

Stellen wir uns folgendes Beispiel vor: ein Punkt in einem Koordinatensystem
soll an der 45$^{\circ}$-Linie gespiegelt werden. Dazu m�ssen einfach
die beiden Koordinaten des Punkts vertauscht werden.

Wenn wir jetzt eine Funktion {\tt ReflectPoint()} in der folgenden Weise schreiben,
so werden wir damit nicht den erhofften Erfolg haben:

\begin{verbatim}
    void ReflectPoint (Point_t point)      /* Does not work! */
    {
        double temp = point.x;
        point.x = point.y;
        point.y = temp;
    }
\end{verbatim}
%
Die Funktion {\tt ReflectPoint()} arbeitet nicht korrekt, weil die �nderungen, die wir
an der Strukturvariablen in der Funktion vornehmen, keine Auswirkungen auf die
Strukturvariable in der aufgerufenen Funktion haben.

Wir m�ssen die Funktion so �ndern, dass wir statt einer Kopie der Struktur einen
Verweis auf die Struktur erhalten (call-by-reference). Dazu muss die Funktion einen Parameter vom
Typ {\tt Point\_t~*ptr} (Pointer auf die Struktur vom Typ {\tt Point\_t}) erhalten und wir 
m�ssen die Adresse des originalen Strukturobjekts an die Funktion �bergeben:


\begin{verbatim}
    void ReflectPoint (Point_t *ptr)
    {
        double temp = ptr->x;
        ptr->x = ptr->y;
        ptr->y = temp;
    }
\end{verbatim}
\index{Pfeiloperator}
Normalerweise benutzen wir den Punktoperator ({\tt .}) um auf die Elemente einer
Struktur zuzugreifen.
Wenn wir aber �ber einen Pointer (Zeiger) auf die Komponenten einer Struktur zugreifen, 
m�ssen wir einen speziellen Operator, den \textbf{Pfeiloperator}  ({\tt ->}) benutzen.
%
%di we are accessing the struct member variables through a pointer 
%we can no longer use the Punktoperator ({\tt .}). Instead we need to use
%the .

%
\index{Adressoperator}
Beim Funktionsaufruf m�ssen wir jetzt nur noch die Adresse der Struktur ermitteln und
der Funktion als Argument mitgeben. 
Wir benutzen dazu den \emph{Adressoperator} ({\tt \&}):

\begin{verbatim}
    PrintPoint (blank);
    ReflectPoint (&blank);
    PrintPoint (blank);
\end{verbatim}
%
Die Ausgabe des Programms sieht folgenderma�en aus und entspricht unseren Erwartungen:

\begin{verbatim}
    (3.0, 4.0)
    (4.0, 3.0)
\end{verbatim}
%
Das Stackdiagramm f�r den Funktionsaufruf sieht folgenderma�en aus:

\index{Diagramm!Stack}
\index{Stackdiagramm}

\vspace{0.1in}
\centerline{\epsfig{figure=figs/stack_point3.pdf, width=6.5cm}}
\vspace{0.1in}
%
Der Parameter {\tt ptr} ist ein Zeiger auf die Struktur  {\tt blank}.  
Pointer werden in einem Stackdiagramm als ein Punkt mit einem 
Pfeil dargestellt. Der Pfeil zeigt direkt auf das Speicherobjekt 
dessen Adresse im Pointer gespeichert ist.

Der wichtige Unterschied, den dieses Diagramm zum Ausdruck bringt,
ist der Fakt, dass alle �nderungen welche innerhalb der Funktion 
{\tt ReflectPoint()} �ber den Pointer  {\tt ptr} gemacht werden, eine
direkte �nderung der Strukturvariablen  {\tt blank} zur Folge hat.

Wenn wir die Adresse einer Struktur an eine Funktion �bergeben 
haben wir mehr M�glichkeiten, da die Funktion die Struktur
selbst ver�ndern kann und nicht nur mit einer Kopie arbeitet.  
Au�erdem ist es etwas schneller, weil der Rechner nicht erst eine Kopie
der ganzen Struktur erzeugen muss.
Auf der anderen Seite ist es weniger sicher und schwieriger nachvollziehbar.
Wir m�ssen selbst nachverfolgen an welchen Stellen eine Struktur
modifiziert wird und gerade wenn wir viele Funktionen benutzen kann das
schnell un�bersichtlich werden. 
Trotzdem werden in C die meisten Strukturen �ber Pointer an 
Funktionen �bergeben. Die in diesem Buch beschriebenen Funktionen
werden deshalb dieser Tradition treu bleiben. 

\section{Rechtecke}
\index{Rechtecke}
\index{struct!Rectangle}

Ok, nachdem wir jetzt also einen Punkt als Struktur 
dargestellt haben, wollen wir noch etwas weiter gehen und
uns �berlegen wie wir ein Rechteck als Datenstruktur beschreiben
k�nnen. Welche Daten sind n�tig, um die geometrische Figur eines
 Rechtecks vollst�ndig zu beschreiben? \\
(Um es uns etwas einfacher zu machen wollen wir davon ausgehen,
dass das Rechteck immer rechtwinklig zu unserem Koordinatensystem
ausgerichtet ist und niemals verdreht. )

Es gibt  mehrere M�glichkeiten: wir k�nnten den Mittelpunkt des
Rechtecks beschreiben (ein Punkt mit zwei Koordinaten) und die
L�nge und Breite des Rechtecks. Eine andere M�glichkeit w�re es
zwei gegen�berliegende Ecken als Punkte zu beschreiben.

Die meisten existierenden Programme geben die obere linke Ecke
sowie die L�nge und die Breite des Rechtecks an.
Um ein solches Rechteck in C zu beschreiben brauchen wir
eine Struktur die einen Punkt {\tt Point\_t}
und zwei Flie�kommazahlen {\tt double} enth�lt:

\begin{verbatim}
    typedef struct 
    {
        Point_t corner;
        double width, height;
    } Rectangle_t;  
\end{verbatim}
%
Wie wir feststellen k�nnne, ist es erlaubt, dass eine Struktur eine andere
Struktur als Komponente enthalten kann. Diese Art des Aufbaus von
komplexeren aus einfacheren Strukturen ist sogar ziemlich verbreitet.

Diese Definition bedeutet aber auch, dass wir bevor wir ein Rechteck
{\tt Rectangle\_t} definieren k�nnen, zuerst einmal einen Punkt {\tt Point\_t}
definieren m�ssen:

\begin{verbatim}
    Point_t corner = { 0.0, 0.0 };
    Rectangle_t box = { corner, 100.0, 200.0 };
\end{verbatim}
%
Diese Programmzeilen erzeugen eine neue {\tt Rectangle\_t} Struktur
und initialisieren die Komponenten. In der folgenden Abbildung 
ist diese Zuweisung als Zustandsdiagramm dargestellt:

\vspace{0.1in}
\centerline{\epsfig{figure=figs/rectangle.pdf, width=6cm}}
\vspace{0.1in}
%
Wenn wir auf die Komponenten {\tt width} und {\tt height} zugreifen wollen, 
so k�nnen wir das in der gewohnten Form tun:

\begin{verbatim}
    box.width = box.width + 50.0;
    printf("%f\n", box.width);
\end{verbatim}
%
Um auf die Komponenten der Struktur {\tt corner} zuzugreifen, k�nnten
wir, falls wir den Wert nur auslesen wollen, eine tempor�re Variable verwenden:

\begin{verbatim}
    Point_t temp = box.corner;
    double x = temp.x;
\end{verbatim}
%
Es ist allerdings viel einfacher den Punktoperator einzusetzen und
�ber die verkettete Angabe der beiden Strukturen direkt auf die Komponente
zuzugreifen:

\begin{verbatim}
    double x = box.corner.x;
\end{verbatim}
%
Um diese Anweisung zu verstehen ist es am Sinnvollsten die
Programmzeile von rechts nach links zu lesen:
 ``Ermittle den Wert {\tt x} in der Struktur {\tt corner} in der Struktur {\tt box}
und weise ihn der lokalen Variable {\tt x} zu.''

Ich sollte vielleicht noch hinzuf�gen, dass man nat�rlich die 
Strukturvariable {\tt box} auch in einem einzigen Programmschritt initialisieren
kann: 
%out that you can, in fact, create the {\tt Point} and the
%{\tt Rectangle} at the same time:

\begin{verbatim}
    Rectangle_t box = { { 0.0, 0.0 }, 100.0, 200.0 };
\end{verbatim}
%
\index{Verschachtelte Struktur}
Die inneren geschweiften Klammern beschreiben die Koordinaten
des Punkts {\tt corner}. Die zwei anderen Werte definieren
{\tt width} und {\tt height} des neuen Rechtecks {\tt box}.
Wir haben damit ein Beispiel f�r eine \emph{verschachtelte Struktur}
geschaffen.
Welche Variante der Initialisierung wir dabei verwenden ist uns �berlassen. 
Ich finde die erst Variante �bersichtlicher, sie bedeutet aber auch mehr
Schreibarbeit f�r den Programmierer.
 




\section{Strukturen als R�ckgabewerte}
\index{struct!as return type}
\index{return}
\index{Anweisung!return}

Es ist m�glich Funktionen zu schreiben die eine Struktur an die
aufrufende Funktion zur�ckgeben.
Wir k�nnen zum Beispiel eine Funktion {\tt FindCenter()} erstellen,
welche ein {\tt Rectangle\_t} als Parameter besitzt und
einen {\tt Point\_t}~Wert zur�ckgibt, welcher die Koordinaten des
Mittelpunkts des Rechtecks enth�lt:

\begin{verbatim}
    Point_t FindCenter (Rectangle_t box)
    {
        double x = box.corner.x + box.width/2;
        double y = box.corner.y + box.height/2;
        Point_t result = {x, y};
        return result;
    }
\end{verbatim}
%
Um diese Funktion aufzurufen, m�ssen wir ein {\tt Rectangle\_t} als
Argument an die Funktion �bergeben (es wird eine Kopie des Werts
der Struktur �bergeben: call-by-value). Den R�ckgabewert der Funktion
weisen wir einer  {\tt Point\_t} Variable zu:

\begin{verbatim}
    Rectangle_t box = { {0.0, 0.0}, 100, 200 };
    Point_t center = FindCenter (box);
    PrintPoint (center);
\end{verbatim}
%
Die Ausgabe dieser Programmzeilen lautet: {\tt (50, 100)}

Wir h�tten nat�rlich auch einen Pointer auf die Struktur an die
Funktion �bergeben k�nnen (call-by-reference). In diesem Fall
h�tte unsere Funktion folgenderma�en definiert werden m�ssen:
\begin{verbatim}
    Point_t FindCenter (Rectangle_t *box)
    {
        double x = box->corner.x + box->width/2;
        double y = box->corner.y + box->height/2;
        Point_t result = {x, y};
        return result;
    }
\end{verbatim}
In diesem Fall m�ssen wir neben dem Parameter nat�rlich auch noch
den Zugriff auf die Komponenten der Struktur anpassen. Das ist
notwendig, weil es sich bei  {\tt box} jetzt um einen Pointer handelt.
Weiterhin muss nat�rlich auch der Funktionsaufruf von {\tt FindCenter()}
ver�ndert werden:

\begin{verbatim}
    Point_t center = FindCenter (&box);
\end{verbatim}

\section {Andere Datentypen als Referenz �bergeben}
\index{Parameter�bergabe}
\index{call by reference}
\index{Pointer}

Wir k�nnen nicht nur Strukturen als Pointer an eine Funktion
�bergeben. Jeder Datentyp den wir bisher gesehen haben, 
kann auch als Pointer in einem Funktionsaufruf benutzt werden.
So k�nnen wir zum Beispiel zwei ganzzahlige Werte in 
der folgenden Funktion direkt vertauschen:

\begin{verbatim}
    void Swap (int *x, int *y)
    {
        int temp = *x;
        *x = *y;
        *y = temp;
    }
\end{verbatim}
%
Beim Funktionsaufruf m�ssen wir die Adressen der Variablen {\tt i} und {\tt j} 
als Argumente der Funktion verwenden:

\begin{verbatim}
    int i = 7;
    int j = 9;
    printf (" i=%i, j=%i\n", i, j);
    Swap (&i, &j);
    printf (" i=%i, j=%i\n", i, j);
\end{verbatim}
%
Die Ausgabe des Programms zeigt, dass die Werte der Variablen 
getauscht wurden.
Sie k�nnen ja selbst ein Stack-Diagramm zeichnen um
sich davon zu �berzeugen, dass die Funktion direkt auf die 
Variablen der aufrufenden Funktion zugreift.\\
Wenn ich die Funktionsparameter {\tt x} und {\tt y} als
normale {\tt int} -Variablen (ohne {\tt *}) deklariert h�tte, w�re die Funktion
{\tt Swap()} nicht korrekt.
Sie w�rde {\tt x} und {\tt y} innerhalb der Funktion vertauschen, h�tte
aber keinen Einfluss auf {\tt i} und {\tt j}.

Wenn wir Werte per \emph{call-by-value} an eine Funktion �bergeben,
dann ist es durchaus �blich hier einen komplexeren Ausdruck
als Argument der Funktion zu verwenden.
Bei \emph{call-by-reference} Funktionen kann aber die Verwendung von Ausdr�cken
zu schwer zu findenden Programmfehlern f�hren. 
Schauen wir uns folgende Programmzeilen an:

\begin{verbatim}
    int i = 7;
    int j = 9;
    Swap (&i, &j+1);         /* WRONG!! */
\end{verbatim}
%
Vermutlich wollte der Programmierer den Wert von {\tt j} vor
dem Tausch der Variablen um {\tt 1} erh�hen.
Allerdings wird hier nicht der Wert von {\tt j} erh�ht, sondern
die Adresse von {\tt j}. Es ist also gar nicht mehr klar, wohin der
Pointer {\tt *y} gerade zeigt.
 
Eine gute Faustregel ist daher als Argumente einer \emph{call-by-reference} Funktion
 nur Variablen zu verwenden und keine Ausdr�cke.
%For now a good rule of thumb is that reference arguments have to be
%variables.
\hint


\section{Glossar}

\begin{description}

\item[Struktur (engl: \emph{structure}):]  Ein zusammengesetztes Datenobjekt, welches
im Gegensatz zu einem Array aus einer Ansammlung von Werten unterschiedlichen Typs bestehen 
kann. Strukturen werden in C durch das Schl�sselwort \texttt{struct} gekennzeichnet.  
% of data grouped together and treated as a single object.

\item[Komponente (engl: \emph{member variable}):]  Eine benanntes Element einer Struktur 
auf das �ber seinen Namen einzeln zugegriffen werden kann.

\item[Verweis (engl: \emph{reference}):]  Ein Wert der auf ein Datenobjekt verweist. Im
Stack-Diagramm werden Verweise als Pfeil von einem Datenobjekt auf ein anderes dargestellt.

\item[Call by value (engl: \emph{call by value}):]  Eine Art der Parameter�bergabe beim Funktionsaufruf. 
Dabei wird der Wert des Arguments in den dazugeh�rigen
Parameter der Funktion kopiert. Die Speicherstelle des Parameters befindet sich innerhalb
der aufgerufenen Funktion und ist komplett unabh�ngig von der aufgerufenen Funktion. 
Dies ist die Standardform der Parameter�bergabe in C.
% of parameter-passing in which the
%value provided as an argument is copied into the corresponding
%parameter, but the parameter and the argument occupy distinct
%locations.

\item[Call by reference (engl: \emph{call by reference}):]  Eine Art der Parameter�bergabe beim Funktionsaufruf. 
Dabei wird an die aufgerufene Funktion ein Verweis �bergeben. �ber diesen Verweis kann
die aufgerufene Funktion direkt auf Werte au�erhalb ihres Speicherbereichs  zugreifen.
%parameter-passing in which
%the parameter is a reference to the argument variable.  Changes
%to the parameter also affect the argument variable.

\index{Struktur}
\index{\texttt{struct}}
\index{Komponente}
\index{Verweis}
\index{call by value}
\index{call by reference}

\end{description}

\section{�bungsaufgaben}
\setcounter{exercisenum}{0}

\ifthenelse {\boolean{German}}{ \begin{exercise}
%
Im Abschnitt~\ref{Structures as parameters} wird die Funktion {\tt PrintPoint()}
definiert. Der Parameter dieser Funktion wird als Wert (Call-by-value) �bergeben.

�ndern Sie die Definition dieser Funktion, so dass nur eine Referenz auf die
auszugebende Variable �bergeben wird (Call-by-reference).
Testen Sie die neu geschriebene Funktion. 

\end{exercise}

\begin{exercise}
Computerspiele werden erst dadurch interessant, dass die Aktionen ihres
Gegenspielers nicht vorhersagbar sind.
Im Kapitel~\ref{Random numbers} haben wir gesehen wie sich Zufallszahlen
in C erzeugen lassen. 


Schreiben Sie ein kleines Spiel, in dem der Computer eine
beliebige Zahl im Bereich von 1 - 20 ausw�hlt und Sie auffordert die gew�hlte
Zahl zu erraten.

Falls ihre Eingabe kleiner ist als der Zufallswert soll der Computer ausgeben:
'Meine Zahl ist gr��er!' und Sie zu einer erneuten Eingabe auffordern.
F�r den Fall, dass ihre Eingabe gr��er ist soll die Ausgabe 'Meine Zahl ist kleiner!'
lauten.

Damit das Programm bei jedem Versuch mit einem neuen Wert startet, muss der
Zufallszahlengenerator am Anfang des Programms neu initialisiert werden 
(siehe Kapitel~\ref{Random seeds}).
Sie k�nnen dazu die Funktion {\tt time()} verwenden, welche bei jedem Aufruf eine
aktualisierte Anzahl eines Sekundenwerts zur�ckgibt.


\begin{verbatim}
    srand(time(NULL));   /*Initialisierung des Zufallszahlengenerators*/
\end{verbatim}

Haben Sie die Zahl richtig erraten soll der Computer ihnen gratulieren
und die Anzahl der ben�tigten Versuche und den aktuellen 'High-Score' ausgeben.

Der Computer speichert dazu den High-Score (die Anzahl der minimal ben�tigen Versuche)
 in einem {\tt struct} zusammen mit ihrem Namen.

Ist der aktuelle High-Score Wert gr��er als die Anzahl ihrer Versuche 
soll ihr Spielergebnis zusammen mit ihrem Namen als High-Score Wert gespeichert werden. 
Dazu fragt die High-Score Funktion Sie nach ihrem Namen. 

Durch Dr�cken der Taste 'q' soll das Programm beendet werden.


\end{exercise}}
{\input{exercises/Exercise_9_english}}



%!TEX root = Main_german.tex

% LaTeX source for textbook ``How to think like a computer scientist''
% Copyright (C) 1999  Allen B. Downey

% This LaTeX source is free software; you can redistribute it and/or
% modify it under the terms of the GNU General Public License as
% published by the Free Software Foundation (version 2).

% This LaTeX source is distributed in the hope that it will be useful,
% but WITHOUT ANY WARRANTY; without even the implied warranty of
% MERCHANTABILITY or FITNESS FOR A PARTICULAR PURPOSE.  See the GNU
% General Public License for more details.

% Compiling this LaTeX source has the effect of generating
% a device-independent representation of a textbook, which
% can be converted to other formats and printed.  All intermediate
% representations (including DVI and Postscript), and all printed
% copies of the textbook are also covered by the GNU General
% Public License.

% This distribution includes a file named COPYING that contains the text
% of the GNU General Public License.  If it is missing, you can obtain
% it from www.gnu.org or by writing to the Free Software Foundation,
% Inc., 59 Temple Place - Suite 330, Boston, MA 02111-1307, USA.


\chapter{Hardwarenahes Programmieren}
\label{Binary}


\section{Bits and Bytes}
\index{bin�re Darstellung von Werten}


Es ist hin und wieder notwendig, nicht nur auf einzelne Variablen im
Speicher des Computers zuzugreifen, sondern sogar die einzelnen Bits
dieser Werte abzufragen, zu verarbeiten oder zu ver�ndern.
Speziell wenn man mit Microcontrollern arbeitet, um 
Regelungs- und Steuerungsaufgaben zu automatisieren, kann es n�tig
sein, den Zustand angeschlossener Ger�te und Sensoren bitgenau zu kennen, 
auszuwerten und gegebenenfalls zu �ndern. 

C bietet daf�r eine Reihe von sogenannten \emph{Bitoperatoren} die 
allerdings trotz ihres Namens nicht direkt mit einzelnen Bitwerten arbeiten
sondern immer auf mehrere Bits gleichzeitig angewendet werden.
Daran ist der Aufbau moderner Computersysteme schuld, welche die Daten
immer parallel zwischen Prozessor und Hauptspeicher �bertragen und anschlie�end
auch verarbeiten. Es w�re viel zu langsam jedes Bits einzeln aus dem Hauptspeicher 
zu laden und in einer Operation des Prozessors zu verarbeiten.
Weiterhin ist es nicht m�glich einzelne Bits �berhaupt sinnvoll zu adressieren.
Der kleinste direkt adressierbare Speicherbereich ist ein Byte und dieses 
besteht aus einer Folge von 8 Bit.
\index{Bit}
\index{Byte}

Es hat sich eingeb�rgert, die zwei Zust�nde eines einzelnen Bits durch die
numerischen Werte 0 oder 1 darzustellen. Daraus ableitend, kann der Wert
eines Bytes entsprechend der Bit-Wertigkeit im Dualsystem als 
dezimaler Wert interpretiert werden.

So entspricht die folgende Bin�rdarstellung des Byte A dem dezimalen Wert \texttt{105} ($2^6+2^5+2^3+2^0$):

% (int, char, long)
% lsb, msb


\unitlength0.1cm

\begin{picture}(80,12)

\put(10,0){\framebox(7,7){\textbf{\textsf{0}}}}
\put(17,0){\framebox(7,7){\textbf{\textsf{1}}}}
\put(24,0){\framebox(7,7){\textbf{\textsf{1}}}}
\put(31,0){\framebox(7,7){\textbf{\textsf{0}}}}
\put(38,0){\framebox(7,7){\textbf{\textsf{1}}}}
\put(45,0){\framebox(7,7){\textbf{\textsf{0}}}}
\put(52,0){\framebox(7,7){\textbf{\textsf{0}}}}
\put(59,0){\framebox(7,7){\textbf{\textsf{1}}}}

\put(13,8.5){{\scriptsize \texttt{$2^7$}}}
\put(20,8.5){{\scriptsize \texttt{$2^6$}}}
\put(27,8.5){{\scriptsize \texttt{$2^5$}}}
\put(34,8.5){{\scriptsize \texttt{$2^4$}}}
\put(41,8.5){{\scriptsize \texttt{$2^3$}}}
\put(48,8.5){{\scriptsize \texttt{$2^2$}}}
\put(55,8.5){{\scriptsize \texttt{$2^1$}}}
\put(62,8.5){{\scriptsize \texttt{$2^0$}}}

\put(70,1.5){{\large \texttt{Byte \textbf{A}}}}

\end{picture}

\section{Operatoren f�r die Arbeit mit Bits}
\index{Bitperatoren}
\index{Operatoren!bitweise}

Die bin�ren Operatoren �hneln in ihrer Funktion stellenweise
den logischen Operatoren aus Kapitel \ref{Logical Operators}.
So existieren in C die drei {\bf Bitoperatoren} \emph{AND} ({\tt \&}), \emph{OR} ({\tt |}) und \emph{NOT}
(\texttt{\textasciitilde}).  

Im Gegensatz zu den logischen Operatoren, arbeiten die bin�ren Operatoren allerdings 
auf der Ebene der einzelne Bits. So liefert
zum Beispiel der Operator f�r die \emph{bitweise Negation}  \texttt{\textasciitilde}
das folgende Bitmuster:

\begin{picture}(80,20)

\put(10,10){\framebox(7,7){\textbf{\textsf{1}}}}
\put(17,10){\framebox(7,7){\textbf{\textsf{0}}}}
\put(24,10){\framebox(7,7){\textbf{\textsf{1}}}}
\put(31,10){\framebox(7,7){\textbf{\textsf{0}}}}
\put(38,10){\framebox(7,7){\textbf{\textsf{1}}}}
\put(45,10){\framebox(7,7){\textbf{\textsf{0}}}}
\put(52,10){\framebox(7,7){\textbf{\textsf{1}}}}
\put(59,10){\framebox(7,7){\textbf{\textsf{0}}}}

\put(13,18.5){{\scriptsize \texttt{$2^7$}}}
\put(20,18.5){{\scriptsize \texttt{$2^6$}}}
\put(27,18.5){{\scriptsize \texttt{$2^5$}}}
\put(34,18.5){{\scriptsize \texttt{$2^4$}}}
\put(41,18.5){{\scriptsize \texttt{$2^3$}}}
\put(48,18.5){{\scriptsize \texttt{$2^2$}}}
\put(55,18.5){{\scriptsize \texttt{$2^1$}}}
\put(62,18.5){{\scriptsize \texttt{$2^0$}}}

\put(70,11.5){{\large \texttt{Byte \textbf{A}}}}

\put(10,0){\framebox(7,7){\textbf{\textsf{0}}}}
\put(17,0){\framebox(7,7){\textbf{\textsf{1}}}}
\put(24,0){\framebox(7,7){\textbf{\textsf{0}}}}
\put(31,0){\framebox(7,7){\textbf{\textsf{1}}}}
\put(38,0){\framebox(7,7){\textbf{\textsf{0}}}}
\put(45,0){\framebox(7,7){\textbf{\textsf{1}}}}
\put(52,0){\framebox(7,7){\textbf{\textsf{0}}}}
\put(59,0){\framebox(7,7){\textbf{\textsf{1}}}}

\put(70,1.5){{\large \texttt{Byte \texttt{\textasciitilde}\textbf{A}}}}

\end{picture}


An jeder Stelle wo zuvor eine 1 stand, steht jetzt eine 0 und
umgekehrt.

Mit Hilfe des \emph{bitweise AND} Operators \texttt{\&} k�nnen
die Bitfolgen zweier Operanden UND-verkn�pft werden:

\begin{picture}(80,31)

\put(10,20){\framebox(7,7){\textbf{\textsf{1}}}}
\put(17,20){\framebox(7,7){\textbf{\textsf{0}}}}
\put(24,20){\framebox(7,7){\textbf{\textsf{1}}}}
\put(31,20){\framebox(7,7){\textbf{\textsf{0}}}}
\put(38,20){\framebox(7,7){\textbf{\textsf{1}}}}
\put(45,20){\framebox(7,7){\textbf{\textsf{0}}}}
\put(52,20){\framebox(7,7){\textbf{\textsf{1}}}}
\put(59,20){\framebox(7,7){\textbf{\textsf{0}}}}

\put(13,28.5){{\scriptsize \texttt{$2^7$}}}
\put(20,28.5){{\scriptsize \texttt{$2^6$}}}
\put(27,28.5){{\scriptsize \texttt{$2^5$}}}
\put(34,28.5){{\scriptsize \texttt{$2^4$}}}
\put(41,28.5){{\scriptsize \texttt{$2^3$}}}
\put(48,28.5){{\scriptsize \texttt{$2^2$}}}
\put(55,28.5){{\scriptsize \texttt{$2^1$}}}
\put(62,28.5){{\scriptsize \texttt{$2^0$}}}

\put(70,21.5){{\large \texttt{Byte \textbf{A}}}}

\put(10,10){\framebox(7,7){\textbf{\textsf{1}}}}
\put(17,10){\framebox(7,7){\textbf{\textsf{1}}}}
\put(24,10){\framebox(7,7){\textbf{\textsf{1}}}}
\put(31,10){\framebox(7,7){\textbf{\textsf{1}}}}
\put(38,10){\framebox(7,7){\textbf{\textsf{0}}}}
\put(45,10){\framebox(7,7){\textbf{\textsf{0}}}}
\put(52,10){\framebox(7,7){\textbf{\textsf{0}}}}
\put(59,10){\framebox(7,7){\textbf{\textsf{0}}}}

\put(70,11.5){{\large \texttt{Byte \textbf{B}}}}

\put(10,0){\framebox(7,7){\textbf{\textsf{1}}}}
\put(17,0){\framebox(7,7){\textbf{\textsf{0}}}}
\put(24,0){\framebox(7,7){\textbf{\textsf{1}}}}
\put(31,0){\framebox(7,7){\textbf{\textsf{0}}}}
\put(38,0){\framebox(7,7){\textbf{\textsf{0}}}}
\put(45,0){\framebox(7,7){\textbf{\textsf{0}}}}
\put(52,0){\framebox(7,7){\textbf{\textsf{0}}}}
\put(59,0){\framebox(7,7){\textbf{\textsf{0}}}}

\put(70,1.5){{\large \texttt{Byte \textbf{A \& }\textbf{B}}}}

\end{picture}

Wie wir erkennen k�nnen, werden die einzelnen
Bits von Byte A und Byte B UND-verkn�pft.
Das Ergebnis der bitweisen UND-Verkn�pfung 
unterscheidet sich also stark von der logischen 
UND-Verkn�pfung.


Mit Hilfe des \emph{bitweise OR} Operators \texttt{|} k�nnen
die Bitfolgen zweier Operanden ODER-verkn�pft werden:

\begin{picture}(80,31)

\put(10,20){\framebox(7,7){\textbf{\textsf{1}}}}
\put(17,20){\framebox(7,7){\textbf{\textsf{0}}}}
\put(24,20){\framebox(7,7){\textbf{\textsf{1}}}}
\put(31,20){\framebox(7,7){\textbf{\textsf{0}}}}
\put(38,20){\framebox(7,7){\textbf{\textsf{1}}}}
\put(45,20){\framebox(7,7){\textbf{\textsf{0}}}}
\put(52,20){\framebox(7,7){\textbf{\textsf{1}}}}
\put(59,20){\framebox(7,7){\textbf{\textsf{0}}}}

\put(13,28.5){{\scriptsize \texttt{$2^7$}}}
\put(20,28.5){{\scriptsize \texttt{$2^6$}}}
\put(27,28.5){{\scriptsize \texttt{$2^5$}}}
\put(34,28.5){{\scriptsize \texttt{$2^4$}}}
\put(41,28.5){{\scriptsize \texttt{$2^3$}}}
\put(48,28.5){{\scriptsize \texttt{$2^2$}}}
\put(55,28.5){{\scriptsize \texttt{$2^1$}}}
\put(62,28.5){{\scriptsize \texttt{$2^0$}}}

\put(70,21.5){{\large \texttt{Byte \textbf{A}}}}

\put(10,10){\framebox(7,7){\textbf{\textsf{1}}}}
\put(17,10){\framebox(7,7){\textbf{\textsf{1}}}}
\put(24,10){\framebox(7,7){\textbf{\textsf{1}}}}
\put(31,10){\framebox(7,7){\textbf{\textsf{1}}}}
\put(38,10){\framebox(7,7){\textbf{\textsf{0}}}}
\put(45,10){\framebox(7,7){\textbf{\textsf{0}}}}
\put(52,10){\framebox(7,7){\textbf{\textsf{0}}}}
\put(59,10){\framebox(7,7){\textbf{\textsf{0}}}}

\put(70,11.5){{\large \texttt{Byte \textbf{B}}}}

\put(10,0){\framebox(7,7){\textbf{\textsf{1}}}}
\put(17,0){\framebox(7,7){\textbf{\textsf{0}}}}
\put(24,0){\framebox(7,7){\textbf{\textsf{1}}}}
\put(31,0){\framebox(7,7){\textbf{\textsf{1}}}}
\put(38,0){\framebox(7,7){\textbf{\textsf{1}}}}
\put(45,0){\framebox(7,7){\textbf{\textsf{0}}}}
\put(52,0){\framebox(7,7){\textbf{\textsf{1}}}}
\put(59,0){\framebox(7,7){\textbf{\textsf{0}}}}

\put(70,1.5){{\large \texttt{Byte \textbf{A | }\textbf{B}}}}

\end{picture}

\pagebreak
Dar�ber hinaus existiert ein Bitoperator, welcher keine  
Entsprechung als logischer Operator hat. 
Dabei handelt es sich um den \emph{Exklusiv-ODER} oder XOR Operator( \verb|^|).
Er besitzt die folgende Wertetabelle: 

%\noindent
%$
%\displaystyle
\[
\begin{array}{c|c|c}
A & B & A \verb|^| B  \\\hline 
 0 & 0 & 0 \\
 1 & 0 & 1 \\
 0 & 1 & 1 \\
 1 & 1 & 0
\end{array}
\]
%$

Dieser Operator hat die interessante Eigenschaft, dass die zweimalige
Ausf�hrung einer XOR Operation mit dem gleichen Operanden wieder zum
originalen Zustand zur�ckf�hrt:

\begin{picture}(80,41)

\put(10,30){\framebox(7,7){\textbf{\textsf{1}}}}
\put(17,30){\framebox(7,7){\textbf{\textsf{0}}}}
\put(24,30){\framebox(7,7){\textbf{\textsf{1}}}}
\put(31,30){\framebox(7,7){\textbf{\textsf{0}}}}
\put(38,30){\framebox(7,7){\textbf{\textsf{1}}}}
\put(45,30){\framebox(7,7){\textbf{\textsf{0}}}}
\put(52,30){\framebox(7,7){\textbf{\textsf{1}}}}
\put(59,30){\framebox(7,7){\textbf{\textsf{0}}}}

\put(13,38.5){{\scriptsize \texttt{$2^7$}}}
\put(20,38.5){{\scriptsize \texttt{$2^6$}}}
\put(27,38.5){{\scriptsize \texttt{$2^5$}}}
\put(34,38.5){{\scriptsize \texttt{$2^4$}}}
\put(41,38.5){{\scriptsize \texttt{$2^3$}}}
\put(48,38.5){{\scriptsize \texttt{$2^2$}}}
\put(55,38.5){{\scriptsize \texttt{$2^1$}}}
\put(62,38.5){{\scriptsize \texttt{$2^0$}}}

\put(70,31.5){{\large \texttt{Byte \textbf{A}}}}

\put(10,20){\framebox(7,7){\textbf{\textsf{1}}}}
\put(17,20){\framebox(7,7){\textbf{\textsf{1}}}}
\put(24,20){\framebox(7,7){\textbf{\textsf{1}}}}
\put(31,20){\framebox(7,7){\textbf{\textsf{1}}}}
\put(38,20){\framebox(7,7){\textbf{\textsf{0}}}}
\put(45,20){\framebox(7,7){\textbf{\textsf{0}}}}
\put(52,20){\framebox(7,7){\textbf{\textsf{0}}}}
\put(59,20){\framebox(7,7){\textbf{\textsf{0}}}}

\put(70,21.5){{\large \texttt{Byte \textbf{B}}}}

\put(10,10){\framebox(7,7){\textbf{\textsf{0}}}}
\put(17,10){\framebox(7,7){\textbf{\textsf{1}}}}
\put(24,10){\framebox(7,7){\textbf{\textsf{0}}}}
\put(31,10){\framebox(7,7){\textbf{\textsf{1}}}}
\put(38,10){\framebox(7,7){\textbf{\textsf{1}}}}
\put(45,10){\framebox(7,7){\textbf{\textsf{0}}}}
\put(52,10){\framebox(7,7){\textbf{\textsf{1}}}}
\put(59,10){\framebox(7,7){\textbf{\textsf{0}}}}

\put(70,11.5){{\large \texttt{Byte \textbf{A \^{}} \textbf{B}}}}

\put(10,0){\framebox(7,7){\textbf{\textsf{1}}}}
\put(17,0){\framebox(7,7){\textbf{\textsf{0}}}}
\put(24,0){\framebox(7,7){\textbf{\textsf{1}}}}
\put(31,0){\framebox(7,7){\textbf{\textsf{0}}}}
\put(38,0){\framebox(7,7){\textbf{\textsf{1}}}}
\put(45,0){\framebox(7,7){\textbf{\textsf{0}}}}
\put(52,0){\framebox(7,7){\textbf{\textsf{1}}}}
\put(59,0){\framebox(7,7){\textbf{\textsf{0}}}}

\put(70,1.5){{\large \texttt{Byte \textbf{A \^{}} \textbf{B \^{}} \textbf{B}}}}

\end{picture}




\section{Verschiebeoperatoren}
\index{Operatoren!bitweise}

Es ist weiterhin m�glich alle Bits in einer Bitfolge
nach links oder nach rechts zu schieben.

Daf�r kommen die Operatoren \texttt{<<} und \texttt{>>} zum Einsatz.
Der \texttt{<<} Operator schiebt alle Bits entsprechend des Werts \texttt{n}
des zweiten Operanden nach links. Die Operation entspricht
einer bin�ren Multiplikation mit $2^n$. 

\begin{picture}(80,20)

\put(10,10){\framebox(7,7){\textbf{\textsf{0}}}}
\put(17,10){\framebox(7,7){\textbf{\textsf{0}}}}
\put(24,10){\framebox(7,7){\textbf{\textsf{1}}}}
\put(31,10){\framebox(7,7){\textbf{\textsf{0}}}}
\put(38,10){\framebox(7,7){\textbf{\textsf{1}}}}
\put(45,10){\framebox(7,7){\textbf{\textsf{0}}}}
\put(52,10){\framebox(7,7){\textbf{\textsf{1}}}}
\put(59,10){\framebox(7,7){\textbf{\textsf{0}}}}

\put(13,18.5){{\scriptsize \texttt{$2^7$}}}
\put(20,18.5){{\scriptsize \texttt{$2^6$}}}
\put(27,18.5){{\scriptsize \texttt{$2^5$}}}
\put(34,18.5){{\scriptsize \texttt{$2^4$}}}
\put(41,18.5){{\scriptsize \texttt{$2^3$}}}
\put(48,18.5){{\scriptsize \texttt{$2^2$}}}
\put(55,18.5){{\scriptsize \texttt{$2^1$}}}
\put(62,18.5){{\scriptsize \texttt{$2^0$}}}

\put(70,11.5){{\large \texttt{Byte \textbf{A} (42)}}}

\put(10,0){\framebox(7,7){\textbf{\textsf{1}}}}
\put(17,0){\framebox(7,7){\textbf{\textsf{0}}}}
\put(24,0){\framebox(7,7){\textbf{\textsf{1}}}}
\put(31,0){\framebox(7,7){\textbf{\textsf{0}}}}
\put(38,0){\framebox(7,7){\textbf{\textsf{1}}}}
\put(45,0){\framebox(7,7){\textbf{\textsf{0}}}}
\put(52,0){\framebox(7,7){\textbf{\textsf{0}}}}
\put(59,0){\framebox(7,7){\textbf{\textsf{0}}}}

\put(70,1.5){{\large \texttt{Byte \textbf{A << 2 } (168)}}}

\end{picture}


Der \texttt{>>} Operator schiebt alle Bits entsprechend des Werts \texttt{n}
des zweiten Operanden nach rechts. Die Operation entspricht
einer bin�ren Division mit $2^n$. 




\begin{picture}(80,20)

\put(10,10){\framebox(7,7){\textbf{\textsf{0}}}}
\put(17,10){\framebox(7,7){\textbf{\textsf{0}}}}
\put(24,10){\framebox(7,7){\textbf{\textsf{1}}}}
\put(31,10){\framebox(7,7){\textbf{\textsf{1}}}}
\put(38,10){\framebox(7,7){\textbf{\textsf{0}}}}
\put(45,10){\framebox(7,7){\textbf{\textsf{0}}}}
\put(52,10){\framebox(7,7){\textbf{\textsf{0}}}}
\put(59,10){\framebox(7,7){\textbf{\textsf{0}}}}

\put(13,18.5){{\scriptsize \texttt{$2^7$}}}
\put(20,18.5){{\scriptsize \texttt{$2^6$}}}
\put(27,18.5){{\scriptsize \texttt{$2^5$}}}
\put(34,18.5){{\scriptsize \texttt{$2^4$}}}
\put(41,18.5){{\scriptsize \texttt{$2^3$}}}
\put(48,18.5){{\scriptsize \texttt{$2^2$}}}
\put(55,18.5){{\scriptsize \texttt{$2^1$}}}
\put(62,18.5){{\scriptsize \texttt{$2^0$}}}

\put(70,11.5){{\large \texttt{Byte \textbf{A} (48)}}}

\put(10,0){\framebox(7,7){\textbf{\textsf{0}}}}
\put(17,0){\framebox(7,7){\textbf{\textsf{0}}}}
\put(24,0){\framebox(7,7){\textbf{\textsf{0}}}}
\put(31,0){\framebox(7,7){\textbf{\textsf{0}}}}
\put(38,0){\framebox(7,7){\textbf{\textsf{1}}}}
\put(45,0){\framebox(7,7){\textbf{\textsf{1}}}}
\put(52,0){\framebox(7,7){\textbf{\textsf{0}}}}
\put(59,0){\framebox(7,7){\textbf{\textsf{0}}}}

\put(70,1.5){{\large \texttt{Byte \textbf{A >> 2 } (12)}}}

\end{picture}


\section{Anwendung bin�rer Operatoren}


Wenn wir Byte A mit dem negierten Byte A UND-verkn�pfen
ist die resultierende Bitfolge komplett auf 0 gesetzt:

\begin{picture}(80,20)

\put(10,10){\framebox(7,7){\textbf{\textsf{1}}}}
\put(17,10){\framebox(7,7){\textbf{\textsf{0}}}}
\put(24,10){\framebox(7,7){\textbf{\textsf{1}}}}
\put(31,10){\framebox(7,7){\textbf{\textsf{0}}}}
\put(38,10){\framebox(7,7){\textbf{\textsf{1}}}}
\put(45,10){\framebox(7,7){\textbf{\textsf{0}}}}
\put(52,10){\framebox(7,7){\textbf{\textsf{1}}}}
\put(59,10){\framebox(7,7){\textbf{\textsf{0}}}}

\put(13,18.5){{\scriptsize \texttt{$2^7$}}}
\put(20,18.5){{\scriptsize \texttt{$2^6$}}}
\put(27,18.5){{\scriptsize \texttt{$2^5$}}}
\put(34,18.5){{\scriptsize \texttt{$2^4$}}}
\put(41,18.5){{\scriptsize \texttt{$2^3$}}}
\put(48,18.5){{\scriptsize \texttt{$2^2$}}}
\put(55,18.5){{\scriptsize \texttt{$2^1$}}}
\put(62,18.5){{\scriptsize \texttt{$2^0$}}}

\put(70,11.5){{\large \texttt{Byte \textbf{A}}}}

\put(10,0){\framebox(7,7){\textbf{\textsf{0}}}}
\put(17,0){\framebox(7,7){\textbf{\textsf{0}}}}
\put(24,0){\framebox(7,7){\textbf{\textsf{0}}}}
\put(31,0){\framebox(7,7){\textbf{\textsf{0}}}}
\put(38,0){\framebox(7,7){\textbf{\textsf{0}}}}
\put(45,0){\framebox(7,7){\textbf{\textsf{0}}}}
\put(52,0){\framebox(7,7){\textbf{\textsf{0}}}}
\put(59,0){\framebox(7,7){\textbf{\textsf{0}}}}

\put(70,1.5){{\large \texttt{Byte \textbf{A \&} \texttt{\textasciitilde}\textbf{A}}}}

\end{picture}


Wir k�nnen diese Eigenschaft benutzen um aus einer
Bitfolge einzelne Bits zu isolieren. Wir benutzen daf�r
eine sogenannte \emph{Bitmaske}. Das ist eine
Bitfolge, bei der das f�r uns interessante Bit den
Wert 1 und alle nicht interessanten Bits den Wert 
0 besitzen. Das resultierende Ergebnis h�ngt damit
nur vom Zustand des �ber die Maske ausgew�hlten
Bits ab:

\begin{picture}(80,31)

\put(10,20){\framebox(7,7){\textbf{\textsf{1}}}}
\put(17,20){\framebox(7,7){\textbf{\textsf{0}}}}
\put(24,20){\framebox(7,7){\textbf{\textsf{1}}}}
\put(31,20){\framebox(7,7){\textbf{\textsf{0}}}}
\put(38,20){\framebox(7,7){\textbf{\textsf{1}}}}
\put(45,20){\framebox(7,7){\textbf{\textsf{0}}}}
\put(52,20){\framebox(7,7){\textbf{\textsf{1}}}}
\put(59,20){\framebox(7,7){\textbf{\textsf{0}}}}

\put(13,28.5){{\scriptsize \texttt{$2^7$}}}
\put(20,28.5){{\scriptsize \texttt{$2^6$}}}
\put(27,28.5){{\scriptsize \texttt{$2^5$}}}
\put(34,28.5){{\scriptsize \texttt{$2^4$}}}
\put(41,28.5){{\scriptsize \texttt{$2^3$}}}
\put(48,28.5){{\scriptsize \texttt{$2^2$}}}
\put(55,28.5){{\scriptsize \texttt{$2^1$}}}
\put(62,28.5){{\scriptsize \texttt{$2^0$}}}

\put(70,21.5){{\large \texttt{Byte \textbf{A}}}}

\put(10,10){\framebox(7,7){\textbf{\textsf{0}}}}
\put(17,10){\framebox(7,7){\textbf{\textsf{0}}}}
\put(24,10){\framebox(7,7){\textbf{\textsf{1}}}}
\put(31,10){\framebox(7,7){\textbf{\textsf{0}}}}
\put(38,10){\framebox(7,7){\textbf{\textsf{0}}}}
\put(45,10){\framebox(7,7){\textbf{\textsf{0}}}}
\put(52,10){\framebox(7,7){\textbf{\textsf{0}}}}
\put(59,10){\framebox(7,7){\textbf{\textsf{0}}}}

\put(70,11.5){{\large \texttt{Maske $\mathbf{2^5}$ (32)}}}

\put(10,0){\framebox(7,7){\textbf{\textsf{0}}}}
\put(17,0){\framebox(7,7){\textbf{\textsf{0}}}}
\put(24,0){\framebox(7,7){\textbf{\textsf{1}}}}
\put(31,0){\framebox(7,7){\textbf{\textsf{0}}}}
\put(38,0){\framebox(7,7){\textbf{\textsf{0}}}}
\put(45,0){\framebox(7,7){\textbf{\textsf{0}}}}
\put(52,0){\framebox(7,7){\textbf{\textsf{0}}}}
\put(59,0){\framebox(7,7){\textbf{\textsf{0}}}}

\put(70,1.5){{\large \texttt{Byte \textbf{A \& }\textbf{32}}}}

\end{picture}


Wenn wir Byte A mit dem negierten Byte A ODER-verkn�pfen
ist die resultierende Bitfolge komplett auf 1 gesetzt.:

\begin{picture}(80,20)

\put(10,10){\framebox(7,7){\textbf{\textsf{1}}}}
\put(17,10){\framebox(7,7){\textbf{\textsf{0}}}}
\put(24,10){\framebox(7,7){\textbf{\textsf{1}}}}
\put(31,10){\framebox(7,7){\textbf{\textsf{0}}}}
\put(38,10){\framebox(7,7){\textbf{\textsf{1}}}}
\put(45,10){\framebox(7,7){\textbf{\textsf{0}}}}
\put(52,10){\framebox(7,7){\textbf{\textsf{1}}}}
\put(59,10){\framebox(7,7){\textbf{\textsf{0}}}}

\put(13,18.5){{\scriptsize \texttt{$2^7$}}}
\put(20,18.5){{\scriptsize \texttt{$2^6$}}}
\put(27,18.5){{\scriptsize \texttt{$2^5$}}}
\put(34,18.5){{\scriptsize \texttt{$2^4$}}}
\put(41,18.5){{\scriptsize \texttt{$2^3$}}}
\put(48,18.5){{\scriptsize \texttt{$2^2$}}}
\put(55,18.5){{\scriptsize \texttt{$2^1$}}}
\put(62,18.5){{\scriptsize \texttt{$2^0$}}}

\put(70,11.5){{\large \texttt{Byte \textbf{A}}}}

\put(10,0){\framebox(7,7){\textbf{\textsf{1}}}}
\put(17,0){\framebox(7,7){\textbf{\textsf{1}}}}
\put(24,0){\framebox(7,7){\textbf{\textsf{1}}}}
\put(31,0){\framebox(7,7){\textbf{\textsf{1}}}}
\put(38,0){\framebox(7,7){\textbf{\textsf{1}}}}
\put(45,0){\framebox(7,7){\textbf{\textsf{1}}}}
\put(52,0){\framebox(7,7){\textbf{\textsf{1}}}}
\put(59,0){\framebox(7,7){\textbf{\textsf{1}}}}

\put(70,1.5){{\large \texttt{Byte \textbf{A | }\texttt{\textasciitilde}\textbf{A}}}}

\end{picture}


Wir k�nnen diese Eigenschaft benutzen um innerhalb einer
Bitfolge einzelne Bits gezielt zu setzen. 
Indem wir in einer Bitfolge die zu setzenden 
Bits mit dem Wert 1 und alle nicht zu ver�ndernden Bits
mit dem Wert 0 kodieren l�sst sich der Wert
einzelner Bits gezielt setzen:

\begin{picture}(80,31)

\put(10,20){\framebox(7,7){\textbf{\textsf{1}}}}
\put(17,20){\framebox(7,7){\textbf{\textsf{0}}}}
\put(24,20){\framebox(7,7){\textbf{\textsf{1}}}}
\put(31,20){\framebox(7,7){\textbf{\textsf{0}}}}
\put(38,20){\framebox(7,7){\textbf{\textsf{1}}}}
\put(45,20){\framebox(7,7){\textbf{\textsf{0}}}}
\put(52,20){\framebox(7,7){\textbf{\textsf{1}}}}
\put(59,20){\framebox(7,7){\textbf{\textsf{0}}}}

\put(13,28.5){{\scriptsize \texttt{$2^7$}}}
\put(20,28.5){{\scriptsize \texttt{$2^6$}}}
\put(27,28.5){{\scriptsize \texttt{$2^5$}}}
\put(34,28.5){{\scriptsize \texttt{$2^4$}}}
\put(41,28.5){{\scriptsize \texttt{$2^3$}}}
\put(48,28.5){{\scriptsize \texttt{$2^2$}}}
\put(55,28.5){{\scriptsize \texttt{$2^1$}}}
\put(62,28.5){{\scriptsize \texttt{$2^0$}}}

\put(70,21.5){{\large \texttt{Byte \textbf{A}}}}

\put(10,10){\framebox(7,7){\textbf{\textsf{0}}}}
\put(17,10){\framebox(7,7){\textbf{\textsf{0}}}}
\put(24,10){\framebox(7,7){\textbf{\textsf{0}}}}
\put(31,10){\framebox(7,7){\textbf{\textsf{1}}}}
\put(38,10){\framebox(7,7){\textbf{\textsf{0}}}}
\put(45,10){\framebox(7,7){\textbf{\textsf{0}}}}
\put(52,10){\framebox(7,7){\textbf{\textsf{0}}}}
\put(59,10){\framebox(7,7){\textbf{\textsf{0}}}}

\put(70,11.5){{\large \texttt{Setze Bit $\mathbf{2^4}$(16)}}}

\put(10,0){\framebox(7,7){\textbf{\textsf{1}}}}
\put(17,0){\framebox(7,7){\textbf{\textsf{0}}}}
\put(24,0){\framebox(7,7){\textbf{\textsf{1}}}}
\put(31,0){\framebox(7,7){\textbf{\textsf{1}}}}
\put(38,0){\framebox(7,7){\textbf{\textsf{1}}}}
\put(45,0){\framebox(7,7){\textbf{\textsf{0}}}}
\put(52,0){\framebox(7,7){\textbf{\textsf{1}}}}
\put(59,0){\framebox(7,7){\textbf{\textsf{0}}}}

\put(70,1.5){{\large \texttt{Byte \textbf{A | }\textbf{16}}}}

\end{picture}

\pagebreak
\subsection{Variablen f�r Bitfolgen}
\todo{Vorzeichen - unsigned}
\todo{Hexadezimal und einfache Umwandlung}
Bin�re Operatoren lassen sich auf beliebige, ganzzahlige Werte anwenden.
In der Regel werden wir dazu Variablen zu verwenden. 
Daher stellt sich nat�rlich sofort die Frage, welchen Typ soll diese Variable
haben?

Diese Frage ist nicht ganz einfach zu beantworten. Ist die Bitfolge nur
8 Bit lang, so verwenden viele Programmierer den Datentyp \texttt{unsigned char}.


ACHTUNG: Der Datentyp \texttt{char} ist in C ein ganz normaler, ganzzahliger Datentyp der auf den
allermeisten Rechnern 8 Bits enth�lt, es k�nnen aber auch mehr oder weniger Bit sein! 
\hint 

Typischerweise wird in C f�r ganzzahlige Werte das \emph{Most Significant Bit} (MSB)
genutzt, um das Vorzeichen des Werts zu speichern. Das MSB ist das Bit mit der h�chsten Wertigkeit
und findet sich in unserer Bitfolge ganz links.
Bitoperatoren wirken aber immer auf alle Bits der Folge gleichzeitig, was dazu f�hrt, dass eine �nderung
des MSB das Vorzeichen des gespeicherten Werts ver�ndert.
Da das in den allermeisten F�llen nicht sinnvoll ist, lassen sich mit dem Schl�sselwort \texttt{unsigned}
vorzeichenlose Variablen definieren.  
\index{MSB|see{Most Significant Bit}}
\index{Most Significant Bit}
\index{unsigned}

Eine besondere Eigenart von C ist es, dass in der Sprache nicht genau
festgelegt wurde wie viele Bit f�r eine Variable vom Typ \emph{int} 
genutzt werden. 
Es sind immer mindestens 16 Bit, je nach Rechnerarchitektur k�nnen es aber auch 32 oder 64 Bit sein.

Diese Ungenauigkeiten in der Sprachbeschreibung haben dazu gef�hrt, dass in neueren Versionen
der Sprache (ab C99) ganzzahlige Datentypen mit fester Breite eingef�hrt wurden.

In der Bibliothek \texttt{stdint.h} wurden eine Reihe neuer Datentypen mit exakter Bit-Anzahl definiert,
die wir in unserem Programm verwenden k�nnen und die uns davor sch�tzen, dass m�glicherweise unser
Programm auf einem Mikrocontroller zu Fehlern f�hrt, obwohl es auf dem PC wunderbar funktioniert hat.
In der folgenden Tabelle habe ich die wichtigsten Typen zusammengefasst:
\index{<stdint.h>}
\index{Header-Datei!stdint.h}
\index{Bibliothek!stdint.h}

\[
\begin{array}{|c|c|c|}\hline \textbf{L�nge} & \textbf{Typ mit Vorzeichen} & \textbf{Typ ohne Vorzeichen} \\
\hline \textrm{8 Bit} & \texttt{int8\_t} & \texttt{uint8\_t} \\
\hline \textrm{16 Bit} & \texttt{ int16\_t} & \texttt{uint16\_t} \\
\hline \textrm{32 Bit} & \texttt{ int32\_t} & \texttt{uint32\_t} \\
\hline \end{array}
\]



\subsection{Erstellung von Bitmasken}
\index{Bitmasken}
Um gezielt einzelne Bits an einer Speicherstelle ver�ndern zu k�nnen, ben�tigen
wir eine so genannte \emph{Bitmaske}. Das ist nichts weiter als eine Variable oder ein
Wert, dessen Bits an den entsprechenden Stellen gesetzt sind, so dass wir damit
gezielte Ver�nderungen einzelner Bits erreichen k�nnen.

Diese Maske k�nnen wir entweder als konstanten Wert in
unserem Programm speichern, oder bei Bedarf dynamisch erzeugen.
Eine beliebte Methode ist es diese Maske durch die gezielte Verschiebung  
einzelner Bits zu generieren.

Wir hatten bereits das MSB kennen gelernt. Ein weiteres wichtiges Bit
ist das  \emph{Least Significant Bit} (LSB). Das ist das am weitesten rechts
stehende Bit einer Bitfolge. Es hat den Wert $2^{0}$ (dezimal 1) und ist damit das einzige Bit mit einem 
ungeraden Wert.
\index{LSB|see{Least Significant Bit}}
\index{Least Significant Bit}

Um die Bitmaske \texttt{my\_mask} zu erzeugen, in der das Bit $2^{5}$ gesetzt ist, k�nnen wir folgenden
Code verwenden:
\begin{verbatim}
    uint8_t my_mask = (1 << 5);
\end{verbatim}

Um mehrere Bits in der Maske zu setzen, verwenden wir den bitweise-ODER Operator. So lassen
sich zum Beispiel Bit $2^{3}$ und $2^{5}$ in einem Befehl setzen:

\begin{verbatim}
    uint8_t my_mask2 = (1 << 3) | (1 << 5);
\end{verbatim}

Eine weitere interessante M�glichkeit ergibt sich, wenn wir den C-Pr�prozessor benutzen,
um f�r die Verschiebeoperation ein Makro zu definieren. Bisher haben wir  \texttt{\#define} nur
daf�r benutzt um konstante Werte zu definieren.
Es ist aber auch m�glich ein Makro mit Argumenten zu definieren:
\index{\#define}
\index{Makro}
\index{Makro!mit Argumenten}

\begin{verbatim}
   #define BIT(x) (1 << (x))
\end{verbatim}
Es wird   das Makro \texttt{BIT(x)} definiert, bei dem der Wert \texttt{x} von dem im Programm
angegebenen Wert abh�ngt. 
Dadurch k�nnen wird die Lesbarkeit unseres Programms sehr stark verbessern.
So ist der Ausdruck:
\begin{verbatim}
    uint8_t my_mask = BIT(5);
\end{verbatim}

identisch zu folgendem Code:

\begin{verbatim}
    uint8_t my_mask = (1 << 5);
\end{verbatim}

\subsection{Verwendung von Bitmasken}

Um das Bit  $2^{5}$ wieder zu l�schen, benutzen wir den 
bitweisen UND-Operator (\texttt{\&}) gemeinsam mit dem 
NOT-Operator (\texttt{\textasciitilde}).

\begin{verbatim}
    my_mask = my_mask & ~BIT(5);
\end{verbatim}

Dazu invertieren wir den zweiten Operanden mit dem bitweisen NOT-Operator, dass hei�t es sind jetzt alle Bits gesetzt,
au�er $2^{5}$. Anschlie�end f�hren wir eine UND-Verkn�pfung diese Bitfolge mit unserer  Maske durch,
was dazu f�hrt, dass Bit  $2^{5}$ gel�scht wird, w�hrend alle anderen Bits ihren aktuellen Wert beibehalten.


Um ein Bit zwischen zwei Zust�nden wechselweise umschalten zu lassen, benutzen
 wir den XOR-Operator (\verb|^|).

\begin{verbatim}
    my_mask = my_mask ^ BIT(5);
\end{verbatim}

Bit $2^{5}$ wird dadurch invertiert. Hatte es vorher den Wert 1, so ist dieser jetzt 0 und umgekehrt.
Damit kann man zum Beispiel auf einem Mikrocontroller eine einfache Blinkschaltung einer angeschlossenen
LED realisieren.

\pagebreak
Mit dem UND-Operator l�sst sich �berpr�fen, ob ein bestimmtes Bit gesetzt ist:

\begin{verbatim}
    if (my_maks & BIT(5))
    {...}
\end{verbatim}

Dieser Ausdruck ist dann wahr, wenn Bit $2^{5}$ gesetzt ist, anderenfalls ist er falsch.

ACHTUNG: Es ist wichtig sich daran zu erinnern, dass Informatiker beim Z�hlen immer mit der
0 anfangen! Wenn wir das LSB ver�ndern wollen, m�ssen wir 
\texttt{BIT(0)} schreiben. Das ist aber eigentlich ganz logisch, denn dieses 
Bit hat ja den Wert $2^{0}$.
 \hint

\section{Glossar}

\begin{description}

\item[Bit (engl: \emph{bit}):]  Ein Bit (\emph{binary digit}) ist die kleinste Informationseinheit.
Ein Bit kann zwei Zust�nde annehmen, die �blicherweise
durch die Werte 0 und 1 dargestellt werden und die Grundlage
des bin�ren Zahlensystems bilden. In Computern 
werden �blicherweise mehrere Bits gleichzeitig bearbeitet.

\item[Byte (engl: \emph{byte}):] Ein Byte ist �blicherweise eine Folge von 8 Bit und
stellt die kleinste adressierbare Einheit in einem Computersystem
dar. Die Festlegung der Gr��e eines Bytes auf 8 Bit hat historische
Gr�nde, weil sich damit die Buchstaben des lateinischen Alphabets 
sowie Ziffern und Sonderzeichen darstellen lassen.

%\item[Dualsystem]

\index{Bit}
\index{Byte}

\end{description}

\section{�bungsaufgaben}

\begin{exercise}

Es ist folgendes Programm gegeben, welches eine Zahl von der Tastatur einliest und
�berpr�ft, ob es sich um eine gerade oder eine ungerade Zahl handelt:

\begin{verbatim}
  #include <stdio.h>
  #include <stdlib.h>
  
  int main(void)
  {
      int eingabe;
      int test;
  
      printf("Geben Sie eine Zahl ein: ");
      test = scanf("%i",&eingabe);
      if (test == 0)
      {
          printf("Fehler: Es wurde keine Zahl eingegeben\n");
          return EXIT_FAILURE;
      }
  
      if (eingabe & 1) 
      {
          printf("ungerade Zahl\n");
      }
      else printf("gerade Zahl\n");
      return EXIT_SUCCESS;
  }      
\end{verbatim}


\begin{enumerate}
	\item
	Erkl�ren Sie, wie das Programm herausfindet ob die eingegebene Zahl gerade oder ungerade ist.
	\item
	Was passiert wenn der Benutzer anstelle einer \emph{Zahl} einen \emph{Buchstaben} auf der Tastatur eingibt?
	Welche Bedeutung hat dabei die Variable \emph{test} im Programm?
	\item
	Versehen Sie das Programm mit geeigneten Kommentaren f�r sp�tere Entwickler.
\end{enumerate} 



\end{exercise}




%\include{Chapter11}
%\include{Chapter12}
%\include{Chapter13}
%\include{Chapter14}

\appendix
%\include{Append1}
\selectlanguage{ngerman}
\chapter{Guter Programmierstil}
\label{Coding Style}
\section{Eine kurze Stilberatung f�r Programmierer}
\index{Stil}
\index{Programmierstil!guter}

%
%In den letzten �blicherweise
%In the last few sections, I used the phrase ``by convention''
%several times to indicate design decisions that are arbitrary
%in the sense that there are no significant reasons to do things
%one way or another, but dictated by convention.

W�hrend der Besch�ftigung mit der Programmiersprache
C werden Sie feststellen, dass es einige Regeln 
gibt die Sie unbedingt beachten m�ssen, w�hrend andere
Regeln und Designentscheidungen eher als 
eine Art stille �bereinkunft zwischen Programmierern getroffen
werden und als die '�bliche' Art und Weise der Programmierung
angesehen werden.

Viele diese Regeln sind willk�rlich getroffen, trotzdem ist es sinnvoll
und vorteilhaft diese Konventionen zu kennen und sich daran zu halten, weil 
sie ihre Programme f�r Sie und andere einfacher lesbar machen
und ihnen helfen Fehler zu vermeiden.
Es k�nnen im wesentlichen drei Arten von Regeln unterschieden werden:


%In these cases, it is to your advantage to be familiar with
%convention and use it, since it will make your programs easier
%for others to understand.  At the same time, it is important to
%distinguish between (at least) three kinds of rules:

\begin{description}

\item[Naturgesetze:] Diese Art von Regeln beschreiben Prinzipien der Logik und der 
Mathematik und gelten damit ebenfalls f�r Programmiersprachen
wie C (oder andere formale Systems). So ist es zum Beispiel nicht
m�glich die Lage und Gr��e eines Rechtecks in einem Koordinatensystem
durch weniger als vier Angaben genau zu beschreiben. Ein weiteres
Beispiel besagt, dass die Addition von zwei nat�rlichen Zahlen dem
Kommutativgesetz unterliegt. Dieser Zusammenhang ergibt sich 
aus der Definition der Addition und hat nichts mit der Programmiersprache 
C zu tun.

\item[Regeln von C:]  
Jede Programmiersprache definiert syntaktische und semantische
Regeln die nicht verletzt werden d�rfen, da sonst das Programm
nicht korrekt �bersetzt und ausgef�hrt werden kann. 
Einige dieser Regeln sind willk�rlich gew�hlt, wie zum Beispiel 
das {\tt =} Symbol, dass den Zuweisungsoperator darstellt und
 {\em nicht} die Gleichheit der Werte.  Andere Regeln widerspiegeln 
 die zugrundeliegenden Beschr�nkungen des Vorgangs der Kompilation
 und Ausf�hrung des Programms.
 So m�ssen zum Beispiel die Typen der Parameter von Funktionen
 explizit spezifiziert werden.
 
 %default typ int
 

\item[Stil und �bereinkunft:] 
Weiterhin existieren eine Reihe von Regeln die nicht durch den
Compiler vorgegeben oder �berpr�ft werden, die aber trotzdem
wichtig daf�r sind, dass Programme fehlerfrei erstellt
werden, gut lesbar sind und durch Sie selbst und durch andere
modifiziert, getestet und erweitert werden k�nnen. 
Beispiele daf�r sind Einr�ckungen und die Anordnung von 
geschweiften Klammern, sowie Konventionen �ber die Benennung
von Variablen, Funktionen und Typen.

\end{description}

In diesem Abschnitt werde ich kurz den Programmierstil zusammenfassen, 
der in diesem Buch verwendet wird. Er lehnt sich lose an die "Nasa C Style Guide" 
\footnote{www.scribd.com/doc/6878959/NASA-C-programming-guide}
an und das Hauptaugenmerk ist dabei auf die gute Lesbarkeit des Codes
gerichtet. Es kommt weniger darauf an Platz zu sparen oder
den Tippaufwand zu minimieren.

%/ref 
Da C eine - f�r eine Programmiersprache - vergleichsweise lange Geschichte aufweist,
haben sich mehrere verschiedene Programmierstile herausgebildet.
Es ist wichtig, dass Sie diese Stile lesen und verstehen k�nnen und 
dass Sie sich in ihrem eigenen Code auf einen Stil festlegen.
Das macht den Programmcode viel zug�nglicher, sollten es einmal notwendig 
werden, dass Sie den Code mit anderen Programmierern austauschen
oder auf Teile ihres Codes zugreifen wollen, den Sie selbst vor einigen Jahren
geschrieben haben.

\section{Konventionen f�r Namen und Regeln f�r die Gro�- und Kleinschreibung}
\label{Conventions for names}

Als generelle Regel sollten Sie sich angew�hnen bedeutungsvolle Namen f�r ihre
Variablen und Funktionen zu verwenden. Idealerweise k�nnen Sie durch die
Verwendung so genannter \emph{sprechender Bezeichner} f�r Funktionen und
Variablen bereits deren Verhalten und Verwendung erkennen. 

\index{Sprechende Bezeichner}

Auch wenn es vielleicht aufw�ndiger ist eine Funktion {\tt FindSubString()}
anstatt {\tt FStr()} zu nennen, so ist doch der erste Name fast selbsterkl�rend
und kann ihnen eine Menge Zeit bei der Fehlersuche und sp�teren Wiederverwendung
des Programms sparen.   

\textbf{Benutzen Sie keine Variablennamen die nur aus einem Buchstaben bestehen!}

�hnlich wie bei Funktionen sollten Sie die Namen ihrer Programmvariablen 
f�r sich selbst sprechen lassen.
Durch einen geeigneten Namen wird von selbst klar welche Werte
in der Variable gespeichert werden.

%Similarly to functions, you should give your variables names that
%speak for themselves and make clear what values will be stored
%by this variable.

Wie zu jeder guten Regel gibt es auch hier einige Ausnahmen:
Programmierer benutzen �blicherweise {\tt i}, {\tt j} und {\tt k} als Z�hlvariablen in Schleifen
und f�r r�umliche Koordinaten werden {\tt x}, {\tt y} und {\tt z} genutzt.

Benutzen Sie diese Konventionen wenn Sie in ihr Programm passen.
Versuchen Sie nicht eigene, neue Konventionen zu erfinden, die nur Sie
selbst verstehen.

Die folgenden Regeln zur Gro�- und Kleinschreibung sollten Sie f�r
die verschiedenen Elemente in ihrem Programm nutzen.
Durch die einheitliche Verwendung eines Stils k�nnen Sie als
Programmierer und Leser eines Programms sehr schnell die Bedeutung
und Verwendung der verschiedenen Elemente bestimmen.
%
%The following capitalization style shold be used for the different elements in your
%program. The consistent use of one style gives the programmer and the reader
%of the source code a quick way to determine the meaning of different items
%in your program:


\begin{description}
\item[variablenNamen: ]  Namen von Variablen werden immer klein geschrieben.
Zusammengesetzte Namen werden dadurch gekennzeichnet, dass der erste Buchstabe
des folgenden Worts gro� geschrieben wird. 
\item[KONSTANTEN: ] verwenden ausschlie�lich Gro�buchstaben. Um Konflikte mit 
bereits definierten Konstanten aus Bibliotheksfunktionen zu vermeiden kann es
notwendig sein einen Prefix wie zum Beispiel {\tt MY\_CONSTANT} zu verwenden.
\item[FunktionsNamen:] beginnen immer mit einem Gro�buchstaben und sollten nach
M�glichkeit ein Verb enthalten welches die Funktion beschreibt (z.B. {\tt SearchString()}). 
Funktionsnamen f�r Testfunktionen sollten mit '{\tt Is}' oder '{\tt Are}' beginnen (z.B. {\tt IsNumber()}).  
\item[NutzerDefinierteTypen\_t:] enden immer mit einem '{\tt \_t}'. Namen f�r Typen
m�ssen gro� geschrieben werden. Dadurch werden Konflikte mit bereits
definierten POSIX Namen vermieden.
\item[pointerNamen\_p:] um Pointer Variablen sichtbar von anderen Variablen
zu unterscheiden sollten Sie Pointer mit einem '{\tt \_p}' enden lassen.
\end{description}

%%
\section{Klammern und Einr�ckungen}

Die gr��te Vielfalt der Stile finden sich in C bei der Positionierung von 
Klammern und Einr�ckungen.
Deren Hauptaufgabe besteht darin den Code optisch zu gliedern und
funktionale Bereiche durch die konsistente Verwendung von 
Einr�ckungen sichtbar voneinander abzugrenzen.

Die einzelnen Stile unterscheiden sich hierbei in der Art und Weise
wie die Klammern mit dem Rest des Kontrollblocks positioniert und einger�ckt
werden. In diesem Kurs wird der so genannte  \emph{BSD/Allman} Stil verwendet, weil 
er den lesbarsten Code produziert.
Bei diesem Stil nimmt der geschriebene Code mehr horizontalen Raum ein als bei 
dem ebenfalls sehr weit verbreiteten K\&R Stil. 
Der \emph{BSD/Allman} Stil macht es allerdings sehr viel einfacher 
alle �ffnenden und schlie�enden Klammern im Blick zu behalten. 

Im folgenden sehen Sie ein Auflistung verschiedener gebr�uchlicher Klammer- und
Einr�ckungsstile. Die Einr�ckungen betragen immer vier Leerzeichen pro Level:

\begin{verbatim}
/*Whitesmiths Style*/
   if (condition)            
       {
       statement1; 
       statement2;
       }
\end{verbatim}

Der Stil ist nach einem fr�hen kommerziellen C Compiler \emph{Whitesmiths C} benannt,
welcher diesen Stil in seinen Programmbeispielen verwendet hat. 
Die Klammern befinden sich auf dem �u�erem Einr�ckungsniveau.

\begin{verbatim}

/*GNU Style*/
   if (condition)
     {
       statement1;
       statement2;
     }
\end{verbatim}

Die Klammern befinden sich 
in der Mitte zwischen inneren und �u�erem Einr�ckungsniveau.

\begin{verbatim}

/*K&R/Kernel Style*/
   if (condition) {
       statement1;
       statement2;
   }
\end{verbatim}

Dieser Stil wurde nach den Programmierbeispielen des Buchs 
\emph{The C Programming Language} von Brian W. Kernighan und 
Dennis Ritchie (die C-Entwickler) benannt. 

Der K\&R Stil ist am schwersten zu lesen. 
Die �ffnende Klammer befindet sich an der �u�ersten 
rechten Seite der Kontrollanweisung
und ist damit schwer zu finden. Die Klammern haben 
unterschiedliche Einr�ckungstiefen.
Trotzdem ist dieser Stil weit verbreitet und viele C-Programme 
nutzen ihn. Sie sollten deshalb
in der Lage sein diesen Code lesen zu k�nnen.

\begin{verbatim}

/*BSD/Allman Style*/
   if (condition)
   {
       statement1;
       statement2;
   }
\end{verbatim}

Die Klammern befinden sich auf dem inneren Einr�ckungsniveau und sind damit leicht
zu finden und zuzuordnen.
Dieser Stil wird f�r alle Beispiele dieses Kurses verwendet.


Wenn Sie Programme schreiben ist es am Wichtigsten sich auf einen Stil 
festzulegen und diesen Stil dann konsequent beizubehalten.
In gr��eren Softwareprojekten sollten sich alle Mitwirkenden auf einen
gemeinsamen Stil einigen.
Moderne Programmierumgebungen wie zum Beispiel Eclipse\footnote{http://www.eclipse.org} 
und Code::Blocks\footnote{http://www.codeblocks.org/} machen es leicht durch automatische Einr�ckungen
einen Stil durchzusetzen.



\section{Layout}

Kommentarbl�cke k�nnen dazu genutzt werden die Funktion des Programms
zu dokumentieren und zus�tzliche Angaben zum Ersteller zu machen. 
Sinnvollerweise werden diese Angaben als erste Angaben noch vor den
Funktionsdeklarationen vorgenommen. 

Einen �hnlichen Kommentarblock k�nnen Sie vor jeder Funktion verwenden
um deren Funktion zu beschreiben.

\begin{verbatim}

/*
 * File:     test.c
 * Author:   Peter Programmer
 * Date:     May, 29th, 2009
 *
 * Purpose: to demonstrate good programming
 *          practise
 * /

#include <stdio.h>
#include <stdlib.h>

/*
 * Main function, input: none, output: 'HelloWorld' 
  */

int main (void)
{
    printf("Hello World!\n");
    return EXIT_SUCCESS;
}
 
\end{verbatim}
\selectlanguage{english}
\include{ASCII_german}
\clearpage
\renewcommand{\indexname}{Stichwortverzeichnis}
\addcontentsline{toc}{chapter}{Stichwortverzeichnis}
\renewcommand{\seename}{siehe}
\printindex

\end{document}



