

\begin{exercise}

Schreiben Sie eine Funktion {\tt LetterHist()}, welche einen String
als Parameter �bernimmt und Ihnen ein Histogramm der Buchstaben in diesem
String liefert.

Das 'nullte' Element des Histogramms soll die Anzahl der {\tt a}'s 
(gemeinsam f�r Gro�- und Kleinschreibung) in dem String enthalten. Das 25. Element
die Anzahl der {\tt z}'s 

{\bf Zusatzaufgabe:}
Ihre L�sung soll den String nur genau einmal durchsuchen.
\end{exercise}


\begin{exercise}
Es existiert eine bestimmte Anzahl Worte bei denen jeder Buchstabe
genau zwei Mal im Wort vorkommt.

Beispiele aus einem Englisch-W�rterbuch enthalten:
\begin {quote}
Abba, Anna, appall, appearer, appeases, arraigning, beriberi,
bilabial, boob, Caucasus, coco, Dada, deed, Emmett, Hannah,
horseshoer, intestines, Isis, mama, Mimi, murmur, noon, Otto, papa,
peep, reappear, redder, sees, Shanghaiings, Toto
\end{quote}

Schreiben Sie eine Funktion {\tt IsDoubleLetterWord()} welche {\tt TRUE}
zur�ck liefert wenn das �bergebene Wort die oben beschriebene Eigenschaft 
aufweist, ansonsten soll {\tt FALSE} zur�ckgegeben werden.
\end{exercise}

\begin{exercise}

Der R�mische Kaiser Julius C�sar soll seine geheimen Botschaften
mit einem einfachen Verschl�sselungsverfahren gesichert haben.
Dazu hat er in seiner Botschaft jeden Buchstaben durch den Buchstaben
ersetzt, der 3 Positionen weiter hinten im Alphabet zu finden ist.

So wurde zum Beispiel aus {\tt a} ein {\tt d} und aus {\tt b} ein {\tt e}.
Die Buchstaben am Ende des Alphabets werden wieder auf den Anfang
abgebildet. So wird aus {\tt z} dann ein {\tt c}.

\begin{enumerate}
\item Schreiben Sie eine Funktion, welche zwei Strings �bernimmt. Einer
der Strings enth�lt die originale Botschaft, in dem anderen String soll
die verschl�sselte Geheimnachricht gespeichert werden.

Der String kann Gro�- und Kleinschreibung sowie Leerzeichen enthalten.
Andere Satzzeichen (Punkt, Komma, etc.) sollen nicht vorkommen.
Die Funktion soll die Buchstaben vor der Verschl�sselung in eine einheitliche
Darstellung umwandeln (Gro�- oder Kleinschreibung). Leerzeichen werden
nicht verschl�sselt.


\item Generalisieren Sie die Verschl�sselungsfunktion, so dass anstelle
der festen Verschiebung um 3 Positionen, Sie die Verschiebung frei
w�hlen k�nnen.

Sie sollten damit in der Lage sein die Nachrichten auch wieder zu entschl�sseln,
indem Sie z.B. mit dem Wert 13 verschl�sseln und mit -13 wieder entschl�sseln.

\end{enumerate}
\end{exercise}
