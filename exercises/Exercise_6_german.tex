
\begin{exercise}\label{infloop}
%changed the condition on the loop so that it will terminate
%(was this *supposed* to be an infinite loop?)
\begin{verbatim}
    void Loop(int n) 
    {
        int i = n;
        while (i > 1) 
        {
            printf ("%i\n",i);
            if (i%2 == 0) 
            {
                i = i/2;
            } 
            else 
            {
                i = i+1;
            }
        }
    }

    int main (void) 
    {
        Loop(10);
        return EXIT_SUCCESS;
    }
\end{verbatim}
%
\begin{enumerate}

\item Zeichnen Sie eine Tabelle welche die Werte der Variablen {\tt i}
und {\tt n} w�hrend der Ausf�hrung der Funktion {\tt Loop()} zeigen. 
Die Tabelle sollte eine Spalte f�r jede Variable und eine Zeile f�r
jede Iteration der {\tt while}-Schleife enthalten.

\item Was gibt dieses Programm aus?  

\end{enumerate}
\end{exercise}

%String Beispiel

%\begin{exercise}
%\begin{enumerate}

%\item Encapsulate the following code fragment, transforming it
%into a function that takes a String as an argument and that
%returns (and doesn't print) the final value of {\tt count}.

%\item In a sentence, describe abstractly what the resulting
%function does.

%\item Assuming that you have already generalized this
%function so that it works on any String, what else could you do to
%generalize it more?

%\end{enumerate}

%\begin{verbatim}
%    char s[] = "((3 + 7) * 2)";
%    int len = strlen(s);

%    int i = 0;
%    int count = 0;

%    while (i < len) 
%    {
%        char c = s[i]);

%        if (c == '(') 
%        {
%           count = count + 1;
%        } 
%        else if (c == ')') 
%        {
%           count = count - 1;
%        }
%        i = i + 1;
%    }

%    printf ("%i\n", count);
%\end{verbatim}
%\end{exercise}


\begin{exercise}

C stellt in der mathematischen Bibliothek die Funktion {\tt pow()} 
zur Verf�gung, welche die Potenz einer reellen Zahl berechnet. 

Schreiben Sie Ihre eigene Version {\tt Power()} dieser Funktion welche zwei
Parameter:  double {\tt x} und integer {\tt n} �bernimmt und das
Resultat der Berechnung $x^n$ zur�ckliefert.  
Ihre Funktion soll die Berechnung iterativ (mit Hilfe einer Schleife) durchf�hren.
\end{exercise}

\begin{exercise} 
%\textbf{Zusatzaufgabe!}

Angenommen Sie haben eine Zahl  $a$, und Sie wollen die 
Quadratwurzel dieser Zahl ermitteln. 

Eine m�gliche Vorgehensweise besteht darin, dass Sie mit
einer ersten groben Sch�tzung, $x_0$, der Antwort beginnen
und diese Sch�tzung mit Hilfe der folgenden Formel verbessern:

\begin{equation}
x_1 = (x_0 + a/x_0) / 2
\end{equation}

Zum Beispiel, suchen wir die Quadratwurzel von 9. Wir beginnen
mit $x_0 = 6$, dann ergibt sich f�r $x_1 = (6 + 9/6) /2 = 15/4 = 3.75$,
welches n�her an der gesuchten L�sung liegt.

Wir k�nnen das Verfahren wiederholen indem wir $x_1$ benutzen um $x_2$
zu berechnen und so weiter...
In diesem Fall ergibt sich $x_2 = 3.075$ und $x_3 = 3.00091$.
Unsere Berechnung konvergiert sehr schnell hin zu der richtigen Antwort (3).

Schreiben Sie eine Funktion {\tt SquareRoot} welche ein {\tt double}
als Parameter �bernimmt und eine N�herung der Quadratwurzel des 
Parameters zur�ckliefert. 
Die Funktion soll dabei den oben beschriebenen Algorithmus benutzen 
und darf nicht die {\tt sqrt()} Funktion der {\tt math.h} Bibliothek verwenden.

Als erste, initiale N�herung sollten Sie $a/2$ verwenden.  
Ihre Funktion soll die Berechnung wiederholen, bis Sie zwei aufeinanderfolgende
N�herungen erhalten, welche um weniger als 0.0001 voneinander abweichen:
mit anderen Worten, bis der Absolutbetrag von $x_n - x_{n-1}$ geringer ist als
0.0001.  F�r die Berechnung des Absolutbetrags k�nnen Sie die 
{\tt fabs()} Funktion der {\tt math.h} Bibliothek verwenden.
\end{exercise}


%\begin{exercise}
%Section~\ref{factorial} presents a recursive function
%that computes the factorial function.
%Write an iterative version of {\tt factorial}.
%\end{exercise}

%\begin{exercise}
%One way to calculate $e^x$ is to use the infinite series expansion

%\begin{equation}
%e^x = 1 + x + x^2 / 2! + x^3 / 3! + x^4 / 4! + ...
%\end{equation}

%If the loop variable is named {\tt i}, then the $i$th term is equal to
%$x^i / i!$.

%\begin{enumerate}

%\item Write a function called {\tt myexp} that adds up the first {\tt n}
%terms of the series shown above.  You can use the {\tt factorial}
%function from Section~\ref{factorial} or your iterative version.

%\item You can make this function much more efficient if you realize that
%in each iteration the numerator of the term is the same as its
%predecessor multiplied by {\tt x} and the denominator is the same as
%its predecessor multiplied by {\tt i}.  Use this observation to
%eliminate the use of {\tt Math.pow} and {\tt factorial}, and check
%that you still get the same result.

%\item Write a function called {\tt check} that takes a single parameter,
%{\tt x}, and that prints the values of {\tt x}, {\tt Math.exp(x)} and
%{\tt myexp(x)} for various values of {\tt x}.  The output should look
%something like:

%\begin{verbatim}
%1.0     2.708333333333333       2.718281828459045
%\end{verbatim}


%the next line used to use \\ in an attempt to escape the backslash character
%but this doesn't work: \\ produces a newline
%according to the LaTeX symbol list
%http://www.tex.ac.uk/tex-archive/info/symbols/comprehensive/symbols-letter.pdf
%\textbackslash is the actual way to escape a backslash
%(but it seems not to respect the \tt font)


%HINT: you can use the String {\tt "\textbackslash t"} to print a tab character
%between columns of a table.

%\item Vary the number of terms in the series (the second argument
%that {\tt check} sends to {\tt myexp}) and see the effect on
%the accuracy of the result.  Adjust this value until the estimated
%value agrees with the ``correct'' answer when {\tt x} is 1.

%\item Write a loop in {\tt main} that invokes {\tt check} with the
%values 0.1, 1.0, 10.0, and 100.0.  How does the accuracy of the
%result vary as {\tt x} varies?  Compare the number of digits of
%agreement rather than the difference between the actual and
%estimated values.

%\item Add a loop in {\tt main} that checks {\tt myexp} with the values
%-0.1, -1.0, -10.0, and -100.0.  Comment on the accuracy.

%\end{enumerate}
%\end{exercise}

%
%\begin{exercise}
%One way to evaluate $e^{-x^2}$ is to use the infinite series expansion

%\begin{equation}
%e^{-x^2} = 1 - 2x + 3x^2/2! - 4x^3/3! + 5x^4/4! - ...
%\end{equation}

%In other words, we need to add up a series of terms where the $i$th
%term is equal to $(-1)^i(i+1) x^i / i!$.  Write a function named {\tt gauss}
%that takes {\tt x} and {\tt n} as arguments and that returns the sum
%of the first {\tt n} terms of the series.  You should not use {\tt
%factorial} or {\tt pow}.



%%changed the condition on the loop so that it will terminate
%%(was this *supposed* to be an infinite loop?)
%\begin{verbatim}
%    void loop (int n) 
%    {
%        int i = n;
%        while (i > 1) 
%        {
%            printf ("%i\n",i);
%            if (i%2 == 0) 
%            {
%                i = i/2;
%            } 
%            else 
%            {
%                i = i+1;
%            }
%        }
%    }

%    int main (void) 
%    {
%        Loop (10);
%    }
%\end{verbatim}
%%
%\begin{enumerate}

%\item Draw a table that shows the value of the variables {\tt i}
%and {\tt n} during the execution of {\tt loop}.  The table should
%contain one column for each variable and one line for each
%iteration.

%\item What is the output of this program?  

%\end{enumerate}
%\end{exercise}

%
%\begin{exercise}
%\begin{enumerate}

%\item Encapsulate the following code fragment, transforming it
%into a function that takes a String as an argument and that
%returns (and doesn't print) the final value of {\tt count}.

%\item In a sentence, describe abstractly what the resulting
%function does.

%\item Assuming that you have already generalized this
%function so that it works on any String, what else could you do to
%generalize it more?

%\end{enumerate}

%\begin{verbatim}
%    char s[] = "((3 + 7) * 2)";
%    int len = strlen(s);

%    int i = 0;
%    int count = 0;

%    while (i < len) 
%    {
%        char c = s[i]);

%        if (c == '(') 
%        {
%           count = count + 1;
%        } 
%        else if (c == ')') 
%        {
%           count = count - 1;
%        }
%        i = i + 1;
%    }

%    printf ("%i\n", count);
%\end{verbatim}
%\end{exercise}

%
%\begin{exercise}
%Let's say you are given a number, $a$, and you want to find
%its square root.  One way to do that is to start with a very
%rough guess about the answer, $x_0$, and then improve
%the guess using the following formula:

%\begin{equation}
%x_1 = (x_0 + a/x_0) / 2
%\end{equation}

%For example, if we want to find the square root of 9, and
%we start with $x_0 = 6$, then $x_1 = (6 + 9/6) /2 = 15/4 = 3.75$,
%which is closer.

%We can repeat the procedure, using $x_1$ to calculate $x_2$,
%and so on.  In this case, $x_2 = 3.075$ and $x_3 = 3.00091$.
%So that is converging very quickly on the right answer (which
%is 3).

%Write a function called {\tt squareRoot} that takes a {\tt double}
%as a parameter and that returns an approximation of the square
%root of the parameter, using this algorithm.  You may not use
%the {\tt sqrt()} function from the {\tt math.h} library.

%As your initial guess, you should use $a/2$.  Your function should
%iterate until it gets two consecutive estimates that differ by
%less than 0.0001; in other words, until the absolute value of
%$x_n - x_{n-1}$ is less than 0.0001.  You can use the built-in
%{\tt abs()} function to calculate the absolute value.
%\end{exercise}

%
%\begin{exercise}
%In Exercise~\ref{ex.power} we wrote a recursive version of {\tt
%power}, which takes a double {\tt x} and an integer {\tt n} and
%returns $x^n$.  Now write an iterative function to perform the same
%calculation.
%\end{exercise}

%\begin{exercise}
%Section~\ref{factorial} presents a recursive function
%that computes the factorial function.
%Write an iterative version of {\tt factorial}.
%\end{exercise}

%\begin{exercise}
%One way to calculate $e^x$ is to use the infinite series expansion

%\begin{equation}
%e^x = 1 + x + x^2 / 2! + x^3 / 3! + x^4 / 4! + ...
%\end{equation}

%If the loop variable is named {\tt i}, then the $i$th term is equal to
%$x^i / i!$.

%\begin{enumerate}

%\item Write a function called {\tt myexp} that adds up the first {\tt n}
%terms of the series shown above.  You can use the {\tt factorial}
%function from Section~\ref{factorial} or your iterative version.

%\item You can make this function much more efficient if you realize that
%in each iteration the numerator of the term is the same as its
%predecessor multiplied by {\tt x} and the denominator is the same as
%its predecessor multiplied by {\tt i}.  Use this observation to
%eliminate the use of {\tt Math.pow} and {\tt factorial}, and check
%that you still get the same result.

%\item Write a function called {\tt check} that takes a single parameter,
%{\tt x}, and that prints the values of {\tt x}, {\tt Math.exp(x)} and
%{\tt myexp(x)} for various values of {\tt x}.  The output should look
%something like:

%\begin{verbatim}
%1.0     2.708333333333333       2.718281828459045
%\end{verbatim}

%%the next line used to use \\ in an attempt to escape the backslash character
%%but this doesn't work: \\ produces a newline
%%according to the LaTeX symbol list
%%http://www.tex.ac.uk/tex-archive/info/symbols/comprehensive/symbols-letter.pdf
%%\textbackslash is the actual way to escape a backslash
%%(but it seems not to respect the \tt font)
%HINT: you can use the String {\tt "\textbackslash t"} to print a tab character
%between columns of a table.

%\item Vary the number of terms in the series (the second argument
%that {\tt check} sends to {\tt myexp}) and see the effect on
%the accuracy of the result.  Adjust this value until the estimated
%value agrees with the ``correct'' answer when {\tt x} is 1.

%\item Write a loop in {\tt main} that invokes {\tt check} with the
%values 0.1, 1.0, 10.0, and 100.0.  How does the accuracy of the
%result vary as {\tt x} varies?  Compare the number of digits of
%agreement rather than the difference between the actual and
%estimated values.

%\item Add a loop in {\tt main} that checks {\tt myexp} with the values
%-0.1, -1.0, -10.0, and -100.0.  Comment on the accuracy.

%\end{enumerate}
%\end{exercise}

%
%\begin{exercise}
%One way to evaluate $e^{-x^2}$ is to use the infinite series expansion

%\begin{equation}
%e^{-x^2} = 1 - 2x + 3x^2/2! - 4x^3/3! + 5x^4/4! - ...
%\end{equation}

%In other words, we need to add up a series of terms where the $i$th
%term is equal to $(-1)^i(i+1) x^i / i!$.  Write a function named {\tt gauss}
%that takes {\tt x} and {\tt n} as arguments and that returns the sum
%of the first {\tt n} terms of the series.  You should not use {\tt
%factorial} or {\tt pow}.

%\end{exercise}
