\begin{exercise}

%\subsubsection*{Deutsche �bersetzung der Aufgabe}
In dieser �bung sollen Sie das Lesen von Programmcode
praktizieren. Sie sollen den Ablauf der Ausf�hrung von Programmen
mit mehreren Funktionen verstehen und nachvollziehen lernen.


\begin{enumerate}

\item 
Was gibt dieses Programm auf dem Bildschirm aus?
Geben Sie pr�zise an wo sich Leerzeichen und Zeilenumbr�che
befinden. 

HINWEIS: Beginnen Sie mit einer verbalen Beschreibung dessen
was die Funktionen {\tt Ping} und {\tt Baffle} tun, wenn sie aufgerufen
werden.

\begin{verbatim}
#include <stdio.h>
#include <stdlib.h>

  void Ping (void) 
  {
    printf (".\n");
  }

  void Baffle (void) 
  {
    printf ("wug");
    Ping ();
  }

 void Zoop (void) 
 {
    Baffle ();    
    printf ("You wugga ");
    Baffle ();
  }

  int main (void) 
  {
    printf ("No, I ");
    Zoop ();
    printf ("I ");
    Baffle ();
    return EXIT_SUCCESS;
  }
\end{verbatim}


\item 
Zeichnen Sie ein Stackdiagram welches den Status des Programms
wiedergibt wenn {\tt Ping} zum ersten Mal aufgerufen wird.

\end{enumerate}
\end{exercise}

%%%%%%%%%%%%%%%%%%%%%%%%%%%%%%

\begin{exercise}

%\subsubsection*{Deutsche �bersetzung der Aufgabe}

In dieser �bung lernen Sie wie man Funktionen mit Parametern 
schreibt und aufruft.

\begin{enumerate}

\item 
Schreiben Sie die erste Zeile einer Funktion mit dem Namen {\tt Zool}.
Die Funktion hat drei Parameter: ein {\tt int} und zwei {\tt char}.

\item 
Schreiben Sie eine Code-Zeile in der Sie {\tt Zool} aufrufen und
die folgenden Werte als Argumente �bergeben: {\tt 11}, den Buchstaben {\tt a}, und 
den Buchstaben {\tt z}.
\end{enumerate}
\end{exercise}


%%%%%%%%%%%%%%%%%%%%%%%%%%%%%%%


\begin{exercise}

%\subsubsection*{Deutsche �bersetzung der Aufgabe}

In dieser �bung werden wir ein Programm aus einer vorigen �bung anpassen
und ver�ndern, so dass eine Funktion mit Parametern zum Einsatz kommt. Starten
mit einer funktionsf�higen Programmversion.
%~\ref{ex.date}.

\begin{enumerate}

\item 
Schreiben Sie eine Funktion mit dem Namen {\tt PrintDateAmerican}
diese hat die folgenden Parameter day, month und year 
und gibt das Datum im amerikanischen Standardformat aus.

\item 
Testen Sie die Funktion indem Sie diese aus {\tt main} heraus aufrufen
und die entsprechenden Parameter als Argumente �bergeben.
Das Ergebnis sollte folgendem Muster entsprechen:
%
\begin{verbatim}
3/29/2009
\end{verbatim}
%
\item 
Nachdem Sie die Funktion {\tt PrintDateAmerican} erfolgreich erstellt und
ausgef�hrt haben, schreiben Sie eine weitere Funktion 
{\tt PrintDateEuropean} welche das Datum im europ�ischen Format
ausgibt.

\end{enumerate}

\end{exercise}

%%%%%%%%%%%%%%%%%%%%%%%%%%%%%%%

\begin{exercise}
\label{ex.multadd}


%\subsubsection*{Deutsche �bersetzung der Aufgabe}

Viele Berechnungen lassen sich �bersichtlich als  ``multadd''
Operation ausf�hren, dazu wird mit drei Operanden folgende Berechnung
durchgef�hrt {\tt a*b + c}.  Einige Prozessoren bieten f�r diesen 
Befehl sogar eine Hardwareimplementierung f�r Gleitkommazahlen.

\begin{enumerate}

\item 
Erstellen Sie ein neues Programm mit dem Namen {\tt Multadd.c}.

\item 
Schreiben Sie eine Funktion {\tt Multadd} welche drei  {\tt doubles}
als Parameter besitzt und  welche das Ergebnis der Multaddition ausgibt.

\item 
Schreiben Sie eine {\tt main} Funktion welche {\tt Multadd} 
durch den Aufruf mit einigen einfachen Parametern testet
und das Ergebnis ausgibt. 
So sollte zum Beispiel f�r die Parameter {\tt 1.0, 2.0, 3.0} als 
Ergebnis {\tt 5.0} ausgegeben werden.

\item 
Benutzen Sie {\tt Multadd} in der {\tt main} Funktion um den folgenden
Wert zu berechnen:
%
\begin{eqnarray*}
& \sin \frac{\pi}{4} + \frac{\cos \frac{\pi}{4}}{2} & 
%\\
%\\
%& \log 10 + \log 20 &
\end{eqnarray*}
%
\item 
Schreiben Sie eine Funktion {\tt Yikes} welche ein {\tt double} als 
Parameter �bernimmt und {\tt Multadd} f�r die Berechnung und
Ausgabe benutzt:
%
\begin{eqnarray*}
x e^{-x} + \sqrt{1 - e^{-x}}
\end{eqnarray*}
%
HINWEIS: Die mathematische Funktion f�r die Berechnung von  $e^x$ lautet {\tt double exp(double x);}.

\end{enumerate}

In der letzten Aufgabe sollen Sie eine Funktion schreiben, welche ihrerseits
eine selbst erstellte Funktion aufruft. Dabei sollten Sie stets daran denken
die erste Funktion ausgiebig zu testen bevor Sie mit der Arbeit an der 
zweiten Funktion beginnen.
Ansonsten kann es vorkommen, dass Sie gleichzeitig zwei Methoden
debuggen m�ssen - ein sehr m�hsames Unterfangen.

Ein weiteres Ziel dieser �bung ist es ein spezielles Problem als Teil einer
allgemeineren Klasse von Problemen zu erkennen. Wenn immer m�glich sollten
Sie versuchen Programme zu entwickeln, die allgemeine Probleme l�sen.


\end{exercise}