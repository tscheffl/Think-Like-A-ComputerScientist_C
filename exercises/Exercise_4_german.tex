
\begin{exercise}
Der erste Vers des Lieds ``99 Bottles of Beer'' lautet:

\begin{quote}
99 bottles of beer on the wall,
99 bottles of beer,
ya' take one down, ya' pass it around,
98 bottles of beer on the wall.
\end{quote}

Die nachfolgenden Verse sind identisch bis auf die Anzahl
der Flaschen. Deren Anzahl nimmt in jedem Vers um eine Flasche
ab, bis schlie�lich der letzte Vers lautet:

\begin{quote}
No bottles of beer on the wall,
no bottles of beer,
ya' can't take one down, ya' can't pass it around,
'cause there are no more bottles of beer on the wall!
\end{quote}
%
Und dann ist diese Lied schlie�lich zu Ende.

Schreiben Sie ein Programm, welches den gesamten Text 
des Lieds ``99 Bottles of Beer'' ausgibt.
Ihr Programm sollte eine rekursive Funktion f�r die Ausgabe
des Liedtextes verwenden.
Sie k�nnen weitere Funktionen verwenden um ihr Programm
zu strukturieren.

W�hrend Sie den Programmcode schreiben und testen sollten
Sie mit einer kleineren Anzahl von Versen beginnen, z.B. 
``3 Bottles of Beer.''

Der Sinn dieser �bung besteht darin ein Problem zu analysieren und
in kleinere, l�sbare Bestandteile zu zerlegen.
Diese kleineren Einheiten lassen sich unabh�ngig und nacheinander
entwickeln und testen und f�hren im Ergebnis zu einer schnelleren
und robusteren L�sung.
\end{exercise}

\begin{exercise}
In C k�nnen Sie die {\tt getchar()} Funktion benutzen um Zeichen von
der Tastatur einzulesen. Diese Funktion stoppt die Ausf�hrung des
Programms und wartet auf eine Eingabe des Benutzers. Die 
{\tt getchar()} Funktion ist vom Typ {\tt int} und
erfordert kein Argument. Sie liefert den ASCII-Code des eingegeben
Zeichens von der Tastatur zur�ck.

Schreiben Sie ein Programm, welches den Benutzer auffordert eine 
Ziffer von 0-9 einzugeben. 

�berpr�fen Sie die Eingabe des Benutzers und geben Sie 
einen Hinweis aus, falls es sich bei dem eingegeben Wert nicht
um eine Zahl handeln sollte. Geben Sie nach erfolgreicher Pr�fung die
Zahl aus.

% Kapitel 5 (Return)
%Schreiben Sie dazu eine Funktion {\tt AsciiToNumber()} welche ein
%{\tt int} als Typ und als Argument besitzt. �bergeben Sie der Funktion
%den eingelesenen Wert und 

\end{exercise}




\begin{exercise}
Fermat's "Letzter Satz" besagt, dass es keine ganzen Zahlen
$a$, $b$ und $c$ gibt, f�r die gilt

\[a^n + b^n = c^n \]
%
au�er f�r den Fall, dass $n=2$.

Schreiben Sie eine Funktion mit dem Namen {\tt CheckFermat()} 
welche vier {\tt int} als Parameter hat ---{\tt a}, {\tt b}, {\tt c} and {\tt n}--- und
welche �berpr�ft, ob Fermats Satz Bestand hat. Sollte sich f�r
$n$ gr��er als 2 herausstellen, dass $a^n + b^n = c^n$,
dann sollte ihr Programm ausgeben: ``Holy smokes, Fermat was wrong!''
In allen anderen F�llen sollte das Programm ausgeben: ``No, that doesn't work.''

Verwenden Sie f�r die Berechnung der Potenzen die Funktion {\tt pow()} aus der 
mathematischen Bibliothek. Diese Funktion �bernimmt zwei {\tt double} als 
Argument. Das erste Argument stellt dabei die Basis und das zweite Argument den
Exponenten der Potenz dar. Die Funktion liefert als Ergebnis wiederum ein {\tt double}.

Um die Funktion in unserem Programm nutzen zu k�nnen m�ssen die Datentypen
angepasst werden (siehe Abschnitt \ref{typecasting}). Dabei wandelt C den Datentyp
 {\tt int}  automatisch in  {\tt double} um. Um einen  {\tt double}  Wert in  {\tt int} zu wandeln
 muss  der Typecast-Operator  {\tt (int)} verwendet werden.  

Zum Beispiel:

\begin{verbatim}
    int x = (int) pow(2, 3);
\end{verbatim}
%
weist  {\tt x} den Wert {\tt 8} zu, weil $2^3 = 8$.
\end{exercise}

