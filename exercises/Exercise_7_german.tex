\begin{exercise}
Schreiben Sie eine  Funktion namens {\tt CheckFactors(int,~int[], int)} mit 3 Parametern.
Die Funktion �bernimmt einen Integerwert {\tt n}, ein Array von Integerwerten
sowie die L�nge des Arrays {\tt len} als drittes Argument. 

Die Funktion  soll {\tt TRUE} zur�ckliefern, falls alle Zahlen in dem �bergebenen Array 
 Faktoren von {\tt n} sind 
(d.h.   {\tt n} durch alle diese Zahlen teilbar ist).
F�r den Fall, dass mindestens eines der Array-Elemente kein Faktor von 
{\tt n} ist soll {\tt FALSE} zur�ckgegeben werden.

HINWEIS: Ermitteln Sie vor dem Aufruf der Funktion {\tt CheckFactors()} die L�nge des Arrays in der {\tt main()} Funktion, siehe dazu \ref{Array length}.
Vergleichen Sie ebenfalls die L�sung der �bungsaufgabe~\ref{ex.isdiv}.
\end{exercise}

%%%%%%%%%%%%%%%%%%%%%%%%%%%%%%%%%%%
\begin{exercise}
Schreiben Sie eine Funktion  {\tt void SetToZero(int[], int)}
welche ein Array von  {\tt int} und die L�nge dieses Arrays �bernimmt und 
anschlie�end dieses Array f�r alle Elemente auf
den Wert {\tt 0} initialisiert.

F�r die Ermittlung der L�nge des Arrays k�nnen Sie die Funktion aus dem
Abschnitt~\ref{Array length} �bernehmen.
Testen Sie die korrekte Implementierung dieser Funktion mit Hilfe
der {\tt PrintArray()} Funktion aus Abschnitt~\ref{Array of random numbers}.
\end{exercise}

%%%%%%%%%%%%%%%%%%%%%%%%%%%%%%%%%%%
\begin{exercise}
Schreiben Sie eine Funktion welche ein Array von {\tt int}, 
die L�nge des Arrays {\tt len} und 
einen {\tt int}  mit dem Namen
{\tt target}  als Argumente �bernimmt.  Die Funktion soll das
Array durchsuchen und den Index zur�ckliefert an dem
{\tt target} zum ersten Mal in dem Array auftritt. Sollte {\tt target} 
nicht in dem Array enthalten sein soll -1 zur�ckgegeben werden.
\end{exercise}

%%%%%%%%%%%%%%%%%%%%%%%%%%%%%%%%%%%
\begin{exercise}
%\textbf{Zusatzaufgabe!}

One not-very-efficient way to sort the elements of an array
is to find the largest element and swap it with the first
element, then find the second-largest element and swap it with
the second, and so on.

\begin{enumerate}

\item Write a function called {\tt IndexOfMaxInRange()} that 
takes an array of integers, finds the
largest element in the given range, and returns {\em its index}.

\item Write a function called {\tt SwapElement()} that takes an
array of integers and two indices, and that swaps the elements
at the given indices.

\item Write a function called {\tt SortArray()} that takes an array of
integers and that uses {\tt IndexOfMaxInRange()} and {\tt SwapElement()}
to sort the array from largest to smallest.

\end{enumerate}
\end{exercise}
