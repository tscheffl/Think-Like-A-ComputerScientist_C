\begin{exercise}
Schreiben Sie eine  Funktion namens {\tt CheckFactors(int,~int[], int)} mit 3 Parametern.
Die Funktion �bernimmt einen Integerwert {\tt n}, ein Array von Integerwerten
sowie die L�nge des Arrays {\tt len} als drittes Argument. 

Die Funktion  soll {\tt TRUE} zur�ckliefern, falls alle Zahlen in dem �bergebenen Array 
 Faktoren von {\tt n} sind 
(d.h.   {\tt n} durch alle diese Zahlen teilbar ist).
F�r den Fall, dass mindestens eines der Array-Elemente kein Faktor von 
{\tt n} ist soll {\tt FALSE} zur�ckgegeben werden.

HINWEIS: Ermitteln Sie vor dem Aufruf der Funktion {\tt CheckFactors()} die L�nge des Arrays in der {\tt main()} Funktion, siehe dazu \ref{Array length}.
Vergleichen Sie ebenfalls die L�sung der �bungsaufgabe~\ref{ex.isdiv}.
\end{exercise}

%%%%%%%%%%%%%%%%%%%%%%%%%%%%%%%%%%%
\pagebreak
\begin{exercise}
Schreiben Sie eine Funktion  {\tt void SetToZero(int[], int)}
welche ein Array von  {\tt int} und die L�nge dieses Arrays �bernimmt und 
anschlie�end dieses Array f�r alle Elemente auf
den Wert {\tt 0} initialisiert.

F�r die Ermittlung der L�nge des Arrays k�nnen Sie die Funktion aus dem
Abschnitt~\ref{Array length} �bernehmen.
Testen Sie die korrekte Implementierung dieser Funktion mit Hilfe
der {\tt PrintArray()} Funktion aus Abschnitt~\ref{Array of random numbers}.
\end{exercise}

%%%%%%%%%%%%%%%%%%%%%%%%%%%%%%%%%%%
\begin{exercise}
Schreiben Sie eine Funktion welche ein Array von {\tt int}, 
die L�nge des Arrays {\tt len} und 
einen {\tt int}  mit dem Namen
{\tt target}  als Argumente �bernimmt.  Die Funktion soll das
Array durchsuchen und den Index zur�ckliefert an dem
{\tt target} zum ersten Mal in dem Array auftritt. Sollte {\tt target} 
nicht in dem Array enthalten sein soll -1 zur�ckgegeben werden.
\end{exercise}

%%%%%%%%%%%%%%%%%%%%%%%%%%%%%%%%%%%
\begin{exercise}
%\textbf{Zusatzaufgabe!}

Seit der Erfindung der Computer werden diese genutzt um Arrays 
mit Daten zu sortieren. Unz�hlige Algorithmen wurden entworfen und
hinsichtlich ihrer Effizienz verglichen. 

Eine nicht-besonders-effizienter Algorithmus l�uft folgenderma�en
ab: Finde das gr��te Element im Array und tausche es mit dem ersten Element.
Finde das zweit-gr��te Element im Array und tausche es mit dem zweiten Element,
und so weiter...

\begin{enumerate}

\item Schreiben Sie eine Funktion {\tt IndexOfMaxInRange()}, welche  
ein Array von ganzen Zahlen (integers) �bernimmt und das
gr��te Element in einem bestimmten Bereich (range) des Arrays findet und seine Position als {\em index} zur�ckliefert.

\item Schreiben Sie eine Funktion {\tt SwapElement()}, welche ein
Array von ganzen Zahlen und zwei Indexe �bernimmt und anschlie�end die Werte der Elemente an
den gegebenen Indexen vertauscht.

\item Schreiben Sie eine Funktion {\tt SortArray()}, welche ein Array von ganzen Zahlen
�bernimmt und die Funktionen {\tt IndexOfMaxInRange()} und {\tt SwapElement()} benutzt
um das Array vom gr��ten zum kleinsten Wert zu sortieren.

\end{enumerate}
\end{exercise}
