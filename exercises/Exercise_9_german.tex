\begin{exercise}
%
Im Abschnitt~\ref{Structures as parameters} wird die Funktion {\tt PrintPoint()}
definiert. Der Parameter dieser Funktion wird als Wert (Call-by-value) �bergeben.

�ndern Sie die Definition dieser Funktion, so dass nur eine Referenz auf die
auszugebende Variable �bergeben wird (Call-by-reference).
Testen Sie die neu geschriebene Funktion. 

\end{exercise}

\begin{exercise}
Computerspiele werden erst dadurch interessant, dass die Aktionen ihres
Gegenspielers nicht vorhersagbar sind.
Im Kapitel~\ref{Random numbers} haben wir gesehen wie sich Zufallszahlen
in C erzeugen lassen. 


Schreiben Sie ein kleines Spiel, in dem der Computer eine
beliebige Zahl im Bereich von 1 - 20 ausw�hlt und Sie auffordert die gew�hlte
Zahl zu erraten.

Falls ihre Eingabe kleiner ist als der Zufallswert soll der Computer ausgeben:
'Meine Zahl ist gr��er!' und Sie zu einer erneuten Eingabe auffordern.
F�r den Fall, dass ihre Eingabe gr��er ist soll die Ausgabe 'Meine Zahl ist kleiner!'
lauten.

Damit das Programm bei jedem Versuch mit einem neuen Wert startet, muss der
Zufallszahlengenerator am Anfang des Programms neu initialisiert werden 
(siehe Kapitel~\ref{Random seeds}).
Sie k�nnen dazu die Funktion {\tt time()} verwenden, welche bei jedem Aufruf eine
aktualisierte Anzahl eines Sekundenwerts zur�ckgibt.


\begin{verbatim}
    srand(time(NULL));   /*Initialisierung des Zufallszahlengenerators*/
\end{verbatim}

Haben Sie die Zahl richtig erraten soll der Computer ihnen gratulieren
und die Anzahl der ben�tigten Versuche und den aktuellen 'High-Score' ausgeben.

Der Computer speichert dazu den High-Score (die Anzahl der minimal ben�tigen Versuche)
 in einem {\tt struct} zusammen mit ihrem Namen.

Ist der aktuelle High-Score Wert gr��er als die Anzahl ihrer Versuche 
soll ihr Spielergebnis zusammen mit ihrem Namen als High-Score Wert gespeichert werden. 
Dazu fragt die High-Score Funktion Sie nach ihrem Namen. 

Durch Dr�cken der Taste 'q' soll das Programm beendet werden.


\end{exercise}