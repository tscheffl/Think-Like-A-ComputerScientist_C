\begin{exercise}
\label{ex.date}


%\paragraph*{Deutsche �bersetzung der Aufgabe:}

\begin{enumerate}

\item 
Erstellen Sie ein neues Programm mit dem Namen {\tt MyDate.c}.
Kopieren Sie dazu die Struktur des "Hello, World"-Programms 
und stellen Sie sicher, dass Sie dieses kompilieren und ausf�hren k�nnen. 

\item
Folgen Sie dem Beispiel in Abschnitt ~\ref{output variables} und definieren 
Sie in dem Programm die folgenden Variablen: {\tt day}, {\tt month}
und {\tt year}.
{\tt day} enth�lt den Tag des Monats, {\tt month} den Monat und {\tt year} das Jahr.
Von welchem Typ sind diese Variablen? 

Weisen Sie den Variablen
Werte zu, welche dem heutigen Datum entsprechen. 

\item
Geben Sie die Werte auf dem Bildschirm aus. Stellen Sie jeden Wert auf einer
eigenen Bildschirmzeile dar. Das ist ein Zwischenschritt, der ihnen dabei hilft
zu �berpr�fen, ob das Programm funktionsf�hig ist.

\item
Modifizieren Sie das Programm dahingehend, dass es das Datum im amerikanischen
Standardformat darstellt: {\tt mm/dd/yyyy}.

\item
Modifizieren Sie das Programm erneut, um eine Ausgabe nach folgendem Muster 
zu erzeugen:
\begin{verbatim}
American format:
3/18/2009
European format:
18.3.2009
\end{verbatim}

\end{enumerate}

Diese �bung soll Ihnen dabei helfen, formatierte Ausgaben von 
Werten unterschiedlicher Datentypen mittels der  {\tt printf} Funktion zu
erzeugen. Weiterhin sollen Sie die kontinuierliche Entwicklung von komplexen
Programmen durch das schrittweise Hinzuf�gen von einigen, wenigen 
Anweisungen erlernen.

\end{exercise}


\begin{exercise}


%\paragraph*{Deutsche �bersetzung der Aufgabe:}
\begin{enumerate}

\item 
Erstellen Sie ein neues Programm mit dem Namen {\tt MyTime.c}. 
In den nachfolgenden Aufgaben werde ich Sie nicht mehr daran
erinnern mit einem kleinen, funktionsf�higen Programm zu beginnen.
Allerdings sollten Sie dieses auch weiterhin tun.

\item 
Folgen Sie dem Beispiel im Abschnitt~\ref{operators} und erstellen
Sie Variablen mit dem Namen {\tt hour}, {\tt minute} und {\tt second}.
Weisen Sie den Variablen Werte zu, welche in etwa der 
aktuellen Zeit entsprechen. 
Benutzen Sie dazu das 24-Stunden Zeitformat.

\item 
Das Programm soll die Anzahl der Sekunden seit Mitternacht berechnen.

\item 
Das Programm soll die Anzahl der noch verbleibenden Sekunden
des Tages berechnen und ausgeben.

\item 
Das Programm soll berechnen, wieviel Prozent des Tages bereits verstrichen
sind und diesen Wert ausgeben.

\item 
Ver�ndern Sie die Werte von  {\tt hour}, {\tt minute} und {\tt second},
um die aktuelle Zeit wiederzugeben. 
�berpr�fen Sie, ob das Programm mit unterschiedlichen Werten korrekt
arbeitet.

\end{enumerate}

In dieser �bung f�hren Sie arithmetische Operationen
durch und beginnen dar�ber nachzudenken, wie 
komplexere Datenobjekte, wie z.B. die Uhrzeit,
als Zusammensetzung von mehreren Werten dargestellt werden
k�nnen.
Weiterhin entdecken Sie m�glicherweise Probleme, die sich aus der Darstellung
und Berechnung mit dem ganzzahligen Datentypen {\tt int} ergeben (Prozentberechnung).
Diese Probleme k�nnen mit der Verwendung von Flie�kommazahlen
umgangen werden (siehe n�chstes Kapitel).  


HINWEIS: Sie k�nnen weitere Variablen benutzen, um
Zwischenergebnisse der Berechnung abzulegen.
Diese Variablen, welche in einer Berechnung genutzt, aber niemals
ausgegeben werden, bezeichnet man auch als tempor�re Variablen.

\end{exercise}
