\begin{exercise}

Sie haben 3 St�cke erhalten und stehen vor der Aufgabe daraus ein
Dreieck zu formen. Diese Aufgabe kann l�sbar oder unl�sbar sein,
je nachdem wie lang die zur Verf�gung stehenden St�cke sind.

Wenn zum Beispiel einer der St�cke 12cm lang ist und die anderen 
Beiden je nur 2cm, so ist klar, dass diese sich nicht in der Mitte treffen
werden.
Es gibt einen einfachen Test, der f�r drei beliebige L�ngen ermittelt,
ob sich ein Dreieck formen l�sst oder nicht: 


\begin{quotation}
``Wenn eine der drei L�ngen gr��er ist als die Summe der
anderen beiden, dann l�sst sich kein Dreieck formen. Ansonsten
ist es m�glich ein Dreieck zu formen.''
\end{quotation}

Schreiben Sie eine Funktion mit dem Namen {\tt IsTriangle}, welche
drei {\tt integer} als Argumente hat und entweder {\tt TRUE} or {\tt FALSE}
zur�ckgibt, abh�ngig davon, ob sich aus St�cken mit der gegebenen L�nge
ein Dreieck formen l�sst oder nicht. 

Der Sinn dieser �bung bestehet darin eine Funktion mit bedingten Abfragen
zu schreiben, welche als Ergebnis einen Wert zur�ckgibt. 
\end{exercise}
%%%%%%%%%%%%%%%%%%%%%%%%%%%%%%%%%%%%%%%%
\begin{exercise}
\label{ex.isdiv}
Schreiben Sie eine Funktion {\tt IsDivisible} welche zwei {\tt integer} Werte,  {\tt n} and {\tt m} als 
Argumente hat und {\tt TRUE} zur�ckgibt, wenn {\tt n} durch {\tt m} teilbar ist. 
Ansonsten soll die Funktion {\tt FALSE} zur�ckgeben.
\end{exercise}
%%%%%%%%%%%%%%%%%%%%%%%%%%%%%%%%%%%%%%%%
\begin{exercise}
Der Sinn der folgenden �bung besteht darin
das Verst�ndnis f�r die Ausf�hrung logischer Operatoren zu sch�rfen und
den Programmablauf in Funktionen mit R�ckgabewerten nachvollziehbar zu machen.
Wie lautet die Ausgabe des folgenden Programms?

\begin{verbatim}
#define TRUE 1
#define FALSE 0

  short IsHoopy (int x)
   {
      short hoopyFlag;
      if (x%2 == 0) 
      {
          hoopyFlag = TRUE;
      } 
      else 
      {
          hoopyFlag = FALSE;
      }
      return hoopyFlag;
  }

  short IsFrabjuous (int x) 
  {
      short frabjuousFlag;
      if (x > 0) 
      {
          frabjuousFlag = TRUE;
      }
      else 
      {
          frabjuousFlag = FALSE;
      }
      return frabjuousFlag;
  }

  int main (void) 
  {
      short flag1 = IsHoopy (202);
      short flag2 = IsFrabjuous (202);
      printf ("%i\n", flag1);
      printf ("%i\n", flag2);
      if (flag1 && flag2) 
      {
          printf ("ping!\n");
      }
      if (flag1 || flag2) 
      {
          printf ("pong!\n");
      }
      return EXIT_SUCCESS;
  }
\end{verbatim}
\end{exercise}
%%%%%%%%%%%%%%%%%%%%%%%%%%%%%%%%%%%%%%%%

\begin{exercise}
Die Entfernung zwischen zwei Punkten  $(x_1, y_1)$ und $(x_2, y_2)$
ist

\[Distance = \sqrt{(x_2 - x_1)^2 + (y_2 - y_1)^2} \]

Schreiben Sie bitte eine Funktion {\tt Distance} welche vier
{\tt double} als Argumente erh�lt---{\tt x1}, {\tt y1}, {\tt x2} und {\tt
y2}---und welche die Entfernung zwischen den Punkten  $(x_1, y_1)$ und $(x_2, y_2)$ zur�ckgibt.

Sie sollen annehmen, dass bereits eine Funktion mit dem Namen {\tt SumSquares}
existiert, welche die Quadrate der Summen berechnet und zur�ckgibt.

Zum Beispiel:

\begin{verbatim}
    double x = SumSquares (3.0, 4.0);
\end{verbatim}
%
w�rde {\tt x} den Wert {\tt 25.0} zuweisen.

Der Sinn dieser �bung besteht darin eine neue Funktion zu schreiben,
welche eine bereits bestehende Funktion aufruft.
Sie sollen nur die eine Funktion {\tt Distance} schreiben.
Lassen Sie die Funktionen {\tt SumSquares} und {\tt main} weg und rufen
Sie {\tt Distance} auch nicht auf!
\end{exercise}

%\begin{exercise}
%The point of this exercise is to practice the syntax of fruitful
%functions.

%\begin{enumerate}

%\item Use your existing solution to Exercise~\ref{ex.multadd} and make sure
%you can still compile and run it.

%\item Transform {\tt multadd} into a fruitful function, so
%that instead of printing a result, it returns it.

%\item Everywhere in the program that {\tt multadd} gets
%invoked, change the invocation so that it stores the
%result in a variable and/or prints the result.

%\item Transform {\tt yikes} in the same way.

%\end{enumerate}
%\end{exercise}


%\begin{exercise}
%The point of this exercise is to use a stack diagram to understand
%the execution of a recursive program.

%\begin{verbatim}

%    int main (void) 
%    {
%        printf (Prod (1, 4));
%    }

%    int Prod (int m, int n) 
%    {
%        if (m == n) 
%        {
%            return n;
%        } 
%        else 
%        {
%            int recurse = Prod (m, n-1);
%            int result = n * recurse;
%            return result;
%        }
%    }

%\end{verbatim}
%%
%\begin{enumerate}

%\item Draw a stack diagram showing the state of the program just
%before the last instance of {\tt prod} completes.
%What is the output of this program?

%\item Explain in a few words what {\tt Prod} does.

%\item Rewrite {\tt prod} without using the temporary variables
%{\tt recurse} and {\tt result}.

%\end{enumerate}
%\end{exercise}


%\begin{exercise}
%The purpose of this exercise is to translate a recursive definition
%into a C function.  The Ackerman function is defined for non-negative
%integers as follows:

%\begin{eqnarray*}
%A(m,n) = \left\{
%\begin{array}{l@{\quad \mathbf{if} \quad }l}
%n+1 & m=0 \\
%A(m-1, 1) & m>0, n=0 \\
%A(m-1, A(m, n-1)) & m>0, n>0 \\
%\end{array} \right.
%\end{eqnarray*}
%%
%Write a function called {\tt ack} that takes two {\tt int}s as
%parameters and that computes and returns the value
%of the Ackerman function.

%Test your implementation of Ackerman by invoking it
%from {\tt main} and printing the return value.  

%WARNING: the return value gets very big very quickly.  You should try it
%only for small values of $m$ and $n$ (not bigger than 2).

%\end{exercise}

%
%\begin{exercise}
%\begin{enumerate}

%\item Create a program called {\tt Recurse.c} and
%type in the following functions:

%\begin{verbatim}
%    /* First: returns the first character of the given String */
%    char First (String s) 
%    {
%        return s.charAt (0);
%    }

%    /* Rest: returns a new String that contains all but the */
%    /* first letter of the given String */
%    String Rest (String s) 
%    {
%        return s.substring (1, s.length());
%    }

%    /* Length: returns the length of the given String */
%    int Length (String s) 
%    {
%        return s.length();
%    }
%\end{verbatim}
%
%\item Write some code in {\tt main} that tests each of these
%functions.  Make sure they work, and make sure you understand
%what they do.

%\item Write a function called {\tt printString} that takes a
%String as a parameter and that prints the letters of the
%String, one on each line.  It should be a {\tt void} function.

%\item Write a function called {\tt printBackward} that does
%the same thing as {\tt printString} but that prints the String
%backwards (one character per line).

%\item Write a function called {\tt reverseString} that takes
%a String as a parameter and that returns a new String as a
%return value.  The new String should contain the same letters
%as the parameter, but in reverse order.  For example, the
%output of the following code

%\begin{verbatim}
%	String backwards = reverseString ("Allen Downey");
%	System.out.println (backwards);
%\end{verbatim}
%%
%should be

%\begin{verbatim}
%yenwoD nellA
%\end{verbatim}

%
%\end{enumerate}
%\end{exercise}

%%%%%%%%%%%%%%%%%%%%%%%%%%%%%%%%%%%%%%%%
\vskip 1em
\begin{exercise}
%\textbf{!! 1. ZUSATZAUFGABE !!}

Erstellen Sie eine neue Programmdatei mit dem Namen {\tt Sum.c},
und geben Sie die folgenden zwei Funktionen ein:

\begin{verbatim}
  int FunctionOne (int m, int n) 
  {
      if (m == n) 
      {
          return n;
      } 
      else 
      {
          return m + FunctionOne (m+1, n);
      }
  }

  int FunctionTwo (int m, int n) 
  {
      if (m == n) 
      {
          return n;
      } 
      else 
      {
          return n * FunctionTwo (m, n-1);
      }
  }
\end{verbatim}
%
\begin{enumerate}

\item F�gen Sie in den Funktionen eine {\tt prinf()}-Anweisung 
hinzu, mit der Sie sofort nach dem Funktionsaufruf den aktuellen
Wert der Funktionsparameter ausgeben. 
Das ist eine n�tzliche Technik um rekursive Programme zu debuggen,
da sich auf diese Weise die Abarbeitung eines Programms besser nachvollziehen
l�sst.
%
%functions so that they print their arguments each time they are
%invoked.  This is a useful technique for debugging recursive
%programs, since it demonstrates the flow of execution.


\item Schreiben Sie in der {\tt main}-Funktion ihres Programms einige
Zeilen um diese Funktionen zu testen (rufen Sie die Funktionen einige
Male mit unterschiedlichen Argumenten auf und lassen Sie sich die R�ckgabewerte
ausgeben, um zu sehen, was die Funktionen machen.

Nutzen Sie eine Kombination aus gezieltem Testen und der Inspektion
des Quellcodes um herauszufinden, was diese Funktionen machen.
Geben Sie den Funktionen einen neuen Namen, aus dem besser hervorgeht, was
die Funktionen machen. F�gen Sie Kommentare zu den Funktionen hinzu um
ihre Funktion allgemeinverst�ndlich zu beschreiben.


\end{enumerate}
\end{exercise}

%\begin{exercise}
%\label{ex.power}
%Write a recursive function called {\tt Power} that
%takes a double {\tt x} and an integer {\tt n} and that
%returns $x^n$.  Hint: a recursive definition of this
%operation is {\tt Power (x, n) = x * Power (x, n-1)}.
%Also, remember that anything raised to the zeroeth power
%is 1.
%\end{exercise}

%%%%%%%%%%%%%%%%%%%%%%%%%%%%%%%%%%%%%%%%
\begin{exercise}
%\textbf{!! 2. ZUSATZAUFGABE !!}

\label{gcd}
(This exercise is based on page 44 of Ableson and Sussman's
{\em Structure and Interpretation of Computer Programs}.)

The following algorithm is known as Euclid's Algorithm because
it appears in Euclid's {\em Elements} (Book 7, ca. 300 B.C.).
It may be the oldest nontrivial algorithm.

The algorithm is based on the observation that, if $r$ is the
remainder when $a$ is divided by $b$, then the common divisors
of $a$ and $b$ are the same as the common divisors of $b$ and $r$.
Thus we can use the equation

\[ gcd (a, b) = gcd (b, r) \]

to successively reduce the problem of computing a GCD to the
problem of computing the GCD of smaller and smaller pairs of integers.
For example,

\[ gcd (36, 20) = gcd (20, 16) = gcd (16, 4) = gcd (4, 0) = 4\]

implies that the GCD of 36 and 20 is 4.  It can be shown
that for any two starting numbers, this repeated reduction eventually
produces a pair where the second number is 0.  Then the GCD is the
other number in the pair.

Write a function called {\tt gcd} that takes two integer parameters and
that uses Euclid's algorithm to compute and return the greatest
common divisor of the two numbers.
\end{exercise}


%\begin{exercise}
%If you are given three sticks, you may or may not be able to arrange
%them in a triangle.  For example, if one of the sticks is 12 inches
%long and the other two are one inch long, it is clear that you will
%not be able to get the short sticks to meet in the middle.  For any
%three lengths, there is a simple test to see if it is possible to form
%a triangle:

%\begin{quotation}
%``If any of the three lengths is greater than the sum of the other two,
%then you cannot form a triangle.  Otherwise, you can.''
%\end{quotation}

%Write a function named {\tt IsTriangle} that it takes three integers as
%arguments, and that returns either {\tt true} or {\tt false},
%depending on whether you can or cannot form a triangle from sticks
%with the given lengths.

%%return values only at next chapter!!!
%The point of this exercise is to use conditional statements to
%write a function that returns a value.
%\end{exercise}

%\begin{exercise}
%\label{ex.isdiv}
%Write a function named {\tt isDivisible} that takes
%two integers, {\tt n} and {\tt m} and that returns {\tt true}
%if {\tt n} is divisible by {\tt m} and {\tt false} otherwise.
%\end{exercise}

%\begin{exercise}
%What is the output of the following program?  The purpose of
%this exercise is to make sure you understand logical operators
%and the flow of execution through fruitful functions.

%\begin{verbatim}
%  int main (void) 
%  {
%      short flag1 = isHoopy (202);
%      short flag2 = isFrabjuous (202);
%      printf ("%i\n", flag1);
%      printf ("%i\n", flag2);
%      if (flag1 && flag2) {
%          printf ("ping!");
%      }
%      if (flag1 || flag2) {
%          printf ("pong!");
%      }
%  }

%  public static boolean isHoopy (int x) {
%      boolean hoopyFlag;
%      if (x%2 == 0) {
%          hoopyFlag = true;
%      } else {
%          hoopyFlag = false;
%      }
%      return hoopyFlag;
%  }

%  public static boolean isFrabjuous (int x) {
%      boolean frabjuousFlag;
%      if (x > 0) {
%          frabjuousFlag = true;
%      } else {
%          frabjuousFlag = false;
%      }
%      return frabjuousFlag;
%  }
%\end{verbatim}

%\end{exercise}

%

%\begin{exercise}
%The distance between two points $(x_1, y_1)$ and $(x_2, y_2)$
%is

%\[Distance = \sqrt{(x_2 - x_1)^2 + (y_2 - y_1)^2} \]

%Please write a function named {\tt Distance} that takes four
%doubles as parameters---{\tt x1}, {\tt y1}, {\tt x2} and {\tt
%y2}---and that prints the distance between the points.

%You should assume that there is a function named {\tt sumSquares}
%that calculates and returns the sum of the squares of its arguments.
%For example:

%\begin{verbatim}
%    double x = SumSquares (3.0, 4.0);
%\end{verbatim}
%%
%would assign the value {\tt 25.0} to {\tt x}.

%The point of this exercise is to write a new function that uses an
%existing one.  You should write only one function: {\tt Distance}.  You
%should not write {\tt SumSquares} or {\tt main} and you should not
%invoke {\tt Distance}.
%\end{exercise}

%\begin{exercise}
%The point of this exercise is to practice the syntax of fruitful
%functions.

%\begin{enumerate}

%\item Use your existing solution to Exercise~\ref{ex.multadd} and make sure
%you can still compile and run it.

%\item Transform {\tt multadd} into a fruitful function, so
%that instead of printing a result, it returns it.

%\item Everywhere in the program that {\tt multadd} gets
%invoked, change the invocation so that it stores the
%result in a variable and/or prints the result.

%\item Transform {\tt yikes} in the same way.

%\end{enumerate}
%\end{exercise}

%
%\begin{exercise}
%The point of this exercise is to use a stack diagram to understand
%the execution of a recursive program.

%\begin{verbatim}

%    int main (void) 
%    {
%        printf (Prod (1, 4));
%    }

%    int Prod (int m, int n) 
%    {
%        if (m == n) 
%        {
%            return n;
%        } 
%        else 
%        {
%            int recurse = Prod (m, n-1);
%            int result = n * recurse;
%            return result;
%        }
%    }

%\end{verbatim}
%%
%\begin{enumerate}

%\item Draw a stack diagram showing the state of the program just
%before the last instance of {\tt prod} completes.
%What is the output of this program?

%\item Explain in a few words what {\tt Prod} does.

%\item Rewrite {\tt prod} without using the temporary variables
%{\tt recurse} and {\tt result}.

%\end{enumerate}
%\end{exercise}

%
%\begin{exercise}
%The purpose of this exercise is to translate a recursive definition
%into a C function.  The Ackerman function is defined for non-negative
%integers as follows:

%\begin{eqnarray*}
%A(m,n) = \left\{
%\begin{array}{l@{\quad \mathbf{if} \quad }l}
%n+1 & m=0 \\
%A(m-1, 1) & m>0, n=0 \\
%A(m-1, A(m, n-1)) & m>0, n>0 \\
%\end{array} \right.
%\end{eqnarray*}
%%
%Write a function called {\tt ack} that takes two {\tt int}s as
%parameters and that computes and returns the value
%of the Ackerman function.

%Test your implementation of Ackerman by invoking it
%from {\tt main} and printing the return value.  

%WARNING: the return value gets very big very quickly.  You should try it
%only for small values of $m$ and $n$ (not bigger than 2).

%\end{exercise}

%
%\begin{exercise}
%\begin{enumerate}

%\item Create a program called {\tt Recurse.c} and
%type in the following functions:

%\begin{verbatim}
%    /* First: returns the first character of the given String */
%    char First (String s) 
%    {
%        return s.charAt (0);
%    }

%    /* Rest: returns a new String that contains all but the */
%    /* first letter of the given String */
%    String Rest (String s) 
%    {
%        return s.substring (1, s.length());
%    }

%    /* Length: returns the length of the given String */
%    int Length (String s) 
%    {
%        return s.length();
%    }
%\end{verbatim}

%\item Write some code in {\tt main} that tests each of these
%functions.  Make sure they work, and make sure you understand
%what they do.

%\item Write a function called {\tt printString} that takes a
%String as a parameter and that prints the letters of the
%String, one on each line.  It should be a {\tt void} function.

%\item Write a function called {\tt printBackward} that does
%the same thing as {\tt printString} but that prints the String
%backwards (one character per line).

%\item Write a function called {\tt reverseString} that takes
%a String as a parameter and that returns a new String as a
%return value.  The new String should contain the same letters
%as the parameter, but in reverse order.  For example, the
%output of the following code

%\begin{verbatim}
%	String backwards = reverseString ("Allen Downey");
%	System.out.println (backwards);
%\end{verbatim}
%%
%should be

%\begin{verbatim}
%yenwoD nellA
%\end{verbatim}

%
%\end{enumerate}
%\end{exercise}

%\begin{exercise}
%\begin{enumerate}

%\item Create a new program called {\tt Sum.c},
%and type in the following two functions.

%\begin{verbatim}
%  int FunctionOne (int m, int n) 
%  {
%      if (m == n) 
%      {
%          return n;
%      } 
%      else 
%      {
%          return m + FunctionOne (m+1, n);
%      }
%  }

%  int FunctionTwo (int m, int n) 
%  {
%      if (m == n) 
%      {
%          return n;
%      } 
%      else 
%      {
%          return n * FunctionTwo (m, n-1);
%      }
%  }
%\end{verbatim}
%%
%\item Write a few lines in {\tt main} to test these functions.
%Invoke them a couple of times, with a few different values,
%and see what you get.  By some combination of testing and
%examination of the code, figure out what these functions do,
%and give them more meaningful names.  Add comments that
%describe their function abstractly.

%\item Add a {\tt prinf} statement to the beginning of both
%functions so that they print their arguments each time they are
%invoked.  This is a useful technique for debugging recursive
%programs, since it demonstrates the flow of execution.

%\end{enumerate}
%\end{exercise}

%\begin{exercise}
%\label{ex.power}
%Write a recursive function called {\tt Power} that
%takes a double {\tt x} and an integer {\tt n} and that
%returns $x^n$.  Hint: a recursive definition of this
%operation is {\tt Power (x, n) = x * Power (x, n-1)}.
%Also, remember that anything raised to the zeroeth power
%is 1.
%\end{exercise}

%
%\begin{exercise}
%\label{gcd}
%(This exercise is based on page 44 of Ableson and Sussman's
%{\em Structure and Interpretation of Computer Programs}.)

%The following algorithm is known as Euclid's Algorithm because
%it appears in Euclid's {\em Elements} (Book 7, ca. 300 B.C.).
%It may be the oldest nontrivial algorithm.

%The algorithm is based on the observation that, if $r$ is the
%remainder when $a$ is divided by $b$, then the common divisors
%of $a$ and $b$ are the same as the common divisors of $b$ and $r$.
%Thus we can use the equation

%\[ gcd (a, b) = gcd (b, r) \]

%to successively reduce the problem of computing a GCD to the
%problem of computing the GCD of smaller and smaller pairs of integers.
%For example,

%\[ gcd (36, 20) = gcd (20, 16) = gcd (16, 4) = gcd (4, 0) = 4\]

%implies that the GCD of 36 and 20 is 4.  It can be shown
%that for any two starting numbers, this repeated reduction eventually
%produces a pair where the second number is 0.  Then the GCD is the
%other number in the pair.

%Write a function called {\tt gcd} that takes two integer parameters and
%that uses Euclid's algorithm to compute and return the greatest
%common divisor of the two numbers.
%\end{exercise}
